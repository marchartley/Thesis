\LetLtxMacro\OldChapter\chapter
\LetLtxMacro\OldSection\section
\LetLtxMacro\OldSubsection\subsection
\LetLtxMacro\OldSubsubsection\subsubsection

% Define \zzcommand for conditional commands
\def\zzcommand#1{\let#1\undefined\newcommand#1}

% \WarningFilter{hyperref}{Token not allowed in a PDF string} % Remove the warnings about \gls in titles or math functions in titles.

\LetLtxMacro{\OldTilde}{\Tilde}
\zzcommand{\Tilde}[1]{
  \StrLen{#1}[\temp]%
  \ifnum\temp>1
    \ThisStyle{%
      \setbox0=\hbox{$\SavedStyle#1$}%
      \stackengine{-.1\LMpt}{$\SavedStyle#1$}{%
        \stretchto{\scaleto{\SavedStyle\mkern.2mu\AC}{.5150\wd0}}{.6\ht0}%
      }{O}{c}{F}{T}{S}%
    }
  \else 
    \OldTilde{#1}
  \fi
}


\LetLtxMacro{\Oldincludegraphics}{\includegraphics}
\RenewDocumentCommand{\includegraphics}{O{} m}{%
	\begin{adjustbox}{max width=\textwidth, max height=\textheight-5\baselineskip}
	    \Oldincludegraphics[#1]{#2}
	\end{adjustbox}
}
\newcommand{\inset}[3][0.2\linewidth]{
    \noindent\stackinset{r}{0.1cm}{b}{0.1cm}{
        \colorbox{white}{
            \includegraphics[width=#1]{#3}
        }
    }
	{\includegraphics{#2}}
}

\newcommand{\teaser}[1]{
    \begin{figure*}[ht]
        \centering
        #1
    \end{figure*}
}

\def\minitoc{
	\etocsettocstyle{\section*{\contentsname}}{}
	\localtableofcontents
	\noindent\textbf{\hyperlink{tocpage}{Back to summary}}
}
% \makeatletter\@addtoreset{chapter}{part}\makeatother%

\sisetup{product-units=repeat}

\emergencystretch=1em

% \setkeys{Gin}{width=\linewidth}

\renewcommand{\cellalign}{cl}
\newcolumntype{H}{>{\setbox0=\hbox\bgroup}c<{\egroup}@{}}

\DeclareCiteCommand{\cite}[\mkbibparens]
  {\usebibmacro{prenote}\usebibmacro{cite:init}}
  {\usebibmacro{citeindex}
  %  \printtext[bibhyperref]{[\thefield{entrykey}] \usebibmacro{cite}}}
  \printtext[bibhyperref]{\usebibmacro{cite}}}
  {\multicitedelim}
  {\usebibmacro{postnote}}

\newbibmacro*{citefullbody}{
  \printnames{labelname}
  \setunit{\addcomma\space}
  \textit{\printfield[]{title}} % remove quotes
  % \textit{\printfield[citetitle]{title}}
  % \setunit{\nameyeardelim}
  (\printfield{year})
}

\DeclareCiteCommand{\citep}
{\usebibmacro{prenote}\usebibmacro{cite:init}}
{\usebibmacro{citeindex}
\printtext[bibhyperref]{\usebibmacro{citefullbody}}}
{\multicitedelim}
{\usebibmacro{postnote}}


\DeclareCiteCommand{\citeProgram}[\mkbibparens]
  {\usebibmacro{prenote}\usebibmacro{cite:init}}
  {\usebibmacro{citeindex}
   \printtext[bibhyperref]{\mkbibemph{\printfield{title}}, \printdate}}
  {\multicitedelim}
  {\usebibmacro{postnote}}

\AtEveryBibitem{\clearfield{url}}
\AtEveryBibitem{\clearfield{note}} % Notes from Mendeley
\newtotcounter{citenum}
\AtEveryBibitem{\stepcounter{citenum}}

\setlength\bibitemsep{1.5\itemsep}


\newenvironment{Itemize}
  {\begin{itemize}[label={}, leftmargin=0em, itemindent=3.0em, itemsep=0em, topsep=0em]} %, left=2em, itemsep=0.5em]} % Custom settings
  {\end{itemize}}
% Define new command \Item that applies italics to the title only
\zzcommand{\Item}[1]{\item \textit{#1}}

\lstset {
    language=C++,
    backgroundcolor=\color{black!5}, % set backgroundcolor
    basicstyle=\footnotesize,% basic font setting
}

\makeglossaries
\DeclareDocumentCommand{\newdualentry}{ O{} O{} m m m m g } {
    \newglossaryentry{gls-#3}{
        name={#5},
        text={#5\if\relax\detokenize{#4}\relax\else\glsadd{#3}\fi},
        plural={\IfValueTF{#7}{#7}{#5s}},
        longplural={\IfValueTF{#7}{#7}{#5s}},
        description={#6},
        #1
    }
    \if\relax\detokenize{#4}\relax
    % nothing
  \else
    \newacronym[longplural={\IfValueTF{#7}{#7}{#5s}}, 
        see={[Glossary:]{gls-#3}},
        #2]{#3}{#4}{#5\glsadd{gls-#3}}
  \fi
    \makeglossaries
}



\zzcommand{\gloss}[1]{\ifglsentryexists{gls-#1}{\glsentryname{gls-#1}\ifcsdef{inheading}{}{{\glsadd{gls-#1}}}}{[XXXX]}}
\zzcommand{\glosses}[1]{\ifglsentryexists{gls-#1}{\glsentrylongpl{gls-#1}\ifcsdef{inheading}{}{{\glsadd{gls-#1}}}}{[XXXX]}}
\zzcommand{\Gloss}[1]{\ifglsentryexists{gls-#1}{\Glsentryname{gls-#1}\ifcsdef{inheading}{}{{\glsadd{gls-#1}}}}{[XXXX]}}
\zzcommand{\Glosses}[1]{\ifglsentryexists{gls-#1}{\Glsentrylongpl{gls-#1}\ifcsdef{inheading}{}{{\glsadd{gls-#1}}}}{[XXXX]}}
\zzcommand{\glsnamefont}[1]{\makefirstuc{#1}}



% \zzcommand{\gloss}[1]{\glsentryname{gls-#1}} % \ifcsdef{inheading}{\glsentryname{gls-#1}}{\ifglsentryexists{#1}{\gls{#1}\glsadd{gls-#1}}{\gls{gls-#1}}}}
% \zzcommand{\glosses}[1]{\glsentrylongpl{gls-#1}} % \ifcsdef{inheading}{\glsentrylongpl{gls-#1}}{\ifglsentryexists{#1}{\glspl{#1}\glsadd{gls-#1}}{\glspl{gls-#1}}}}
% \zzcommand{\Gloss}[1]{\Glsentryname{gls-#1}} % \ifcsdef{inheading}{\Glsentryname{gls-#1}}{\ifglsentryexists{#1}{\Gls{#1}\glsadd{gls-#1}}{\Gls{gls-#1}}}}
% \zzcommand{\Glosses}[1]{\Glsentrylongpl{gls-#1}} % \ifcsdef{inheading}{\Glsentrylongpl{gls-#1}}{\ifglsentryexists{#1}{\Glspl{#1}\glsadd{gls-#1}}{\Glspl{gls-#1}}}}
% \zzcommand{\glsnamefont}[1]{\makefirstuc{#1}}

% \let\oldchapter\chapter
% \renewcommand{\chapter}[1]{
%   \begingroup
%   \newcommand{\inheading}{}
%   \oldchapter{#1}
%   \endgroup
% }

\renewcommand{\eqref}[1]{\cref{#1}}

\zzcommand{\to}{\mapsto}


\zzcommand{\subsubsubsection}[1]{\bigskip\noindent\textit{#1}\nopagebreak\par}
\zzcommand{\subsubsubsubsection}[1]{\bigskip\noindent\textbf{#1}~}

% \wrapFig{img_path}{width}{label}{caption}
\zzcommand{\wrapFig}[4]{
  \wrapFigL{#1}{#2}{#3}{#4}
}

\zzcommand{\wrapFigLR}[5]{
  \def\tempwidth{#2}%
  \ifx\tempwidth\empty
    \def\tempwidth{0.25}%
  \fi
  \begin{wrapfigure}{#5}{\tempwidth \linewidth}
    \includegraphics[width=\linewidth]{#1}
    \expandafter\ifx\expandafter\relax\detokenize{#4}\relax\else\caption{#4}\fi
    \expandafter\ifx\expandafter\relax\detokenize{#3}\relax\else\label{#3}\fi
  \end{wrapfigure}
}

\zzcommand{\wrapFigL}[4]{
  \wrapFigLR{#1}{#2}{#3}{#4}{L}
}
\zzcommand{\wrapFigR}[4]{
  \wrapFigLR{#1}{#2}{#3}{#4}{R}
}

% \zzcommand{\break}{
%     % {\centering $*$~$*$~$*$ \par}
%     \bigskip\bigskip\bigskip
% }

\zzcommand{\abstract}
{
    \section*{Abstract}
}

\zzcommand{\shortAbstract}[1]
{
  % \bigskip\noindent\textbf{Abstract} 
  \noindent
  % \par
  \textit{#1}
  \bigskip
}