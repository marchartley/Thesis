\LetLtxMacro\OldChapter\chapter
\LetLtxMacro\OldSection\section
\LetLtxMacro\OldSubsection\subsection
\LetLtxMacro\OldSubsubsection\subsubsection

\hbadness = 99999

% Define \zzcommand for conditional commands
\def\zzcommand#1{\let#1\undefined\newcommand#1}

% \tensor{expression}
\zzcommand{\tensor}[1]{\mathbf{#1}}
% \mtrx{matrix name}
\zzcommand{\mtrx}[1]{\mathbf{#1}}
% \field{field name}
\zzcommand{\field}[1]{\mathbf{#1}}

% \WarningFilter{hyperref}{Token not allowed in a PDF string} % Remove the warnings about \gls in titles or math functions in titles.

\LetLtxMacro{\OldTilde}{\Tilde}
\zzcommand{\Tilde}[1]{
  \StrLen{#1}[\temp]%
  \ifnum\temp>1
    \ThisStyle{%
      \setbox0=\hbox{$\SavedStyle#1$}%
      \stackengine{-.1\LMpt}{$\SavedStyle#1$}{%
        \stretchto{\scaleto{\SavedStyle\mkern.2mu\AC}{.5150\wd0}}{.6\ht0}%
      }{O}{c}{F}{T}{S}%
    }
  \else 
    \OldTilde{#1}
  \fi
}


\LetLtxMacro{\Oldincludegraphics}{\includegraphics}
\RenewDocumentCommand{\includegraphics}{O{} m}{%
	\begin{adjustbox}{max width=\textwidth, max height=\textheight-5\baselineskip}
	    \Oldincludegraphics[#1]{#2}
	\end{adjustbox}
}
% \inset{inset size}{path to main image}{path to inset image}
\zzcommand{\inset}[3][0.2\linewidth]{
    \noindent\stackinset{r}{0.1cm}{b}{0.1cm}{
        \colorbox{white}{
            \includegraphics[width=#1]{#3}
        }
    }
	{\includegraphics{#2}}
}

\let\origfigure\figure
\let\endorigfigure\endfigure

\renewenvironment{figure}[1][tbph]{%
    \origfigure[#1]%
    \centering
}{%
    \endorigfigure
}

% \teaser{/* Inside commands of the figure* environment */}
\zzcommand{\teaser}[1]{
    \begin{figure*}[ht]
        \centering
        #1
    \end{figure*}
}

% \minitoc
\def\minitoc{
	\etocsettocstyle{\section*{\contentsname}}{}
	\localtableofcontents
	\noindent\textbf{\hyperlink{tocpage}{Back to summary}}
}
% \makeatletter\@addtoreset{chapter}{part}\makeatother%

\sisetup{product-units=repeat}

\emergencystretch=1em

% \setkeys{Gin}{width=\linewidth}

\zzcommand{\cellalign}{cl}
\newcolumntype{H}{>{\setbox0=\hbox\bgroup}c<{\egroup}@{}}

\DeclareCiteCommand{\cite}[\mkbibparens]
  {\usebibmacro{prenote}\usebibmacro{cite:init}}
  {\usebibmacro{citeindex}%
  \printtext[bibhyperref]{\usebibmacro{cite}}}
  {\multicitedelim}
  {\usebibmacro{postnote}}

\newbibmacro*{citefullbody}{
  \printnames{labelname}%
  \setunit{\addcomma\space}%
  \textit{\printfield[]{title}}~(\printfield{year})
}

\DeclareCiteCommand{\citep}
{\usebibmacro{prenote}\usebibmacro{cite:init}}
{\usebibmacro{citeindex}%
\printtext[bibhyperref]{\usebibmacro{citefullbody}}}
{\multicitedelim}
{\usebibmacro{postnote}}


\DeclareCiteCommand{\citeProgram}[\mkbibparens]
  {\usebibmacro{prenote}\usebibmacro{cite:init}}
  {\usebibmacro{citeindex}\printtext[bibhyperref]{\mkbibemph{\printfield{title}},~\printdate}}
  {\multicitedelim}
  {\usebibmacro{postnote}}

\AtEveryBibitem{\clearfield{url}}
\AtEveryBibitem{\clearfield{note}} % Notes from Mendeley
\newtotcounter{citenum}
\AtEveryBibitem{\stepcounter{citenum}}

\setlength\bibitemsep{1.5\itemsep}

\newcounter{AltTextImageCurrentSide}

\newenvironment{Itemize}
{    
  \setcounter{AltTextImageCurrentSide}{0}
  \begin{itemize}[label={}, leftmargin=0em, itemindent=3.0em, itemsep=0em, topsep=0em] %, left=2em, itemsep=0.5em]} % Custom settings
}
{\end{itemize}}

% Define new command \Item that applies italics to the title only
% \Item{text}
\zzcommand{\Item}[1]{\item \textit{#1}}

% \AltTextImageR{item content}{image path}{caption}{label}
\zzcommand{\AltTextImageR}[4]{% 
  \setcounter{AltTextImageCurrentSide}{1}
  \begin{minipage}[t]{0.65\linewidth}
    #1
  \end{minipage}%
  \hfill % Add horizontal space between text and image
  \begin{minipage}[t]{0.30\linewidth}
    \includegraphics[width=\linewidth, valign=t]{#2}
    \captionof{figure}{#3}
    \label{#4}
  \end{minipage}%
}
% \AltTextImageL{item content}{image path}{caption}{label}
\zzcommand{\AltTextImageL}[4]{% 
  \setcounter{AltTextImageCurrentSide}{0}
  \begin{minipage}[t]{0.30\linewidth}
    \includegraphics[width=\linewidth, valign=t]{#2}
    \captionof{figure}{#3}
    \label{#4}
  \end{minipage}%
  \hfill % Add horizontal space between text and image
  \begin{minipage}[t]{0.65\linewidth}
      #1
  \end{minipage}%
}
% \AltTextImage{item content}{image path}{caption}{label}
\zzcommand{\AltTextImage}[4]
{
  \stepcounter{AltTextImageCurrentSide}  % Increment the counter
  \ifodd\value{AltTextImageCurrentSide}  % Check if the counter is odd
      \AltTextImageR{#1}{#2}{#3}{#4}
  \else
      \AltTextImageL{#1}{#2}{#3}{#4}
  \fi
}

\lstset {
    language=C++,
    backgroundcolor=\color{black!5}, % set backgroundcolor
    basicstyle=\footnotesize,% basic font setting
}

\makeglossaries
\DeclareDocumentCommand{\newdualentry}{ O{} O{} m m m m g } {
    \newglossaryentry{gls-#3}{
        name={#5},
        text={#5\if\relax\detokenize{#4}\relax\else\glsadd{#3}\fi},
        plural={\IfValueTF{#7}{#7}{#5s}},
        longplural={\IfValueTF{#7}{#7}{#5s}},
        description={#6},
        #1
    }
    \if\relax\detokenize{#4}\relax
    % nothing
  \else
    \newacronym[longplural={\IfValueTF{#7}{#7}{#5s}}, 
        see={[Glossary:]{gls-#3}},
        #2]{#3}{#4}{#5\glsadd{gls-#3}}
  \fi
    \makeglossaries
}


% \gloss{entry name}
\zzcommand{\gloss}[1]{\ifglsentryexists{gls-#1}{\glsentryname{gls-#1}\ifcsdef{inheading}{}{{\glsadd{gls-#1}}}}{[XXXX]}}
% \glosses{entry name}
\zzcommand{\glosses}[1]{\ifglsentryexists{gls-#1}{\glsentrylongpl{gls-#1}\ifcsdef{inheading}{}{{\glsadd{gls-#1}}}}{[XXXX]}}
% \Gloss{entry name}
\zzcommand{\Gloss}[1]{\ifglsentryexists{gls-#1}{\Glsentryname{gls-#1}\ifcsdef{inheading}{}{{\glsadd{gls-#1}}}}{[XXXX]}}
% \Glosses{entry name}
\zzcommand{\Glosses}[1]{\ifglsentryexists{gls-#1}{\Glsentrylongpl{gls-#1}\ifcsdef{inheading}{}{{\glsadd{gls-#1}}}}{[XXXX]}}
\zzcommand{\glsnamefont}[1]{\makefirstuc{#1}}

% \eqref{label}
\zzcommand{\eqref}[1]{\cref{#1}}

\zzcommand{\to}{\mapsto}


% \subsubsubsection{title}
\zzcommand{\subsubsubsection}[1]{\bigskip\noindent\textit{#1}\nopagebreak\par}
% \subsubsubsubsection{title}
\zzcommand{\subsubsubsubsection}[1]{\bigskip\noindent\textbf{#1}~}

% \wrapFig{img_path}{width}{label}{caption}
\zzcommand{\wrapFig}[4]{
  \wrapFigL{#1}{#2}{#3}{#4}
}

% \wrapFigLR{img_path}{width}{label}{caption}{|L,R|}
\zzcommand{\wrapFigLR}[5]{
  \def\tempwidth{#2}%
  \ifx\tempwidth\empty
    \def\tempwidth{0.25}%
  \fi
  \begin{wrapfigure}{#5}{\tempwidth \linewidth}
    \includegraphics[width=\linewidth]{#1}
    \expandafter\ifx\expandafter\relax\detokenize{#4}\relax\else\caption{#4}\fi
    \expandafter\ifx\expandafter\relax\detokenize{#3}\relax\else\label{#3}\fi
    \vspace{-1 \baselineskip}
  \end{wrapfigure}
}

% \wrapFigL{img_path}{width}{label}{caption}
\zzcommand{\wrapFigL}[4]{
  \wrapFigLR{#1}{#2}{#3}{#4}{L}
}
% \wrapFigR{img_path}{width}{label}{caption}
\zzcommand{\wrapFigR}[4]{
  \wrapFigLR{#1}{#2}{#3}{#4}{R}
}

\zzcommand{\abstract}
{
    \section*{Abstract}
}

% \shortAbstract{text}
\zzcommand{\shortAbstract}[1]
{
  \noindent
  \textit{#1}
  \bigskip
}

\zzcommand{\smallConclusion}
{
  \bigskip
}
  
\zzcommand{\midConclusion}
{
  \bigskip\bigskip\bigskip
}

\ExplSyntaxOn
\NewDocumentCommand{\appendtographicspath}{m}
 {
  \tl_if_exist:cF { Ginput@path } { \tl_new:c { Ginput@path } }
  \tl_gput_right:cn {Ginput@path} { #1 }
 }
\NewDocumentCommand{\prependtographicspath}{m}
 {
  \tl_if_exist:cF { Ginput@path } { \tl_new:c { Ginput@path } }
  \tl_gput_left:cn {Ginput@path} { #1 }
 }
\ExplSyntaxOff

% \resetgraphicspath
\zzcommand{\resetgraphicspath}
{
  \graphicspath{}
}