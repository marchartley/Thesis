\hbadness = 99999

% Define \zzcommand for conditional commands
\def\zzcommand#1{\let#1\undefined\newcommand#1}

% Macros to hold current titles
% \currentsection
\zzcommand{\currentsection}{}
% \currentsubsection
\zzcommand{\currentsubsection}{}
% \currentsubsubsection
\zzcommand{\currentsubsubsection}{}
\makeatletter

% Save original commands
\let\oldsection\section
\let\oldsubsection\subsection
\let\oldsubsubsection\subsubsection

\makeatletter
\gdef\currentsection{}
\gdef\currentsubsection{}
\gdef\currentsubsubsection{}
\zzcommand{\section}[1]{
  \gdef\currentsection{}
  \gdef\currentsubsection{}
  \gdef\currentsubsubsection{}
  \ifthenelse{\equal{#1}{*}}%
  {\@sectionstar}%
  {\@sectionnostar{#1}}%
  % \@ifstar{\@sectionstar}{\@sectionnostar}
}
\zzcommand{\@sectionnostar}[1]{%
  \gdef\currentsection{#1}%
  \oldsection{#1}%
}
\zzcommand{\@sectionstar}[1]{%
  \gdef\currentsection{#1}
  \oldsection*{#1}%
}
\zzcommand{\subsection}[1]{
  \gdef\currentsubsection{}
  \gdef\currentsubsubsection{}
  \ifthenelse{\equal{#1}{*}}%
  {\@subsectionstar}
  {\@subsectionnostar{#1}}
}
\zzcommand{\@subsectionnostar}[1]{%
  \gdef\currentsubsection{#1}%
  \oldsubsection{#1}%
}
\zzcommand{\@subsectionstar}[1]{%
  \gdef\currentsubsection{#1}%
  \oldsubsection*{#1}%
}
\zzcommand{\subsubsection}[1]{
  \gdef\currentsubsubsection{}
  \ifthenelse{\equal{#1}{*}}%
  {\@subsubsectionstar}
  {\@subsubsectionnostar{#1}}
}
\zzcommand{\@subsubsectionnostar}[1]{%
  \gdef\currentsubsubsection{#1}%
  \oldsubsubsection{#1}%
}
\zzcommand{\@subsubsectionstar}[1]{%
  \gdef\currentsubsubsection{#1}%
  \oldsubsubsection*{#1}%
}
\makeatother

% \tensor{expression}
\zzcommand{\tensor}[1]{\mathbf{#1}}
% \mtrx{matrix_name}
\zzcommand{\mtrx}[1]{\mathbf{#1}}
% \field{field_name}
\zzcommand{\field}[1]{\mathbf{#1}}

\captionsetup{width=\linewidth}

% \WarningFilter{hyperref}{Token not allowed in a PDF string} % Remove the warnings about \gls in titles or math functions in titles.

\LetLtxMacro{\OldTilde}{\Tilde}
\zzcommand{\Tilde}[1]{
  \StrLen{#1}[\temp]%
  \ifnum\temp>1
    \ThisStyle{%
      \setbox0=\hbox{$\SavedStyle#1$}%
      \stackengine{-.1\LMpt}{$\SavedStyle#1$}{%
        \stretchto{\scaleto{\SavedStyle\mkern.2mu\AC}{.5150\wd0}}{.6\ht0}%
      }{O}{c}{F}{T}{S}%
    }
  \else 
    \OldTilde{#1}
  \fi
}

% \Bar{expression}
\zzcommand{\Bar}[1]{\overline{#1}}


\LetLtxMacro{\Oldincludegraphics}{\includegraphics}
\RenewDocumentCommand{\includegraphics}{O{} m}{%
	\begin{adjustbox}{max width=\linewidth, max height=\textheight-5\baselineskip}
	    \Oldincludegraphics[#1]{#2}
	\end{adjustbox}
}
% \inset{inset_size}{path_to_main_image}{path_to_inset_image}
\zzcommand{\inset}[3][0.2\linewidth]{
    \noindent\stackinset{r}{0.1cm}{b}{0.1cm}{
        \colorbox{white}{
            \includegraphics[width=#1]{#3}
        }
    }
	{\includegraphics{#2}}
}

\let\origfigure\figure
\let\endorigfigure\endfigure

\renewenvironment{figure}[1][tbph]{%
    \origfigure[#1]%
    \centering
}{%
    \endorigfigure
}

% \teaser{/* Inside commands of the figure* environment */}
\zzcommand{\teaser}[1]{
    \begin{figure*}[ht]
        \centering
        #1
    \end{figure*}
}

% \minitoc
\def\minitoc{
	\etocsettocstyle{\section*{\contentsname}}{}
	\localtableofcontents
	\noindent\textbf{\hyperlink{tocpage}{$\uparrow$ Back to summary}}
}
% \makeatletter\@addtoreset{chapter}{part}\makeatother%

\sisetup{product-units=repeat}

\emergencystretch=1em

\zzcommand{\cellalign}{cl}
\newcolumntype{H}{>{\setbox0=\hbox\bgroup}c<{\egroup}@{}}

\DeclareCiteCommand{\cite}[\mkbibparens]
  {\usebibmacro{prenote}\usebibmacro{cite:init}}%
  {\usebibmacro{citeindex}\usebibmacro{countcite}{\thefield{entrykey}}%
  \printtext[bibhyperref]{\usebibmacro{cite}}}%
  {\multicitedelim}%
  {\usebibmacro{postnote}}

\newbibmacro*{citefullbody}{%
  \printnames{labelname}%
  \setunit{\addcomma\space}%
  \textit{\printfield[]{title}}~(\printfield{year})%
}

\DeclareCiteCommand{\citep}
{\usebibmacro{prenote}\usebibmacro{cite:init}}%
{\usebibmacro{citeindex}\usebibmacro{countcite}{\thefield{entrykey}}%
\printtext[bibhyperref]{\usebibmacro{citefullbody}}}%
{\multicitedelim}%
{\usebibmacro{postnote}}


\DeclareCiteCommand{\citeProgram}[\mkbibparens]
  {\usebibmacro{prenote}\usebibmacro{cite:init}}
  {\usebibmacro{citeindex}\usebibmacro{countcite}{\thefield{entrykey}}\printtext[bibhyperref]{\mkbibemph{\printfield{title}},~\printdate}}
  {\multicitedelim}
  {\usebibmacro{postnote}}

\AtEveryBibitem{\clearfield{url}}
\AtEveryBibitem{\clearfield{doi}}
\AtEveryBibitem{\clearfield{note}} % Notes from Mendeley
\newtotcounter{citenum}
\AtEveryBibitem{\stepcounter{citenum}}


\renewbibmacro*{title}{%
{}
{\printtext[title]{\mkbibbold{\printfield[title]{title}}}\newunit}}

\DefineBibliographyStrings{english}{
  backrefpage = {Cited on page},
  backrefpages = {Cited on pages},
}



\setlength\bibitemsep{1.2\itemsep}

\newcounter{AltTextImageCurrentSide}

% \begin{Itemize}
%     \Item{$1} $2
% \end{Itemize}
\newenvironment{Itemize}
{    
  \setcounter{AltTextImageCurrentSide}{0}
  \begin{itemize}[label={}, leftmargin=\parindent, itemindent=1.0em, itemsep=0em, topsep=0em] %, left=2em, itemsep=0.5em]} % Custom settings
}
{\end{itemize}\bigbreak}

% Define new command \Itemthat applies italics to the title only
% \Item{text}
\zzcommand{\Item}[1]{%
  \ifblank{#1}%
    {\item $\bullet$ }% Case when #1 is empty
    {\item \textit{#1}}% Case when #1 is not empty
}

\newlength{\currentparindent}
\newlength{\contentheightofalttextimage}
\newlength{\contentheightofalttextimagecaption}
\newsavebox{\alttextbox}

\ExplSyntaxOn
\newdimen\l_tmpa_img_width
\newdimen\l_tmpa_img_height
\newdimen\l_tmpa_ratio
\newdimen\l_tmpa_min_height
\newdimen\l_tmpa_final_height
\ExplSyntaxOff

\ExplSyntaxOn
% \AltTextImageR{item_content}{placeholder.pdf}{caption}{label}
\zzcommand{\AltTextImageR}[4]{% 
  \setcounter{AltTextImageCurrentSide}{1}
  \setlength{\currentparindent}{\parindent}%
  % \newsavebox{\alttextbox}
    \sbox{\alttextbox}{\vbox{\hsize=0.65\linewidth #1}}%
    \setlength{\contentheightofalttextimage}{\ht\alttextbox}%

    \sbox{\alttextbox}{\vbox{\hsize=0.3\linewidth #3\quad}}%
    \setlength{\contentheightofalttextimagecaption}{\ht\alttextbox}%
  \noindent
  \begin{minipage}[t]{0.65\linewidth}
    \setlength{\parindent}{\currentparindent}
    #1
  \end{minipage}%
  \hfill % Add horizontal space between text and image
  \begin{minipage}[t]{0.30\linewidth}    
    \centering
    % Size: \the\contentheightofalttextimage
    % Parse comma-separated images and stack them
    \seq_set_split:Nnn \l_tmpa_seq { , } { #2 }%
    \int_set:Nn \l_tmpa_int { \seq_count:N \l_tmpa_seq }%
    \dim_set:Nn \l_tmpa_dim {
      \dim_max:nn {
        .5 \contentheightofalttextimage
      }{
        (\contentheightofalttextimage / \l_tmpa_int) - (\contentheightofalttextimagecaption / 2)
      }
    }
    \seq_map_inline:Nn \l_tmpa_seq {%
      \includegraphics[width=\linewidth, height=\l_tmpa_dim, keepaspectratio, valign=t]{##1}\\%
    }%
    % \includegraphics[width=\linewidth, height=\contentheightofalttextimage, keepaspectratio, valign=t]{#2}
    \ifblank{#3}{}{
      \captionsetup{font=footnotesize}
      \captionsetup{width=\linewidth}
      \captionsetup{justification=RaggedRight}
      \captionof{figure}{#3}
    }
    \label{#4}
  \end{minipage}%
}
% \AltTextImageL{item_content}{placeholder.pdf}{caption}{label}
\zzcommand{\AltTextImageL}[4]{% 
  \setcounter{AltTextImageCurrentSide}{0}
  \setlength{\currentparindent}{\parindent}%
    \sbox{\alttextbox}{\vbox{\hsize=0.65\linewidth #1}}%
    \setlength{\contentheightofalttextimage}{\ht\alttextbox}%

    \sbox{\alttextbox}{\vbox{\hsize=0.3\linewidth #3\quad}}%
    \setlength{\contentheightofalttextimagecaption}{\ht\alttextbox}%
  \noindent
  \begin{minipage}[t]{0.30\linewidth}
    \centering
    % Size: \the\contentheightofalttextimagecaption
    % Parse comma-separated images and stack them
    \seq_set_split:Nnn \l_tmpa_seq { , } { #2 }%
    \int_set:Nn \l_tmpa_int { \seq_count:N \l_tmpa_seq }%
    \dim_set:Nn \l_tmpa_dim {
      \dim_max:nn {
        .5 \contentheightofalttextimage
      }{
        (\contentheightofalttextimage / \l_tmpa_int) - (\contentheightofalttextimagecaption / 2)
      }
    }
    % \dim_set:Nn \l_tmpa_dim { (\contentheightofalttextimage / \l_tmpa_int)  - (\contentheightofalttextimagecaption / 2) }%
    \seq_map_inline:Nn \l_tmpa_seq {%
      \includegraphics[width=\linewidth, height=\l_tmpa_dim, keepaspectratio, valign=t]{##1}\\%
    }%
    % \includegraphics[width=\linewidth, height=\contentheightofalttextimage, keepaspectratio, valign=t]{#2}
    \ifblank{#3}{}{
      \captionsetup{font=footnotesize}
      \captionsetup{width=\linewidth}
      \captionsetup{justification=RaggedRight}
      \captionof{figure}{#3}
    }
    \label{#4}
  \end{minipage}%
  \hfill % Add horizontal space between text and image
  \begin{minipage}[t]{0.65\linewidth}
    \setlength{\parindent}{\currentparindent}
    #1
  \end{minipage}%
}
\ExplSyntaxOff
% \AltTextImage{item_content}{placeholder.pdf}{caption}{label}
\zzcommand{\AltTextImage}[4]
{
  \stepcounter{AltTextImageCurrentSide}  % Increment the counter
  \ifodd\value{AltTextImageCurrentSide}  % Check if the counter is odd
      \AltTextImageR{#1}{#2}{#3}{#4}
  \else%
      \AltTextImageL{#1}{#2}{#3}{#4}
  \fi
}
\ExplSyntaxOff

% \AltTextImage{item_content}{placeholder.pdf}{caption}{label}
\zzcommand{\AltTextImage}[4]
{
  \stepcounter{AltTextImageCurrentSide}  % Increment the counter
  \ifodd\value{AltTextImageCurrentSide}  % Check if the counter is odd
      \AltTextImageR{#1}{#2}{#3}{#4}
  \else%
      \AltTextImageL{#1}{#2}{#3}{#4}
  \fi
}

\lstset {
    language=C++,
    backgroundcolor=\color{black!5}, % set backgroundcolor
    basicstyle=\footnotesize,% basic font setting
}

\makeglossaries
\DeclareDocumentCommand{\newdualentry}{ O{} O{} m m m m g } {
    \newglossaryentry{gls-#3}{
        name={#5},
        text={#5\if\relax\detokenize{#4}\relax\else\glsadd{#3}\fi},
        plural={\IfValueTF{#7}{#7}{#5s}},
        longplural={\IfValueTF{#7}{#7}{#5s}},
        description={#6},
        #1
    }
    \if\relax\detokenize{#4}\relax
    % nothing
  \else
    \newacronym[longplural={\IfValueTF{#7}{#7}{#5s}}, 
        see={[Glossary:]{gls-#3}},
        #2]{#3}{#4}{#5\glsadd{gls-#3}}
  \fi
    \makeglossaries
}


% \gloss{entry_name}
\zzcommand{\gloss}[1]{\ifglsentryexists{gls-#1}{\glsentryname{gls-#1}\ifcsdef{inheading}{}{{\glsadd{gls-#1}}}}{[XXXX]}}
% \glosses{entry_name}
\zzcommand{\glosses}[1]{\ifglsentryexists{gls-#1}{\glsentrylongpl{gls-#1}\ifcsdef{inheading}{}{{\glsadd{gls-#1}}}}{[XXXX]}}
% \Gloss{entry_name}
\zzcommand{\Gloss}[1]{\ifglsentryexists{gls-#1}{\Glsentryname{gls-#1}\ifcsdef{inheading}{}{{\glsadd{gls-#1}}}}{[XXXX]}}
% \Glosses{entry_name}
\zzcommand{\Glosses}[1]{\ifglsentryexists{gls-#1}{\Glsentrylongpl{gls-#1}\ifcsdef{inheading}{}{{\glsadd{gls-#1}}}}{[XXXX]}}
\zzcommand{\glsnamefont}[1]{\makefirstuc{#1}}

% \eqref{label}
\zzcommand{\eqref}[1]{\cref{#1}}
% \to 
\zzcommand{\to}{\mapsto}


% \subsubsubsection{title}
\zzcommand{\subsubsubsection}[1]{\bigskip\noindent\textit{#1}\nopagebreak\par}
% \subsubsubsubsection{title}
\zzcommand{\subsubsubsubsection}[1]{\bigskip\noindent\textbf{#1}~}

% \wrapFig{img_path}{width}{label}{caption}
\zzcommand{\wrapFig}[4]{
  \wrapFigL{#1}{#2}{#3}{#4}
}

% \wrapFigLR{img_path}{width}{label}{caption}{|L,R|}
\zzcommand{\wrapFigLR}[5]{
  \def\tempwidth{#2}%
  \ifx\tempwidth\empty
    \def\tempwidth{0.25}%
  \fi
  \begin{wrapfigure}{#5}{\tempwidth \linewidth}
    \includegraphics[width=\linewidth]{#1}
    \expandafter\ifx\expandafter\relax\detokenize{#4}\relax\else\caption{#4}\fi
    \expandafter\ifx\expandafter\relax\detokenize{#3}\relax\else\label{#3}\fi
    \vspace{-1 \baselineskip}
  \end{wrapfigure}
}

% \wrapFigL{img_path}{width}{label}{caption}
\zzcommand{\wrapFigL}[4]{
  \wrapFigLR{#1}{#2}{#3}{#4}{L}
}
% \wrapFigR{img_path}{width}{label}{caption}
\zzcommand{\wrapFigR}[4]{
  \wrapFigLR{#1}{#2}{#3}{#4}{R}
}

\zzcommand{\abstract}
{
    \section*{Abstract}
}

% \shortAbstract{text}
\zzcommand{\shortAbstract}[1]
{
  \noindent
  \textit{#1}
  \bigskip
}

\zzcommand{\smallConclusion}
{
  \bigskip
}
  
\zzcommand{\midConclusion}
{
  \bigskip\bigskip\bigskip
}

\ExplSyntaxOn
\NewDocumentCommand{\appendtographicspath}{m}
 {
  \tl_if_exist:cF { Ginput@path } { \tl_new:c { Ginput@path } }
  \tl_gput_right:cn {Ginput@path} { #1 }
 }
\NewDocumentCommand{\prependtographicspath}{m}
 {
  \tl_if_exist:cF { Ginput@path } { \tl_new:c { Ginput@path } }
  \tl_gput_left:cn {Ginput@path} { #1 }
 }
\ExplSyntaxOff

% \resetgraphicspath
\zzcommand{\resetgraphicspath}
{
  \graphicspath{}
}

% \hide{text}
\zzcommand{\hide}[1] {}




% Counter for citations per section
\newcounter{sectioncites}
\newcounter{subsectioncites}
\newcounter{subsubsectioncites}

% Macro to store list of citation keys used in current section
\zzcommand{\sectioncitedkeyslist}{,} % Start with a leading comma
\zzcommand{\subsectioncitedkeyslist}{,} % Start with a leading comma
\zzcommand{\subsubsectioncitedkeyslist}{,} % Start with a leading comma

% Reset at each section
\zzcommand{\resetsectioncites}{%
  \setcounter{sectioncites}{0}%
  \zzcommand{\sectioncitedkeyslist}{,}%
  \resetsubsectioncites
}
\zzcommand{\resetsubsectioncites}{%
  \setcounter{subsectioncites}{0}%
  \zzcommand{\subsectioncitedkeyslist}{,}%
  \resetsubsubsectioncites
}
\zzcommand{\resetsubsubsectioncites}{%
  \setcounter{subsubsectioncites}{0}%
  \zzcommand{\subsubsectioncitedkeyslist}{,}%
}
% Helper: checks if key is in the list, and adds if not
\zzcommand{\adduniquecite}[1]{%
  \def\currentkey{#1}%
  \IfSubStr{\sectioncitedkeyslist}{,\currentkey,}{\def\foundkey{1}}{\def\foundkey{0}}%
  \ifnum\foundkey=0%
    \stepcounter{sectioncites}%
    \xappto{\sectioncitedkeyslist}{,\currentkey,}%
  \fi%
  \IfSubStr{\subsectioncitedkeyslist}{,\currentkey,}{\def\foundkey{1}}{\def\foundkey{0}}%
  \ifnum\foundkey=0%
    \stepcounter{subsectioncites}%
    \xappto{\subsectioncitedkeyslist}{,\currentkey,}%
  \fi%
  \IfSubStr{\subsubsectioncitedkeyslist}{,\currentkey,}{\def\foundkey{1}}{\def\foundkey{0}}%
  \ifnum\foundkey=0%
    \stepcounter{subsubsectioncites}%
    \xappto{\subsubsectioncitedkeyslist}{,\currentkey,}%
  \fi%
}

% Custom bibmacro to do the tracking per entry
\newbibmacro*{countcite}[1]{%
  \adduniquecite{#1}%
}
\zzcommand{\showsectioncites}{%
  \notblank{\currentsection}{%
  \ifdef{\thesectioncites}{%
    \ifnum\thesectioncites=0%
    \else%
      \textcolor{white}{\texttransparent{0.0}{~\thesectioncites~(\currentsection)}}%
    \fi%
  }{}}{}%
}
\zzcommand{\showsubsectioncites}{%
  \notblank{\currentsubsection}{%
  \ifdef{\thesubsectioncites}{%
    \ifnum\thesubsectioncites=0%
    \else%
      \textcolor{white}{\texttransparent{0.0}{~\thesubsectioncites~(\currentsubsection)}}%
    \fi%
  }{}}{}%
}
\zzcommand{\showsubsubsectioncites}{%
  \notblank{\currentsubsubsection}{%
  \ifdef{\thesubsubsectioncites}{%
    \ifnum\thesubsubsectioncites=0%
    \else%
      \textcolor{white}{\texttransparent{0.0}{~\thesubsubsectioncites~(\currentsubsubsection)}}%
    \fi%
  }{}}{}%
}

\pretocmd{\section}{\showsubsubsectioncites\showsubsectioncites\showsectioncites\resetsectioncites}{}{}
\pretocmd{\subsection}{\showsubsubsectioncites\showsubsectioncites\resetsubsectioncites}{}{}
\pretocmd{\subsubsection}{\showsubsubsectioncites\resetsubsubsectioncites}{}{}








% Holds user-defined row width
\newcommand{\fitrowwidth}{\linewidth}
\newcommand{\fitimageoptions}{}

% autofitgraphics with [options]{img1,img2,...}
\makeatletter
\NewDocumentCommand{\autofitgraphics}{O{} m}{%
  \renewcommand{\fitrowwidth}{\linewidth}%
  \def\fitimageoptions{}%
  \def\cleanopts{}%

  \def\tempa{#1}%
  \@for\opt:=\tempa\do{%
    \IfBeginWith{\opt}{width=}%
      {\StrBehind{\opt}{=}[{\tempwidth}]\renewcommand{\fitrowwidth}{\tempwidth}}%
      {\ifx\cleanopts\@empty
         \edef\cleanopts{\opt}%
       \else
         \edef\cleanopts{\cleanopts,\opt}%
       \fi
      }%
  }%
  \edef\fitimageoptions{\cleanopts}%
  \fitNimages{#2}%
}
\newcounter{fitCapIndex}
\NewDocumentCommand{\autofitcaptions}{O{} m}{%
  \ifblank{#1}
    {\def\fitraiseamount{2ex}} % default
    {\def\fitraiseamount{#1}}  % use passed value
  \setcounter{fitCapIndex}{0}%
  \begin{adjustbox}{width=\fitrowwidth}%
    \renewcommand{\do}[1]{%
      \stepcounter{fitCapIndex}%
      \pgfmathsetmacro{\relwidth}{\csname fitRatio\thefitCapIndex\endcsname / \fitTotalRatio}%
      \raisebox{\fitraiseamount}{\parbox[t]{\relwidth\linewidth}{\centering ##1}}%
    }%
    \docsvlist{#2}%
  \end{adjustbox}%
  \vspace{-\fitraiseamount}%
}
\makeatother

\newcounter{fitImgCount}%
\newcounter{fitImgIndex}%
\newcommand{\fitNimages}[1]{%
  \setcounter{fitImgCount}{0}
  \setcounter{fitImgIndex}{0}
  \begingroup
  
  \def\fitTotalRatio{0}%
  \global\let\fitWidths\empty
  \def\fitImgList{}%

  % Copy input to a clean list for repeated processing
  \renewcommand{\do}[1]{%
    \edef\fitImgList{\fitImgList,##1}%
  }%
  \docsvlist{#1}%

  % First pass: calculate total aspect ratio and store individual ratios
  \renewcommand{\do}[1]{%
    \stepcounter{fitImgCount}%
    \sbox0{\includegraphics{##1}}%
    \edef\imgWidth{\the\wd0}%
    \edef\imgHeight{\the\ht0}%
    \pgfmathsetmacro{\imgRatio}{\imgWidth/\imgHeight}%
    \pgfmathparse{\fitTotalRatio + \imgRatio}%
    \xdef\fitTotalRatio{\pgfmathresult}%
    \expandafter\xdef\csname fitRatio\thefitImgCount\endcsname{\imgRatio}%
  }%
  \docsvlist{#1}%

  % Second pass: output images with proportional widths
  \setcounter{fitImgIndex}{0}%
  \begin{adjustbox}{width=\fitrowwidth}%
  \renewcommand{\do}[1]{%
    \stepcounter{fitImgIndex}%
    \pgfmathsetmacro{\relwidth}{\csname fitRatio\thefitImgIndex\endcsname / \fitTotalRatio}%
    \expandafter\includegraphics\expandafter[\fitimageoptions, width=\relwidth\linewidth]{##1}%
  }%
  \docsvlist{#1}%
  \end{adjustbox}%
  \endgroup
}

% \delete{text_to_remove}
\zzcommand{\delete}[1]
{\remove{#1}}
\zzcommand{\comment}[1]
{\par \highlight{~!~!~!~!~!~!~!~!~!~!~} #1 \highlight{~!~!~!~!~!~!~!~!~!~!~} \\}
\zzcommand{\Replace}{\replace}
\zzcommand{\Comment}{\comment}
\zzcommand{\Add}{\add}
\zzcommand{\Delete}{\delete}
\zzcommand{\revrepl}{\replace}
\zzcommand{\revnote}{\comment}
