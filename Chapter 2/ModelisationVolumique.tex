\chapter{Volumetric terrain modeling}
\label{chap:volumic-modeling}
\minitoc

\section{Introduction}
[THIS MIGHT BE THE ABSTRACT OF THE PART]

The output of the semantic terrain generation is a set of \glosses{EnvObj} containing a \gloss{Skeleton} defined as points, curves or regions. From this very coarse representation, we want to create the 3D geometry to visualize the real terrain. This parametric geometry representation of the \glosses{Skeleton} pushes us to use the implicit surface representation to model it. We extended this idea to work with implicit volumes with material information.

Working with volumes instead of surfaces allows to keep the information of which terrain features occupies which points in space. Adding to this the notion of material information has many advantages: first, we can render different textures procedurally on the ground, the features and even the inside if needed. Secondly, this information can be used in post-processes like the erosion simulation step, where this information enables to weather more or less some parts, or add effects at the interface between multiple materials. Finally, including "invisible" materials like air or water allows to represent 3D objects like tunnels the same way as solid objects.

% - Volumetric modeling is important for representing 3D structures \\
% - Allows for the representation of cavities, arches, overlays, etc. \\
% - The concept of materials allows for including much more information for the following parts: enhancement and rendering \\
% ** Enhancement (e.g., erosion) needs to know the type of soil at the surface and subsurface to be realistic \\
% ** Rendering needs to know the material at the surface to correctly display textures \\
% - ...

\section{Implicit terrains with materials}
\label{sec:volumic-modeling_implicit-terrain-with-materials}

Material information in the terrain ground provides useful hints for the generation process. Recent works in terrain generation make use of construction trees to aggregate terrain features in a landscape. Construction trees allow to store the sparsely the features for a very low memory cost. While the implicit modeling has been studied for a long time now \cite{Turk2001}, only recently has the . In 3D modeling, the model of the BlobTree\cite{Schmidt2006} has been often used. More studies about blending functions \cite{Barthe2004,Bernhardt2010,DeGroot2014,Vaillant2013,Angles2017} enabled to find new ways to smoothly integrate the different nodes of the construction trees. In terrain generation, the use of construction trees have been used in 2.5D for adding and blending features together in order to represent a large scene \cite{Genevaux2015,Guerin2016}, but also in 3D to represent smaller features \cite{Paris2021a}. However, in these methods, the material information is lost and we consider that everything that is defined by the construction tree is composed of a single ground type. 

We extended the idea of construction trees for 3D terrains by extending the operator nodes with positional information. While the problem of material definition inside overlapping objects remains an open problem, we propose solutions to tackle this problematic.

% \subsection{Material density}
% % - Scalar value across the field \\
% - ...

% \subsubsection{Material granularity}
% - ...

% \subsubsection{Soil triangle}
% - ...

\subsection{Tree definition}
- Tree structure \\
** Leaves \\
*** Implicit volumes \\
*** Material information \\
*** Bounded support \\
** Nodes \\
*** Not always unary or binary \\
*** Blending functions \\
*** Positional information \\
** Evaluation at one point \\
*** From leaves to trunk \\
- ... 

\section{Primitives}
The leaves of the construction tree are implicit primitives. An implicit primitive is a scalar function $f: \R^3 \to \R$ for which we typically consider the surface at any point $\p$ satisfying $f(\p) = 0$. We consider that we are inside the volume for $f(\p) < 0$ and outside for $f(\p) > 0$. This definition include many possible functions like voxel grids (explicit function) or implicit volumes.

We use the primitives to represent the presence of a feature at a given position, and as such we limited the domain of the function to be $f: \R^3 \to [0, 1]$ where a value closer to 1 is the presence of the given feature and 0 its absence. A common mapping from the "classic" function to the restrained function may look as $f_{\text{restrained}} = \sigmoid (- f_{\text{classic}})$ with the sigmoid function 
\begin{align}
    \sigmoid(x) = \frac{1}{1+e^{-t(x - x_0)}}
\end{align}    
with $t$ being the logistic growth rate and $x_0$ an offset of the function, the $x$ value of the function's midpoint. We typically set $x_0 = 0$ to center symmetry point of the function at $x = 0$. With this definition of the scalar function, the surface previously set at $f_{\text{classic}} = 0$ is transfered to $f_{\text{restrained}} = 0.5$. Keeping our values between 0 and 1 allows us to use more tools as to manipulate the content of our construction trees. By using the logit function 
\begin{align}
    \operatorname{logit}(x)=\frac{1}{t}\ln\left(\frac{x}{1-x}\right)+x_0
\end{align}
we can get back to the classic implicit formulations. As the exponential function quickly grow, we usually consider $\sigmoid(x)=1$ for $x \geq \frac{6}{t}$ and $\sigmoid(x)=0$ for $x \leq -\frac{6}{t}$.

We include an extra information in the node: the composing material. Typically, we represent the material as an index to a set of predefined materials like sand, rock, soil, water, etc... In our representation each primitive is associated to a unique material. 

\subsection{Scalar functions}
The scalar functions associated to primitives are defined as $f: \R^3 \to \R$. We usually associate points and curves to the function in order to define parametric volumes.

Examples of scalar functions may include:
\begin{itemize}
    \item Spheres: the parametric equation of the sphere is the most common model, using its center $c$ and radius $\radius$. $f(\p) = (c - \p)^2 - \radius^2$.
    \item Tubes: using a parametric curve $\curve$ and a radius $\radius$, we can define tubes, that can represent arches or tunnels, depending on the material associated. $f(\p) = (\curve^* - \p)^2 - \radius^2$ with $\curve^*$ defined as the closest point of the curve from $\p$.
\end{itemize}

Scalar functions are not limited to simple equations. The tunnels digged in \cref{chap:karsts}, for example, use the implicit swept volume formulation \cite{Schroeder1994}, providing a parametric path and a parametric 2D shape to carve the ground. The later is also parametrized to evolve along the curve path in order to have an entry shape and an output shape that can be different.

\subsection{Material information}
The material information given for a primitive inform the evaluator what type of material should be present at any point $\p$ that satisfy $f(\p) < 0$.

Now we also like to define the density of this material inside the primitive, such that in the case where two or more primitives are present in a single point, we can find a unique final material $\material$. In our work we considered the density function $\sigma: \R^3 \to \R$ to be completely related to the scalar function $f$ such as $\sigma = -f$. While not realistic, this default value is sufficient for most cases.

\section{Operators}
In the construction tree structure, nodes represent operators. Unary operators are effective to translate, rotate or scale their child node, but many other manipulations are useful. On the other hand, $n$-ary nodes, containing two or more child nodes, aggregate their children like CSG trees with their union, soustraction and difference operations, but in a smooth manner, that we call blending.

\subsection{Unary nodes}
An unary node has a unique child. They are used to apply transformations on their child, whever it is an implicit primitive or another node. In CSG, we often use unary operators to apply linear transformation on an object, such as a translation, a rotation or a scaling operation. 

Other non-linear transformations may be applied on the child as twisting and warping operations. The twist, usually defined as 
\begin{align}
    T(x, y, z) = \left( x \cos(\theta(z)) - y \sin(\theta(z)), x \sin(\theta(z)) + y \cos(\theta(z)), z \right)
\end{align}
with $\theta(z)$ a scalar function defining the rotation strength function of $z$, resulting in a helice shape, while the warping operator would be defined as 
\begin{align}
    W(x,y,z) = (x,y,z) + F(x, y, z)
\end{align}
with $F$ a vector field that define the warp direction and strength.
    

Many other operators may be proposed, such as a noise function $f(\p) = f_A(\p) + \noise(\p)$.

\subsection{$n$-ary nodes}
Our scenes may contain a large amount of elements and as such, a binary construction tree representation can become quickly deep and not ideal. We then focused the construction of $n$-ary trees, where each node can contain an undefined number $n$ of children with $n \geq 2$.

\subsubsection{Blending functions}
The primitive functions are defined on $f_{\text{primitive}}: \R^3 \to [0,1]$ and we want to keep the result of our blending functions in the same domain $f_{\text{blending}}: [0,1]^2 \to [0,1]$, providing us possible simplifications on the blending functions, which is a good thing as the inclusion of the materials, on the other hand, adds complexities.

First, we need to take into account that some materials are "solid" (sand, rock) and some are "invisible" (water, air). In the CSG paradigm, adding a solid and an invisible object would be seen as the substraction operation. In our work, we consider

- Symmetric blending: \\
** Analogy easy with the 2.5D version \\
** With the form $f(\p) = \sqrt[n]{f_A(\p)^n + f_B(\p)^n}$. \\
** We use the same function to determine the amount of each material $\sigma_A$ and $\sigma_B$. \\
** But now, we need to define which material is present where. So nodes contains a vector of materials, such that multiple materials can be defined as each point. This multiple materials information is kept, we will see different solutions to render or collapse this vector in the next section.

- Asymmetric blending: \\
** A bit more complex than the 2.5D version. \\
** Replacing: \\
*** [FIND FUNCTION] \\
** One-way blending: \\
*** [FIND FUNCTION] \\
- ...

\subsubsection{Placement functions}
- Fixed = "max" in 3D \\
** A "free" object, or not dependent on environment for example \\
In this case, we can evaluate intuitively $G(f_A(\p), f_B(\p))$
- Stacking = addition in 3D \\
** "Liquid" positionning based on object A's surface \\
The equation for evaluating a position $\p$ of the function $f_B$ stacked over $f_A$ becomes $G(f_A(\p), f_B(\p - \begin{bmatrix}0 & 0 & \arg \max_z f_A(\p)\end{bmatrix}))$ \\
- Roof = obj B is mirrored on Z and placed at above surface \\
** Say obj A is a cave (or "hole object", like tunnel), place obj B (a stalctite for example) on the roof of the tunnel. \\
The equation becomes $G(f_A(\p), f_B(\p - \begin{bmatrix}0 & 0 & \arg \max_z f_B(\p) + \arg \max_z f_A(\p)\end{bmatrix}))$. \\
- Ground = obj B is placed at lower surface \\
** Say obj A is a cave (or "hole object", like tunnel), place obj B (a rock for example) on the ground of the tunnel. \\
The equation becomes $G(f_A(\p), f_B(\p - \begin{bmatrix}0 & 0 & \arg \min_z f_A(\p)\end{bmatrix}))$. \\
- ...

\section{Returning a material}
As we are mixing together multiple primitives with different materials, some may want to collapse the possible materials to a single material. Discretizing the implicit models to a layer-based representation or a material-voxel grid, for example, require a single material for a point in space. No single answer can be found as we will always lose information by doing so. In the backend, the information about each quantities of material may be stored.

\subsection{Material usage}
Storing all the materials that compose a single point in space have many usages. Determining properties such as density, porosity, granularity may be important indicators for future processes. While natural processes like corrosion is applied at the molecular level, keeping the information of the composition of the ground can allow to simulate natural phenomena much easily.

However, when rendering the surface of a 3D model or a terrain, we usually need to reduce the number to textures to blend to one or few. Another use case of the collapse: if we want to transform an implicit model to a material-voxel grid, for example, a unique material may be affected to each voxel.

\subsubsection{Defining the final material}
If we are in a situation in which we need to reduce the number of materials to a unique material, we have multiple solutions. Each solution has inevitably drawbacks.
\begin{itemize}
    \item Select the material, or the $n$ materials, with the highest amount. This simple solution requires close to no computation and is intuitive.
    \item Classify materials in a taxonomy. This allows many new options to implement a voting system.
\end{itemize}

\subsubsection{Usage example: material transformation}
As multiple materials may be present at a single position $\p$, we can apply transformation on these materials. We may use the formulation of chemical reactions to describe the phenomena.

An example of transformation can include $1 \text{dirt} + 1 \text{water} \to 2 \text{mud}$ to describe the mixing of soft materials with liquids may transform this place in a muddy floor. $1 \text{water current} + 1 \text{rock} \to 0.5 \text{sand}$ simulates the effect of water erosion on solid ground, approximating the desaggregation of rocks into granular materials in addition to corrosion, as the equation is not balanced. 

\section{Graphical representation of \glosses{EnvObj}}
\label{sec:volumic-modeling_graphic-representation-env-objects}

As presented in \cref{chap:semantic-representation}, the \glosses{EnvObj} we used to define our sparse terrain are symbolics and have no geometric representation. It is still possible to translate the semantic representation into a 3D representation by using a construction tree of the form :

- Root \\
- Water primitive : 1 large block set at $z = \Wlevel$ \\
- "Ground" primitives that are fixed at $z = 0$, aggregated with $G_\text{fixed}$ \\
- "Water" primitives aggregated with water primitive, aggregated with $G_\text{roof}$ \\
- "Surface" primitives aggregated with $G_\text{stack}$ to be automatically placed on top of the surface. \\

While the constuction trees are mainly aimed at combining implicit surfaces, the definition of the primitives can easily be extended to include meshes and implicit volumes. In the rest of this thesis, we will most of the time use implicit volumes for elements that shape the ground and meshes for elements that populate the surface, like vegetation. An exmple of underground 3D feature, karst systems, presented in \cref{chap:karsts}, may require an exception as we will have it a fixed position from the surface. As such, we would apply the $G_\text{roof}(f_\text{surface}, f_\text{karst})$ positionning operator in order to fix the entrance of the karst at the surface.

The implicit volumes used are mainly combinations of tubes, spheres, and boxes, with the inclusion of noise unary nodes to add rugosity at the surface.