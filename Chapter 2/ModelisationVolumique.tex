\chapter{Volumetric Terrain Modeling [MAYBE TO BE MOVED TO INTRO]}
\label{chap:volumic-modeling}
\minitoc

- Volumetric modeling is important for representing 3D structures \\
- Allows for the representation of cavities, arches, overlays, etc. \\
- The concept of materials allows for including much more information for the following parts: amplification and rendering \\
** Amplification (e.g., erosion) needs to know the type of soil at the surface and subsurface to be realistic \\
** Rendering needs to know the material at the surface to correctly display textures \\
- ...

\section{Implicit Terrains with Materials}
\label{sec:volumic-modeling_implicit-terrain-with-materials}
- ...

\subsection{Material Density}
- ...

\subsubsection{Material Granularity}
- ...

\subsubsection{Soil Triangle}
- ...

\subsection{Scalar Functions}
- ...

\subsection{Blending Functions}
- ...

\subsection{Placement Functions}
- ...

\subsection{Material Usage}
- ...

\subsubsection{Defining the Final Material}
- ...

\subsubsection{Post-processing: Material Transformation}
- ...

\section{Graphical Representation of Environmental Objects}
\label{sec:volumic-modeling_graphic-representation-env-objects}
- Implicit surfaces \\
- Meshes \\
- ...
