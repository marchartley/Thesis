\part{Modelisation}

\chapter*{Abstract}
\label{chap:modelisation-abstract}
- Methods completely different \\
** Physical phenomena are different \\
** => Simulation/generation methods must be specific for one landscape \\
- Methods are developed at different instants of the thesis work, \\
** We will see different procedural generation domains: \\
*** Analytical solutions (coral islands \#1) \\
*** Deep Neural Networks (coral islands \#2) \\
*** 3D user interaction (karst networks) \\
- Deep Learning tends to replace procedural methods in 2D domain, \\
** But still too complex for 3D models \\
*** (lack of data, lack of research interest for now) \\
** Still requires a lot of data, which, is (while not easy) possible using aerial images, but is way too sparse with underwater landscapes or underground biomes. \\
- We want to control the area in which an element is modeled in the terrain \\
** Because that's how we define them in the previous chapter. \\
** Thus we cannot use simple random noise methods => no bounds \\
*** Only solution is to use falloff maps, but meh... \\
- As such, we want to keep the skeleton of our elements using primitives (points, curves, regions) \\
** => Much easier to add constraints / manipulate primitives in a "procedural generation" way. \\
- Warning: usage of Deep Learning is at the limit of "procedural generation" and is not considered as part of it by the whole community. \\
(Complete with the Reddit poll). \\
- ... 


\chapter{Volumetric Terrain Modeling [MAYBE TO BE MOVED TO INTRO]}
\label{chap:volumic-modeling}
\minitoc

- Volumetric modeling is important for representing 3D structures \\
- Allows for the representation of cavities, arches, overlays, etc. \\
- The concept of materials allows for including much more information for the following parts: amplification and rendering \\
** Amplification (e.g., erosion) needs to know the type of soil at the surface and subsurface to be realistic \\
** Rendering needs to know the material at the surface to correctly display textures \\
- ...

\section{Implicit terrains with materials}
\label{sec:volumic-modeling_implicit-terrain-with-materials}
- ...

\subsection{Material density}
- ...

\subsubsection{Material granularity}
- ...

\subsubsection{Soil triangle}
- ...

\subsection{Scalar functions}
- ...

\subsection{Blending functions}
- ...

\subsection{Placement functions}
- ...

\subsection{Material usage}
- ...

\subsubsection{Defining the final material}
- ...

\subsubsection{Post-processing: material transformation}
- ...

\section{Graphical Representation of \glosses{EnvObj}}
\label{sec:volumic-modeling_graphic-representation-env-objects}
- Implicit surfaces \\
- Meshes \\
- ...


\chapter{Automatic Generation of Coral Islands}
\label{chap:coral-island}
\minitoc

\section{Introduction}
\label{sec:coral-island_introduction}
- Definition of coral islands \\
** Different types of coral islands \\
*** Here, volcanic islands \\
- Presentation of corals \\
- Difference from regular landscapes \\
** Concept of corals \\
*** Long-term evolution (island) and short-term evolution (corals) \\
**** Geological process of island subsidence \\
- ...

\subsection{Multiple theories}
- Complexity of Coral Reef Ecosystems \\
** Biological Diversity \\
*** Varied species of corals and their differing growth patterns \\
*** Interaction with marine flora and fauna \\
** Environmental Factors \\
*** Influence of water temperature, salinity, and light availability \\
*** Impact of ocean currents and wave action \\
- Geological Processes \\
** Dynamic Nature of Earth's Crust \\
*** Plate tectonics and volcanic activity \\
*** Subsidence and uplift processes \\
** Variations in Sea Levels \\
*** Historical fluctuations due to glacial and interglacial periods \\
*** Current sea level rise and its impact on coral reefs \\
- Historical and Technological Context \\
** Historical Developments in Marine Science \\
*** Early exploration and observations by naturalists like Darwin and Wallace \\
*** Advances in geological and oceanographic methods \\
** Technological Advancements \\
*** Development of deep-sea exploration tools \\
*** Use of sonar mapping, core sampling, and seismic surveys \\
- Limitations of Early Theories \\
** Inadequate Data and Observation \\
*** Limited access to deep-sea environments in the 19th and early 20th centuries \\
*** Reliance on surface observations and anecdotal evidence \\
** Evolving Scientific Understanding \\
*** Changes in scientific paradigms and theories over time \\
*** Integration of new findings and methodologies \\
- Different Perspectives and Disciplines \\
** Geological vs. Biological Perspectives \\
*** Focus on geological processes like subsidence and sea level change \\
*** Emphasis on biological factors such as coral growth and reproduction \\
** Interdisciplinary Approaches \\
*** Collaboration between geologists, marine biologists, and oceanographers \\
*** Diverse methodologies leading to different interpretations \\
- Regional and Case-Specific Variations \\
** Geographical Differences \\
*** Variations in reef types and formations across different regions \\
*** Influence of local environmental conditions and geological settings \\
** Case Studies of Specific Islands \\
*** Unique formation histories of individual atolls and coral islands \\
*** Examples from the Pacific, Indian, and Atlantic Oceans \\
- Ongoing Research and Discoveries \\
** New Findings and Theories \\
*** Continuous exploration leading to new hypotheses and models \\
*** Advances in climate science impacting understanding of historical sea levels \\
** Reevaluation of Existing Theories \\
*** Critical assessment of long-standing theories with new data \\
*** Adaptation and refinement of theories over time

\subsubsection{Theory 1: Subsidence Theory}
- Origin and proponents \\
** Charles Darwin (The Structure and Distribution of Coral Reefs, 1842) \\
** Alfred Russel Wallace \\
- Mechanism \\
** Initial formation around a volcanic island (fringing reef) \\
** Gradual sinking of the volcanic island (subsidence) \\
** Transition to a barrier reef with a lagoon \\
** Complete submersion of the volcanic island leading to atoll formation \\
- Supporting evidence \\
** Observations from the HMS Beagle voyage \\
** Modern geological surveys and core samples

\subsubsection{Theory 2: Growth on Submarine Mountains Theory}
- Origin and proponents \\
** John Murray \\
** Alexander Agassiz \\
- Mechanism \\
** Coral colonization on underwater mountains (guyots) \\
** Vertical growth of corals towards the sea surface \\
** Formation of atolls without the need for subsidence \\
- Supporting evidence \\
** Studies on deep-sea coral formations \\
** Distribution of guyots and seamounts in tropical regions

\subsubsection{Theory 3: Sea Level Change Theory}
- Origin and proponents \\
** Reginald Daly (The Coral Reef Problem, 1915) \\
- Mechanism \\
** Impact of glacial and interglacial periods on sea levels \\
** Lowered sea levels exposing coral reefs, allowing vertical growth \\
** Rising sea levels creating conditions for atoll formation \\
- Supporting evidence \\
** Geological records of sea level changes \\
** Correlation with coral reef growth periods

\subsubsection{Theory 4: Erosion and Sedimentation Theory}
- Origin and proponents \\
** Maurice Ewing \\
** William Donn \\
- Mechanism \\
** Erosion of coral reefs by waves and currents \\
** Accumulation of coral debris forming islands \\
** Continuous process of erosion and sediment deposition \\
- Supporting evidence \\
** Sediment analysis around coral reefs \\
** Observations of island formation from coral rubble

\subsubsection{Theory 5: Platform Reef Theory}
- Origin and proponents \\
** William Morris Davis \\
- Mechanism \\
** Coral growth on stable continental or insular platforms \\
** Formation of reef structures (platform reefs) \\
** Development into coral islands when conditions are favorable \\
- Supporting evidence \\
** Studies on platform reefs and their distribution \\
** Geological stability analysis of coral platforms

\subsubsection{Comparative Analysis}
- Similarities and Differences \\
** Common themes: coral growth, geological activity, sea level changes \\
** Differences in primary mechanisms (subsidence vs. growth on seamounts) \\
- Complementary Aspects \\
** Integration of multiple theories for a comprehensive understanding \\
** Case studies showing multiple processes at work 

\subsubsection{Modern Advances and Technologies}
- Geological Surveys \\
** Core sampling techniques \\
** Seismic and sonar mapping \\
- Environmental Studies \\
** Impact of climate change on coral growth and sea levels \\
** Conservation efforts and their implications for island formation

\subsection{Darwinian theory}
- Several theories \\
- Difficulty in studying environments \\
** Use of observations \\
- Theory refuted by \cite{Droxler2021} \\
** But it's too early to judge \\
** Practical for our case. \\
- ...


\subsection{Overview}
- Proposed tool \\
** Sketching the island \\
** Sketching the profile \\
** Wind simulation \\
- Automation of examples \\
** Data augmentation \\
- ...

\section{Related works}
\label{sec:coral-island_related-works}
- Perlin + falloff \\
** But lacks control \\
- Uplift [SCHOTT UPLIFT] \cite{Cordonnier2016,Cordonnier2017a} \\
** But we propose a method that limits the required interaction \\
- Sketching [PEYTAVIE SKETCHING, EMILIEN FIRST PERSON] \cite{Gain2009} \\
** Added radial constraint to simplify interaction \\
- Geological modeling \cite{Patel2021} \\
** For us, it's hard to know what's happening underground (?) \\
- Unlike the literature, integration of the underwater part \\
** Integration of observation data into our process \\
*** -> Typical shape and profile of an island \\
** Difficult to integrate physical simulations due to uncertainty \\
- Layer-based generation could improve the realism of examples by considering layer/environment interactions. \\
- ...

\section{Example generation}
\label{sec:coral-island_example-generation}
- We use Darwin's theory in our case because generation is relatively simple. \\
- 2 steps: \\
** Generation of the island \\
** Generation of the reef \\
- Numerous assumptions: \\
** Island has a relatively round shape \\
** Coral grows to a constant height around the island \\
** All "features" are radial \\
** Deformations of islands caused by winds and waves \\
** Islands are independent of each other \\
** The profile is relatively identical all around the island \\
- ...

\subsubsection{Input}
- Sketching of the island from above + profile view \\
- Wind strength \\
- Water level \\
- Subsidence rate \\
- ...

\subsubsection{"Simulation"}
- Subsidence calculated by simple scaling \\
** Note: No geological consistency \\
*** Here, using a zero level to scale in Z \\
*** No consideration of different soil materials \\
- Reef growth \\
** Does not consider any weather events \\
** Considers the "keep up" strategy (as opposed to "give up" and "catch up") [large simplification] \\
- Wind deformation \\
** using directly $f(\p) = f(\warp(\p))$. \\
- ...

\subsubsection{Output}
- Height map of the island's surface \\
- Island zones and coral zones \\
- Possibility to recalculate ground height and coral height \\
- ...

\subsection{Closed form of coral growth}
- Proposes a closed-form solution for surface calculation \\
- Calculation of ground height: \\
** Calculation of height by revolution around the origin point \\
** Deformation of the revolution profile using the top-down sketch \\
** Deformation of the height field by the wind map. \\
- Calculation of growth: \\
** On the initial heightmap, \\
*** Any surface $z_{min} < h(p) < z_{max}$ becomes coral \\
*** Calculation of the "low" contour $h(p) = z_{min}$ and "high" contour $h(p) = z_{max}$ \\
*** ... \\
- Deformation of the map using the wind map: \\
** ... \\
- ...

\subsection{Labeling of the map}
- Using top-down sketch: \\
** Features "Mountain", "Island borders", "Beach" are radial $(\theta, r)$, so we can label each point of the map as the "next" feature \\
- Using coral simulation information: \\
** Provides the labels "Lagoon" and "Reef \{begin, peak, end\}".
- The height map is directly associated to the feature map \\
- This is perfect to feed a cGAN. \\
- ... 

\subsection{Automation}
- Take advantage of the radial nature of the features \\
- Some features can be optional (mountains) \\
- Deformation of the feature lines \\ 
** Influence on the radius: $\Tilde \radius(\theta) = \radius(\theta) + \noise(\theta)$ with $\noise$ continuous noise function $2\pi$-periodic. \\
- ...

\section{cGAN}
\label{sec:coral-island_cGAN}
- Conditional GAN: A type of Generative Adversarial Network (GAN) where the generation process is conditioned on additional information. \\
- ...

\subsection{Definition of cGAN}
- Two Networks: Consists of two neural networks, a Generator and a Discriminator, which compete against each other. \\
** Generator: Takes both random noise and additional information (like class labels or data) to produce synthetic data. \\
** Discriminator: Evaluates whether a given data instance is real (from the actual dataset) or fake (produced by the Generator), while also considering the additional information. \\
** Additional Information: This can be labels, data from other modalities, or any other contextual information that guides the generation process. \\
- Training Process: The Generator tries to create realistic data to fool the Discriminator, while the Discriminator tries to correctly classify data as real or fake based on both the data and additional information. \\
- Objective: The goal is to improve the Generator’s ability to produce realistic data that matches the given conditions and to enhance the Discriminator’s ability to distinguish between real and fake data. \\
- Applications: Used in various domains including image-to-image translation, text-to-image synthesis, and other tasks where generating data based on specific conditions is required. \\
- ...

\subsection{Why cGAN?}
- Flexibility of input \\
- Moving beyond the "radial" input condition \\
- Output even for "inconsistent" data (e.g., ocean in an island) \\
- No math, geometry, geology, or complicated things to master (hehe) \\
- ... 

\subsection{Training}
- ...

\subsubsection{Use of synthetic data}
+ Problem with synthetic data \\
- ...

\subsubsection{Data augmentation}
- ...

\subsection{Model usage}
- ...

\subsubsection{Generation from sketch}
- ...

\subsubsection{Interactive times}
- ...

\subsubsection{Realism}
- ...


\graphicspath{{"./Chapter 2/Figures/"}}

\chapter{Generation of karst networks}
\label{chap:karsts}
\minitoc

- ...

\section{Introduction}
\label{sec:karst_introduction}
- Definition: complex subterranean systems formed primarily in soluble rocks like limestone, dolomite, and gypsum. \\
- Importance: significant for water resources, unique ecosystems, and land use management. \\
- Formation and development \\
** Chemical weathering \\
*** Study the role of rainwater acidity (carbonic acid) in dissolving carbonate minerals. \\
** Underground drainage \\
*** Analyze the development of subterranean drainage systems, including caves, tunnels, and sinkholes. \\
** Speleogenesis \\
*** Examine the process of cave formation, especially at or below the water table. \\
- Key features of karst networks \\
** Caves and caverns \\
*** Document major cave systems and their formation processes. \\
** Sinkholes (dolines) \\
*** Identify areas prone to sinkhole formation and study their causes. \\
** Disappearing streams and springs \\
*** Trace the path of streams that vanish into the ground and re-emerge. \\
** Karst valleys \\
*** Investigate valleys formed by the collapse of underground voids. \\
** Stalactites and stalagmites \\
*** Study the formation of these features within caves. \\
- Hydrology of karst networks \\
** Aquifers \\
*** Assess the productivity and structure of karst aquifers. \\
** Groundwater flow \\
*** Model the rapid and turbulent flow of water through karst systems. \\
** Vulnerability to contamination \\
*** Evaluate the risks and sources of contamination in karst aquifers. \\
- Ecology of karst environments \\
** Unique habitats \\
*** Research cave-dwelling species (troglobites) and their adaptations. \\
** Biodiversity hotspots \\
*** Identify and document biodiversity hotspots in karst regions. \\
- Human interaction with karst networks \\
** Water resources \\
*** Study the dependence of regions on karst aquifers for freshwater. \\
** Tourism \\
*** Assess the impact of tourism on karst landscapes and caves. \\
** Land use challenges \\
*** Develop guidelines for construction and land use planning in karst regions. \\
** Environmental concerns \\
*** Identify and mitigate pollution and land degradation impacts. \\
- Examples of karst regions \\
** The Mammoth Cave System (USA): world's longest cave system. \\
** The Guilin Karst (China): unique limestone peaks and river systems. \\
** The Dinaric Karst (Balkans): extensive cave systems and karst phenomena. \\
- Research and study in karst science \\
** Speleology \\
*** Promote the scientific study of caves and karst features. \\
** Geohydrology \\
*** Conduct studies on water flow in karst systems for resource management. \\
** Geomorphology \\
*** Research the development of karst landforms over geological timescales. \\
- Action steps \\
** Field surveys and mapping \\
*** Conduct detailed field surveys and create maps of karst features. \\
** Hydrological studies \\
*** Implement hydrological modeling and water quality testing. \\
** Ecological research \\
*** Carry out biodiversity assessments and ecological studies in karst habitats. \\
** Human impact analysis \\
*** Study the impact of human activities on karst systems and develop mitigation strategies. \\
** Education and outreach \\
*** Educate local communities and stakeholders about the importance of karst networks and sustainable practices. \\
- Goals \\
** Conservation \\
*** Preserve unique karst ecosystems and landscapes. \\
** Sustainable development \\
*** Ensure that human activities in karst regions are sustainable and minimize environmental impacts. \\
** Scientific advancement \\
*** Promote research and understanding of karst processes and features.

\section{Related works}
\label{sec:karsts_related-works}
- State of the art: \\
** \cite{Paris2021} \\
*** In our case, we do not rely on a graph and path finder, which allows us to compute sections of the karst on the fly (no need to know the whole network to find paths) \\
** \cite{Pytel2015} \\
*** Based on voxels, looks plausible, with a low number of parameters, but take way too long (\~10 to 20 minutes per generation) \\
*** Which makes it unfit for user interaction \\
** \cite{Collon2015,Collon2017} \\
- With some imagination, we can see the shape of trees: \\
** [Prusinkiewicz : ANY] \\
** \cite{Runions2008} \\
- Or miscelleanous networks: \\
** \cite{Galin2010, DiasFernandes2018} \\
- Maybe even look for lychen or ant colonies \\
- ...

\section{Space colonization}
\label{sec:karsts_space-colonization}
- Not really a tree, but... \\
** Some networks are branching \\
** Some have low number of cycles \\
- SC is easy to manipulate for an user \\
- For these reasons, we will consider the use of SC. \\
- Description of the algorithm: \\
** Definition of a root and sinks \\
** Evolution from the root toward closest sinks \\
** Branching [WHEN NEEDED] \\
- ...

\section{Our method}
\label{sec:karsts_our-method}
- Based on classic SC \\
** Merging branches \\
** Closing paths based on angle and distances to create cycles \\
- Adding width to the branches \\
** Destroying paths when width too small \\
- Leaves as chambers/cavities \\
- ... 

\section{Modeling}
\label{sec:karsts_modeling}
- Tunnels: \\
** The output of SC is a set of paths \\
** \cite{Paris2021} provides a classification of tunnel shapes (\cref{fig:karsts_tunnel-classif}). \\
- ...

\begin{figure}
    \includegraphics{KarstClassificationParis}
    \caption{Classification of tunnel shapes}
    \label{fig:karsts_tunnel-classif}
\end{figure}

\section{User control}
\label{sec:karsts_user-control}
- Importance of keeping user control \\
** Shape of a karst is close to randomness \\
** Want to be predictible \\
- Manipulation of control points \\
** Source \\
** Sink \\
- Paths tortuosity \\
- Inclusion of soil properties in the formation of paths, using the gradient of: \\
** Humidity, \\
** Porosity \\
- Real-time editing \\
** Allows for precise manipulation \\
- ...

\section{Results}
\label{sec:karsts_results}
- ... 

\section{Conclusion}
\label{sec:karsts_conclusion}
- ...