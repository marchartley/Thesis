\chapter{Influence sur la génération}
\label{chap:influence-on-env-objects}
\minitoc


\section{Introduction}
\label{sec:influence-on-env-objects_introduction}
This document aims to formalize a terrain generation method developed during my thesis.

The main issue this method seeks to address is the difficulty for a user to describe the environments they want to generate. By proposing a hierarchical structure in the description of elements to define, the environments exhibit adaptive representation granularity, catering both to the user's needs and hardware capabilities.

The key element of the method is the "biotope," defined as "a habitat defined by relatively uniform physical and chemical characteristics." (Source: Wikipedia)

In this context, we define the characteristics of a geographical region as well as the sub-biomes included within it.

Although this method was conceived with the aim of generating underwater environments, many examples given in this document will use elements present in surface environments, purely for simplification purposes.

\section{Glossary}
\label{sec:influence-on-env-objects_glossary}
Initially, this document will present definitions of various terms used in the description of the method. By doing so, we hope to eliminate any ambiguity in the upcoming explanations. Definitions may appear in certain sections of this document if deemed too specific.

\textbf{Biotope}: The main element of the method used. The biotope (or, as I might often write, the biome) is a geographical region defined by topographic and geomorphological characteristics. The notion of biotope used does not distinguish between different scales (the biosphere and the ecosystem of a rock are represented uniformly), nor does it distinguish between living and mineral elements. The biotope is represented in two different ways in the terrain generation process: the model and the instance.

\textbf{Model (of biotope)}: The model (or biotope model, if the context requires) is a biological description of the geographical area. It remains an abstract, or literary, form of the biotope. It is defined by a list of characteristics related to topography (description of the relief, description of the shape of the region) and geology (type of soil, for example). One can also list the sub-biomes that compose it, specifying the quantity, proportion, chances of appearance, etc., of each sub-biome. This form is the user input into the system, but during the terrain generation process, the model's role is to create instances of itself to concretize the structure to be generated. In the document illustrations, if a model is represented in a graph, it will be symbolized by its name in quotation marks (e.g., "Forest," "Lagoon").

\textbf{Instance (of biotope)}: The instance is the concrete form of the model. Unlike the model, the instance takes place in space with fixed characteristics (unlike the probabilistic characteristics of the model). Through generation rules and living conditions defined by the user and the system, the biotope instance has a geometric shape and neighboring relationships with other instances. An instance, like its original model, consists of sub-biomes. The instance remains linked to its original model. To distinguish the instance from the model in the document illustrations, if represented in a graph, the instance will be symbolized by the model's name followed by a hash and an instance number (e.g., Forest \#1, Lagoon \#28).

\textbf{Generation Rule}: Generation rules represent a list of obligations or prohibitions, including living conditions and adjacency rules.

\textbf{Living Condition}: In our definition, a biotope can only exist under certain environmental conditions in a specific place. The conditions can be multiple and will be listed in a subsequent chapter of this document. Vegetative elements, for example, have strong constraints in terms of sunlight exposure, the "snow" biotope requires a low temperature condition, and some biotopes like sand cannot exist on steep terrain. Living conditions are therefore important properties to express in biotope models. These conditions can be expressed probabilistically.

\textbf{Adjacency Rules}: Adjacency rules define how biotope instances can be arranged. There are three types of rules to express adjacency between two biotopes: prohibition, possibility, and obligation. Two models linked by an adjacency prohibition cannot generate instances with a common border. Conversely, a model linked to another by an obligation constraint can only generate an instance if it shares a border with an instance of the linked model. The possibility constraint is the default, where the instance is indifferent to the presence or absence of a common border. It is conceivable that the notion of "possibility" could be associated with a probability in the future, allowing for example, a lake to often be adjacent to an urban area, but an urban area to rarely be near a desert. For now, adjacency rules are bilateral: a rule that applies from biotope A to biotope B also applies from B to A. In the rest of the document, adjacency rules will mainly be represented by graphs (undirected) with the nodes as the models to be generated, solid edges representing the possibility of adjacency, dotted edges symbolizing an adjacency obligation (rarely used), and the lack of an edge indicating an adjacency prohibition.

\textbf{Adjacency Graph}: The adjacency graph, although similar to the representation of adjacency rules, this time represents the concrete topological adjacency between the generated instances (not the models). The adjacency graph consists of nodes representing each instance generated by the models and the edges representing the existence of a common border. Edges can only be present if the two biotope instances see their model linked either by an obligation or possibility of adjacency constraint. By definition, this graph must be planar. In this document, adjacency graphs will therefore be represented by graphs where the nodes are biotope instances (following the instance graphic convention), and an edge represents a topological adjacency.

\textbf{Region (of biotope)}: The region is the space occupied in a geometric sense by a biotope instance once represented in space. Among the important properties, one can note that a region has a shape and an area.

\textbf{Boundaries}: The boundary is the geometric adjacency of two regions representing biotope instances.

\textbf{Adjacent Regions}: Two regions are adjacent if they share a common edge. In a successful generation, a region shares a common boundary with regions represented by instances linked to the current region instance in the adjacency graph. We can then distinguish three types of connections between instances: correctly adjacent instances are linked in the adjacency graph and are adjacent regions, over-adjacent instances are not linked in the adjacency graph and are adjacent regions, and under-adjacent instances are linked in the adjacency graph and are not adjacent regions. The latter two types are maladjacencies.

\textbf{Biotope Characteristics}: Biotopes possess characteristics common at all scales, whether they represent geological or biological elements. The characteristics are defined in the models in probabilistic forms. Instances inherit a unique and fixed value for each characteristic from the original model. The characteristics notably define the shape the generated region will take, the relief, etc.

\textbf{Sub-biotope}: Each biotope can include one or more sub-biomes. This is how an environment can be defined on the scale of a planet as well as on the scale of a rock. The biotope specifies the number of instances to generate for each sub-biome, or the proportion of the surface to cover. The parent biotope provides an approximate description of its environment, but with each sub-biome, the description is refined.

\textbf{Recursive Tree}: The proposed method has a strong recursive nature through its representation in biotopes and sub-biomes. The biotope construction tree is thus a recursive tree of biotopes composed at the root of a biotope to be generated and whose nodes (and leaves) represent sub-biomes. The parent-child relationship defines the inclusion of the child among the sub-biomes of the parent node. The tree describing a biotope can then be reused to describe a larger biotope, up to a planetary scale or larger.

\textbf{User Action}: The user plays an important role in the terrain generation process. By user action, we mean any action performed during the process. User actions are applied to biotope instances: adding, replacing, or deleting an instance, or modifying the characteristics and living conditions of an instance.

\textbf{Primitives}: Primitives are simple geometric representations that can be combined to represent more complex structures. Among these, we can notably find: the point, the curve, the sphere, and the 3D model.

\section{Environment description}
\label{sec:influence-on-env-objects_environment-description}
In this method, an environment is defined by a biotope model for which characteristics such as the general shape of its region, the biotope's relief, or the representative primitive of the biotope are specified, as well as living conditions such as altitude, temperature, and necessary luminosity for its survival. These characteristics are uniform across the entire biotope region; the only way to vary them is to add sub-biomes in the definition, which will take the form of sub-regions of the parent. Sub-biomes have the same types of characteristics and living conditions, which they can define in their own way, as well as their own sub-biomes.

Thus, the environment description is carried out recursively until the level of detail is sufficient for the modeler.

A biotope model can be the sub-biome of several different biotope models, offering the possibility to define complex environments quickly.

\subsection{Biotope characteristics}
The biotope model allows the user to describe the biotope's appearance to be generated. This description is probabilistic, and instances derived from the model each define a unique value for each characteristic respecting the probability law described by the model. The list of characteristics includes:
\begin{Itemize}
	\Item{} Region size: This represents the area of the surface (without considering relief) that the region occupies in space.
	\Item{} Altitude (min and max): These two characteristics represent the altitudes the biotope can reach.
	\Item{} Relief (frequency and amplitude): These characteristics describe the terrain's gradient. High relief amplitude symbolizes significant disturbances on the ground, unlike low amplitude, which
	
	indicates the ground is relatively flat.
	\Item{} Primitive (type, dimension, position): The primitive determines the basic element represented by the biotope.
	\Item{} Shape: The biotope region's shape in space.
\end{Itemize}

To facilitate the user's work, default values are proposed for each biotope, respecting each type of primitive.

\subsection{Living conditions}
The living conditions of a biotope define the specific properties necessary for the biotope to survive. To be viable, the biotope must respect these conditions. These properties may relate to the following conditions:
\begin{Itemize}
	\Item{} Luminosity (min and max)
	\Item{} Temperature (min and max)
	\Item{} Soil type
	\Item{} Exposure (direction, minimum and maximum angle)
\end{Itemize}

Like characteristics, default living conditions are proposed to the user.

\subsection{Adjacency rules}
Three types of rules express adjacency between two biotopes: prohibition, possibility, and obligation. These rules define how biotope instances can be arranged spatially.

\begin{Itemize}
	\Item{} Prohibition: Two models linked by an adjacency prohibition cannot generate instances with a common border.
	\Item{} Possibility: Instances can be adjacent, but it is not required.
	\Item{} Obligation: Instances must share a common border.
\end{Itemize}

\section{Generation process}
\label{sec:influence-on-env-objects_generation-process}
The terrain generation process involves creating instances from biotope models while respecting their characteristics, living conditions, and adjacency rules. This section details the steps of the process.

\subsection{Instance creation}
Each biotope model creates instances with fixed values for each characteristic derived from the probabilistic descriptions in the model. These instances are then positioned in space according to their characteristics and living conditions.

\subsection{Placement and adjustment}
Instances are placed in the environment, respecting their adjacency rules. If instances violate adjacency constraints, adjustments are made to either reposition instances or modify their characteristics within acceptable limits.

\subsection{User interaction}
Users can intervene in the generation process by adding, replacing, or deleting instances, as well as modifying instance characteristics and living conditions. These actions allow for fine-tuning the generated environment.

\subsection{Validation and refinement}
The final step involves validating the generated environment by checking for over- or under-adjacencies and making necessary refinements to ensure all instances adhere to their constraints and living conditions.

\section{Conclusion}
\label{sec:influence-on-env-objects_conclusion}
This method provides a structured approach to terrain generation by leveraging biotopes and their hierarchical organization. By defining biotopes probabilistically and respecting adjacency rules and living conditions, it is possible to generate diverse and realistic environments with varying levels of detail.

Future work may include refining probabilistic models for characteristics and adjacency rules, as well as exploring the use of probabilistic adjacency possibilities to further enhance the flexibility and realism of generated terrains.