\chapter{Introduction}
\label{chap:introduction}
% \minitoc

% \section{Procedural generation}
% \label{sec:introduction_procedural-generation}
Over the past 50 years, terrain generation has emerged as an increasingly active domain inside the field of computer graphics. As the demand for more realistic, faster, and automated processes has grown, terrain generation techniques have evolved to meet these needs. As these objectives get progressively achieved, a new trend has started to emmerge in which focus is pushed toward user control. The focus of this work is on one specific branch of terrain generation: the interactive creation and modeling of landscapes.

The utility of landscape generation goes beyond purely scientific pursuits, finding applications in diverse fields such as biology, geology, and robotics. Additionally, it has become a central tool in the entertainment industry, particularly in video games and cinema, where creating realistic and dynamic environments is key for immersive experiences. The ability to generate landscapes that help in understanding natural rules and testing hypotheses makes this the ideal tool for both researchers and industries.

What sets this work apart is its emphasis on underwater landscapes or "seascapes", an area still overviewed but is important in domains such as marine biology, oceanography, and underwater robotics. Furthermore, the extension of these techniques to new areas of the entertainment industry, such as video games and cinema, presents new opportunities for innovation.

However, this exploration of the domain comes with significant difficulties. Finding a balance between automation and the user's creative desires is essential, as is the ability to isolate variables and manage scaling effectively. 
% There are three main approaches to generating virtual worlds: simulations, user interaction/modeling, and procedural modeling. Procedural modeling, in particular, is noteworthy for its potential in "automatic generation," where content is created using functions independently of user actions. For example, Perlin noise might be used for terrain generation, random noise for rock modeling, or specific algorithms for texture generation.
The central question guiding this research is: "How can we efficiently guide the user in the creation of virtual content along the production process line to maintain as much control as possible over the final product?" This concept of "guiding" rather than "replacing" the user is, from my point of view, fundamental, as no machine can truly know better than a human what the final product should look like. The goal is to present algorithms that can be used flexibly within a production pipeline, adapting to many terrain representations, fluid solvers, landscape types, user's hardwares, or objectives, whether the final use is real-time rendering, realism, or animation.

Maintaining control over the creative process is essential. Each generated result should feel unique, meet the user's expectations, be explainable, and be easily correctible without requiring a complete regeneration. This approach ensures that the user remains at the center of the creative process, with the tools and algorithms serving to enhance their objectives, rather than replacing the users themselves.

\section{Contributions and outlines}
This thesis explores the procedural generation of underwater environments, with a particular focus on coral reef islands. Our contributions are organized into three complementary parts, covering the creation, structuring, and physical evolution of underwater terrains.

% This thesis proposes several contributions in different branches of terrain generation. We will divide these works in three parts: the semantic representation of environments, the modeling of specific features and the terrain enhancement. Those three parts are presented in the chronological order in which a modeler would create his work: first drawing the esquisse of his artwork to see what it could look like coarsely at the end, then produce each element one by one, and finally add the final details that improve the whole. 

% This research follows the same order for terrain generation, beginning from the development of an abstract representation that bridges the gap between computer science and Earth science, offering a generalization of the desired landscape type, with a particular focus on underwater environments, but also applicable to terrestrial landscapes (\cref{chap:semantic-representation}). Secondly, we will develop on the problematic of 3D modeling of some natural features, and more specifically on coral reef islands (\cref{chap:coral-island}) and karst networks (\cref{chap:karsts}). Finally, we propose an erosion method that amplify the realism of the virtual terrain (\cref{chap:erosion}).
% %  a new particle-based erosion simulation method is introduced in \cref{chap:erosion}. This method is agnostic to the terrain representation, making it lightweight, fast, and easy to implement. 

% The overarching aim is to provide maximum control to the user, not only during the generation process but also in making corrections to details upstream, ensuring that the final product aligns with the user's vision.

\subsubsubsection{Automatic generation of coral reef islands}
In \cref{chap:coral-island}, we propose a user-guided method for the procedural creation of coral reef islands. Users sketch the island shape and elevation from two projections and define a wind field that simulates long-term environmental deformation. The system models coral growth and outputs heightmaps that are further used to train a conditional generative adversarial network (cGAN) for diversified generation. This method enables rapid and controllable generation of varied reef island configurations.

\subsubsubsection{Semantic terrain representation}
% [TODO]
% The work is heavily oriented towards underwater landscape generation, developed in collaboration with marine biologists to ensure scientific accuracy and relevance. 
% This interdisciplinary approach enriches the modeling process, ensuring that the generated landscapes reflect real-world biological and geological phenomena.
The semantic representation we present in \cref{chap:semantic-representation} aims to design the features of a terrain with an abstraction the 3D aspect of the surface. We will introduce \glosses{EnvObj} and their simplified representation from the real world, which we used to obtain a symbolic representation of the terrain features, biotic and abiotic, that are present in the scene. Using symbolism allows us to focus on the interactions between the different elements of an environment without the high computational needs of running an accurate multiphysics simulation. Moreover, the simplified representation used allows the user to manipulate the shape of the final terrain, without having to choose a specific terrain representation.

% \subsubsubsection{Modeling}
% As the global layout of the terrain is defined, the 3D modeling of the terrain surface can begin. Many methods have been proposed in the computer graphics community to shape specific terrain features using procedural methods. We will propose two new contributions to these methods. 

% In \cref{chap:coral-island}, we present a method to model a coral reef island from a sketch by using the properties that have been observed in travel journals, such as the radiality, the global structure of the visual features, the typical profile. The method is simple but constraint. We then use the procedural paradigm of sketching to create a large synthetic dataset of terrains, that is fed in a deep learning generative model to unlock the limitations of the initial sketching. This pipeline shows that we can first use the given properties to create a simplified model and then, by simply fine-tuning an off-the-shelf pretrained GAN model, remove these constraints to improve the user's possibilities.

% In \cref{chap:karsts}, we propose a new algorithm for the generation of karst networks. This geomorphological feature has been studied previously with a geologist point of view. We provide a new viewpoint on the problem by focusing on the non-expert user control to shape the structure of the network by using a directed acyclic graph, which closely resembles a tree structure. This method is enhanced by a fractal generation approach with cycles, allowing for iterative generation of complex landscapes.

\subsubsubsection{Erosion simulation}
To increase the realism and the visual impact of the generated synthetic landscapes, the use of terrain enhancement techniques are often required. In \cref{chap:erosion}, we tackled the specific challenge of running erosion simulations, a type of enhancement that mimics the effects of water, wind and erosive forces on a virtual terrain to improve the belivability of the final landscape. A particle-based erosion method is proposed, designed to be generalizable for flexibility on multiple levels, and its implementation is oriented towards speed and parallelization. The main flexibility of our method is to be applicable to multiple terrain representations, but also to be agnostic to the fluid solver used, and to be generalized for both landscapes and seascapes.





% - Active domain for the last 50 years \\
% - Computer graphics more and more proiminent \\
% - Need to be always faster, realistic, automatic \\
% - Terrain generation affected the same way \\
% - Our focus on one branch: the landscape \\
% - Useful domain for: \\
% ** Biology \\
% ** Geology \\
% ** Robotics \\
% ** But also entertaining industries \\
% *** Video games \\
% *** Cinema \\
% - Because helps understanding rules and observing hypotheses \\
% - Novelty is the underwater aspect \\
% - Useful for: \\
% ** Marine biology \\
% ** Oceanology \\
% ** Underwater robotics \\
% ** And by extension, new fields of entertaining industries: \\
% *** Video games \\
% *** Cinema \\
% - Big challenges: \\
% ** Balance between automation and user desires \\
% ** Isolation of variables \\
% ** Scaling \\
% - 3 main ways to generate worlds: \\
% ** Simulations \\
% ** User interaction/modeling \\
% ** Procedural modeling \\
% *** This one is tricky, maybe "automatic generation" \\
% *** I mean: "Content creation using functions, independent of the user actions" (eg. using Perlin noise for a terrain, random noise for a rock modeling, specific texture generation, etc...) \\
% - Main question: \\
% ** "How to efficiently guide the user in the creation of virtual content along the production process line to keep as much control on the final product?" \\
% *** "Guiding" as opposed to "replace": I believe no machine can know better than a human what he wants \\
% *** "Production process line": we are presenting algorithms that can be used in a pipeline \\
% **** We want each component of the thesis to be as flexible as possible to the user workspace: \\
% ***** Any terrain representation \\
% ***** Any fluid solver \\
% ***** Any landscape type \\
% ***** Any hardware \\
% ***** Any objective (real-time rendering, realistic, animated, etc...) \\
% *** "As much control": 
% **** Each result should feel unique \\
% **** Each result should be expected by the user \\
% **** Each result should be explainable \\
% **** Each result should be correctible(?) -> without having to restart everything \\
% - ...

% \section{Contributions and outlines}
% \label{sec:introduction_contribution-plan}
% - Chronological order of terrain generation \\
% - Proposes an abstract representation halfway between computing and terrain expertise \\
% ** Offering a generalization of the desired landscape type (underwater, but also terrestrial) \\
% - Proposes new types of landscapes (karsts and coral reef islands) \\
% ** Maintaining the notion of sparseness (implicit volumes) \\
% - Proposes a particle-based erosion simulation method \\
% ** Terrain representation agnostic \\
% ** Lightweight, fast, easy to implement \\
% - Aims to keep maximum control for the user \\ 
% ** In the generation process, but also to correct details upstream

% \subsection{Semantics}
% - Work oriented towards underwater generation \\
% - Collaboration with a marine biologist \\
% - ...

% \subsection{Modeling}
% - Generation of some landscape elements still new (karsts referring to Axel Paris, but coral reef islands new) \\
% - Karst networks represented in a highly user-friendly manner \\
% ** Viewing karsts as a directed acyclic graph => close to tree structure \\
% ** Enhanced method for fractal generation with cycles (generation in multiple iterations) \\
% - Coral reef islands using an interpretation of Darwin's theory \\
% ** Based on observations \\
% ** Based on travel journals \\
% - ... 

% \subsection{Enhancement}
% - Increasing realism by adding details \\
% - Particle-based erosion method \\
% ** Generalization for flexibility \\
% ** Speed, parallelization \\
% - Towards a continuous erosion method. \\
% - ... 
