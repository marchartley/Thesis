\chapter{Introduction}
\label{chap:introduction}

Over the past 50 years, terrain generation has emerged as an increasingly active domain inside the field of computer graphics. As the demand for more realistic, faster, and automated processes has grown, terrain generation techniques have evolved to meet these needs. As these objectives get progressively achieved, a new trend has started to emmerge in which focus is pushed toward user control. The focus of this work is on one specific branch of terrain generation: the interactive creation and modeling of landscapes.

The utility of landscape generation goes beyond purely scientific pursuits, finding applications in diverse fields such as biology, geology, and robotics. Additionally, it has become a central tool in the entertainment industry, particularly in video games and cinema, where creating realistic and dynamic environments is key for immersive experiences. The ability to generate landscapes that help in understanding natural rules and testing hypotheses makes this the ideal tool for both researchers and industries.

What sets this work apart is its emphasis on underwater landscapes or "seascapes", an area still overviewed but is important in domains such as marine biology, oceanography, and underwater robotics. Furthermore, the extension of these techniques to new areas of the entertainment industry, such as video games and cinema, presents new opportunities for innovation.

However, this exploration of the domain comes with significant difficulties. Finding a balance between automation and the user's creative desires is essential, as is the ability to isolate variables and manage scaling effectively. 
% There are three main approaches to generating virtual worlds: simulations, user interaction/modeling, and procedural modeling. Procedural modeling, in particular, is noteworthy for its potential in "automatic generation," where content is created using functions independently of user actions. For example, Perlin noise might be used for terrain generation, random noise for rock modeling, or specific algorithms for texture generation.
The central question guiding this research is: "How can we efficiently guide the user in the creation of virtual content along the production process line to maintain as much control as possible over the final product?" This concept of "guiding" rather than "replacing" the user is, from my point of view, fundamental, as no machine can truly know better than a human what the final product should look like. The goal is to present algorithms that can be used flexibly within a production pipeline, adapting to many terrain representations, fluid solvers, landscape types, user's hardwares, or objectives, whether the final use is real-time rendering, realism, or animation.

Maintaining control over the creative process is essential. Each generated result should feel unique, meet the user's expectations, be explainable, and be easily correctible without requiring a complete regeneration. This approach ensures that the user remains at the center of the creative process, with the tools and algorithms serving to enhance their objectives, rather than replacing the users themselves.



\section{Challenges in environment generation for underwater robotics}

This thesis is developed alongside the creation of an underwater marine biology observation robot. In this context, virtual environments are essential since field experiments are costly and prone to failure. Digital twins of the robot and its environment allow simulations of material failure and rare or unpredictable scenarios.

To accurately model a robotic mission, the simulator needs the geometry of the terrain, fauna, and flora that the device and its sensors might encounter. It also requires intrinsic material parameters to simulate sensor behavior and external forces acting on the robot.

Scene geometry must be scaled appropriately to match both the robot, which operates at centimeter precision, and the mission area, which may cover thousands of square meters. Underwater environments also demand consideration of the third dimension. The materials used in the simulation need to respond correctly to light and sound waves, accounting for porosity, granularity, and reflectance, which can distort sensor signals. Water currents strongly affect navigation and require advanced modeling for realistic and interactive simulation.

Designing virtual test fields involves creating surfaces and obstacles, and also the ability to replay simulations with identical or slightly altered configurations. Adding randomness while preserving the designer’s intent presents an additional challenge in maintaining control while introducing variability.


\section{Challenges in environment generation for biology}

Virtual environments play a crucial role in biological research by enabling the observation and study of interactions within ecosystems. Biology relies heavily on phenomenology, where understanding emerges from observing how entities behave and interact under specific conditions.

Simulating these interactions allows scientists to test and refine their theories. Models can reveal unexpected behaviors, highlight gaps in understanding, and uncover hidden correlations. This iterative process between theory, simulation, and fieldwork is central to advancing biological knowledge.

To be useful, simulations must closely follow biological rules while also incorporating controlled randomness. This balance allows for realistic behavior while making room for discovering novel phenomena. Human-Computer Interaction should help bridge the gap between biological expertise and computational tools, enabling domain experts to contribute effectively without introducing bias.

These environments must also operate across multiple spatial scales. Simulations should cover large areas to reflect full ecosystems while still capturing fine-grained interactions between closely located entities.


\section{Challenges of environment generation in Computer Graphics}

Overall, this work lies in the domain of Computer Graphics (CG), specifically in the modeling and interaction branches. 

Designing effective interactive tools requires careful consideration of both user needs and technical constraints. This involves exploring how users interact with terrain elements, defining intuitive controls, and ensuring the interface supports a creative yet efficient workflow. Behind these tools lie critical implementation decisions that directly affect usability and performance. Choosing the right data structures, for instance, determines how well the system handles large or complex environments. The mathematical and physical models used must strike a balance between realism and computational cost, especially in scenarios where interaction must remain fluid and responsive. Algorithms must also be parallelizable to take advantage of modern hardware, enabling real-time updates and feedback. Lastly, simplifying certain aspects of natural phenomena is often necessary to reduce computational overhead while maintaining visual and behavioral plausibility. These layered decisions, from interaction design to algorithmic efficiency, ensure that the system remains both user-centered and technically robust.




\section{Contributions and outlines}
This thesis explores the procedural generation of underwater environments, with a particular focus on coral reef islands. Our contributions are organized into three complementary parts, covering the creation, structuring, and physical evolution of underwater terrains.

% \subsubsubsection{Background}
In \cref{chap:background} we will first introduce the fondamentals of procedural terrain generation. A description of the different terrain representations is provided, as well as an overview of coral biology and coral reefs formation.


% \subsubsubsection{Automatic generation of coral reef islands}
\AltTextImageR{
    In \cref{chap:coral-island}, we propose a user-guided method for the procedural creation of coral reef islands. Users sketch the island shape and elevation from two projections and define a wind field that simulates long-term environmental deformation. The system models coral growth and outputs heightmaps that are further used to train a conditional generative adversarial network (cGAN) for diversified generation. This method enables rapid and controllable generation of varied reef island configurations.
}{Chapter 1/figures/3_results.png}{}{}


% \subsubsubsection{Semantic terrain representation}
\AltTextImageR{
    The semantic representation we present in \cref{chap:semantic-representation} aims to design the features of a terrain with an abstraction the 3D aspect of the surface. We will introduce \glosses{EnvObj} and their simplified representation from the real world, which we used to obtain a symbolic representation of the terrain features, biotic and abiotic, that are present in the scene. Using symbolism allows us to focus on the interactions between the different elements of an environment without the high computational needs of running an accurate multiphysics simulation. Moreover, the simplified representation used allows the user to manipulate the shape of the final terrain, without having to choose a specific terrain representation.
}{Chapter 2/figures/Canyon5.png}{}{}


% \subsubsubsection{Erosion simulation}
\AltTextImageR{
    To increase the realism and the visual impact of the generated synthetic landscapes, the use of terrain enhancement techniques are often required. In \cref{chap:erosion}, we tackled the specific challenge of running erosion simulations, a type of enhancement that mimics the effects of water, wind and erosive forces on a virtual terrain to improve the belivability of the final landscape. A particle-based erosion method is proposed, designed to be generalizable for flexibility on multiple levels, and its implementation is oriented towards speed and parallelization. The main flexibility of our method is to be applicable to multiple terrain representations, but also to be agnostic to the fluid solver used, and to be generalized for both landscapes and seascapes.
}{Chapter 3/results/karst.pdf}{}{}





% - Active domain for the last 50 years \\
% - Computer graphics more and more proiminent \\
% - Need to be always faster, realistic, automatic \\
% - Terrain generation affected the same way \\
% - Our focus on one branch: the landscape \\
% - Useful domain for: \\
% ** Biology \\
% ** Geology \\
% ** Robotics \\
% ** But also entertaining industries \\
% *** Video games \\
% *** Cinema \\
% - Because helps understanding rules and observing hypotheses \\
% - Novelty is the underwater aspect \\
% - Useful for: \\
% ** Marine biology \\
% ** Oceanology \\
% ** Underwater robotics \\
% ** And by extension, new fields of entertaining industries: \\
% *** Video games \\
% *** Cinema \\
% - Big challenges: \\
% ** Balance between automation and user desires \\
% ** Isolation of variables \\
% ** Scaling \\
% - 3 main ways to generate worlds: \\
% ** Simulations \\
% ** User interaction/modeling \\
% ** Procedural modeling \\
% *** This one is tricky, maybe "automatic generation" \\
% *** I mean: "Content creation using functions, independent of the user actions" (eg. using Perlin noise for a terrain, random noise for a rock modeling, specific texture generation, etc...) \\
% - Main question: \\
% ** "How to efficiently guide the user in the creation of virtual content along the production process line to keep as much control on the final product?" \\
% *** "Guiding" as opposed to "replace": I believe no machine can know better than a human what he wants \\
% *** "Production process line": we are presenting algorithms that can be used in a pipeline \\
% **** We want each component of the thesis to be as flexible as possible to the user workspace: \\
% ***** Any terrain representation \\
% ***** Any fluid solver \\
% ***** Any landscape type \\
% ***** Any hardware \\
% ***** Any objective (real-time rendering, realistic, animated, etc...) \\
% *** "As much control": 
% **** Each result should feel unique \\
% **** Each result should be expected by the user \\
% **** Each result should be explainable \\
% **** Each result should be correctible(?) -> without having to restart everything \\
% - ...

% \section{Contributions and outlines}
% \label{sec:introduction_contribution-plan}
% - Chronological order of terrain generation \\
% - Proposes an abstract representation halfway between computing and terrain expertise \\
% ** Offering a generalization of the desired landscape type (underwater, but also terrestrial) \\
% - Proposes new types of landscapes (karsts and coral reef islands) \\
% ** Maintaining the notion of sparseness (implicit volumes) \\
% - Proposes a particle-based erosion simulation method \\
% ** Terrain representation agnostic \\
% ** Lightweight, fast, easy to implement \\
% - Aims to keep maximum control for the user \\ 
% ** In the generation process, but also to correct details upstream

% \subsection{Semantics}
% - Work oriented towards underwater generation \\
% - Collaboration with a marine biologist \\
% - ...

% \subsection{Modeling}
% - Generation of some landscape elements still new (karsts referring to Axel Paris, but coral reef islands new) \\
% - Karst networks represented in a highly user-friendly manner \\
% ** Viewing karsts as a directed acyclic graph => close to tree structure \\
% ** Enhanced method for fractal generation with cycles (generation in multiple iterations) \\
% - Coral reef islands using an interpretation of Darwin's theory \\
% ** Based on observations \\
% ** Based on travel journals \\
% - ... 

% \subsection{Enhancement}
% - Increasing realism by adding details \\
% - Particle-based erosion method \\
% ** Generalization for flexibility \\
% ** Speed, parallelization \\
% - Towards a continuous erosion method. \\
% - ... 
