\chapter{Introduction}
\minitoc

- ...

\section{Génération procédurale}
- ...

\subsection{Définition}
- ...

\subsubsection{Définition officielle}
- ...

\subsubsection{Ma définition}
- ...

\subsection{Historique}
- ...

\subsection{Modèles représentés}
- ...

\subsubsection{Bruit}
- ...

\subsubsection{Automates cellulaires}
- ...

\subsubsection{Réseaux de neuronnes}
- ...

\subsubsection{Modélisation de phénomènes physiques}
- ...

\subsection{Interaction utilisateur}
- ...

\subsubsection{Balance réalisme-rapidité-control}
- Principale problématique de la génération de terrains \\
- Explication réalisme \\
** Paysages générés sont proches de ce qui se retrouve dans la réalité \\
** La génération intègre les processus naturels pour être réaliste \\
** Demande beaucoup de simulations physiques, de connaissances experts \\
** Utile dans les applications de simulation de catastrophes naturelles, par exemple \\
- Explication rapidité \\
** Génération la plus rapide possible, l'objectif étant la génération en temps-réel \\
** J. Gain classifie: temps-réel (< 30ms), interactif (< 3s), proche de interactif (< 5min) et long terme. \\
** Utile dans les jeux vidéos "infinity-scroll", par exemple\\
- Explication control \\
** Cherche à satisfaire les demandes de l'utilisateur \\
** Grosse problématique étant les demandes "impossibles" de l'utilisateur. \\
** Utile dans la plupart des applications de la gen. proc. : accélerer le travail d'artistes, par exemple \\
- ...

\subsubsection{Régénération}
- Actions manuelles \\
- Problématiques de la régénération \\
** Quoi regénérer ? \\
** Comment régénérer ? \\
** Problèmes des interactions utilisateur ? \\
- ...



\section{Representation de terrains}
- ...

\subsection{Terrains 2.5D}
- ...

\subsubsection{Cartes de hauteur}
- ...

\subsubsection{Fonctions de hauteur}
- ...

\subsection{Terrains 3D}
- Besoin de notions 3D \\
** Information geologique \\
** Données volumiques \\
- ...

\subsubsection{Problématiques principales}
- Mémoire \\
- Visualisation \\
- Modifications \\
- Conversion entre representations \\
** Pertes d'informations \\
*** Propagation d'erreur sur la géométrie (approximations sur les normales, résolution Z, surface, etc...) \\
*** Perte d'informations sous-terraines \\
- ...


\subsubsection{Types, définitions, avantages, inconvéniants}
- Grilles de voxels \\
- Piles de matériaux \\
- Maillage \\
- Surfaces implicites \\
- ...

\subsection{Autres modèles}
- Notion de sémantique \\
- ...

\subsection{Paysages sous-marins}
- Données 3D \\
** Paysages coralliens remplis de vide \\
** Beaucoup de cavités (caves, grottes, reseaux karstiques) \\
- Données interdisciplinaires \\
** Plutôt commun avec generation de terrains\\
** => validation géologique avec des experts \\
** Pour sous-marin, \\
*** Moins d'experts, \\
*** Plus d'incertitudes \\
*** Plutôt basé sur des observations \\
*** Peu de données (paysages coralliens < 0.1\% des océans), pour gros impact biologique (25\% biodiversité marine) \\
** Mélange de géologie, biologie, hydrologie et physique (des fluides, notamment) \\
- Besoin de multi-échelles \\
** Pas limité au sous-marin \\
** Intégrer de gros éléments (montagnes) avec des petits éléments (végétation). \\
** LOD \\
- ...



\section{Paysages coralliens (partie biologique)}
- Historique découverte des récifs de corail \\
- Iles, barrières, atolls \\
- Théories des atolls \\
- Utilité dans la biodiversité \\
- Menaces, protection, importance de les comprendre \\
- ...


\section{Création du prototype}
- C++23 et Qt5.12 \\
- OpenGL 4.6 et GLSL \\
- Marching Cubes sur geometry shader \\
** Mauvaise idée, mais justifier le pourquoi \\
- Rendus :\\
** Avec le prototype : \\
*** Triplanar texture \\
*** Résultats temps-réels \\
*** Textures selon matériaux \\
** Avec Unreal Engine 5 : \\
*** Maillages statiques \\
*** Ajout de végétation procédurale avec plugin [NOM DU PLUGIN] et océan avec [NOM DU PLUGIN] \\
** Avec Blender 4.1 \\
*** Maillages statiques \\
*** Utilisation de scripts plus facile \\
- ...

\section{Contributions et plan}
- Ordre chronologique de la génération de terrains \\
- Propose une representation abstraite à mi-chemin entre informatique et expertise terrain \\
** Offrant une généralisation du type de paysage souhaité (sous-marin, mais aussi terrestre) \\
- Proposition de nouveaux types de paysages (karsts et iles coralliennes) \\
** Conservation de la notion de parcimonie (volumes implicites) \\
- Propose une méthode de simulation d'érosion par particules \\
** Agnostique de la représentation de terrain \\
** Légère, rapide, simple d'implémentation \\
- Objectif de garder un maximum de controle pour l'utilisateur \\ 
** Dans le processus de génération, mais aussi pour rectifier en amont des détails

\subsection{Sémantique}
- Travail orienté sur la génération sous-marine \\
- Collaboration avec un biologiste marin \\
- 

\subsection{Modélisation}
- Génération de certains éléments de paysages encore nouveaux (karsts à référer à Axel Paris, mais iles coralliennes nouveau) \\
- Réseaux karstiques représentés de manière hautement user-friendly \\
** Voir les karsts comme un graphe orienté acyclique => proche de la structure arbre \\
** Augmentation de la méthode pour génération fractale avec cycles (génération en plusieurs itérations) \\
- Iles coralliennes en utilisant une interprétation de la théorie de Darwin \\
** Basée sur des observations \\
** Basé sur des carnets de voyages

\subsection{Amplification}
- Augmenter le réalisme par l'ajout de détails \\
- Méthode d'érosion basé sur l'utilisation de particules \\
** Généralisation pour flexibilité \\
** Rapidité, parallelisation \\
- Vers une méthode d'érosion continue.