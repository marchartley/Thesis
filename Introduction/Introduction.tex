\chapter{Introduction}
\label{chap:introduction}
% \minitoc
- ...

% \section{Procedural generation}
% \label{sec:introduction_procedural-generation}
- Active domain for the last 50 years \\
- Computer graphics more and more proiminent \\
- Need to be always faster, realistic, automatic \\
- Terrain generation affected the same way \\
- Our focus on one branch: the landscape \\
- Useful domain for: \\
** Biology \\
** Geology \\
** Robotics \\
** But also entertaining industries \\
*** Video games \\
*** Cinema \\
- Because helps understanding rules and observing hypotheses \\
- Novelty is the underwater aspect \\
- Useful for: \\
** Marine biology \\
** Oceanology \\
** Underwater robotics \\
** And by extension, new fields of entertaining industries: \\
*** Video games \\
*** Cinema \\
- Big challenges: \\
** Balance between automation and user desires \\
** Isolation of variables \\
** Scaling \\
- 3 main ways to generate worlds: \\
** Simulations \\
** User interaction/modeling \\
** Procedural modeling \\
*** This one is tricky, maybe "automatic generation" \\
*** I mean: "Content creation using functions, independent of the user actions" (eg. using Perlin noise for a terrain, random noise for a rock modeling, specific texture generation, etc...) \\
- Main question: \\
** "How to efficiently guide the user in the creation of virtual content along the production process line to keep as much control on the final product?" \\
*** "Guiding" as opposed to "replace": I believe no machine can know better than a human what he wants \\
*** "Production process line": we are presenting algorithms that can be used in a pipeline \\
**** We want each component of the thesis to be as flexible as possible to the user workspace: \\
***** Any terrain representation \\
***** Any fluid solver \\
***** Any landscape type \\
***** Any hardware \\
***** Any objective (real-time rendering, realistic, animated, etc...) \\
*** "As much control": 
**** Each result should feel unique \\
**** Each result should be expected by the user \\
**** Each result should be explainable \\
**** Each result should be correctible(?) -> without having to restart everything \\
- ...

\section{Prototype creation}
\label{sec:introduction_prototype}
- C++23 and Qt5.12 \\
- OpenGL 4.6 and GLSL \\
- Marching Cubes on geometry shader \\
** Bad idea, but justify why \\
- Renderings: \\
** With the prototype: \\
*** Marching Cubes on geometry shader \\
*** Triplanar texture \\
*** Real-time results \\
*** Textures based on materials \\
** With Unreal Engine 5: \\
*** Static meshes \\
*** Added procedural vegetation with plugin [PLUGIN NAME] and ocean with [PLUGIN NAME] \\
** With Blender 4.1: \\
*** Static meshes \\
*** Easier script usage \\
- ...

\section{Contributions and outlines}
\label{sec:introduction_contribution-plan}
- Chronological order of terrain generation \\
- Proposes an abstract representation halfway between computing and terrain expertise \\
** Offering a generalization of the desired landscape type (underwater, but also terrestrial) \\
- Proposes new types of landscapes (karsts and coral islands) \\
** Maintaining the notion of sparseness (implicit volumes) \\
- Proposes a particle-based erosion simulation method \\
** Terrain representation agnostic \\
** Lightweight, fast, easy to implement \\
- Aims to keep maximum control for the user \\ 
** In the generation process, but also to correct details upstream

\subsection{Semantics}
- Work oriented towards underwater generation \\
- Collaboration with a marine biologist \\
- ...

\subsection{Modeling}
- Generation of some landscape elements still new (karsts referring to Axel Paris, but coral islands new) \\
- Karst networks represented in a highly user-friendly manner \\
** Viewing karsts as a directed acyclic graph => close to tree structure \\
** Enhanced method for fractal generation with cycles (generation in multiple iterations) \\
- Coral islands using an interpretation of Darwin's theory \\
** Based on observations \\
** Based on travel journals \\
- ... 

\subsection{Amplification}
- Increasing realism by adding details \\
- Particle-based erosion method \\
** Generalization for flexibility \\
** Speed, parallelization \\
- Towards a continuous erosion method. \\
- ... 
