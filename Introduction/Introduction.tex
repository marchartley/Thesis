\chapter{Introduction}
\label{chap:introduction}
\minitoc

- ...

\section{Génération procédurale}
\label{sec:introduction_procedural-generation}
- ...

\subsection{Définition}
- ...

\subsubsection{Définition officielle}
- ...

\subsubsection{Ma définition}
- ...

\subsection{Historique}
- ...

\subsection{Modèles représentés}
- ...

\subsubsection{Bruit}
- ...

\subsubsection{Automates cellulaires}
- ...

\subsubsection{Réseaux de neuronnes}
- ...

\subsubsection{Modélisation de phénomènes physiques}
- ...

\subsection{Interaction utilisateur}
- ...

\subsubsection{Balance réalisme-rapidité-control}
- Principale problématique de la génération de terrains \\
- Explication réalisme \\
** Paysages générés sont proches de ce qui se retrouve dans la réalité \\
** La génération intègre les processus naturels pour être réaliste \\
** Demande beaucoup de simulations physiques, de connaissances experts \\
** Utile dans les applications de simulation de catastrophes naturelles, par exemple \\
- Explication rapidité \\
** Génération la plus rapide possible, l'objectif étant la génération en temps-réel \\
** J. Gain classifie: temps-réel (< 30ms), interactif (< 3s), proche de interactif (< 5min) et long terme. \\
** Utile dans les jeux vidéos "infinity-scroll", par exemple\\
- Explication control \\
** Cherche à satisfaire les demandes de l'utilisateur \\
** Grosse problématique étant les demandes "impossibles" de l'utilisateur. \\
** Utile dans la plupart des applications de la gen. proc. : accélerer le travail d'artistes, par exemple \\
- ...

\subsubsection{Régénération}
- Actions manuelles \\
- Problématiques de la régénération \\
** Quoi regénérer ? \\
** Comment régénérer ? \\
** Problèmes des interactions utilisateur ? \\
- Fichier de stockage d'actions JSON (?)
- ...



\section{Representation de terrains}
\label{sec:introduction_terrain-representations}
- ...

\subsection{Terrains 2.5D}
- ...

\subsubsection{Cartes de hauteur}
- ...

\subsubsection{Fonctions de hauteur}
- ...

\subsection{Terrains 3D}
- Besoin de notions 3D \\
** Information geologique \\
** Données volumiques \\
- ...

\subsubsection{Problématiques principales}
- Mémoire \\
- Visualisation \\
- Modifications \\
- Conversion entre representations \\
** Pertes d'informations \\
*** Propagation d'erreur sur la géométrie (approximations sur les normales, résolution Z, surface, etc...) \\
*** Perte d'informations sous-terraines \\
- ...


\subsubsection{Types, définitions, avantages, inconvéniants}
- Grilles de voxels \\
- Piles de matériaux \\
- Maillage \\
- Surfaces implicites \\
- ...

\subsection{Autres modèles}
- Notion de sémantique \\
- ...

\subsection{Paysages sous-marins}
- Données 3D \\
** Paysages coralliens remplis de vide \\
** Beaucoup de cavités (caves, grottes, reseaux karstiques) \\
- Données interdisciplinaires \\
** Plutôt commun avec generation de terrains\\
** => validation géologique avec des experts \\
** Pour sous-marin, \\
*** Moins d'experts, \\
*** Plus d'incertitudes \\
*** Plutôt basé sur des observations \\
*** Peu de données (paysages coralliens < 0.1\% des océans), pour gros impact biologique (25\% biodiversité marine) \\
** Mélange de géologie, biologie, hydrologie et physique (des fluides, notamment) \\
- Besoin de multi-échelles \\
** Pas limité au sous-marin \\
** Intégrer de gros éléments (montagnes) avec des petits éléments (végétation). \\
** LOD \\
- ... 

\subsubsection{Simulations de fluides}
- Très important dans génération procédurale de terrain \\
- Permet de justifier la géophysique d'une simulation/génération. \\
- Solutions assez rapide en 2D (PIC, FLIP, Stable Fluids, SPH, etc...) \\
- Mais devient beaucoup plus lourd et memoire intensif en 3D
- ...



\section{Paysages coralliens (partie biologique)}
\label{sec:introduction_biology}
- Historique découverte des récifs de corail \\
- Iles, barrières, atolls \\
- Théories des atolls \\
- Utilité dans la biodiversité \\
- Menaces, protection, importance de les comprendre \\
- ...

\section{Géométrie et structures de données}
\label{sec:introduction_geometry-datastructures}
- Présentation des structures utilisées \\
- ... 

\subsection{Géométrie}
- ... 

\subsubsection{Points}
- Definis dans l'espace 3D sous forme $\left( x, y, z \right)^T$. \\
- Quand projeté en 2D, $z = 0$ implicite. \\
- Représentés dans le manuscrit comme: $\p \in \R^3$ \\
- ...

\subsubsection{Courbes}
- Fonction paramétrique $\curve: [0, 1] \to \R^3$. \\
- Sauf mention contraire, utilisation de Centripetal Catmull–Rom spline [CITE CATMULL 1974]: \\
** Let $\p_i$ denote a point. For a curve segment $\curve$ defined by points $\p_0$, $\p_1$, $\p_2$, $\p_3$ and knot sequence $t_0$, $t_1$, $t_2$, $t_3$, the centripetal Catmull-Rom spline can be produced by:
\begin{align}
    \curve(t) = \frac{t_2 - t}{t_2 - t_1} B_1 + \frac{t - t_1}{t_2 - t_1} B_2
\end{align}
where
\begin{align}
    B_1(t) &= \frac {t_2 - t}{t_2 - t_0} A_1(t) + \frac{t - t_0}{t_2 - t_0} A_2(t) \\
    B_2(t) &= \frac{t_3 - t}{t_3 - t_1} A_2(t) + \frac{t - t_1}{t_3 - t_1} A_3(t) \\
    A_1(t) &= \frac{t_1 - t}{t_1 - t_0} \p_0 + \frac{t - t_0}{t_1 - t_0} \p_1 \\
    A_2(t) &= \frac{t_2 - t}{t_2 - t_1} \p_1 + \frac{t - t_1}{t_2 - t_1} \p_2 \\
    A_3(t) &= \frac{t_3 - t}{t_3 - t_2} \p_2 + \frac{t - t_2}{t_3 - t_2} \p_3
\end{align}
and
\begin{align}
    t_{i + 1} = \sqrt{ \left(x_{i+1} - x_i \right) + \left(y_{i+1} - y_i \right) +  \left(z_{i+1} - z_i \right)}^\alpha + t_i
\end{align}
in which $\alpha$ ranges from 0 to 1 for knot parameterization, and $i = 0, 1, 2, 3$ with $t_0 = 0$. For centripetal Catmull-Rom spline, the value of $\alpha$ is 
0.5. When $\alpha = 0$, the resulting curve is the standard uniform Catmull-Rom spline; when $\alpha = 1$, the result is a chordal Catmull-Rom spline. \\
** We will keep $\alpha = 0.5$ for all the work in this manuscript, as it felt like a good compromise between smoothness and control on the curve. The value of $\alpha$ has not been studied deeply. \\
- Avantages: Centripetal Catmull-Rom spline has several desirable mathematical properties compared to the original and the other types of Catmull-Rom formulation. First, it will not form loop or self-intersection within a curve segment. Second, cusp will never occur within a curve segment. Third, it follows the control points more tightly. [COPIE COLLE WIKIPEDIA] \\
- De plus, nous n'utilisons pas de "handles" (invisible control points) like for Bézier curves. At the cost of a little user control, I feel the that the use is simplified. \\
- Le calcul des derivées premières et secondes (tangeante et normale) sont rapides à calculer. \\
- ...

\subsection{Structures de données}
- ...

\subsubsection{Grilles 3D}
- Dans ce manuscrit, toutes grilles 3D définies en float32 signé. \\
- Pas optimial, notamment pour representé voxels binaires (utilisation de 32x plus de mémoire et calcul que nécessaire), mais flexible... \\
- Grilles de voxels stockées sous forme de listes de sous-grilles 3D ("modifications locales") pour naviguer en undo-redo. Evaluation d'une cellule par somme des sous-grilles. \\
- ...


\section{Création du prototype}
\label{sec:introduction_prototype}
- C++23 et Qt5.12 \\
- OpenGL 4.6 et GLSL \\
- Marching Cubes sur geometry shader \\
** Mauvaise idée, mais justifier le pourquoi \\
- Rendus :\\
** Avec le prototype : \\
*** Marching Cubes sur geometry shader \\
*** Triplanar texture \\
*** Résultats temps-réels \\
*** Textures selon matériaux \\
** Avec Unreal Engine 5 : \\
*** Maillages statiques \\
*** Ajout de végétation procédurale avec plugin [NOM DU PLUGIN] et océan avec [NOM DU PLUGIN] \\
** Avec Blender 4.1 \\
*** Maillages statiques \\
*** Utilisation de scripts plus facile \\
- ...

\section{Contributions et plan}
\label{sec:introduction_contribution-plan}
- Ordre chronologique de la génération de terrains \\
- Propose une representation abstraite à mi-chemin entre informatique et expertise terrain \\
** Offrant une généralisation du type de paysage souhaité (sous-marin, mais aussi terrestre) \\
- Proposition de nouveaux types de paysages (karsts et iles coralliennes) \\
** Conservation de la notion de parcimonie (volumes implicites) \\
- Propose une méthode de simulation d'érosion par particules \\
** Agnostique de la représentation de terrain \\
** Légère, rapide, simple d'implémentation \\
- Objectif de garder un maximum de controle pour l'utilisateur \\ 
** Dans le processus de génération, mais aussi pour rectifier en amont des détails

\subsection{Sémantique}
- Travail orienté sur la génération sous-marine \\
- Collaboration avec un biologiste marin \\
- 

\subsection{Modélisation}
- Génération de certains éléments de paysages encore nouveaux (karsts à référer à Axel Paris, mais iles coralliennes nouveau) \\
- Réseaux karstiques représentés de manière hautement user-friendly \\
** Voir les karsts comme un graphe orienté acyclique => proche de la structure arbre \\
** Augmentation de la méthode pour génération fractale avec cycles (génération en plusieurs itérations) \\
- Iles coralliennes en utilisant une interprétation de la théorie de Darwin \\
** Basée sur des observations \\
** Basé sur des carnets de voyages

\subsection{Amplification}
- Augmenter le réalisme par l'ajout de détails \\
- Méthode d'érosion basé sur l'utilisation de particules \\
** Généralisation pour flexibilité \\
** Rapidité, parallelisation \\
- Vers une méthode d'érosion continue.