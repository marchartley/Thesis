\chapter{Introduction}
\label{chap:introduction}
% \minitoc

- ...

\section{Procedural generation}
\label{sec:introduction_procedural-generation}
- ...

\section{Prototype creation}
\label{sec:introduction_prototype}
- C++23 and Qt5.12 \\
- OpenGL 4.6 and GLSL \\
- Marching Cubes on geometry shader \\
** Bad idea, but justify why \\
- Renderings: \\
** With the prototype: \\
*** Marching Cubes on geometry shader \\
*** Triplanar texture \\
*** Real-time results \\
*** Textures based on materials \\
** With Unreal Engine 5: \\
*** Static meshes \\
*** Added procedural vegetation with plugin [PLUGIN NAME] and ocean with [PLUGIN NAME] \\
** With Blender 4.1: \\
*** Static meshes \\
*** Easier script usage \\
- ...

\section{Contributions and outlines}
\label{sec:introduction_contribution-plan}
- Chronological order of terrain generation \\
- Proposes an abstract representation halfway between computing and terrain expertise \\
** Offering a generalization of the desired landscape type (underwater, but also terrestrial) \\
- Proposes new types of landscapes (karsts and coral islands) \\
** Maintaining the notion of sparseness (implicit volumes) \\
- Proposes a particle-based erosion simulation method \\
** Terrain representation agnostic \\
** Lightweight, fast, easy to implement \\
- Aims to keep maximum control for the user \\ 
** In the generation process, but also to correct details upstream

\subsection{Semantics}
- Work oriented towards underwater generation \\
- Collaboration with a marine biologist \\
- 

\subsection{Modeling}
- Generation of some landscape elements still new (karsts referring to Axel Paris, but coral islands new) \\
- Karst networks represented in a highly user-friendly manner \\
** Viewing karsts as a directed acyclic graph => close to tree structure \\
** Enhanced method for fractal generation with cycles (generation in multiple iterations) \\
- Coral islands using an interpretation of Darwin's theory \\
** Based on observations \\
** Based on travel journals

\subsection{Amplification}
- Increasing realism by adding details \\
- Particle-based erosion method \\
** Generalization for flexibility \\
** Speed, parallelization \\
- Towards a continuous erosion method.
