\chapter{Introduction}
\label{chap:introduction}
\minitoc

- ...

\section{Procedural Generation}
\label{sec:introduction_procedural-generation}
- ...

\subsection{Definition}
- ...

\subsubsection{Official Definition}
- ...

\subsubsection{My Definition}
- ...

\subsection{History}
- ...

\subsection{Models Represented}
- ...

\subsubsection{Noise}
- ...

\subsubsection{Cellular Automata}
- ...

\subsubsection{Neural Networks}
- ...

\subsubsection{Physical Phenomena Modeling}
- ...

\subsection{User Interaction}
- ...

\subsubsection{Realism-Speed-Control Balance}
- Main issue in terrain generation \\
- Explanation of realism \\
** Generated landscapes are close to what is found in reality \\
** Generation incorporates natural processes to be realistic \\
** Requires extensive physical simulations, expert knowledge \\
** Useful in applications such as natural disaster simulations \\
- Explanation of speed \\
** Fastest possible generation, aiming for real-time generation \\
** J. Gain classifies: real-time (< 30ms), interactive (< 3s), near-interactive (< 5min), and long-term. \\
** Useful in "infinity-scroll" video games, for example \\
- Explanation of control \\
** Aims to meet user demands \\
** Major issue being "impossible" user demands \\
** Useful in most procedural generation applications: speeding up artists' work, for example \\
- ...

\subsubsection{Regeneration}
- Manual actions \\
- Issues with regeneration \\
** What to regenerate? \\
** How to regenerate? \\
** Problems with user interactions? \\
- Action storage file in JSON (?) \\
- ...

\section{Terrain Representation}
\label{sec:introduction_terrain-representations}
- ...

\subsection{2.5D Terrains}
- ...

\subsubsection{Height Maps}
- ...

\subsubsection{Height Functions}
- ...

\subsection{3D Terrains}
- Need for 3D concepts \\
** Geological information \\
** Volumetric data \\
- ...

\subsubsection{Main Issues}
- Memory \\
- Visualization \\
- Modifications \\
- Conversion between representations \\
** Information loss \\
*** Error propagation on geometry (approximations on normals, Z resolution, surface, etc.) \\
*** Loss of subsurface information \\
- ...

\subsubsection{Types, Definitions, Advantages, Disadvantages}
- Voxel grids \\
- Material stacks \\
- Meshes \\
- Implicit surfaces \\
- ...

\subsection{Other Models}
- Concept of semantics \\
- ...

\subsection{Underwater Landscapes}
- 3D Data \\
** Coral landscapes filled with voids \\
** Many cavities (caves, grottos, karst networks) \\
- Interdisciplinary Data \\
** Rather common with terrain generation \\
** => Geological validation with experts \\
** For underwater, \\
*** Fewer experts, \\
*** More uncertainties \\
*** Based more on observations \\
*** Few data (coral landscapes < 0.1% of oceans), for significant biological impact (25% marine biodiversity) \\
** Mix of geology, biology, hydrology, and physics (especially fluid dynamics) \\
- Need for multi-scale \\
** Not limited to underwater \\
** Integrate large elements (mountains) with small elements (vegetation) \\
** LOD \\
- ...

\subsubsection{Fluid Simulations}
- Very important in procedural terrain generation \\
- Allows justifying the geophysics of a simulation/generation \\
- Quite fast solutions in 2D (PIC, FLIP, Stable Fluids, SPH, etc.) \\
- But becomes much heavier and memory-intensive in 3D \\
- ...

\section{Coral Reefs (Biological Aspects)}
\label{sec:introduction_biology}
- Historical discovery of coral reefs \\
- Islands, barriers, atolls \\
- Atoll theories \\
- Importance in biodiversity \\
- Threats, protection, importance of understanding them \\
- ...

\section{Geometry and Data Structures}
\label{sec:introduction_geometry-datastructures}
- Presentation of used structures \\
- ...

\subsection{Geometry}
- ...

\subsubsection{Points}
- Defined in 3D space as $\left( x, y, z \right)^T$. \\
- When projected in 2D, $z = 0$ is implicit. \\
- Represented in the manuscript as: $\p \in \R^3$ \\
- ...

\subsubsection{Curves}
- Parametric function $\curve: [0, 1] \to \R^3$. \\
- Unless otherwise specified, use of Centripetal Catmull–Rom spline [CITE CATMULL 1974]: \\
** Let $\p_i$ denote a point. For a curve segment $\curve$ defined by points $\p_0$, $\p_1$, $\p_2$, $\p_3$ and knot sequence $t_0$, $t_1$, $t_2$, $t_3$, the centripetal Catmull-Rom spline can be produced by:
\begin{align}
    \curve(t) = \frac{t_2 - t}{t_2 - t_1} B_1 + \frac{t - t_1}{t_2 - t_1} B_2
\end{align}
where
\begin{align}
    B_1(t) &= \frac {t_2 - t}{t_2 - t_0} A_1(t) + \frac{t - t_0}{t_2 - t_0} A_2(t) \\
    B_2(t) &= \frac{t_3 - t}{t_3 - t_1} A_2(t) + \frac{t - t_1}{t_3 - t_1} A_3(t) \\
    A_1(t) &= \frac{t_1 - t}{t_1 - t_0} \p_0 + \frac{t - t_0}{t_1 - t_0} \p_1 \\
    A_2(t) &= \frac{t_2 - t}{t_2 - t_1} \p_1 + \frac{t - t_1}{t_2 - t_1} \p_2 \\
    A_3(t) &= \frac{t_3 - t}{t_3 - t_2} \p_2 + \frac{t - t_2}{t_3 - t_2} \p_3
\end{align}
and
\begin{align}
    t_{i + 1} = \sqrt{ \left(x_{i+1} - x_i \right)^2 + \left(y_{i+1} - y_i \right)^2 +  \left(z_{i+1} - z_i \right)^2 }^\alpha + t_i
\end{align}
where $\alpha$ ranges from 0 to 1 for knot parameterization, and $i = 0, 1, 2, 3$ with $t_0 = 0$. For centripetal Catmull-Rom spline, the value of $\alpha$ is 
0.5. When $\alpha = 0$, the resulting curve is the standard uniform Catmull-Rom spline; when $\alpha = 1$, the result is a chordal Catmull-Rom spline. \\
** We will keep $\alpha = 0.5$ for all the work in this manuscript, as it felt like a good compromise between smoothness and control on the curve. The value of $\alpha$ has not been studied deeply. \\
- Advantages: Centripetal Catmull-Rom spline has several desirable mathematical properties compared to the original and other types of Catmull-Rom formulations. First, it will not form loops or self-intersections within a curve segment. Second, cusps will never occur within a curve segment. Third, it follows the control points more tightly. [COPY PASTE WIKIPEDIA] \\
- Additionally, we do not use "handles" (invisible control points) like for Bézier curves. At the cost of a little user control, I feel the use is simplified. \\
- The calculation of first and second derivatives (tangent and normal) is quick. \\
- ...

\subsection{Data Structures}
- ...

\subsubsection{3D Grids}
- In this manuscript, all 3D grids are defined as signed float32. \\
- Not optimal, especially for representing binary voxels (uses 32x more memory and computation than necessary), but flexible... \\
- Voxel grids stored as lists of sub-grids ("local modifications") to navigate undo-redo. Cell evaluation by summing sub-grids. \\
- ...

\section{Prototype Creation}
\label{sec:introduction_prototype}
- C++23 and Qt5.12 \\
- OpenGL 4.6 and GLSL \\
- Marching Cubes on geometry shader \\
** Bad idea, but justify why \\
- Renderings: \\
** With the prototype: \\
*** Marching Cubes on geometry shader \\
*** Triplanar texture \\
*** Real-time results \\
*** Textures based on materials \\
** With Unreal Engine 5: \\
*** Static meshes \\
*** Added procedural vegetation with plugin [PLUGIN NAME] and ocean with [PLUGIN NAME] \\
** With Blender 4.1: \\
*** Static meshes \\
*** Easier script usage \\
- ...

\section{Contributions and Plan}
\label{sec:introduction_contribution-plan}
- Chronological order of terrain generation \\
- Proposes an abstract representation halfway between computing and terrain expertise \\
** Offering a generalization of the desired landscape type (underwater, but also terrestrial) \\
- Proposes new types of landscapes (karsts and coral islands) \\
** Maintaining the notion of sparseness (implicit volumes) \\
- Proposes a particle-based erosion simulation method \\
** Terrain representation agnostic \\
** Lightweight, fast, easy to implement \\
- Aims to keep maximum control for the user \\ 
** In the generation process, but also to correct details upstream

\subsection{Semantics}
- Work oriented towards underwater generation \\
- Collaboration with a marine biologist \\
- 

\subsection{Modeling}
- Generation of some landscape elements still new (karsts referring to Axel Paris, but coral islands new) \\
- Karst networks represented in a highly user-friendly manner \\
** Viewing karsts as a directed acyclic graph => close to tree structure \\
** Enhanced method for fractal generation with cycles (generation in multiple iterations) \\
- Coral islands using an interpretation of Darwin's theory \\
** Based on observations \\
** Based on travel journals

\subsection{Amplification}
- Increasing realism by adding details \\
- Particle-based erosion method \\
** Generalization for flexibility \\
** Speed, parallelization \\
- Towards a continuous erosion method.
