\resetgraphicspath
\appendtographicspath{{./Introduction/figures}{./Introduction/figures/Robotics}{./Introduction/figures/Pipeline}}

\chapter{Introduction}
\label{chap:introduction}
The planet is currently experiencing an ecological crisis marked by global warming, the accelerated loss of biodiversity, and the degradation of ecosystems. The oceans, which cover more than 70\% of the Earth's surface, play a crucial role in climate regulation and in sustaining life on Earth \cite{Pendleton2020,Visbeck2018}. Among them, coral reefs occupy a unique place: although they cover only a tiny portion of the ocean floor, they are home to around 25\% of known marine species. These living structures play an essential role in protecting coasts from erosion, support fisheries that are vital for millions of people, and underpin tourist activities that are crucial for many island economies \cite{Ferrario2014,Spalding2017,Plaisance2011}.

In this context, biodiversity monitoring has become a strategic priority. Tracking the dynamics of marine communities makes it possible to identify threats rapidly, assess the effectiveness of marine protected areas, and adapt conservation policies \cite{HoeghGuldberg2017,Yuan2024}. Yet underwater monitoring remains challenging: traditional methods based on divers or visual counts are costly, limited in time and space, and subject to observation bias \cite{Edgar2004}. In coral reefs, where conditions are complex (low light, irregular currents, fragile habitats), reliable and frequent data are indispensable for understanding and anticipating ecological change \cite{Maslin2021}.

It is against this context that BUBOT (Better Understanding Biodiversity changes thanks to new Observation Tools) was launched, an interdisciplinary project led by LIRMM (University of Montpellier, CNRS) in collaboration with MARBEC (marine ecology), Espace-Dev (geography and anthropology), CUFR Mayotte, Universidade Lúrio in Mozambique, and the University of California, Davis. BUBOT combines robotics, computer vision, ecology and the social sciences to improve the understanding of biodiversity change and to provide robust tools for the long-term monitoring of the reefs of Mayotte, the Sparse Islands, and the Mozambique Channel. Among its major achievements is the development of the REMI underwater robot, designed to automate ecological surveys.

Using a robot offers several decisive advantages. Unlike divers, REMI can explore great depths or hazardous areas, collect standardised data over long periods, and operate in conditions that would be impossible or risky for a human \cite{GonzalezGarcia2020}. Its stereo video sensors and artificial intelligence algorithms make it possible to identify species automatically and to analyse reef structure with increased precision and reproducibility \cite{Villon2020,Saleh2022}.

REMI is an underwater robotic platform developed at LIRMM. Equipped with autonomous navigation systems, acoustic and optical sensors, and on-board AI modules, it has been tested in a range of environments such as Mayotte and the Mediterranean. These campaigns demonstrated its potential but also highlighted the complexity of missions in the natural environment: rugged topography, unpredictable currents, variable turbidity, and technical limitations such as power autonomy or underwater communications.

Testing in robotics generally follows a multi-stage process \cite{GonzalezGarcia2020}. First, unit tests verify the proper functioning of each component (motors, cameras, sealing systems). Next come pool tests, easier to organise but far from real conditions: fresh water does not test resistance to salt, artificial waves often reduce to simple sinusoids, there are neither complex ocean currents nor realistic turbidity, and there is no fauna or flora. Navigation in natural currents cannot be properly evaluated in a swimming pool. Finally, field tests provide validation under real conditions but pose serious challenges: high financial and logistical costs, mobilisation of large teams, dependence on a particular site (finding a fault, an arch or a given type of fauna is not guaranteed), and risk of loss or damage to the robot. Recovering a failed robot may require risky human intervention and prematurely end a mission if spare parts are not available. Moreover, each field mission allows the testing of only a single combination of conditions, leaving many potential situations unexplored.

REMI's navigation must also contend with specific obstacles. A canyon should be followed down to the bottom, but a perpendicular fault must be avoided to prevent collision with the opposite wall. Arches and overhangs require avoiding unwarranted ascents to the surface in the presence of a rock ceiling. Caves present an even more complex case: although the walls may appear distant, the robot must recognise that it is trapped and turn back. The presence of mobile animals further complicates decision-making: a school of fish can appear as a moving wall that the robot must interpret without endangering itself or the fauna. Coral reefs themselves pose problems: their porosity can mislead range sensors; algae such as \emph{Posidonia}, which are difficult to detect, can damage the propellers; and fine sand can be lifted into suspension by the thrusters, completely obscuring optical sensors.

\AltTextImage{
    These challenges illustrate the difficulty of reproducing in real conditions all the situations that REMI may encounter. Hence the growing interest in realistic virtual environments instead of real environment in robot control loop (\cref{fig:intro-hil-Hereau2022}). The procedural generation of underwater environments would make it possible to simulate canyons, arches, fauna, turbulence, or turbidity conditions that are almost impossible to bring together in a single mission. It reduces costs, avoids the risk of equipment loss, and accelerates the development of decision-making algorithms.
}{HIL_Hereau.pdf}{A generic control loop structure for a mobile robot \cite{HereauThesis}}{fig:intro-hil-Hereau2022}


\begin{figure}
    \autofitgraphics[]{user-to-simu-pipeline.pdf}
    \caption{The global objective of this project is to bridge a pipeline from large-scale landscape generation to robot-scale environment generation. }
    \label{fig:intro-user-to-simu-pipeline}
\end{figure}



These constraints directly motivate the need for controllable, realistic virtual testing environments, hence this work focuses on the modelling of virtual terrains which are used to incorporate a digital underwater observation robot, providing a playground to verify and validate its behaviour under controlled and controllable environment and obstacles. The global pipeline of our work presented in \cref{fig:intro-user-to-simu-pipeline} keeps the user in the environment creation from start to end, allowing for refinement on the terrain generation outputs from the large-scale scope to smaller-scale scope, until details at robot-scale matches. [PHRASE A RETOURNER PLUSIEURS FOIS... ]

Over the past 50 years, terrain generation has emerged as an increasingly active domain within the field of computer graphics \cite{Fournier1982,Musgrave1989,Miller1986,Galin2019}. As the demand for realistic and automated processes has grown to support the creation of ever-larger and more detailed landscapes effortlessly, terrain generation techniques have evolved accordingly. As these goals are progressively achieved, a new trend shifts the focus towards greater user control. This work is focused on a specific branch of terrain generation: interactive creation and modelling of landscapes.

Landscape generation finds applications in diverse fields such as biology, geology, and robotics \cite{Tzachor2023,Chen2023,Gerigk2025,Rudin2022}. Additionally, it has become a central tool in the entertainment industry, particularly in video games and cinema, where creating realistic and dynamic environments is key for immersive experiences. The ability to generate landscapes that help in understanding natural rules and testing hypotheses makes this the ideal tool for both researchers and industries.

However, this exploration of the domain comes with significant challenges. In the case of underwater environments, finding a balance between automation and the user's creative desires is particularly complex: the visual and physical rules governing seascapes (such as coral growth patterns, sediment deposition, erosion by currents, or the interaction of light with water) differ markedly from those of terrestrial landscapes. Managing scaling is also more challenging in this context, as underwater landscapes often involve vast bathymetric structures while simultaneously requiring fine-scale detail (e.g., coral colonies, rocks, or vegetation) to remain scientifically relevant or visually convincing.

The central question guiding this research is: "How can we efficiently guide the user in the creation of virtual content along the production process line to maintain as much control as possible over the final product?". This concept of "guiding" rather than "replacing" the user is, from my point of view, fundamental, as no machine can truly know better than a human what the final product should look like. The goal is to present algorithms that can be used flexibly within a production pipeline, adapting to many terrain representations, fluid solvers, landscape types, users' hardware, or objectives, whether the final use is real-time rendering, realism, or animation.

Maintaining control over the creative process is essential. Each generated result should feel unique, meet the user's expectations, be explainable, and be easily correctable without requiring a complete regeneration. This approach ensures that the user remains at the centre of the creative process, with the tools and algorithms serving to enhance their objectives, rather than replacing the users themselves.

What sets this work apart is its emphasis on underwater landscapes or "seascapes", an area only slightly touched upon in Computer Graphics, but important in domains such as marine biology, oceanography, and underwater robotics. Furthermore, the extension of these techniques to new areas for the entertainment industry, such as video games and cinema, opens the door to novel visual experiences and storytelling techniques that take full advantage of the unique aesthetic and physical characteristics of underwater environments.




\section{Underwater landscape modelling challenges}
Underwater landscapes, or "seascapes", encompass the complex and dynamic terrains found beneath the ocean surface. Unlike landscapes above water level, which we will call "aerial landscapes", these environments are shaped not only by geological forces but also by biological activity, hydrodynamic processes, and chemical interactions. Coral reef islands represent a particularly intricate subset of these environments, where the terrain is actively constructed and modified by living organisms, primarily reef-building corals. The spatial and structural complexity of such systems, which span multiple scales from individual coral polyps to entire reef platforms, poses unique challenges for their representation and simulation.

In the context of procedural terrain generation, the modelling of underwater landscapes demands fundamentally different assumptions and techniques from those used for land-based terrains. The submerged setting alters the influence of gravity, light, and fluid dynamics, while also introducing significant observational constraints. Furthermore, the scarcity of high-resolution and volumetric data capturing the structural complexity and biological processes of coral reef systems poses a major challenge for data-driven or example-based modelling approaches. In this context, procedural generation offers an alternative by synthesising underwater landscapes from first principles, guided by interdisciplinary understanding of geological, biological, and hydrodynamic processes. This section reviews the key environmental, observational, and modelling challenges associated with seascapes, with a focus on those that motivate and constrain procedural terrain generation for coral reef island systems.


\subsection{Physical environment differences}
The underwater environment differs fundamentally from terrestrial settings in the physical forces that shape landscape formation, the conditions for biological growth, and the constraints imposed on sensing and modelling. These differences directly impact the assumptions underlying terrain generation and limit the applicability of methods developed for aerial landscapes.

First, although gravity remains a dominant force in geomorphological evolution, provoking sediment transport and slope stability, buoyancy significantly alters the behaviour of materials and organisms underwater. The net effect is a reduction in apparent weight and in gravity-driven surface processes, particularly at smaller scales, where water movement becomes a more dominant shaping force. Hydrodynamics plays a central role in shaping the morphology of coral reef systems by redistributing sediments, constraining reef growth zones, and sculpting reef edges and channels \cite{Lowe2015}.

Second, the optical properties of seawater impose strict environmental constraints. Light attenuation limits the available area for photosynthesis, which in turn defines vertical growth boundaries for reef-building corals \cite{Huston1985}. It also constrains the applicability of remote sensing and optical measurement techniques, limiting their effectiveness to shallow, clear-water environments.

Third, underwater terrains are shaped not only by physical processes but also by biological construction and erosion. Reef-building organisms contribute directly to topographic complexity through accretion, while bioeroder species like parrotfish, sponges, and urchins actively degrade and remodel the substrate \cite{Perry2013}.

Finally, these coupled physical-biological interactions result in terrain features that are not only highly complex, but also scale-dependent and dynamic. Overhangs, cavities, and porous structures restrict standard heightmap-based representations. Furthermore, many features emerge from feedback loops between hydrodynamics, sedimentation, and biological growth, further complicating attempts to generalise aerial terrain modelling techniques to underwater environments.


\subsection{Observation and 3D data acquisition challenges}
Despite growing interest in seafloor mapping, high-resolution, volumetric datasets of coral reef environments remain scarce. Most available data are limited to bathymetric or altimetric surface representations, typically acquired through sonar or satellite-derived methods, which offer only 2.5D elevation maps and lack information on vertical and sub-surface complexity. Features such as overhangs, internal cavities, and fine-scale biological structures are not captured, limiting the fidelity of directly observed data.

Moreover optical or active remote sensing techniques, such as underwater photogrammetry or bathymetric LIDAR, are constrained to shallow and optically clear waters, and environments that are safe for the diver or robot carrying the sensor. These methods are ineffective in turbid or deep environments, which characterise large portions of reef systems, particularly mesophotic and fore-reef zones. As a result, consistent 3D coverage across depth gradients is complex and time taking with current technology.

In situ surveys are limited by logistical, financial, and environmental factors. Diver-based surveys, effective at fine scales, are labour-intensive, spatially limited, and can disturb local fauna, thereby introducing bias in biological assessments. Remotely operated vehicles (ROVs) offer greater depth access but require tethering and constant supervision. Autonomous underwater vehicles (AUVs), in contrast, have shown promise in collecting low-disturbance imagery and acoustic data in a more scalable manner \cite{GonzalezRivero2016,Modasshir2018}, and represent a potential pathway for systematic 3D surveying in the future.


\subsection{Environmental dynamics and interdisciplinary integration}
Coral reef terrains develop in relation to underlying geological structures and sedimentary processes. Substrate stability and morphology are influenced by sediment supply which controls the distribution of fine sediments in lagoons versus coarser rubble zones on reef fronts \cite{Montaggioni2005}. Tectonic uplift or subsidence modulates relative sea level, determining exposure intervals that drive the formation of reef terraces and platform architecture \cite{Hopley2014}. Geomorphologic events, such as submarine landslides or turbidity currents on reef slopes, and wave-driven abrasion, further reshape bathymetry at meso- to macroscale.

Hydrodynamic regimes strongly influence sediment transport, mechanical stress, and ecological viability in reef environments. Persistent currents and wave action shaping erosion and deposition patterns on slopes and flats, shape reef crests, and drive overtopping and flushing in lagoonal areas \cite{Lowe2009}. Episodic storms or large swell events can produce rapid morphological change via scour, sediment redistribution, and coral breakage, punctuating longer-term growth trajectories. The evolving reef morphology modifies local flow fields which in turn influence subsequent sediment dynamics and biological processes.

In parallel to erosion processes driven by geological, climatic and hydrological events, the organic nature of coral reef continuously remodels and regenerates the substrate. However, abiotic factors such as light attenuation, temperature regimes, and water chemistry constrain both biological growth and sedimentary behaviour. Light attenuation due to turbidity and depth limits photosynthesis, setting depth boundaries for coral accretion \cite{Kirk1994}. Nutrient concentrations from terrestrial runoff can shift community composition towards macroalgae or favour coral health. Turbidity governs sediment deposition on reef surfaces.

Effectively integrating geological, ecological, hydrodynamic, and chemical knowledge requires the reconciling of diverse data types and spatiotemporal resolutions. Geological data often spans over millennial time scales and kilometre space scale, whereas ecological observations like colony growth rates are studied in years and counted in metres; hydrodynamic models and sensor time series each have their own spatiotemporal resolutions.

The complex, dynamic nature of reef environments motivates the use of process-informed procedural generation rather than example-based interpolation. Algorithms should embody process-based constraints such as depth-dependent accretion limits, susceptibility to storm-induced degradation, and form-flow feedbacks while allowing staged synthesis to yield realistic structural patterns. Simplified feedback loops, such as adjusting local growth likelihood based on simulated flow attenuation, can capture essential interactions without full physical simulation. Given data scarcity, procedural outputs should support exploration across parameter spaces or alternative scenarios, making assumptions transparent.

\subsection{Procedural modelling challenges}
\cite{ParisThesis} classified geological structures in four spatial scales: the \emph{megascale} describes landforms spanning over more than 100km such as continents or mountain ranges, the \emph{macroscale} between 1km and 100km (e.g., islands, rivers or cave networks), the \emph{mesoscale} ranging from 10m to 1km contains most complex structures (e.g., arches, ravines, cliffs, ...) and finally the \emph{microscale} for features between 10cm and 10m, (e.g., sand ripples, ventifacts, rocks). Generally, we consider as \emph{large-scale} features that can be seen from far away while \emph{small-scale} elements requires to be within arm's length to be observed.

Coral reef island landscapes exhibit structural complexity across scales ranging from millimetre- to kilometre-scale features, and procedural generation must support seamless transitions among these scales. At the microscale, individual coral morphologies, crevice networks, and sediment textures determine fine habitat details, whereas at the mesoscale, colony aggregations, reef crests, and lagoon basins define intermediate morphology. Macroscale features include the overall platform shape, island emergence, and large geomorphic structures such as reef rims or terrace outlines.

Unlike aerial landscapes' height fields, reef environments contain overhangs, cavities, and porous frameworks created by biological accretion and bioerosion. Procedural methods must therefore synthesise plausible volumetric geometry rather than relying solely on 2.5D surfaces. This could be achieved through combinations of multiple surface and volume representations, the introduction of rule-based solid modelling such as the use of L-systems for branching coral forms \cite{Prusinkiewicz1992, Abela2015}, and the stochastic placement of "voids" to mimic bioerosion. A procedural pipeline may first establish a coarse reef framework from depth-dependent envelope shapes and then instantiate volumetric modules representing coral colonies and cavities according to ecological rules or bioerosion simulations. The main challenge for such method is to integrate ecological rules and maintain computational tractability.

Physical processes such as hydrodynamic forces and sediment transport impose constraints on feasible reef morphologies. While full computational fluid dynamics (CFD) coupling for procedural generation is typically impractical at large scales, hybrid approaches could incorporate simplified physics-based rules. For example, a priori knowledge of relationships between reef shape and prevailing wave energy informs the geometry of grooves-and-spurs. Sediment deposition rules from in situ observation can guide the formation of lagoon floors or backreef accumulations. Such hybrid modelling ensures that generated forms remain within plausible physical bounds.

Process-informed generation relies on encoding ecological constraints and interactions. Coral growth algorithms may simulate accretion under light-dependent rules according to environmental parameters. Agent-based models can represent competition for space, where faster-growing but fragile forms may dominate in sheltered zones, whereas robust morphologies persist in high-energy areas \cite{Anthony2000}. Simulated bioerosion events introduce heterogeneity and temporal variability. Although full temporal simulation may be computationally expensive, stochastic modelling or staged synthesis (e.g., iterative growth phases punctuated by disturbance events) may yield realistic geometry.

An effective procedural modelling framework for coral reef islands begins by defining a macroscopic envelope (i.e., reef and island outline) based on generic geomorphic templates and parameters such as island size and shape, sea level, or tectonic parameters (\cref{chap:coral-island}). Subsequent stages instantiate mesoscale reef structures by populating the base environment with ecologically informed semantic entities (\cref{chap:semantic-representation}). Microscale details are then added by applying procedural textures or controlled erosion simulations (\cref{chap:erosion}).













% Over the past 50 years, terrain generation has emerged as an increasingly active domain within the field of computer graphics \cite{Fournier1982,Musgrave1989,Miller1986,Galin2019}. As the demand for realistic and automated processes has grown to support the creation of ever-larger and more detailed landscapes effortlessly, terrain generation techniques have evolved accordingly. As these goals are progressively achieved, a new trend shifts the focus towards greater user control. This work is focused on a specific branch of terrain generation: interactive creation and modelling of landscapes.

% Landscape generation finds applications in diverse fields such as biology, geology, and robotics \cite{Tzachor2023,Chen2023,Gerigk2025,Rudin2022}. Additionally, it has become a central tool in the entertainment industry, particularly in video games and cinema, where creating realistic and dynamic environments is key for immersive experiences. The ability to generate landscapes that help in understanding natural rules and testing hypotheses makes this the ideal tool for both researchers and industries.

% However, this exploration of the domain comes with significant challenges. In the case of underwater environments, finding a balance between automation and the user's creative desires is particularly complex: the visual and physical rules governing seascapes (such as coral growth patterns, sediment deposition, erosion by currents, or the interaction of light with water) differ markedly from those of terrestrial landscapes. Managing scaling is also more challenging in this context, as underwater landscapes often involve vast bathymetric structures while simultaneously requiring fine-scale detail (e.g., coral colonies, rocks, or vegetation) to remain scientifically relevant or visually convincing.

% The central question guiding this research is: "How can we efficiently guide the user in the creation of virtual content along the production process line to maintain as much control as possible over the final product?". This concept of "guiding" rather than "replacing" the user is, from my point of view, fundamental, as no machine can truly know better than a human what the final product should look like. The goal is to present algorithms that can be used flexibly within a production pipeline, adapting to many terrain representations, fluid solvers, landscape types, users' hardware, or objectives, whether the final use is real-time rendering, realism, or animation.

% Maintaining control over the creative process is essential. Each generated result should feel unique, meet the user's expectations, be explainable, and be easily correctable without requiring a complete regeneration. This approach ensures that the user remains at the centre of the creative process, with the tools and algorithms serving to enhance their objectives, rather than replacing the users themselves.

% What sets this work apart is its emphasis on underwater landscapes or "seascapes", an area only slightly touched upon in Computer Graphics, but important in domains such as marine biology, oceanography, and underwater robotics. Furthermore, the extension of these techniques to new areas for the entertainment industry, such as video games and cinema, opens the door to novel visual experiences and storytelling techniques that take full advantage of the unique aesthetic and physical characteristics of underwater environments.

% This thesis is developed alongside the creation of an autonomous underwater marine biology observation robot display in \cref{fig:intro-REMI} \cite{Maslin2021}. In this context, virtual environments are essential since field experiments are costly and prone to failure. Digital twins of the robot and its environment allow robotics simulations of material malfunctions and rare or unpredictable scenarios.

% \section{Underwater environment generation for robotics}

% \AltTextImage{
%     To accurately model robotic mission, robotics simulators (such as Gazebo and Isaac Sim) need the geometry of the terrain, fauna, and flora the robot might encounter. They also need material parameters to realistically simulate sensor behavior and external forces acting on the robot.
% % To accurately model a robotic mission, robotics simulators (such as Gazebo and Isaac Sim) needs the geometry of the terrain, fauna, and flora that the device and its sensors might encounter. It also requires intrinsic material parameters to simulate sensor behaviour and external forces acting on the robot.

% Scene geometry must be accurately scaled, from centimetre-level robot precision to mission areas spanningthousands of square metres. Underwater environments also require considering depth. Simulated materials should realistically respond to light and sound, taking into account porosity, granularity and reflectance. These factors significantly affect sensor's accuracy.
% % Scene geometry must be scaled appropriately to match both the robot, which operates at centimetre precision, and the mission area, which may cover thousands of square metres. Underwater environments also demand consideration of the third dimension (depth). The materials used in the simulation need to respond correctly to light and sound, accounting for porosity, granularity, and reflectance, which can distort sensor signals. 
% Water currents strongly affect navigation. Realistic and interactive simulation thus requires advanced currents modelling.

% }{REMI-robot.png}{Our underwater semi-AUV robot REMI executing a transect in Mayotte's lagoon.}{fig:intro-REMI}

% Designing virtual test fields involves creating surfaces and obstacles, and the ability to replay simulations with identical or slightly modified setups. Adding randomness while preserving the designer's intent presents another challenge. % in maintaining control while introducing variability.

% \section{Underwater environment generation for biology}

% Virtual environments play a crucial role in biological research by enabling the observation and study of interactions within ecosystems. Biology relies heavily on phenomenology, where understanding emerges from observing how entities behave and interact under specific conditions.

% % \AltTextImage{
% Simulating these interactions allows scientists to test and refine their theories. Models can reveal unexpected behaviours, highlight gaps in understanding, and uncover hidden correlations. This iterative process between theory, simulation, and fieldwork is central to advancing biological knowledge.

% To be useful, simulations must closely follow biological rules while also incorporating controlled randomness. This balance allows for realistic behaviour while making room for discovering novel phenomena. Human-Computer Interaction should help bridge the gap between biological expertise and computational tools, enabling domain experts to contribute effectively without introducing bias.

% % }{Eccormier-fauna-flora.jpg}{}{fig:intro-challenges-bio}

% These environments must also operate across multiple spatial scales. Simulations should cover large areas to reflect full ecosystems while still capturing fine-grained interactions between closely located entities.

% \section{Underwater environment generation in Computer Graphics}

% Overall, this work lies in the domain of Computer Graphics (CG), specifically in the fields of modelling and interaction.

% Designing effective interactive tools requires careful consideration of both user needs and technical constraints. This involves exploring how users interact with terrain elements, defining intuitive controls, and ensuring the interface supports a creative yet efficient workflow. Behind these tools lie critical implementation decisions that directly affect usability and performance. Choosing the most adapted data structures, for instance, determines how well the system handles large or complex environments. The mathematical and physical models used must strike a balance between realism and computational cost, especially in scenarios where interaction must remain fluid and responsive. Algorithms must also be parallelisable to take advantage of modern hardware, enabling real-time updates and feedback. Lastly, simplifying certain aspects of natural phenomena is often necessary to reduce computational overhead while maintaining visual and behavioural plausibility. These layered decisions, from interaction design to algorithmic efficiency, ensure that the system remains both user-centred and technically robust.




\section{Contributions and outlines}
This thesis explores the procedural generation of underwater environments, with a particular focus on coral reef islands. Our contributions are organised into three complementary parts, covering the creation, structuring, and physical evolution of underwater terrains.

% \subsubsubsection{Background}
In \cref{chap:background} we will first introduce the fundamentals of procedural terrain generation. A description of the different terrain representations is provided, as well as an overview of coral biology and coral reef formation.

% \subsubsubsection{Automatic generation of coral reef islands}
\AltTextImageR{
    In \cref{chap:coral-island}, we propose a user-guided method for the procedural creation of coral reef islands as we see in \cref{fig:intro-example-cgan}. Users sketch the island shape and elevation from two projections and define a wind field that simulates long-term environmental deformation. The system models coral growth and outputs heightmaps that are further used to train a conditional generative adversarial network (cGAN) for diversified generation. This method enables fast and controllable generation of varied reef island configurations.
}{Introduction/figures/cGAN-example.png}{Example of island generated in \cref{chap:coral-island}. }{fig:intro-example-cgan}

% \subsubsubsection{Semantic terrain representation}
\AltTextImageR{
    The semantic representation we present in \cref{chap:semantic-representation} aims to design the features of a terrain with an abstraction of the 3D aspect of the surface. We introduce \glosses{EnvObj} and their simplified representation from the real world, which we used to obtain a symbolic representation of the terrain features, biotic and abiotic, that are present in the scene (such as the canyon, rocks and corals in \cref{fig:intro-example-env-obj}). Using symbolism allows us to focus on the interactions between the different elements to generate a plausible ecosystem without the high computational needs of running an accurate multiphysics simulation. Moreover, the simplified representation used allows the user to manipulate the layout of the final terrain without having to choose a specific terrain representation.
}{Chapter 2/figures/Canyon5.png}{Example of ecosystem generated in \cref{chap:semantic-representation}. }{fig:intro-example-env-obj}

% \subsubsubsection{Erosion simulation}
\AltTextImageR{
    To increase the realism and the visual impact of the generated synthetic landscapes, the use of terrain enhancement techniques is often required. In \cref{chap:erosion}, we tackled the specific challenge of running erosion simulations, a type of enhancement that mimics the effects of water, wind, and erosive forces on a virtual terrain to improve the believability of the final landscape. A particle-based erosion method is proposed, designed to be generalisable for flexibility on multiple scales, and its implementation is oriented towards speed and parallelisation. The main flexibility of our method is to be applicable to multiple terrain representations, agnostic to the fluid solver used, and generalised for both landscapes and seascapes. \cref{fig:intro-example-erosion} illustrates our method simulating coastal erosion, operating at the air-water interface on a voxel grid.
}{Chapter 3/results/karst.pdf}{Example of sea caves generated in \cref{chap:erosion}. }{fig:intro-example-erosion}