\resetgraphicspath
\appendtographicspath{{./Introduction/figures}}

\chapter*{Résumé long}
% Au cours des cinquante dernières années, la génération de terrains est devenue un domaine de plus en plus actif au sein de l'informatique graphique. À mesure que la demande augmente pour des processus plus réalistes, plus rapides et automatisés, les techniques de génération de terrains ont évolué pour répondre à ces exigences. Alors que ces objectifs sont progressivement atteints, une nouvelle tendance émerge, mettant l'accent sur le contrôle utilisateur. Ce travail se concentre sur une branche spécifique de la génération de terrains~: la création interactive et la modélisation de paysages.

% L'utilité de la génération de paysages dépasse les seuls objectifs scientifiques, trouvant des applications dans des domaines variés tels que la biologie, la géologie et la robotique. De plus, elle est devenue un outil central dans l'industrie du divertissement, en particulier dans les jeux vidéo et le cinéma, où la création d'environnements réalistes et dynamiques est devenue essentielle pour proposer des expériences immersives. La capacité à générer des paysages aidant à comprendre les lois naturelles et à tester des hypothèses fait de cette approche un outil idéal tant pour les chercheurs que pour les industries.


% Ce travail se concentre sur les paysages sous-marins, un domaine encore peu exploré mais important dans des secteurs comme la biologie marine, l'océanographie et la robotique sous-marine. Par ailleurs, l'extension de ces techniques à de nouveaux domaines de l'industrie du divertissement, tels que les jeux vidéo et le cinéma, présente de nouvelles opportunités d'innovation.

% Cependant, cette exploration du domaine s'accompagne de difficultés significatives. Trouver un équilibre entre l'automatisation et les désirs créatifs de l'utilisateur est essentiel, tout comme la capacité à gérer efficacement la mise à l'échelle.
% % Il existe trois approches principales pour générer des mondes virtuels~: les simulations, l'interaction/modélisation par l'utilisateur, et la modélisation procédurale. La modélisation procédurale, en particulier, se distingue par son potentiel de "génération automatique", où le contenu est créé à l'aide de fonctions indépendamment des actions de l'utilisateur. Par exemple, on peut utiliser le bruit de Perlin pour la génération de terrains, du bruit aléatoire pour la modélisation de roches, ou des algorithmes spécifiques pour la génération de textures.
% La question centrale guidant cette recherche est~: "Comment pouvons-nous guider efficacement l'utilisateur dans la création de contenu virtuel au long de la chaîne de production afin de conserver autant que possible le contrôle sur le produit final ?" Ce concept de "guider" plutôt que de "remplacer" l'utilisateur est, de mon point de vue, fondamental, car aucune machine ne peut réellement savoir mieux qu'un humain ce à quoi doit ressembler le produit final. L'objectif est de proposer des algorithmes utilisables de manière flexible au sein d'une chaîne de production, s'adaptant à de nombreuses représentations de terrain, solveurs de fluides, types de paysages, acceleration matérielle de l'utilisateur, ou objectifs, que l'usage final soit le rendu en temps réel, le réalisme ou l'animation.

% Maintenir le contrôle sur le processus créatif est essentiel. Chaque résultat généré doit paraître unique, répondre aux attentes de l'utilisateur, être explicable et facilement corrigible sans nécessiter une régénération complète. Cette approche garantit que l'utilisateur reste au centre du processus créatif, les outils et algorithmes servant à renforcer ses objectifs, plutôt qu'à le remplacer lui-même.

% \section*{Défis dans la génération d'environnements pour la robotique sous-marine}

% Cette thèse se développe parallèlement à la création d'un robot d'observation pour la biologie marine sous-marine. Dans ce contexte, les environnements virtuels sont indispensables, car les expérimentations sur le terrain sont coûteuses et sujettes à échec. Les jumeaux numériques du robot et de son environnement permettent de simuler des défaillances matérielles ainsi que des scénarios rares ou imprévisibles.

% Pour modéliser avec précision une mission robotique, le simulateur a besoin de la géométrie du terrain, de la faune et de la flore que l'appareil et ses capteurs pourraient rencontrer. Il requiert également les paramètres matériels intrinsèques pour simuler le comportement des capteurs et les forces externes agissant sur le robot.

% La géométrie de la scène doit être mise à l'échelle de manière appropriée pour correspondre à la fois au robot, qui opère avec une précision centimétrique, et à la zone de mission, qui peut couvrir des milliers de mètres carrés. Les environnements sous-marins exigent aussi de prendre en compte la troisième dimension, la verticalité. Les matériaux simulés doivent réagir correctement aux ondes lumineuses et sonores, en tenant compte de la porosité, de la granularité et de la réflectance, qui peuvent altérer les signaux des capteurs. Les courants marins affectent fortement la navigation et nécessitent une modélisation avancée pour une simulation réaliste et interactive.

% La conception de bancs d'essai virtuels implique la création de surfaces et d'obstacles, ainsi que la possibilité de rejouer des simulations avec des configurations identiques ou légèrement modifiées. Introduire de l'aléa tout en préservant l'intention du concepteur représente un défi supplémentaire pour maintenir le contrôle tout en apportant de la variabilité.

% \section*{Défis dans la génération d'environnements pour la biologie}

% Les environnements virtuels jouent un rôle crucial en recherche biologique en permettant l'observation et l'étude des interactions au sein des écosystèmes. La biologie repose largement sur la phénoménologie, où la compréhension émerge de l'observation du comportement et des interactions des entités dans des conditions spécifiques.

% Simuler ces interactions permet aux scientifiques de tester et d'affiner leurs théories. Les modèles peuvent révéler des comportements inattendus, mettre en évidence des lacunes dans la compréhension et dévoiler des corrélations cachées. Ce processus itératif entre théorie, simulation et observation est central pour faire progresser les connaissances en biologie.

% Pour être utiles, les simulations doivent suivre de près les règles biologiques tout en intégrant un aléa contrôlé. Cet équilibre permet un comportement réaliste tout en laissant place à la découverte de phénomènes nouveaux. L'interaction homme-machine doit contribuer à combler le fossé entre l'expertise biologique et les outils informatiques, permettant aux experts du domaine de contribuer efficacement sans introduire de biais.

% Ces environnements doivent également fonctionner à plusieurs échelles spatiales. Les simulations doivent couvrir de vastes zones pour refléter des écosystèmes complets tout en capturant des interactions fines entre des entités proches.

% \section*{Défis de la génération d'environnements en informatique graphique}

% Globalement, ce travail se situe dans le domaine de l'informatique graphique, spécifiquement dans les branches de la modélisation et de l'interaction.

% La conception d'outils interactifs efficaces nécessite une prise en compte soignée des besoins des utilisateurs et des contraintes techniques. Cela implique d'explorer comment les utilisateurs interagissent avec les éléments de terrain, de définir des contrôles intuitifs et de garantir que l'interface soutient une capacité de conception à la fois créatif et efficace. Derrière ces outils se cachent des décisions d'implémentation critiques qui affectent directement l'utilisabilité et les performances. Le choix des structures de données, par exemple, détermine la manière dont le système gère des environnements vastes ou complexes. Les modèles mathématiques et physiques employés doivent trouver un équilibre entre réalisme et coût de calcul, surtout dans des scénarios où l'interaction doit rester fluide et réactive. Les algorithmes doivent également pouvoir être parallélisés pour tirer parti des accélérations matérielles modernes, permettant des mises à jour et des retours en temps réel. Enfin, simplifier certains aspects des phénomènes naturels est souvent nécessaire pour réduire la charge de calcul tout en conservant une plausibilité visuelle. Ces décisions à plusieurs niveaux, de la conception de l'interaction à l'efficacité algorithmique, assurent que le système reste à la fois centré sur l'utilisateur et techniquement robuste.

% \section*{Contributions et plan}
% Cette thèse explore la génération procédurale d'environnements sous-marins, avec un focus particulier sur les îles dôtés de récifs coralliens. Nos contributions sont organisées en trois parties complémentaires, couvrant la création, la structuration et l'évolution physique des terrains sous-marins.

% % \subsubsubsection{Contexte}
% Avant tout, dans le \cref{chap:background}, nous introduirons les fondamentaux de la génération procédurale de terrains. Une description des différentes représentations de terrains est fournie, ainsi qu'un aperçu de la biologie corallienne et de la formation des récifs coralliens.

% % \subsubsubsection{Génération automatique d'îles de récifs coralliens}
% Dans \cref{chap:coral-island}, nous proposons une méthode guidée par l'utilisateur pour la création procédurale d'îles de récifs coralliens. Les utilisateurs esquissent la forme de l'île et l'élévation à partir de deux projections et définissent un champ vectoriel de vent, simulant la déformation du terrain à long terme. Le système modélise la croissance du récif corallien et génère des cartes de hauteur utilisées ensuite pour entraîner un \textit{conditional Generative Adversarial Network} (cGAN) pour une génération diversifiée. Cette méthode permet une génération rapide et contrôlable d'îles.

% % \subsubsubsection{Représentation sémantique du terrain}
% La représentation sémantique que nous présentons dans \cref{chap:semantic-representation} vise à concevoir les caractéristiques d'un terrain avec une abstraction de l'aspect 3D de la surface. Nous introduirons les objets environementaux et leur représentation simplifiée à partir du monde réel, que nous avons utilisée pour obtenir une représentation symbolique des caractéristiques du terrain, biotiques et abiotiques, présentes dans la scène. L'utilisation du symbolisme permet de se concentrer sur les interactions entre les différents éléments d'un environnement sans les coûts de calcul élevés d'une simulation physique et chimique précise. De plus, la représentation simplifiée utilisée permet à l'utilisateur de manipuler la forme du terrain final sans avoir à choisir une représentation de terrain spécifique.

% % \subsubsubsection{Simulation d'érosion}
% Pour augmenter le réalisme et l'impact visuel des paysages synthétiques générés, l'utilisation de techniques d'amélioration de terrain est souvent requise. Dans \cref{chap:erosion}, nous avons abordé le défi spécifique d'exécuter des simulations d'érosion, un type d'amélioration qui imite les effets de l'eau, du vent et des forces érosives sur un terrain virtuel pour augmenter la crédibilité du paysage final. Une méthode d'érosion basée sur des particules est proposée, conçue pour être généralisable afin de proposer une implémentation privilégiant rapidité et parallelisme et offrant de la flexibilité à plusieurs niveaux~: applicable à plusieurs représentations de terrain, mais aussi agnostique par rapport au solveur de fluide utilisé, et  généralisable tant aux paysages terrestres qu'aux paysages marins.
Au cours des 50 dernières années, la génération de terrains est devenue un domaine de plus en plus actif dans le champ de l'informatique graphique \cite{Fournier1982,Musgrave1989,Miller1986,Galin2019}. Avec la demande croissante de procédés réalistes et automatisés permettant de créer des paysages toujours plus vastes et détaillés sans effort humain, les techniques de génération de terrains ont évolué en conséquence. Alors que ces objectifs sont progressivement atteints, une nouvelle tendance déplace l'attention vers un meilleur contrôle utilisateur. Notre travail se concentre sur une branche spécifique de la génération de terrains~: la création et la modélisation interactive de paysages.

La génération de paysages trouve des applications dans des domaines variés tels que la biologie, la géologie et la robotique \cite{Tzachor2023,Chen2023,Gerigk2025,Rudin2022}. Elle est également devenue un outil central dans l'industrie du divertissement, notamment dans les jeux vidéo et le cinéma, où la création d'environnements réalistes et dynamiques est essentielle pour des expériences immersives. La capacité à générer des paysages permettant de mieux comprendre les règles naturelles et de tester des hypothèses en fait un outil idéal pour les chercheurs comme pour les industriels.

Cependant, l'exploration de ce domaine s'accompagne de défis importants. Dans le cas des environnements sous-marins, trouver un équilibre entre automatisation et créativité de l'utilisateur est particulièrement complexe~: les règles visuelles et physiques qui régissent les paysages marins (comme la croissance des coraux, le dépôt de sédiments, l'érosion par les courants ou encore l'interaction de la lumière avec l'eau) diffèrent fortement de celles des paysages terrestres. La gestion des échelles est également essentielle dans ce contexte, car les paysages sous-marines impliquent souvent de vastes structures bathymétriques tout en exigeant simultanément des détails fins (par ex. la présence de colonies de coraux, rochers, végétation) pour rester scientifiquement pertinentes ou visuellement convaincantes.

La question centrale guidant cette recherche est~: "Comment peut-on guider efficacement l'utilisateur dans la création de contenu virtuel tout au long de la chaîne de production afin de maintenir le plus de contrôle possible sur le produit final ?". Ce concept de "guider" plutôt que de "remplacer" l'utilisateur est, de mon point de vue, fondamental, car aucune machine ne peut réellement savoir mieux qu'un humain à quoi le produit final devrait ressembler. L'objectif est de présenter des algorithmes pouvant être utilisés de manière flexible dans une chaîne de production, capables de s'adapter à de nombreuses représentations de terrain, solveurs de fluides, types de paysages, configurations matérielles ou objectifs des utilisateurs, que l'usage final soit le rendu temps réel, le réalisme ou l'animation.

Maintenir le contrôle sur le processus créatif est essentiel. Chaque résultat généré doit sembler unique, répondre aux attentes de l'utilisateur, être explicable et facilement corrigeable sans nécessiter une régénération complète. Cette approche garantit que l'utilisateur reste au centre du processus créatif, les outils et algorithmes servant à renforcer ses objectifs, et non à le remplacer.

Ce qui distingue ce travail est son accent sur les paysages sous-marins, un domaine peu exploré en informatique graphique, mais important en biologie marine, océanographie et robotique sous-marine. De plus, l'extension de ces techniques vers de nouveaux horizons dans l'industrie du divertissement ouvre la voie à de nouvelles expériences visuelles et narrations qui exploitent pleinement les caractéristiques esthétiques et physiques uniques des environnements sous-marins.

Cette thèse est développée en parallèle de la création d'un robot autonome d'observation de biologie marine sous-marine \cite{Maslin2021}. Dans ce contexte, les environnements virtuels sont essentiels car les expériences de terrain sont coûteuses et sujettes à l'échec. Les jumeaux numériques du robot et de son environnement permettent de simuler des défaillances matérielles et des scénarios rares ou imprévisibles.

\section*{Génération d'environnements sous-marins pour la robotique}

% \AltTextImage{
    Pour modéliser avec précision une mission robotique, les simulateurs de robotique (tels que Gazebo et Isaac Sim) ont besoin de la géométrie du terrain, de la faune et de la flore que le robot pourrait rencontrer. Ils nécessitent également les propriétés des matériaux afin de simuler de manière réaliste le comportement des capteurs et les forces externes agissant sur le robot.

La géométrie de la scène doit être correctement mise à l'échelle, depuis la précision au centimètre pour le robot jusqu'aux zones de mission couvrant des milliers de mètres carrés. Les environnements sous-marins exigent aussi de prendre en compte la profondeur. Les matériaux simulés doivent répondre de manière réaliste à la lumière et à l'acoustique en tenant compte de la porosité, de la granularité et de la réflectance, ces facteurs affectant considérablement la précision des capteurs.
Les courants marins influencent fortement la navigation. Une simulation réaliste et interactive nécessite donc une modélisation avancée des courants.
% }{REMI-robot.png}{Notre robot semi-AUV REMI exécutant un transect dans le lagon de Mayotte.}{fig:intro-REMI}

La conception de terrains virtuels de test implique la création de surfaces et d'obstacles, ainsi que la possibilité de rejouer des simulations avec des configurations identiques ou légèrement modifiées. Ajouter de l'aléatoire tout en préservant l'intention du concepteur représente veritable défi.

\section*{Génération d'environnements sous-marins pour la biologie}

Les environnements virtuels jouent un rôle crucial dans la recherche biologique en permettant l'observation et l'étude des interactions au sein des écosystèmes. La biologie repose fortement sur la phénoménologie, où la compréhension émerge de l'observation de la manière dont les entités se comportent et interagissent dans des conditions spécifiques.

La simulation de ces interactions permet aux scientifiques de tester et d'affiner leurs théories. Les modèles peuvent révéler des comportements inattendus, mettre en évidence des lacunes dans la compréhension et dévoiler des corrélations cachées. Ce processus itératif entre théorie, simulation et travail de terrain est central dans l'avancée des connaissances biologiques.

Pour être utiles, les simulations doivent suivre de près les règles biologiques tout en intégrant un aléatoire contrôlé. Cet équilibre permet un comportement réaliste tout en laissant la place à la découverte de phénomènes nouveaux. L'Interaction Homme-Machine doit aider à combler le fossé entre expertise biologique et outils computationnels, afin de permettre aux experts du domaine de contribuer efficacement sans introduire de biais.

Ces environnements doivent également fonctionner sur plusieurs échelles spatiales. Les simulations doivent couvrir de vastes zones pour refléter les écosystèmes dans leur ensemble tout en capturant les interactions fines entre entités proches.

\section*{Génération d'environnements sous-marins en informatique graphique}

Globalement, ce travail se situe dans le domaine de l'informatique graphique, et plus précisément dans les champs de la modélisation et de l'interaction.

La conception d'outils interactifs efficaces nécessite une attention particulière aux besoins des utilisateurs et aux contraintes techniques. Cela implique d'explorer la manière dont les utilisateurs interagissent avec les éléments de terrain, de définir des contrôles intuitifs et de s'assurer que l'interface soutient un flux de travail à la fois créatif et efficace. Derrière ces outils se cachent des choix d'implémentation critiques qui affectent directement l'utilisabilité et la performance. Par exemple, le choix des structures de données détermine la capacité du système à gérer des environnements vastes ou complexes. Les modèles mathématiques et physiques utilisés doivent trouver un équilibre entre réalisme et coût computationnel, notamment dans des scénarios où l'interaction doit rester fluide et réactive. Les algorithmes doivent également être parallélisables afin de tirer parti du matériel moderne, permettant des mises à jour et un retour visuel en temps réel. Enfin, simplifier certains aspects des phénomènes naturels est souvent nécessaire pour réduire la charge computationnelle tout en maintenant une plausibilité visuelle et comportementale. Ces décisions, réparties sur plusieurs niveaux allant de la conception de l'interaction à l'efficacité algorithmique, garantissent que le système reste à la fois centré sur l'utilisateur et techniquement robuste.

\section*{Contributions et plan}

Cette thèse explore la génération procédurale d'environnements sous-marins, avec une attention particulière portée aux îles coralliennes. Nos contributions sont organisées en trois parties complémentaires, couvrant la création, la structuration et l'évolution physique des terrains sous-marins.

Dans le chapitre~\ref{chap:background}, nous introduisons d'abord les bases de la génération procédurale de terrains. Une description des différentes représentations de terrain est fournie, ainsi qu'un aperçu de la biologie des coraux et de la formation des récifs coralliens.

% \AltTextImageR{
    Dans le chapitre~\ref{chap:coral-island}, nous proposons une méthode guidée par l'utilisateur pour la création procédurale d'îles coralliennes. L'utilisateur esquisse la forme et l'élévation de l'île à partir de deux projections et définit un champ de deformation simulant l'évolution de l'île à long terme. Le système modélise la croissance corallienne et produit des cartes de hauteur, utilisées ensuite pour entraîner un réseau antagoniste génératif conditionnel (cGAN) pour une génération rapide et contrôlable par l'utilisateur de configurations variées d'îles récifales.
% }{Introduction/figures/cGAN-example.png}{Exemple d'île générée dans le chapitre~\ref{chap:coral-island}. }{fig:intro-example-cgan}

% \AltTextImageR{
    La représentation sémantique que nous présentons dans le chapitre~\ref{chap:semantic-representation} vise à concevoir des terrains avec une abstraction de l'aspect 3D de la surface. Nous introduisons les "\glosses{EnvObj}" et leur représentation simplifiée du monde réel, que nous utilisons pour obtenir une représentation symbolique des caractéristiques du terrain, biotiques et abiotiques, présentes dans la scène. L'utilisation du symbolisme permet de se concentrer sur les interactions entre les différents éléments pour générer un écosystème plausible sans les besoins computationnels élevés d'une simulation physique précise. De plus, la représentation simplifiée utilisée permet à l'utilisateur de manipuler la disposition du terrain final sans avoir à choisir une représentation de terrain spécifique.
% }{Chapter 2/figures/Canyon5.png}{Exemple d'écosystème généré dans le chapitre~\ref{chap:semantic-representation}. }{fig:intro-example-env-obj}

% \AltTextImageR{
    Pour accroître le réalisme et l'impact visuel des paysages synthétiques générés, l'utilisation de techniques d'enrichissement du terrain est souvent nécessaire. Dans le chapitre~\ref{chap:erosion}, nous avons abordé le défi spécifique des simulations d'érosion, un type d'enrichissement qui imite les effets de l'eau, du vent et des forces érosives sur un terrain virtuel afin d'améliorer la crédibilité du paysage final. Une méthode d'érosion basée sur les particules est proposée, conçue pour être généralisable afin de rester flexible à plusieurs échelles, et dont l'implémentation est orientée vers la vitesse et la parallélisation. La principale flexibilité de notre méthode réside dans sa capacité à s'appliquer à plusieurs représentations de terrain, indépendamment du solveur de fluides utilisé, et à être généralisée aussi bien pour les paysages terrestres que marins.
% }{Chapter 3/results/karst.pdf}{Exemple de grottes marines générées dans le chapitre~\ref{chap:erosion}. }{fig:intro-example-erosion}








\chapter{Introduction}
\label{chap:introduction}

% \teaser{
%     \autofitgraphics[]{TeaserChap1.png, TeaserChap2.png, TeaserChap3.png}
% }

Over the past 50 years, terrain generation has emerged as an increasingly active domain within the field of computer graphics \cite{Fournier1982,Musgrave1989,Miller1986,Galin2019}. As the demand for realistic and automated processes has grown to support the creation of ever-larger and more detailed landscapes effortlessly, terrain generation techniques have evolved accordingly. As these goals are progressively achieved, a new trend shifts the focus towards greater user control. This work is focused on a specific branch of terrain generation: interactive creation and modelling of landscapes.

Landscape generation finds applications in diverse fields such as biology, geology, and robotics \cite{Tzachor2023,Chen2023,Gerigk2025,Rudin2022}. Additionally, it has become a central tool in the entertainment industry, particularly in video games and cinema, where creating realistic and dynamic environments is key for immersive experiences. The ability to generate landscapes that help in understanding natural rules and testing hypotheses makes this the ideal tool for both researchers and industries.

However, this exploration of the domain comes with significant challenges. In the case of underwater environments, finding a balance between automation and the user's creative desires is particularly complex: the visual and physical rules governing seascapes (such as coral growth patterns, sediment deposition, erosion by currents, or the interaction of light with water) differ markedly from those of terrestrial landscapes. Managing scaling is also more challenging in this context, as underwater landscapes often involve vast bathymetric structures while simultaneously requiring fine-scale detail (e.g., coral colonies, rocks, or vegetation) to remain scientifically relevant or visually convincing.

The central question guiding this research is: "How can we efficiently guide the user in the creation of virtual content along the production process line to maintain as much control as possible over the final product?". This concept of "guiding" rather than "replacing" the user is, from my point of view, fundamental, as no machine can truly know better than a human what the final product should look like. The goal is to present algorithms that can be used flexibly within a production pipeline, adapting to many terrain representations, fluid solvers, landscape types, users' hardware, or objectives, whether the final use is real-time rendering, realism, or animation.

Maintaining control over the creative process is essential. Each generated result should feel unique, meet the user's expectations, be explainable, and be easily correctable without requiring a complete regeneration. This approach ensures that the user remains at the centre of the creative process, with the tools and algorithms serving to enhance their objectives, rather than replacing the users themselves.

What sets this work apart is its emphasis on underwater landscapes or "seascapes", an area only slightly touched upon in Computer Graphics, but important in domains such as marine biology, oceanography, and underwater robotics. Furthermore, the extension of these techniques to new areas for the entertainment industry, such as video games and cinema, opens the door to novel visual experiences and storytelling techniques that take full advantage of the unique aesthetic and physical characteristics of underwater environments.

This thesis is developed alongside the creation of an autonomous underwater marine biology observation robot display in \cref{fig:intro-REMI} \cite{Maslin2021}. In this context, virtual environments are essential since field experiments are costly and prone to failure. Digital twins of the robot and its environment allow robotics simulations of material malfunctions and rare or unpredictable scenarios.

\section{Underwater environment generation for robotics}

\AltTextImage{
    To accurately model robotic mission, robotics simulators (such as Gazebo and Isaac Sim) need the geometry of the terrain, fauna, and flora the robot might encounter. They also need material parameters to realistically simulate sensor behavior and external forces acting on the robot.
% To accurately model a robotic mission, robotics simulators (such as Gazebo and Isaac Sim) needs the geometry of the terrain, fauna, and flora that the device and its sensors might encounter. It also requires intrinsic material parameters to simulate sensor behaviour and external forces acting on the robot.

Scene geometry must be accurately scaled, from centimetre-level robot precision to mission areas spanningthousands of square metres. Underwater environments also require considering depth. Simulated materials should realistically respond to light and sound, taking into account porosity, granularity and reflectance. These factors significantly affect sensor's accuracy.
% Scene geometry must be scaled appropriately to match both the robot, which operates at centimetre precision, and the mission area, which may cover thousands of square metres. Underwater environments also demand consideration of the third dimension (depth). The materials used in the simulation need to respond correctly to light and sound, accounting for porosity, granularity, and reflectance, which can distort sensor signals. 
Water currents strongly affect navigation. Realistic and interactive simulation thus requires advanced currents modelling.

}{REMI-robot.png}{Our underwater semi-AUV robot REMI executing a transect in Mayotte's lagoon.}{fig:intro-REMI}

Designing virtual test fields involves creating surfaces and obstacles, and the ability to replay simulations with identical or slightly modified setups. Adding randomness while preserving the designer's intent presents another challenge. % in maintaining control while introducing variability.

\section{Underwater environment generation for biology}

Virtual environments play a crucial role in biological research by enabling the observation and study of interactions within ecosystems. Biology relies heavily on phenomenology, where understanding emerges from observing how entities behave and interact under specific conditions.

% \AltTextImage{
Simulating these interactions allows scientists to test and refine their theories. Models can reveal unexpected behaviours, highlight gaps in understanding, and uncover hidden correlations. This iterative process between theory, simulation, and fieldwork is central to advancing biological knowledge.

To be useful, simulations must closely follow biological rules while also incorporating controlled randomness. This balance allows for realistic behaviour while making room for discovering novel phenomena. Human-Computer Interaction should help bridge the gap between biological expertise and computational tools, enabling domain experts to contribute effectively without introducing bias.

% }{Eccormier-fauna-flora.jpg}{}{fig:intro-challenges-bio}

These environments must also operate across multiple spatial scales. Simulations should cover large areas to reflect full ecosystems while still capturing fine-grained interactions between closely located entities.

\section{Underwater environment generation in Computer Graphics}

Overall, this work lies in the domain of Computer Graphics (CG), specifically in the fields of modelling and interaction.

Designing effective interactive tools requires careful consideration of both user needs and technical constraints. This involves exploring how users interact with terrain elements, defining intuitive controls, and ensuring the interface supports a creative yet efficient workflow. Behind these tools lie critical implementation decisions that directly affect usability and performance. Choosing the most adapted data structures, for instance, determines how well the system handles large or complex environments. The mathematical and physical models used must strike a balance between realism and computational cost, especially in scenarios where interaction must remain fluid and responsive. Algorithms must also be parallelisable to take advantage of modern hardware, enabling real-time updates and feedback. Lastly, simplifying certain aspects of natural phenomena is often necessary to reduce computational overhead while maintaining visual and behavioural plausibility. These layered decisions, from interaction design to algorithmic efficiency, ensure that the system remains both user-centred and technically robust.




\section{Contributions and outlines}
This thesis explores the procedural generation of underwater environments, with a particular focus on coral reef islands. Our contributions are organised into three complementary parts, covering the creation, structuring, and physical evolution of underwater terrains.

% \subsubsubsection{Background}
In \cref{chap:background} we will first introduce the fundamentals of procedural terrain generation. A description of the different terrain representations is provided, as well as an overview of coral biology and coral reef formation.

% \subsubsubsection{Automatic generation of coral reef islands}
\AltTextImageR{
    In \cref{chap:coral-island}, we propose a user-guided method for the procedural creation of coral reef islands as we see in \cref{fig:intro-example-cgan}. Users sketch the island shape and elevation from two projections and define a wind field that simulates long-term environmental deformation. The system models coral growth and outputs heightmaps that are further used to train a conditional Generative Adversarial Network (cGAN) for diversified generation. This method enables fast and controllable generation of varied reef island configurations.
}{Introduction/figures/cGAN-example.png}{Example of island generated in \cref{chap:coral-island}. }{fig:intro-example-cgan}

% \subsubsubsection{Semantic terrain representation}
\AltTextImageR{
    The semantic representation we present in \cref{chap:semantic-representation} aims to design the features of a terrain with an abstraction of the 3D aspect of the surface. We introduce \glosses{EnvObj} and their simplified representation from the real world, which we used to obtain a symbolic representation of the terrain features, biotic and abiotic, that are present in the scene (such as the canyon, rocks and corals in \cref{fig:intro-example-env-obj}). Using symbolism allows us to focus on the interactions between the different elements to generate a plausible ecosystem without the high computational needs of running an accurate multiphysics simulation. Moreover, the simplified representation used allows the user to manipulate the layout of the final terrain without having to choose a specific terrain representation.
}{Chapter 2/figures/Canyon5.png}{Example of ecosystem generated in \cref{chap:semantic-representation}. }{fig:intro-example-env-obj}

% \subsubsubsection{Erosion simulation}
\AltTextImageR{
    To increase the realism and the visual impact of the generated synthetic landscapes, the use of terrain enhancement techniques is often required. In \cref{chap:erosion}, we tackled the specific challenge of running erosion simulations, a type of enhancement that mimics the effects of water, wind, and erosive forces on a virtual terrain to improve the believability of the final landscape. A particle-based erosion method is proposed, designed to be generalisable for flexibility on multiple scales, and its implementation is oriented towards speed and parallelisation. The main flexibility of our method is to be applicable to multiple terrain representations, agnostic to the fluid solver used, and generalised for both landscapes and seascapes. \cref{fig:intro-example-erosion} illustrates our method simulating coastal erosion, operating at the air-water interface on a voxel grid.
}{Chapter 3/results/karst.pdf}{Example of sea caves generated in \cref{chap:erosion}. }{fig:intro-example-erosion}