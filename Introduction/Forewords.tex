\clearpage
\pagebreak

\section*{Abstract}

This thesis, entitled \textit{"Procedural terrain generation for underwater environments"},explores the specialized area of procedural terrain generation, specifically targeting underwater settings. Procedural terrain generation remains a vibrant research area within computer graphics, particularly as advancements in simulation, rendering, and interaction techniques have fostered increased collaboration with terrain experts across various disciplines.

Virtualizing the physical world enables users to observe and interact with it in ways that enhance understanding and break down the boundaries between scientific fields. Terrain science, by its nature, brings together a diverse range of experts, including geologists, oceanologists, physicists, meteorologists, biologists, roboticists, computer scientists, and 3D artists. This thesis focuses on involving users in the creation of virtual worlds through fast and controllable algorithms, with a particular emphasis on underwater environments.

The thesis is structured into three parts, guiding the reader from the initial design of a landscape through to its 3D modeling and final refinement.

In the first part, we introduce a formal approach to terrain design, allowing environments to be conceived in semantic terms, abstracting away from the complexities of 3D geometry and data structures.

The second part focuses on translating these conceptual environments into 3D form. We present new models for generating and modeling coral reef islands and karst networks.

Finally, in the third part, we concentrate on enhancing the realism of these 3D terrains through physical simulations of erosion processes. We demonstrate a flexible and controllable method for simulating the long-term effects of water and wind on both terrestrial and marine landscapes.







% This thesis, entitled \textit{"Procedural terrain generation for underwater environments"}, as its name suggests, focuses on the topic of procedural terrain generation with the specificity to tackle the tackle underwater environments. Terrain generation is still an open research area in computer graphics as, with the emergence of simulation, rendering, and interaction techniques, the entire field is experiencing increased collaboration with terrain experts. 
% Virtualization of the physical world helps users to see and manipulate it, allowing for a better understanding of it and bringing together numerous experts, gradually breaking down the boundaries of scientific disciplines. 
% Terrain science brings together, in almost a direct manner, geologists, oceanologists, physicists, meteorologists, biologists, roboticists, computer scientists, 3D artists, and more. We focused our work on the inclusion of the user in the generation process of virtual worlds through fast and controlable algorithms, with the possibility to dive underwater.

% This thesis is divided into three parts, guiding the user from the design of a landscape, its 3D modeling, until its finalization, step by step.
% Firstly, we will propose a formalization of terrain designing, allowing for the conception of environments in a semantic sense, abstracting away from 3D geometry and data structures.

% Secondly, we will see how to give 3D form to these environments. We will propose new models for the generation and modeling of coral reef islands and karst networks.

% In the third part, we will focus on adding realism to 3D terrains through physical simulations of erosion processes. We will demonstrate a flexible and controllable method to imitate the long-term effects of water and wind on both terrestrial and marine landscapes.

\textbf{Keywords:} terrain representation, procedural generation, physical simulations, user interaction

% \section*{Résumé}
% Cette thèse porte sur le thème de la génération procédurale de terrains en milieux sous-marins. La génération de terrain est un domaine de recherche encore ouvert en informatique graphique car, avec l'emergence des techniques de simulation, de rendu et d'interaction, le domaine entier connait un accroissement de collaboration avec les experts terrains. Amener le monde physique dans un ordinateur et permettre à son utilisateur de voir et manipuler ce monde virtuel permet de mieux le comprendre et de réunir de nombreuses experts ensemble, brisant petit à petit les frontières des disciplines scientifiques. La science des terrains rassemble, de manière presque directe, géologues, oceanologues, physiciens, météorologues, biologistes, roboticiens, informaticiens, artistes, etc... 

% Cette thèse se divise en trois parties amenant l'utilisateur de la conception d'un paysage jusqu'à sa finalisation, étape par étape, en gardant le control en tout point. 
% Premièrement, nous proposerons une formalisation d'esquissage de terrains, permettant la conception d'environnements dans un sens sémantique, permettant de s'abstraire de la géometrie et de structures de données. 

% Secondement, nous verrons comment donner forme 3D à ces esquisses. Nous proposerons de nouveaux algorithmes pour la génération et modélisation d'iles coralliennes et de réseaux karstiques.

% En troisième partie, nous nous interesserons à l'ajout de réalisme sur terrains existants au travers de simulations physiques de processus d'érosion. Nous montrerons une méthode flexible et controlable pour imiter les effets de l'eau et du vents à très long terme sur un paysage terrestre et marin.

% \textbf{Mots-clés :} représentation de terrain, génération procédurale, simulations physiques, interaction utilisateur