\chapter{State of the art}
\label{chap:state-of-the-art}
\minitoc

- ...

\section{Procedural generation}
\label{sec:state-of-the-art_procedural-generation}
- ...

\subsection{Definition}
- Algorithmic Content Creation: Procedural generation refers to the process of creating data algorithmically rather than manually. This technique is widely used in fields such as computer graphics, simulations, and game development. \\
- Generative Models: Utilizes predefined rules or algorithms to generate complex structures or systems. Examples include generating landscapes, textures, or entire worlds. \\
- Data Independence: The generated content is not explicitly stored but created in real-time or on-the-fly, allowing for more extensive and dynamic content creation. \\
- Applications: Commonly applied in video games for creating vast, varied, and detailed environments without requiring large amounts of storage space. \\
- Automated Data Creation: Procedural generation is a technique used to automatically create complex and diverse content through algorithmic processes. This involves defining a set of rules or procedures that can produce varied outputs. \\
- Dynamic and Flexible: Allows for the generation of content that can adapt and change in response to different inputs or conditions, making it highly suitable for applications requiring variability and adaptability. \\
- Integration of Algorithms and Data: Combines mathematical models, noise functions, and other algorithms to create realistic or abstract content. This can include generating landscapes, textures, or entire ecosystems. \\
- Real-World Relevance: Involves incorporating aspects of real-world phenomena or simulations (e.g., erosion patterns in terrain generation) to produce more believable and interactive environments. \\
- User-Driven Customization: Provides mechanisms for users to influence or guide the content generation process, allowing for customized or user-specific outcomes. \\
- Scalability and Efficiency: Offers scalability in content generation by producing large amounts of data with relatively low computational and storage costs compared to manually crafted content. \\
- Deterministic vs. Stochastic: \\
** Deterministic: The output is predictable and repeatable given the same input parameters. \\
** Stochastic: Incorporates randomness, leading to varied outputs even with the same initial conditions. \\
- Rule-Based Systems: Utilizes predefined rules or systems to generate content, allowing for controlled and structured outputs. \\
- Iteration and Refinement: Often involves iterative processes where initial results are refined or adjusted based on additional rules or parameters. \\
- Advantages \\
** Efficiency: Reduces the need for extensive manual creation, saving time and resources. \\
** Variability: Capable of generating a wide range of unique outputs from the same set of rules. \\
** Adaptability: Can easily adapt to changes in requirements or user input, allowing for dynamic content creation. \\
** Storage: Minimizes storage needs by generating content on-the-fly rather than storing large datasets. \\
- Challenges \\
** Complexity: Developing and fine-tuning procedural generation algorithms can be complex and require careful balancing of parameters. \\
** Realism vs. Performance: Achieving a balance between realistic content and computational performance can be challenging. \\
** User Expectations: Meeting user expectations for content quality and variety can be difficult, especially in interactive applications.

\subsection{History}
- Early Developments \\
** Mathematical Foundations: \\
*** Algorithmic Foundations: Early uses of procedural generation can be traced back to mathematical and algorithmic theories in the mid-20th century. Key concepts included randomness and noise functions. \\
*** Noise Functions: The introduction of noise functions, such as Perlin noise (1983), provided a method for generating smooth, natural-looking randomness in computer graphics. \\
** Early Computer Graphics: \\
*** Simplex Noise: Developed by Ken Perlin in 1985, Simplex noise was a significant advancement, providing a more computationally efficient and visually pleasing alternative to Perlin noise. \\
*** Fractals: In the 1980s, fractal geometry was introduced as a method for generating self-similar structures, influencing terrain generation and procedural modeling. \\
- Video Games and Interactive Media \\
** Early Video Games: \\
*** Rogue (1980): The early example of procedural generation in game level design, with randomly generated dungeons and item placement. \\
*** Elite (1984): Procedural generation for each planet creates unique characteristics such as the planet's name, its position in space, its economic model, and the types and quantities of resources available. This approach ensures a diverse and expansive universe, enhancing the game's replayability and exploration aspects.
** Procedural Content Generation (PCG): \\
*** Dwarf Fortress (2006): A significant milestone in procedural generation for games, known for its deep and complex world generation that includes civilizations, histories, and intricate ecosystems. \\
*** Spore (2008): Considered as the first game heavily using the technology of procedural generation, as no texture, music or animation are stored in the game \\
*** Minecraft (2011): Revolutionized procedural generation in gaming by generating vast, open worlds with diverse biomes and features. \\
- Applications in Simulation and Graphics \\
** Scientific Simulations: \\
*** Terrain Generation: Used in geological simulations and virtual landscapes for scientific research and training. \\
*** Fluid Dynamics: Procedural methods were applied to simulate realistic fluid behavior in various scientific and engineering contexts. \\
** Computer Graphics and Animation: \\
*** Film and Visual Effects: Procedural generation techniques have been used in films for creating complex visual effects, landscapes, and textures that would be time-consuming to model manually. \\
*** Real-Time Rendering: The advent of real-time graphics and game engines (e.g., Unreal Engine, Unity) leveraged procedural generation for dynamic content creation and environmental diversity. \\
- Technological Advancements \\
** Algorithmic Improvements: \\
*** Advanced Noise Functions: Developments in noise functions, such as Worley noise and new variations of Perlin noise, improved the quality and efficiency of procedural generation. \\
*** Machine Learning and AI: Recent advances in machine learning and artificial intelligence are being integrated with procedural generation to create more complex and adaptive content. \\
** Computing Power: \\
*** Increased Processing Power: As computing power has increased, so has the capability to generate more complex and detailed procedural content in real time. \\
*** Parallel Computing: Use of parallel processing and GPUs to accelerate procedural generation processes, enabling real-time applications and high-resolution simulations. \\
- Current Trends and Future Directions \\
** Hybrid Approaches: Combining procedural generation with manual design to achieve a balance between complexity and artistic control. \\
** User-Driven Content: Integration of user input and customization options to allow players or users to influence procedural content generation. \\
** Advanced Simulations: Use of procedural generation in advanced simulations for training, virtual reality, and scientific research, including climate modeling and urban planning. 


\subsection{Models represented}
Noise: Techniques like Perlin noise and Simplex noise for creating natural variations. \\
Cellular Automata: Models like Conway's Game of Life for generating patterns and structures. \\
Neural Networks: Use of deep learning to generate and refine procedural content. \\
Physical Phenomena Modeling: Simulating natural processes like erosion and sedimentation for realistic terrain features. 

- Noise \\
** Perlin Noise: \\
*** Overview: Developed by Ken Perlin in 1983, Perlin noise is a gradient noise function used to generate smooth, coherent patterns that resemble natural phenomena. \\
*** Applications: Commonly used in texture generation, terrain modeling, and procedural landscapes to create realistic, natural-looking variations. \\
*** Characteristics: Provides continuous and smooth randomness, making it ideal for creating natural-looking terrains and textures. \\
** Simplex Noise: \\
*** Overview: Introduced by Ken Perlin in 2001 as an improvement over Perlin noise. It is more computationally efficient and produces fewer directional artifacts. \\
*** Applications: Used in procedural texture generation, terrain creation, and simulations where higher performance and quality are needed. \\
*** Characteristics: Offers a higher degree of visual coherence and efficiency in high-dimensional space compared to Perlin noise. \\
** Worley Noise: \\
*** Overview: Also known as Voronoi noise, Worley noise generates patterns based on the distance between points in a space, producing cellular structures. \\
*** Applications: Suitable for creating textures such as stone or marble, and for modeling natural phenomena like cloud patterns and cellular structures. \\
*** Characteristics: Produces a pattern of cells with distinct boundaries, useful for generating irregular, organic textures. \\
- Cellular Automata \\
** Basic Concepts: \\
*** Definition: Cellular automata are discrete, grid-based models where each cell evolves based on a set of rules applied to its neighbors. \\
*** History: Originating in the 1940s with John von Neumann and Stanislaw Ulam, cellular automata have been used to model complex systems and processes. \\
** Applications in Procedural Generation: \\
*** Terrain Generation: Used to simulate natural processes like erosion, sediment deposition, and landform evolution. Examples include generating cave systems and cave-like features. \\
*** Pattern Formation: Employed to create various patterns and textures, such as forests, vegetation, and city layouts. \\
** Notable Models: \\
*** Game of Life: A famous example of cellular automata, used to demonstrate how simple rules can lead to complex, self-organizing patterns. \\
*** Langton's Ant: A simpler automaton that produces interesting and complex behavior from basic rules, demonstrating the potential of cellular automata for generating diverse patterns. \\
- Neural Networks \\
** Overview: \\
*** Artificial Neural Networks (ANNs): Computational models inspired by the human brain, used to learn patterns and generate data. \\
*** Generative Models: Types of neural networks, such as Generative Adversarial Networks (GANs) and Variational Autoencoders (VAEs), are specifically designed for content generation. \\
** Applications: \\
*** Texture and Terrain Generation: Neural networks can be trained to generate realistic textures and terrains based on training data, producing highly detailed and varied results. \\
*** Image Synthesis: Used to generate photorealistic images, including landscapes, environments, and even character models in video games and simulations. \\
** Key Techniques: \\
*** GANs (Generative Adversarial Networks): Consist of two neural networks, a generator and a discriminator, that work against each other to create realistic outputs. \\
*** VAEs (Variational Autoencoders): Learn to encode and decode data in a way that allows for the generation of new, similar data instances. \\
- Physical Phenomena Modeling \\
** Overview: \\
*** Definition: Models that simulate real-world physical processes to generate natural features and behaviors in procedural content. \\
*** Types: Includes models for fluid dynamics, erosion, sediment transport, and other geological and environmental processes. \\
** Applications: \\
*** Terrain Generation: Simulating erosion, sediment deposition, and other geological processes to create realistic terrain features such as mountains, valleys, and riverbeds. \\
*** Environmental Simulations: Used in simulations for weather patterns, climate modeling, and natural disaster scenarios. \\
** Key Models: \\
*** Erosion Models: Simulate the effects of weathering and erosion on terrain, often using algorithms that mimic natural processes like water flow and wind. \\
*** Sediment Transport: Models that simulate the movement and deposition of sediment, contributing to realistic terrain formation and changes.

\subsection{User interaction}
\subsubsection{Realism-Speed-Control Balance}
Realism: Achieving accurate representation of real-world features, useful in simulations and visualizations. \\
Speed: Real-time generation needs, classified by J. Gain's metrics (real-time, interactive, near-interactive, long-term). \\
Control: Meeting user demands and customization, addressing the challenge of “impossible” requests.

- Realism: \\
** Definition: The degree to which generated content mimics real-world characteristics, including visual accuracy, physical processes, and natural patterns. \\
** Applications: Critical in scenarios where realistic simulations are required, such as natural disaster simulations, scientific visualizations, and training environments. \\
** Techniques: \\
*** Physical Simulations: Implementing realistic physical processes (e.g., erosion, sediment transport) to enhance authenticity. \\
*** Expert Knowledge: Integrating domain-specific expertise (e.g., geology, ecology) to ensure accuracy in generated content. \\
** Challenges: \\
*** Complexity: Balancing detail and realism can lead to complex simulations requiring substantial computational resources. \\
*** Expertise: Ensuring realism may necessitate expert input, which can be resource-intensive. \\
- Speed: \\
** Definition: The efficiency of generating content within acceptable time constraints, categorized into different levels based on response time. \\
** Categories: \\
*** Real-Time: Generation occurring in less than 30 milliseconds, essential for interactive applications and gaming (e.g., procedural environments in VR). \\
*** Interactive: Generation in less than 3 seconds, suitable for user-driven customization in games and simulations. \\
*** Near-Interactive: Generation in less than 5 minutes, applicable for applications where some delay is acceptable (e.g., large-scale simulations). \\
*** Long-Term: Generation that takes longer, often used for precomputed content or less time-sensitive applications (e.g., offline rendering). \\
** Techniques: \\
*** Optimization: Using efficient algorithms and parallel processing to speed up generation processes. \\
*** Level of Detail (LOD): Adjusting detail based on the user's distance or focus to balance performance and visual quality. \\
- Control: \\
** Definition: The extent to which users can influence or direct the procedural generation process to meet their specific needs or preferences. \\
** Applications: \\
*** Customization: Allowing users to modify parameters or settings to tailor generated content to their preferences (e.g., terrain features, game worlds). \\
*** Artistic Control: Providing tools for artists and designers to guide the procedural generation process and integrate artistic elements. \\
** Challenges: \\
*** User Demands: Addressing the challenge of fulfilling diverse and sometimes contradictory user expectations. \\
*** Complexity: Balancing user control with the complexity of procedural systems to avoid overwhelming users or producing unrealistic results. 

\subsubsection{Interaction mechanisms}
- Parameter Adjustment: \\
** Sliders and Controls: User interfaces that allow adjustment of parameters such as terrain height, texture type, or weather conditions. \\
** Real-Time Feedback: Immediate visual or functional feedback based on user input, enhancing the interactive experience. \\
- Guided Creation: \\
** Templates and Presets: Offering predefined templates or presets that users can modify to fit their needs while ensuring a coherent base for the generated content. \\
** Assistive Tools: Providing tools or wizards that guide users through the generation process, helping them understand and control the outcome. \\
- Direct Manipulation: \\
** Interactive Tools: Tools that allow users to directly manipulate the generated content, such as sculpting terrain or painting textures. \\
** Live Updates: Changes made by the user are reflected in real-time, enabling immediate validation and adjustment. \\
\subsubsection{Challenges and considerations}
- Balancing Complexity and Usability: Ensuring that complex procedural generation systems remain user-friendly and accessible, even when offering deep customization options. \\
- Performance Impact: Managing the performance implications of user interactions, especially in real-time applications where frequent updates are needed. \\
- Feedback and Iteration: Providing adequate feedback to users to help them understand the impact of their interactions and iterate on their designs effectively.
% - ...

% \subsubsection{Realism-speed-control balance}
% - Main issue in terrain generation \\
% - Explanation of realism \\
% ** Generated landscapes are close to what is found in reality \\
% ** Generation incorporates natural processes to be realistic \\
% ** Requires extensive physical simulations, expert knowledge \\
% ** Useful in applications such as natural disaster simulations \\
% - Explanation of speed \\
% ** Fastest possible generation, aiming for real-time generation \\
% ** J. Gain classifies: real-time (< 30ms), interactive (< 3s), near-interactive (< 5min), and long-term. \\
% ** Useful in "infinity-scroll" video games, for example \\
% - Explanation of control \\
% ** Aims to meet user demands \\
% ** Major issue being "impossible" user demands \\
% ** Useful in most procedural generation applications: speeding up artists' work, for example \\
% - ...

\subsubsection{Regeneration}
- Manual Actions: How users can influence or direct the regeneration process. \\
- Issues with Regeneration: \\
** What to regenerate: Criteria for selecting parts of the terrain to refresh. \\
** How to regenerate: Techniques and algorithms used for regeneration. \\
- Problems with User Interactions: Handling user inputs and maintaining consistency. \\
- Action Storage: Possible use of formats like JSON to store and manage regeneration actions.

- Manual Actions \\
** User-Initiated Regeneration: \\
*** Triggering Regeneration: Users can manually trigger regeneration processes, such as refreshing a terrain or updating a landscape to reflect new parameters or conditions. \\
*** Interface Controls: Tools and options within user interfaces that allow users to initiate regeneration, such as buttons or commands for reloading or updating generated content. \\
** Customizable Parameters: \\
*** Adjustment of Variables: Users can adjust parameters or settings before triggering regeneration to influence the outcome. This includes changing terrain features, texture types, or simulation conditions. \\
*** Preview and Validation: Offering preview modes or validation checks to help users understand potential outcomes before finalizing regeneration. \\
- Issues with Regeneration \\
** What to Regenerate: \\
*** Selective Regeneration: Deciding whether to regenerate the entire model or specific parts. For example, only regenerating terrain features while keeping other elements intact. \\
*** Scope and Scale: Determining the extent of regeneration, whether it involves minor updates or complete overhauls of generated content. \\
** How to Regenerate: \\
*** Algorithmic Approaches: Different methods for regenerating content, such as reapplying algorithms or using new procedural rules to alter or update the existing content. \\
*** Incremental vs. Complete Regeneration: \\
*** Incremental: Applying changes gradually or updating only parts of the generated content to minimize disruption. \\
*** Complete: Full regeneration from scratch, which may be necessary for substantial changes or when previous results are no longer valid. \\
- Problems with User Interactions: \\
** Consistency: Ensuring that regeneration maintains or improves the consistency and quality of generated content, avoiding artifacts or inconsistencies. \\
** Feedback Handling: Managing user feedback and requests for regeneration, which can vary in scope and detail. \\
** Performance: Addressing the impact on performance and resource usage when regenerating content, especially in real-time or interactive applications. \\
- Action Storage \\
** Storage Formats: \\
*** JSON: Using JSON (JavaScript Object Notation) files to store regeneration actions and parameters. JSON provides a flexible, readable format for saving and retrieving procedural settings. \\
*** Other Formats: Potential use of alternative formats such as XML, binary files, or custom formats tailored to specific needs or systems. \\
** Tracking Changes: \\
*** Action History: Maintaining a history of user actions and regeneration events, allowing users to revert to previous states or track changes over time. \\
*** Versioning: Implementing version control for procedural generation settings, ensuring that different versions of generated content can be managed and compared. \\
** Restoration and Undo: \\
*** Undo Mechanisms: Providing options for users to undo recent regeneration actions or revert to previous states, enhancing flexibility and user control. \\
*** Restoration of Defaults: Allowing users to restore default settings or regeneration conditions if custom changes lead to unsatisfactory results. \\
- Implementation Considerations \\
** Efficiency: Designing algorithms and systems to handle regeneration efficiently, minimizing computational overhead and ensuring responsive updates. \\
** User Experience: Creating intuitive interfaces and feedback mechanisms to facilitate smooth user interactions with regeneration processes. \\
** Error Handling: Implementing robust error handling and recovery mechanisms to address issues that may arise during regeneration, such as unexpected results or system failures.
% \subsubsection{Regeneration}
% - Manual actions \\
% - Issues with regeneration \\
% ** What to regenerate? \\
% ** How to regenerate? \\
% ** Problems with user interactions? \\
% - Action storage file in JSON (?) \\
% - ...

\section{Terrain representation}
\label{sec:state-of-the-art_terrain-representations}
- ...

\subsection{2.5D terrains}
- ...

\subsubsection{Height maps}
- ...

\subsubsection{Height functions}
- ...

\subsection{3D terrains}
- Need for 3D concepts \\
** Geological information \\
** Volumetric data \\
- ...

\subsubsection{Main issues}
- Memory \\
- Visualization \\
- Modifications \\
- Conversion between representations \\
** Information loss \\
*** Error propagation on geometry (approximations on normals, Z resolution, surface, etc.) \\
*** Loss of subsurface information \\
- ...

\subsubsection{Types, definitions, advantages, disadvantages}
- Voxel grids \\
- Material stacks \\
- Meshes \\
- Implicit surfaces \\
- Implicit materials \\
- ...

\subsection{Other models}
- Concept of semantics \\
- ...

\subsection{Underwater landscapes}
- 3D Data: \\
** Coral Landscapes: Addressing challenges in modeling coral reefs and underwater features. \\
** Cavities: Representing underwater caves and karst formations. \\
- Interdisciplinary Data: \\
** Geological Validation: Integrating expert knowledge for accurate modeling. \\
** Challenges: Fewer experts and more uncertainties in underwater environments. \\
** Data Scarcity: Limited data availability for detailed underwater landscapes. \\
- Need for Multi-Scale: \\
** Integration of large and small elements. \\
** Level of Detail (LOD): Techniques for managing detail across different scales.

\subsubsection{3D data}
- Coral Landscapes: \\
** Complex Structures: Coral reefs feature complex 3D structures with intricate patterns including branching corals, coral mounds, and encrusting forms. These structures often have a high degree of variability and detail. \\
** Void Spaces: Coral reefs contain many voids and cavities such as caves, grottos, and karst networks, adding to the complexity of modeling these environments. \\
** Challenges: Modeling coral reefs requires accurately representing the porous and irregular nature of coral formations, which can be challenging due to the high level of detail and variability. \\
- Geological Features: \\
** Sedimentary Layers: Representing sedimentary layers and underwater geological formations such as seamounts and underwater ridges. \\
** Volcanic Activity: Incorporating features such as underwater volcanos and hydrothermal vents, which affect the landscape and ecosystem. \\
- Measurement and Data Collection: \\
** Sonar and LIDAR: Using sonar (e.g., multi-beam echo sounders) and LIDAR (Light Detection and Ranging) to collect detailed 3D data of underwater terrain. \\
** Remote Sensing: Utilizing remote sensing technologies to map and model underwater landscapes, especially in areas that are difficult to access.

\subsubsection{Interdisciplinary data}
- Geological Validation: \\
** Expert Consultation: Collaborating with geologists and marine scientists to validate and refine models based on real-world observations and data. \\
** Data Accuracy: Ensuring that the generated models accurately reflect real underwater geological and biological features. \\
- Biological Data: \\
** Coral and Marine Life: Integrating data on coral species, marine biodiversity, and ecosystem dynamics to create realistic and biologically accurate representations. \\
** Ecological Impact: Considering the impact of various biological factors on the terrain, such as coral growth patterns and marine erosion. \\
- Challenges: \\
** Data Scarcity: Limited availability of high-resolution data for some underwater environments, especially in less studied or remote areas. \\
** Integration: Combining geological, biological, and hydrological data effectively to create comprehensive and accurate models. 

\subsubsection{Multi-scale modeling}
- Large vs. Small Scale Elements: \\
** Macro Features: Incorporating large-scale features such as underwater mountains, ridges, and large coral formations. \\
** Micro Features: Modeling small-scale elements such as individual coral polyps, marine vegetation, and detailed sedimentary textures. \\
- Level of Detail (LOD): \\
** Adaptive LOD: Using techniques to adjust the level of detail based on user interaction or view distance to balance performance and visual fidelity. \\
** Detail Preservation: Ensuring that both macro and micro-scale features are accurately represented without losing important details during scaling or zooming. \\
- Visualization Techniques: \\
** Hydrographic Mapping: Employing specialized visualization techniques to represent underwater features clearly and accurately. \\
** Texture and Lighting: Using textures and lighting models that simulate underwater conditions, such as light absorption and scattering in water.

% - 3D Data \\
% ** Coral landscapes filled with voids \\
% ** Many cavities (caves, grottos, karst networks) \\
% - Interdisciplinary Data \\
% ** Rather common with terrain generation \\
% ** => Geological validation with experts \\
% ** For underwater, \\
% *** Fewer experts, \\
% *** More uncertainties \\
% *** Based more on observations \\
% *** Few data (coral landscapes < 0.1\% of oceans), for significant biological impact (25\% marine biodiversity) \\
% ** Mix of geology, biology, hydrology, and physics (especially fluid dynamics) \\
% - Need for multi-scale \\
% ** Not limited to underwater \\
% ** Integrate large elements (mountains) with small elements (vegetation) \\
% ** LOD \\
% - ...

\subsection{Fluid simulations}
- Definition and Purpose: \\
** Overview: Explanation of fluid simulations and their role in representing natural phenomena in terrains, including water flow, erosion, and sediment transport. \\
** Importance: Impact of accurate fluid simulations on creating realistic terrain and environmental models.\\
\subsubsection{Types of fluid simulations}
- 2D Fluid Simulations: \\
** Particle-In-Cell (PIC): \\
*** Concept: Combines particles with a grid to simulate fluid dynamics. \\
*** Applications: Used for simpler simulations and visualizations. \\
** Fluid-Implicit Particle (FLIP): \\
*** Concept: A hybrid method that combines particle and grid approaches for better accuracy and efficiency. \\
*** Applications: Suitable for capturing complex fluid behaviors in 2D environments. \\
** Stable Fluids: \\
*** Concept: A grid-based method for simulating stable, incompressible fluids with less computational complexity. \\
*** Applications: Often used in interactive applications where real-time performance is crucial.\\
- 3D Fluid Simulations: \\
** Smoothed Particle Hydrodynamics (SPH): \\
*** Concept: A particle-based method where fluid properties are represented by particles that interact based on smoothing kernels. \\
*** Applications: Useful for highly detailed simulations of fluids, including interactions with terrain. \\
** Grid-Based Methods: \\
*** Concept: Methods like Marker-And-Cell (MAC) and Navier-Stokes equations applied on a grid to simulate fluid behavior. \\
*** Applications: Used for detailed and accurate 3D simulations, including large-scale environments. \\
** Hybrid Methods: \\
*** Concept: Combining grid and particle approaches to leverage the strengths of both methods.\\
*** Applications: Balancing detail and performance in complex simulations. 

\subsubsection{Applications in terrain representation}
- Erosion and Sediment Transport: \\
** Simulation of Erosion: How fluid simulations model erosion processes affecting terrain features such as riverbeds and coastlines. \\
** Sediment Movement: Modeling the transport and deposition of sediment, contributing to realistic terrain evolution. \\
- Water Flow and Hydrology: \\
** Surface Water Dynamics: Representing the flow of water across terrain surfaces, including rivers, lakes, and wetlands. \\
** Subsurface Flow: Simulating groundwater movement and interactions with terrain features. \\
- Interactive Environments: \\
** Real-Time Simulations: Implementing fluid simulations in interactive applications, such as video games and virtual environments, where dynamic water interactions are crucial. 

\subsubsection{Challenges and considerations}
- Computational Resources: \\
** Performance Trade-Offs: Balancing simulation accuracy with computational efficiency, especially for real-time applications. \\
** Hardware Requirements: The need for powerful processors and memory to handle complex 3D fluid simulations. \\
- Accuracy vs. Realism: \\
** Detail vs. Performance: Finding the right balance between detailed fluid dynamics and the practical limitations of simulation resources. \\
** Simulation Artifacts: Addressing potential artifacts or inaccuracies in simulations that can impact realism. \\
- Integration with Terrain Models: \\
** Interaction with Terrain: Ensuring fluid simulations integrate seamlessly with terrain models, including handling interactions such as fluid erosion or deposition. \\
** Data Consistency: Maintaining consistency between simulated fluid behaviors and terrain features for accurate representations. 

\subsubsection{Recent developments and future trends}
- Advancements in Algorithms: \\
** New Techniques: Emerging algorithms and methods that enhance the accuracy and efficiency of fluid simulations. \\
** Real-Time Improvements: Innovations aimed at improving real-time performance and interactivity in fluid simulations. \\
- Integration with AI and Machine Learning: \\
** AI-Enhanced Simulations: Using machine learning to improve fluid simulation accuracy and adaptivity. \\
** Predictive Models: Leveraging AI to predict and simulate complex fluid behaviors based on historical data. 

% - Very important in procedural terrain generation \\
% - Allows justifying the geophysics of a simulation/generation \\
% - Quite fast solutions in 2D (PIC, FLIP, Stable Fluids, SPH, etc.) \\
% - But becomes much heavier and memory-intensive in 3D \\
% - ...

\section{Coral reefs (biological aspects)}
\label{sec:state-of-the-art_biology}
- Historical Discovery: Evolution of knowledge about coral reefs. \\
- Types of Coral Reefs: \\
** Islands, Barriers, Atolls: Different forms and structures of coral reefs. \\
- Atoll Theories: Historical and scientific theories explaining the formation of atolls. \\
- Importance in Biodiversity: Coral reefs as critical ecosystems with high marine biodiversity. \\
- Threats and Protection: Current threats to coral reefs and conservation efforts.

\subsection{Historical discovery of coral reefs}
- ... 
\subsubsection{Early observations}
- Ancient Knowledge: Initial observations by ancient civilizations (e.g., Greeks, Romans) and their interpretations of coral structures. \\
- Exploration Era: The role of early explorers and naturalists (e.g., Captain James Cook, Alexander von Humboldt) in documenting coral reefs and their biodiversity.
\subsubsection{Scientific discovery}
- 19th Century Advances: Key contributions from scientists such as Charles Darwin, who developed theories on coral reef formation. \\
- 20th Century Developments: Advances in marine biology and oceanography that improved understanding of coral reef ecosystems.

\subsection{Coral reef structure and formation}
- ... 
\subsubsection{Coral anatomy}
- Coral Polyps: Basic building blocks of coral reefs, their structure, and function. Each polyp is a tiny, soft-bodied organism that secretes calcium carbonate to form the reef structure. \\
- Coral Colonies: How individual polyps form colonies and contribute to the growth of the reef structure. 
\subsubsection{Types of coral reefs}
- Fringing Reefs: \\
** Definition: Reefs that are directly attached to a coastline, extending out from the shore. \\
** Characteristics: Shallow waters, often with a narrow reef crest and slope. \\
- Barrier Reefs: \\
** Definition: Reefs that run parallel to the coastline but are separated by a lagoon or deep channel. \\
** Characteristics: Typically found farther from the shore, with a more pronounced lagoon and deeper water. \\
- Atolls: \\
** Definition: Circular or oval reefs that encircle a lagoon, often formed around a submerged volcanic island. \\
** Characteristics: Reefs form a ring around a central lagoon, with no land in the center. 
\subsubsection{Reef building processes}
- Calcium Carbonate Deposition: The process by which corals and other organisms secrete calcium carbonate to build the reef structure. \\
- Bioerosion: The natural process by which reef structures are eroded by marine organisms such as parrotfish and sea urchins.

\subsection{Importance in biodiversity}
- ... 
\subsubsection{Species diversity}
- Marine Life: Coral reefs are among the most diverse ecosystems in the world, supporting thousands of marine species including fish, invertebrates, and algae. \\
- Endemism: Many species are found exclusively in coral reef environments, contributing to global biodiversity.
\subsubsection{Ecosystem services}
- Habitat Provision: Coral reefs provide essential habitats for a wide range of marine species, including commercially important fish and invertebrates. \\
- Nutrient Cycling: Reefs play a critical role in nutrient cycling, supporting the productivity of marine ecosystems. 
\subsubsection{Economic and cultural value}
- Fishing and Tourism: Coral reefs support important fisheries and generate significant revenue through tourism and recreational activities. \\
- Cultural Significance: Many coastal communities have cultural and spiritual connections to coral reefs, incorporating them into traditions and practices.

\subsection{Threats to coral reefs}
- ... 
\subsubsection{Climate change}
- Coral Bleaching: Caused by elevated sea temperatures, leading to the expulsion of symbiotic algae (zooxanthellae) and subsequent coral bleaching. \\
- Ocean Acidification: Reduced calcification rates due to increased \ch{CO2} levels, impacting coral growth and reef structure.
\subsubsection{Pollution}
- Nutrient Runoff: Increased nutrient levels from agricultural and urban runoff can lead to algal blooms that smother corals and disrupt reef balance. \\
- Marine Debris: Pollution from plastics and other debris that can damage coral reefs and harm marine life.
\subsubsection{Overfishing}
- Depletion of Fish Stocks: Overfishing can lead to the loss of key reef species and disrupt ecological balance. \\
- Destructive Fishing Practices: Practices such as blast fishing and cyanide fishing cause direct physical damage to reefs and harm marine biodiversity. 
\subsubsection{Coastal development}
- Habitat Destruction: Development activities such as dredging, land reclamation, and construction can destroy or degrade coral reef habitats. \\
- Sedimentation: Increased sedimentation from coastal construction and deforestation can smother corals and reduce light availability.

\subsection{Conservation efforts}
- ... 
\subsubsection{Marine Protected Areas (MPAs)}
- Designated Zones: Areas where human activities are regulated or restricted to protect coral reefs and promote recovery. \\
- Success Stories: Examples of successful MPA implementations and their positive impacts on reef health and biodiversity.
\subsubsection{Restoration projects}
- Coral Gardening: Techniques for growing and replanting corals to restore damaged reef areas. \\
- Artificial Reefs: Creation of artificial structures to provide new habitats for marine life and promote reef recovery.
\subsubsection{Community involvement}
- Local Engagement: Involving local communities in reef conservation through education, stewardship, and sustainable practices. \\
- Citizen Science: Encouraging public participation in monitoring and research efforts to support reef conservation. 
\subsubsection{Policy and legislation}
- International Agreements: Global and regional agreements aimed at protecting coral reefs and addressing climate change impacts (e.g., the Convention on Biological Diversity). \\
- National Policies: Policies and regulations at the national level to safeguard coral reefs and manage marine resources sustainably. 

\subsection{Future research and monitoring}
- ... 
\subsubsection{Innovative technologies}
- Remote Sensing: Use of satellite imagery and drones to monitor reef health and track changes over time. \\
- Genomics: Applying genetic research to understand coral resilience and adaptation to environmental stressors. 
\subsubsection{Adaptive management}
- Dynamic Strategies: Developing flexible management approaches that can adapt to changing conditions and emerging threats. \\
- Collaboration: Promoting collaboration among scientists, policymakers, and local communities to address complex challenges facing coral reefs.
\subsubsection{Education and awareness}
- Public Outreach: Raising awareness about the importance of coral reefs and the actions individuals can take to support conservation efforts. \\
- Training Programs: Providing training for scientists, managers, and local stakeholders to enhance reef management and restoration practices.



\section{Implicit terrains with materials}
\label{sec:state-of-the-art_implicit-terrain-with-materials}
- Volumetric modeling is important for representing 3D structures \\
- Allows for the representation of cavities, arches, overlays, etc. \\
- The concept of materials allows for including much more information for the following parts: enhancement and rendering \\
** Enhancement (e.g., erosion) needs to know the type of soil at the surface and subsurface to be realistic \\
** Rendering needs to know the material at the surface to correctly display textures \\
- ...

\subsection{Material density}
- ...

\subsubsection{Material granularity}
- ...

\subsubsection{Soil triangle}
- ...

\subsection{Scalar functions}
- ...

\subsection{Blending functions}
- ...

\subsection{Placement functions}
- ...

\subsection{Material usage}
- ...

\subsubsection{Defining the final material}
- ...

\subsubsection{Post-processing: material transformation}
- ...
