\section*{Lay summary (EN)}
This thesis contributes to the development of underwater robots for studying marine ecosystems. Since field missions are costly and difficult to organize, it proposes the creation of realistic virtual environments to test and validate these robots before deployment. The work explores the procedural generation of underwater worlds inspired by coral reefs, combining digital sketching, machine learning, and biological knowledge. It introduces methods to build large-scale coral landscapes, simulate the living organisms that inhabit them, and reproduce erosion and aging processes of marine terrains. These interactive and modular tools keep the user at the center of creation while integrating geological, biological, and oceanographic constraints. By bridging robotics, ecology, and computer graphics, this research opens new perspectives for virtual underwater exploration and robotic validation.



\section*{Résumé vulgarisé (FR)}
Cette thèse s'inscrit dans le développement de robots sous-marins pour l'étude des écosystèmes marins. Comme les missions en mer sont coûteuses et difficiles à organiser, elle propose de créer des environnements virtuels réalistes pour tester et valider ces robots avant leur déploiement. Le travail explore la génération procédurale d'univers sous-marins inspirés des récifs coralliens, en combinant dessin numérique, apprentissage automatique et connaissances biologiques. Il introduit des méthodes pour construire de vastes paysages coralliens, simuler la vie qui les peuple et reproduire l'érosion et le vieillissement des fonds marins. Ces outils interactifs et modulaires placent l'utilisateur au centre du processus tout en intégrant les contraintes géologiques, biologiques et océanographiques. En réunissant robotique, écologie et informatique graphique, cette recherche ouvre de nouvelles perspectives pour l'exploration sous-marine virtuelle et la validation robotique.