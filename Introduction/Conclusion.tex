\chapter{Conclusion}
% \addcontentsline{toc}{chapter}{Conclusion}

% In this thesis we propose novel contributions in the field of procedural environment generation for underwater scenes. We will briefly summarize these contributions then propose future research perspectives.

Procedural generation of virtual terrains has matured over decades, yet user control remains a central challenge. This thesis has addressed this challenge in the context of underwater environments, with particular attention to coral reef islands and broader implications for biology, geology and graphics. The guiding question has been how to support the user in the creation process rather than replace creative intent, while ensuring flexibility, interactivity and modularity.

\paragraph{User-guided learning-based approach} We introduced in \cref{chap:coral-island} a user-guided procedural method for coral reef island generation that combines sketch-based specification of island shape with wind-driven deformation and a conditional generative adversarial network for diversity. This approach avoids reliance on large real-world elevation datasets, achieves training in approximately one hour on procedurally generated data, and enables inference in milliseconds. This method demonstrates how neural expressivity can be harnessed to lift procedural constraints and present a more flexible interface for designers. It is important to note that the use case was simple enough for a neural network as the generation rules are distance-based, meaning that the network may simply learn to create distance maps; more complex scenarios are possibly not as fast to learn.

\paragraph{Semantic ecosystem modelling} \cref{chap:semantic-representation} proposed a semantic, entity-based ecosystem modelling framework based on phenomenological principles. By representing biotic and abiotic elements as symbolic objects with user-definable interactions, this work trades physics and biological fidelity for computation speed. The abstraction permits rule definitions informed by domain insights without committing to heavy simulation, thereby offering a lightweight yet extensible means for environment design across multiple representations. The quality and fidelity of the generation is directly dependent on rule definitions, shifting the responsibility into the user's hands, which is far from ideal in a user-oriented application.

\paragraph{Particle-based erosion} In \cref{chap:erosion}, we presented a particle-based erosion framework operating on 2.5D and 3D terrain representations. Each particle is simulated independently along its trajectory, so that the process may be halted at any point without yielding inconsistent partial states. This representation-agnostic and parallelisable method permits interactive control of erosion and can be combined with more accurate submodules if higher fidelity is required. The modular design ensures that improvements in flow modelling or material response can be integrated without altering the overall pipeline. Inter-particle interaction removal is a large physical simplification, trading accuracy with simulation time.

\section*{Future work and research perspectives}

Future work and research perspectives span several directions that extend the foundations established in this thesis. These directions aim to deepen the modular, user-centric approach to terrain generation by incorporating recent advances in data-driven modelling, procedural inference, and simulation surrogates. Each perspective addresses a distinct aspect of the pipeline. While technically diverse, these perspectives share a common goal: to enhance user control, reduce reliance on manual tuning or exhaustive simulation, and promote adaptability across creative and scientific domains.

\paragraph{Inverse procedural generation} An interesting direction would be to infer procedural rules from an observed terrain and its semantic elements by inverting the symbolic framework of \cref{chap:semantic-representation}. The aim is to analyse a single scene, and abstract it as \glosses{EnvObj} to hypothesise placement, implicit geometry \cite{Guerin2016a} and interaction rules that could generate similar patterns \cite{Stava2010,Stava2014}. Scene analysis involves mapping observed entities and spatial distributions \cite{Emilien2015a} into the symbolic vocabulary. Rule hypothesis then explores candidate constructs, followed by forward simulation to validate and adjust parameters until regenerated outputs resemble the original scene. Once validated, the inferred rules can be used to extend the terrain beyond observed boundaries, simulate temporal evolution, or perform inpainting.

\paragraph{User-data-driven generation} We could develop generative models that learn terrain design style from a limited archive of prior creations of a single user by building on the sketch-based cGAN method of \cref{chap:coral-island} or through neuroevolution \cite{Stanley2005,Cortes2024} in association with the generation rules inferred on the semantic representation of the user terrains. Generating label maps from point, curve and region \glosses{EnvObj} may provide a way to produce a dataset with data generation and data augmentation. Once trained, freehand sketching of a label map generates the \glosses{EnvObj} to populate the landscape, which stays conform to prior creations. The challenge is to propose a representation scheme for the cGAN architecture that outputs \glosses{EnvObj}' skeleton and properties.

\paragraph{Learned erosion model} A final work proposed from this thesis is the learned approximation of erosion from \cref{chap:erosion} to enable fast terrain ageing with controllable fidelity. Provided an initial flow field \cite{Tompson2017} and terrain as a voxelised image and the resulting grid after erosion iterations with varied particle counts, the training of a 3D cGAN \cite{Ongun2018} or Transformer \cite{Vaswani2017} could result in a learned-based erosion model. If stable enough, this model would enable interpolation directly on a desired number of particles and wind flow direction in the way of a Nerf \cite{Mildenhall2020}. Providing a model for each possible \gloss{EnvObj} of a terrain could be a means to generate each 3D model plausibly.


% \midConclusion

% The research perspectives outlined above build directly on the thematic contributions and modular philosophy of this thesis. By advancing user-data-driven generation, inverse procedural generation and analytic erosion models, future work can deepen and extend the foundations established in this thesis. Interdisciplinary collaboration remains essential, drawing on expertise in machine learning, human-computer interaction, computer vision, geology, ecology and related fields. The modular abstractions and human-in-the-loop design principles presented in this thesis offer a groundwork for these investigations. Pursuing these directions is expected to yield more robust, flexible and user-centered terrain generation tools that serve both creative and scientific objectives. 




% \section*{Contributions}
% \paragraph{Association of procedural generation with deep learning methods}  Deep learning and machine learning have been expanding domains for the last few decades, and the Computer Graphics community has been brittle to incorporate these fields' methods. Through our work (\cref{chap:coral-island}), we trained a deep learning method in association with sketch-based modeling to improve usability, without replacing the fundamental importance of modeling methods for data-sparse domains such as, in our case, coral reef islands.

% We have seen that this wrapper around procedural modeling can improve performences and interaction, allowing for user inputs with less constraints, and less restrictions. This interface between the complex requirements of a procedural method and the needs for intuitive UI for the user is shown to be easy to put in place, with the illustration of the out-of-the-box pix2pix model. 

% We limit our scope to output 2.5D height fields, but the acceleration of fluid simulations with convolutional neural networks suggests that even the most complex fields may be candidates to this type of associations. That the main challenge for completely including neural networks (NN) in CG works may be the design of intuitive controls as inputs of the NN.


% \paragraph{Ecosystem simulation for terrain modeling} % \paragraph{User-oriented generation}  User Interaction is a whole field of Computer Graphics and is in constant evolution since the begining of computer science. Procedural generation, and more specifically procedural terrain generation, has been relying on random generators and the increase of computation power to produce infinite content or include physical simulations, focusing on automatic generation to produce quantity. More recently, authoring has taken more importance in the field, showing an increasing concern for user-oriented generation over automatic generation, which we may summarize as the desire to obtain quality over quantity.

% In \cref{chap:coral-island} we introduced learning-based methods through the use of cGAN as an interface between curve-based modeling algorithms and drawing-based interaction. The neural network is trained on procedural data that follows user-designed parameters specifically proposed for data-sparse domain, in our case, coral reef islands. We then introduced in \cref{chap:semantic-representation} 