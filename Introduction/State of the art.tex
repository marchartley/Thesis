\chapter{State of the art}
\label{chap:state-of-the-art}
\minitoc

- ...

\section{Procedural generation}
\label{sec:state-of-the-art_procedural-generation}
- ...

\subsection{Definition}
- ...

\subsubsection{Official definition}
- ...

\subsubsection{My definition}
- ...

\subsection{History}
- ...

\subsection{Models represented}
- ...

\subsubsection{Noise}
- ...

\subsubsection{Cellular automata}
- ...

\subsubsection{Neural networks}
- ...

\subsubsection{Physical phenomena modeling}
- ...

\subsection{User interaction}
- ...

\subsubsection{Realism-speed-control balance}
- Main issue in terrain generation \\
- Explanation of realism \\
** Generated landscapes are close to what is found in reality \\
** Generation incorporates natural processes to be realistic \\
** Requires extensive physical simulations, expert knowledge \\
** Useful in applications such as natural disaster simulations \\
- Explanation of speed \\
** Fastest possible generation, aiming for real-time generation \\
** J. Gain classifies: real-time (< 30ms), interactive (< 3s), near-interactive (< 5min), and long-term. \\
** Useful in "infinity-scroll" video games, for example \\
- Explanation of control \\
** Aims to meet user demands \\
** Major issue being "impossible" user demands \\
** Useful in most procedural generation applications: speeding up artists' work, for example \\
- ...

\subsubsection{Regeneration}
- Manual actions \\
- Issues with regeneration \\
** What to regenerate? \\
** How to regenerate? \\
** Problems with user interactions? \\
- Action storage file in JSON (?) \\
- ...

\section{Terrain representation}
\label{sec:state-of-the-art_terrain-representations}
- ...

\subsection{2.5D terrains}
- ...

\subsubsection{Height maps}
- ...

\subsubsection{Height functions}
- ...

\subsection{3D terrains}
- Need for 3D concepts \\
** Geological information \\
** Volumetric data \\
- ...

\subsubsection{Main issues}
- Memory \\
- Visualization \\
- Modifications \\
- Conversion between representations \\
** Information loss \\
*** Error propagation on geometry (approximations on normals, Z resolution, surface, etc.) \\
*** Loss of subsurface information \\
- ...

\subsubsection{Types, definitions, advantages, disadvantages}
- Voxel grids \\
- Material stacks \\
- Meshes \\
- Implicit surfaces \\
- Implicit materials \\
- ...

\subsection{Other models}
- Concept of semantics \\
- ...

\subsection{Underwater landscapes}
- 3D Data \\
** Coral landscapes filled with voids \\
** Many cavities (caves, grottos, karst networks) \\
- Interdisciplinary Data \\
** Rather common with terrain generation \\
** => Geological validation with experts \\
** For underwater, \\
*** Fewer experts, \\
*** More uncertainties \\
*** Based more on observations \\
*** Few data (coral landscapes < 0.1\% of oceans), for significant biological impact (25\% marine biodiversity) \\
** Mix of geology, biology, hydrology, and physics (especially fluid dynamics) \\
- Need for multi-scale \\
** Not limited to underwater \\
** Integrate large elements (mountains) with small elements (vegetation) \\
** LOD \\
- ...

\subsubsection{Fluid simulations}
- Very important in procedural terrain generation \\
- Allows justifying the geophysics of a simulation/generation \\
- Quite fast solutions in 2D (PIC, FLIP, Stable Fluids, SPH, etc.) \\
- But becomes much heavier and memory-intensive in 3D \\
- ...

\section{Coral reefs (biological aspects)}
\label{sec:state-of-the-art_biology}
- Historical discovery of coral reefs \\
- Islands, barriers, atolls \\
- Atoll theories \\
- Importance in biodiversity \\
- Threats, protection, importance of understanding them \\
- ...


\section{Implicit terrains with materials}
\label{sec:state-of-the-art_implicit-terrain-with-materials}
- Volumetric modeling is important for representing 3D structures \\
- Allows for the representation of cavities, arches, overlays, etc. \\
- The concept of materials allows for including much more information for the following parts: amplification and rendering \\
** Amplification (e.g., erosion) needs to know the type of soil at the surface and subsurface to be realistic \\
** Rendering needs to know the material at the surface to correctly display textures \\
- ...

\subsection{Material density}
- ...

\subsubsection{Material granularity}
- ...

\subsubsection{Soil triangle}
- ...

\subsection{Scalar functions}
- ...

\subsection{Blending functions}
- ...

\subsection{Placement functions}
- ...

\subsection{Material usage}
- ...

\subsubsection{Defining the final material}
- ...

\subsubsection{Post-processing: material transformation}
- ...
