% \chapter{State of the art}
% \label{chap:state-of-the-art}
% \minitoc

% - ...

\section{Procedural generation}
\label{sec:state-of-the-art_procedural-generation}
Procedural generation is a method used in computer science to create data algorithmically rather than manually, enabling the automatic generation of large amounts of content with minimal human input. This approach is crucial in fields such as game development, where it helps create expansive, varied worlds, and in simulations or data generation where diverse scenarios or datasets are needed. The process typically involves defining rules and algorithms that dictate how content is generated, ensuring it meets specific criteria and patterns, often incorporating randomness through noise functions to produce unique results each time. Incorporating adjustable parameters and customizable rules in the algorithms allows users to influence the characteristics and outcomes of the generated content. This method enhances efficiency, creativity, and scalability in digital content creation, but the main challenges are ensuring the generated content is both diverse and coherent and achieving a balance between speed, realism and control to meet desired design and quality.

\subsection{Definition}
Procedural generation is a powerful technique for creating data algorithmically, rather than manually. This method is extensively used in areas such as computer graphics, simulations, and game development. Essentially, it involves using predefined rules or algorithms to generate complex structures or systems. For example, it can be employed to create landscapes, textures, or even entire worlds.

One of the main benefits of procedural generation is its data independence. The content generated is created in real-time or on-the-fly, rather than being stored explicitly. This allows for the creation of extensive and dynamic content without the need for large amounts of storage space.

In practical applications, procedural generation is commonly used in video games. It enables the creation of vast, varied, and detailed environments efficiently, without requiring extensive storage. This automated approach involves defining a set of rules or procedures that produce diverse outputs, making the content creation process both dynamic and flexible. The generated content can adapt and change in response to different inputs or conditions, which is particularly valuable for applications needing variability and adaptability.

The integration of algorithms and data is a key aspect of procedural generation. It combines mathematical models, noise functions, and other algorithms to produce both realistic and abstract content. This might involve generating natural features like landscapes or textures, or even entire ecosystems. Incorporating elements of real-world phenomena, such as erosion patterns in terrain generation, can make the environments more believable and interactive.

Moreover, procedural generation allows for user-driven customization. Users can influence or guide the content generation process, leading to customized or user-specific outcomes. This feature, combined with the scalability and efficiency of procedural methods, means that large amounts of data can be produced with relatively low computational and storage costs compared to manually crafted content.

Procedural generation can be categorized into deterministic and stochastic systems. Deterministic systems produce predictable and repeatable outputs given the same input parameters, while stochastic systems introduce randomness, leading to varied outputs even with identical initial conditions.

Rule-based systems are another aspect of procedural generation, using predefined rules to generate content. This approach allows for controlled and structured outputs. Typically, procedural generation involves iterative processes, where initial results are refined or adjusted based on additional rules or parameters.

The advantages of procedural generation include efficiency, as it reduces the need for extensive manual creation, and variability, as it can produce a wide range of unique outputs from the same set of rules. Additionally, it is adaptable, easily responding to changes in requirements or user input, and it minimizes storage needs by generating content on-the-fly.

However, there are challenges associated with procedural generation. The complexity of developing and fine-tuning algorithms can be significant, requiring careful balancing of parameters. Striking a balance between realistic content and computational performance can also be difficult. Furthermore, meeting user expectations for content quality and variety, especially in interactive applications, can be a challenge.

\subsection{History}
Early developments in procedural generation can be traced back to the mid-20th century, grounded in mathematical and algorithmic theories. This period saw the introduction of key concepts like randomness and noise functions, which laid the foundation for procedural techniques. One notable advancement was the introduction of noise functions, such as Perlin noise in 1983. This method allowed for the generation of smooth, natural-looking randomness in computer graphics.

Building on this, Ken Perlin developed Simplex noise in 1985. Simplex noise represented a significant improvement over Perlin noise by offering a more computationally efficient and visually pleasing alternative. During the same era, fractal geometry emerged, introducing the concept of generating self-similar structures. This had a substantial impact on terrain generation and procedural modeling, providing a new way to create intricate and repeating patterns.

As procedural generation began to find applications in video games and interactive media, its impact became more pronounced. In 1980, the game "Rogue" showcased procedural generation in its level design, featuring randomly generated dungeons and item placements. This was followed by "Elite" in 1984, which utilized procedural generation to create unique characteristics for each planet, such as names, positions in space, economic models, and resource availability. This approach contributed to a diverse and expansive game universe, enhancing replayability and exploration.

The evolution continued with significant milestones in procedural content generation (PCG). "Dwarf Fortress," released in 2006, became renowned for its deep and complex world generation, which included detailed civilizations, histories, and ecosystems. "Spore," released in 2008, pushed the boundaries further by using procedural generation extensively, as the game did not store textures, music, or animations. "Minecraft," released in 2011, revolutionized procedural generation in gaming by creating vast, open worlds with a variety of biomes and features.

In addition to video games, procedural generation found applications in simulations and computer graphics. For scientific simulations, it was used in terrain generation for geological studies and virtual landscapes, as well as in fluid dynamics to simulate realistic fluid behaviors. In the realm of computer graphics and animation, procedural techniques became essential for creating complex visual effects, landscapes, and textures in films, which would be labor-intensive to model manually. The advent of real-time graphics and game engines like Unreal Engine and Unity further leveraged procedural generation to create dynamic content and diverse environments.

Technological advancements have played a crucial role in the evolution of procedural generation. Improvements in noise functions, such as Worley noise and new variations of Perlin noise, enhanced both the quality and efficiency of procedural methods. Recent developments in machine learning and artificial intelligence have been integrated with procedural generation to produce more complex and adaptive content. The increase in computing power has also been pivotal, allowing for the generation of more detailed and complex procedural content in real time. Additionally, the use of parallel processing and GPUs has accelerated these processes, enabling real-time applications and high-resolution simulations.

Looking to the future, current trends in procedural generation include hybrid approaches that combine procedural techniques with manual design to balance complexity and artistic control. User-driven content is becoming more prevalent, allowing players and users to influence procedural content generation. Moreover, procedural generation is increasingly being applied in advanced simulations for areas such as training, virtual reality, and scientific research, including climate modeling and urban planning.

\subsection{Models represented}
Procedural generation encompasses various models and techniques that create complex and natural-looking content. These models range from mathematical functions to advanced neural networks and physical simulations, each contributing uniquely to the generation of diverse and realistic environments.

One foundational technique in procedural generation is noise. Perlin noise, developed by Ken Perlin in 1983, is a gradient noise function designed to produce smooth, coherent patterns that mimic natural phenomena. Its continuous and smooth randomness makes it particularly suitable for generating textures, terrains, and procedural landscapes that appear natural. Building on this, Simplex noise, introduced by Perlin in 2001, offers a more computationally efficient alternative with fewer directional artifacts. It is favored for its improved visual coherence and efficiency in higher-dimensional spaces, making it ideal for procedural texture generation and terrain creation.

Another variant of noise, Worley noise—also known as Voronoi noise—creates patterns based on the distance between points in a space. This technique produces cellular structures with distinct boundaries, making it useful for generating textures like stone or marble, and for modeling natural patterns such as cloud formations and cellular structures.

In addition to noise functions, cellular automata provide another approach to procedural generation. These are discrete, grid-based models where each cell evolves according to a set of rules applied to its neighbors. Originating in the 1940s with John von Neumann and Stanislaw Ulam, cellular automata have been used to model complex systems and processes. They are particularly effective for simulating natural processes such as erosion and sediment deposition, which can be employed to generate cave systems and other landforms. Cellular automata are also used to create various patterns and textures, including forests, vegetation, and city layouts. Notable examples include Conway's Game of Life, which demonstrates how simple rules can lead to complex, self-organizing patterns, and Langton's Ant, which showcases how basic rules can produce diverse and intricate behaviors.

Neural networks represent a more recent advancement in procedural generation. Artificial Neural Networks (ANNs), inspired by the human brain, learn patterns and generate data. Generative models, including Generative Adversarial Networks (GANs) and Variational Autoencoders (VAEs), are designed specifically for content generation. GANs use two neural networks—a generator and a discriminator—that compete with each other to produce realistic outputs. VAEs, on the other hand, encode and decode data to generate new, similar instances. These models are applied to generate realistic textures, terrains, and photorealistic images, enhancing the detail and variety of content in video games and simulations.

Finally, physical phenomena modeling uses simulations of real-world processes to generate natural features and behaviors. This includes models for fluid dynamics, erosion, sediment transport, and other geological and environmental processes. These models are crucial for creating realistic terrain features such as mountains, valleys, and riverbeds, as well as for environmental simulations that include weather patterns, climate modeling, and natural disaster scenarios. Erosion models, for instance, simulate the effects of weathering and erosion on terrain, using algorithms that mimic natural processes like water flow and wind. Sediment transport models simulate the movement and deposition of sediment, contributing to the realistic formation and evolution of terrain.

Together, these models and techniques form a comprehensive toolkit for procedural generation, enabling the creation of diverse, realistic, and dynamic content across various applications.

\subsection{User interaction}
\subsubsection{Realism-Speed-Control Balance}
User interaction is a crucial aspect of procedural generation, affecting how users balance realism, speed, and control over generated content.

Realism refers to the extent to which generated content accurately represents real-world characteristics, such as visual details, physical processes, and natural patterns. This is especially important in simulations, visualizations, and training environments where authenticity is critical. Techniques to enhance realism include physical simulations, which model processes like erosion and sediment transport, and the integration of expert knowledge from fields such as geology and ecology. However, achieving realism can be challenging due to the complexity of detailed simulations and the need for specialized expertise, which can demand substantial computational resources and inputs.

Speed is about the efficiency of content generation within acceptable time constraints. While no official categorisation has been set, James Gain proposed to describe levels of speed based on response time: \\
Real-Time: Generation in less than 30 milliseconds, essential for interactive applications like VR environments. \\
Interactive: Generation in under 3 seconds, suitable for user-driven customization in games and simulations. \\
Near-Interactive: Generation in less than 5 minutes, applicable for larger-scale simulations where some delay is acceptable. \\
Long-Term: Generation that takes longer, often used for precomputed content or offline rendering.

Optimizing speed involves using efficient algorithms and parallel processing. Almost all recent works achieve fast execution time thanks to high parallelisation on GPU. Other types of algorithm may rely on a refinement paradigm, generating a coarse result at first and then iteratively adding finer details, such that the global output shape is available long time before the real final result.

Control refers to how much users can influence or direct the procedural generation process to meet their specific needs or preferences. This includes:

Customization: Allowing users to modify parameters like terrain features or game worlds.
Artistic Control: Providing tools for artists and designers to guide the generation process and incorporate artistic elements.
Challenges in control involve managing diverse and sometimes conflicting user expectations, and balancing the complexity of procedural systems to avoid overwhelming users or producing unrealistic results.

\subsubsection{Interaction Mechanisms}
Interaction Mechanisms are the tools and methods used to facilitate user interaction with procedural generation:

Parameter Adjustment: User interfaces like sliders and controls enable adjustments to parameters such as terrain height or texture type. Real-time feedback ensures users see the immediate impact of their adjustments.
Guided Creation: Templates and presets offer predefined starting points that users can modify. Assistive tools or wizards help guide users through the generation process, making it easier to understand and control the outcome.
Direct Manipulation: Interactive tools allow users to directly manipulate generated content, such as sculpting terrain or painting textures. Live updates reflect changes in real-time, facilitating immediate validation and adjustment.

\subsubsection{Challenges and Considerations}
Challenges and Considerations include balancing complexity and usability to ensure that sophisticated procedural systems remain user-friendly. Performance impact is another concern, especially in real-time applications where frequent updates are required. Effective feedback and iteration mechanisms are crucial for helping users understand the impact of their interactions and refine their designs.

\subsubsection{Regeneration}
The regeneration of procedurally generated models is an issue often overlooked. As the user is almost satisfied with its generation, he may decide to adjust slightly the parameters used. The challenges of the regeneration are multiple: How to let the user explore the \gloss{ParamSpace} of the model freely? How to ensure that a small change in a variable induce a small change in the result? How to handle the edition from direct manipulations after regeneration? Those question arise directly from the realism-speed-control balance of our procedural model.

Users can manually start the regeneration process to refresh or update a model, such as a terrain or landscape. This allows the content to reflect new parameters or conditions as specified by the user. The generation speed has a strong influence on the possibility for the user to fine tune parameters as algorithms that can be executed in real-time or interactive-time keep the user active in his work. Long execution times force him to find optimal parameters beforehand in order to create a satisfying model faster than if it was hand-made, which, when not completely explicit, may feel impossible. 
The speed can be improved by different strategies:  \\
- Modifications that have a local incidence on the resulting model requires only a local regeneration of the output. Limiting the spatial scope of an interaction is an efficient mean to keep in the same time a way to explore the \gloss{ParamSpace} of the algorithm while keeping the predictability of the outcome by avoiding global changes from a local edition. \\
- Live preview of the outcome can be displayed in a coarse computation, 

Interface Controls: Various tools and options in the user interface enable users to initiate regeneration. This may include buttons or commands designed to reload or update the generated content based on user inputs.
Customizable Parameters:

Adjustment of Variables: Before initiating regeneration, users can adjust parameters or settings to influence the outcome. This might involve changing terrain features, texture types, or simulation conditions to achieve the desired results.
Preview and Validation: Offering preview modes or validation checks helps users understand potential outcomes before finalizing the regeneration. This feature ensures that users can visualize changes and make adjustments as needed.
Issues with Regeneration
What to Regenerate:

Selective Regeneration: Deciding whether to regenerate the entire model or only specific parts is crucial. For example, users may choose to regenerate just the terrain features while leaving other elements intact to preserve consistency.
Scope and Scale: Determining how extensive the regeneration should be is essential. This could involve minor updates or a complete overhaul of the generated content, depending on the changes required.
How to Regenerate:

Algorithmic Approaches: Various methods can be used for regeneration, including reapplying existing algorithms or introducing new procedural rules. These methods alter or update the existing content based on user inputs or new conditions.
Incremental vs. Complete Regeneration:
Incremental: Gradual updates or changes applied to only parts of the generated content to minimize disruption and maintain consistency.
Complete: Full regeneration from scratch may be necessary for significant changes or when previous results are no longer valid, ensuring a fresh start if needed.
Problems with User Interactions
Consistency: Ensuring that regeneration maintains or enhances the consistency and quality of generated content is vital. This involves avoiding artifacts or inconsistencies that could impact the user experience or the realism of the content.

Feedback Handling: Effectively managing user feedback and requests for regeneration, which can vary in scope and detail, is important. Providing clear feedback mechanisms helps users understand how their inputs affect the generation process.

Performance: Addressing performance implications is crucial, particularly in real-time or interactive applications where frequent updates are necessary. Efficiently handling resource usage and computational demands during regeneration is key to maintaining a smooth experience.

Action Storage
Storage Formats:

JSON: JavaScript Object Notation (JSON) files can be used to store regeneration actions and parameters. JSON is flexible and readable, making it a suitable format for saving and retrieving procedural settings.
Other Formats: Alternatives such as XML, binary files, or custom formats may be used depending on specific needs or systems.
Tracking Changes:

Action History: Maintaining a history of user actions and regeneration events allows users to revert to previous states or track changes over time. This feature supports iterative development and refinement of generated content.
Versioning: Implementing version control for procedural generation settings helps manage different versions of generated content, allowing users to compare and manage variations effectively.
Restoration and Undo:

Undo Mechanisms: Providing options for users to undo recent regeneration actions or revert to previous states enhances flexibility and control. This allows users to correct mistakes or adjust their approach as needed.
Restoration of Defaults: Allowing users to restore default settings or regeneration conditions if custom changes lead to unsatisfactory results ensures that users can revert to a baseline if needed.
Implementation Considerations
Efficiency: Designing algorithms and systems to handle regeneration efficiently is crucial. This involves minimizing computational overhead and ensuring that updates are responsive and timely.

User Experience: Creating intuitive interfaces and feedback mechanisms facilitates smooth user interactions with the regeneration process. Ensuring that users can easily understand and control the regeneration helps improve their overall experience.

Error Handling: Implementing robust error handling and recovery mechanisms addresses potential issues that may arise during regeneration, such as unexpected results or system failures. Effective error management ensures that users can continue working without significant disruptions.

\section{Terrain representation}
\label{sec:state-of-the-art_terrain-representations}
- ...

\subsection{2.5D terrains}
- ...

\subsubsection{Height maps}
- ...

\subsubsection{Height functions}
- ...

\subsection{3D terrains}
- Need for 3D concepts \\
** Geological information \\
** Volumetric data \\
- ...

\subsubsection{Main issues}
- Memory \\
- Visualization \\
- Modifications \\
- Conversion between representations \\
** Information loss \\
*** Error propagation on geometry (approximations on normals, Z resolution, surface, etc.) \\
*** Loss of subsurface information \\
- ...

\subsubsection{Types, definitions, advantages, disadvantages}
- Voxel grids \\
- Material stacks \\
- Meshes \\
- Implicit surfaces \\
- Implicit materials \\
- ...

\subsection{Other models}
- Concept of semantics \\
- ...

\subsection{Underwater landscapes}
- 3D Data: \\
** Coral Landscapes: Addressing challenges in modeling coral reefs and underwater features. \\
** Cavities: Representing underwater caves and karst formations. \\
- Interdisciplinary Data: \\
** Geological Validation: Integrating expert knowledge for accurate modeling. \\
** Challenges: Fewer experts and more uncertainties in underwater environments. \\
** Data Scarcity: Limited data availability for detailed underwater landscapes. \\
- Need for Multi-Scale: \\
** Integration of large and small elements. \\
** Level of Detail (LOD): Techniques for managing detail across different scales.

\subsubsection{3D Data}
- Coral Landscapes: \\
** Complex Structures: Coral reefs feature complex 3D structures with intricate patterns including branching corals, coral mounds, and encrusting forms. These structures often have a high degree of variability and detail. \\
** Void Spaces: Coral reefs contain many voids and cavities such as caves, grottos, and karst networks, adding to the complexity of modeling these environments. \\
** Challenges: Modeling coral reefs requires accurately representing the porous and irregular nature of coral formations, which can be challenging due to the high level of detail and variability. \\
- Geological Features: \\
** Sedimentary Layers: Representing sedimentary layers and underwater geological formations such as seamounts and underwater ridges. \\
** Volcanic Activity: Incorporating features such as underwater volcanoes and hydrothermal vents, which affect the landscape and ecosystem. \\
- Measurement and Data Collection: \\
** Sonar and LIDAR: Using sonar (e.g., multi-beam echo sounders) and LIDAR (Light Detection and Ranging) to collect detailed 3D data of underwater terrain. \\
** Remote Sensing: Utilizing remote sensing technologies to map and model underwater landscapes, especially in areas that are difficult to access.

\subsubsection{Interdisciplinary Data}
- Geological Validation: \\
** Expert Consultation: Collaborating with geologists and marine scientists to validate and refine models based on real-world observations and data. \\
** Data Accuracy: Ensuring that the generated models accurately reflect real underwater geological and biological features. \\
- Biological Data: \\
** Coral and Marine Life: Integrating data on coral species, marine biodiversity, and ecosystem dynamics to create realistic and biologically accurate representations. \\
** Ecological Impact: Considering the impact of various biological factors on the terrain, such as coral growth patterns and marine erosion. \\
- Challenges: \\
** Data Scarcity: Limited availability of high-resolution data for some underwater environments, especially in less studied or remote areas. \\
** Integration: Combining geological, biological, and hydrological data effectively to create comprehensive and accurate models. 

\subsubsection{Multi-Scale Modeling}
- Large vs. Small Scale Elements: \\
** Macro Features: Incorporating large-scale features such as underwater mountains, ridges, and large coral formations. \\
** Micro Features: Modeling small-scale elements such as individual coral polyps, marine vegetation, and detailed sedimentary textures. \\
- Level of Detail (LOD): \\
** Adaptive LOD: Using techniques to adjust the level of detail based on user interaction or view distance to balance performance and visual fidelity. \\
** Detail Preservation: Ensuring that both macro and micro-scale features are accurately represented without losing important details during scaling or zooming. \\
- Visualization Techniques: \\
** Hydrographic Mapping: Employing specialized visualization techniques to represent underwater features clearly and accurately. \\
** Texture and Lighting: Using textures and lighting models that simulate underwater conditions, such as light absorption and scattering in water.

% - 3D Data \\
% ** Coral landscapes filled with voids \\
% ** Many cavities (caves, grottos, karst networks) \\
% - Interdisciplinary Data \\
% ** Rather common with terrain generation \\
% ** => Geological validation with experts \\
% ** For underwater, \\
% *** Fewer experts, \\
% *** More uncertainties \\
% *** Based more on observations \\
% *** Few data (coral landscapes < 0.1\% of oceans), for significant biological impact (25\% marine biodiversity) \\
% ** Mix of geology, biology, hydrology, and physics (especially fluid dynamics) \\
% - Need for multi-scale \\
% ** Not limited to underwater \\
% ** Integrate large elements (mountains) with small elements (vegetation) \\
% ** LOD \\
% - ...

\subsection{Fluid simulations}
- ... 
\subsubsection{Introduction to Fluid Simulations}
- Definition and Purpose: \\
** Overview: Explanation of fluid simulations and their role in representing natural phenomena in terrains, including water flow, erosion, and sediment transport. \\
** Importance: Impact of accurate fluid simulations on creating realistic terrain and environmental models.\\
\subsubsection{Types of Fluid Simulations}
- 2D Fluid Simulations: \\
** Particle-In-Cell (PIC): \\
*** Concept: Combines particles with a grid to simulate fluid dynamics. \\
*** Applications: Used for simpler simulations and visualizations. \\
** Fluid-Implicit Particle (FLIP): \\
*** Concept: A hybrid method that combines particle and grid approaches for better accuracy and efficiency. \\
*** Applications: Suitable for capturing complex fluid behaviors in 2D environments. \\
** Stable Fluids: \\
*** Concept: A grid-based method for simulating stable, incompressible fluids with less computational complexity. \\
*** Applications: Often used in interactive applications where real-time performance is crucial.\\
- 3D Fluid Simulations: \\
** Smoothed Particle Hydrodynamics (SPH): \\
*** Concept: A particle-based method where fluid properties are represented by particles that interact based on smoothing kernels. \\
*** Applications: Useful for highly detailed simulations of fluids, including interactions with terrain. \\
** Grid-Based Methods: \\
*** Concept: Methods like Marker-And-Cell (MAC) and Navier-Stokes equations applied on a grid to simulate fluid behavior. \\
*** Applications: Used for detailed and accurate 3D simulations, including large-scale environments. \\
** Hybrid Methods: \\
*** Concept: Combining grid and particle approaches to leverage the strengths of both methods.\\
*** Applications: Balancing detail and performance in complex simulations. 

\subsubsection{Applications in Terrain Representation}
- Erosion and Sediment Transport: \\
** Simulation of Erosion: How fluid simulations model erosion processes affecting terrain features such as riverbeds and coastlines. \\
** Sediment Movement: Modeling the transport and deposition of sediment, contributing to realistic terrain evolution. \\
- Water Flow and Hydrology: \\
** Surface Water Dynamics: Representing the flow of water across terrain surfaces, including rivers, lakes, and wetlands. \\
** Subsurface Flow: Simulating groundwater movement and interactions with terrain features. \\
- Interactive Environments: \\
** Real-Time Simulations: Implementing fluid simulations in interactive applications, such as video games and virtual environments, where dynamic water interactions are crucial. 

\subsubsection{Challenges and Considerations}
- Computational Resources: \\
** Performance Trade-Offs: Balancing simulation accuracy with computational efficiency, especially for real-time applications. \\
** Hardware Requirements: The need for powerful processors and memory to handle complex 3D fluid simulations. \\
- Accuracy vs. Realism: \\
** Detail vs. Performance: Finding the right balance between detailed fluid dynamics and the practical limitations of simulation resources. \\
** Simulation Artifacts: Addressing potential artifacts or inaccuracies in simulations that can impact realism. \\
- Integration with Terrain Models: \\
** Interaction with Terrain: Ensuring fluid simulations integrate seamlessly with terrain models, including handling interactions such as fluid erosion or deposition. \\
** Data Consistency: Maintaining consistency between simulated fluid behaviors and terrain features for accurate representations. 

\subsubsection{Recent Developments and Future Trends}
- Advancements in Algorithms: \\
** New Techniques: Emerging algorithms and methods that enhance the accuracy and efficiency of fluid simulations. \\
** Real-Time Improvements: Innovations aimed at improving real-time performance and interactivity in fluid simulations. \\
- Integration with AI and Machine Learning: \\
** AI-Enhanced Simulations: Using machine learning to improve fluid simulation accuracy and adaptivity. \\
** Predictive Models: Leveraging AI to predict and simulate complex fluid behaviors based on historical data. 

% - Very important in procedural terrain generation \\
% - Allows justifying the geophysics of a simulation/generation \\
% - Quite fast solutions in 2D (PIC, FLIP, Stable Fluids, SPH, etc.) \\
% - But becomes much heavier and memory-intensive in 3D \\
% - ...

\section{Coral reefs (biological aspects)}
\label{sec:state-of-the-art_biology}
- Historical Discovery: Evolution of knowledge about coral reefs. \\
- Types of Coral Reefs: \\
** Islands, Barriers, Atolls: Different forms and structures of coral reefs. \\
- Atoll Theories: Historical and scientific theories explaining the formation of atolls. \\
- Importance in Biodiversity: Coral reefs as critical ecosystems with high marine biodiversity. \\
- Threats and Protection: Current threats to coral reefs and conservation efforts.

\subsection{Historical Discovery of Coral Reefs}
- ... 
\subsubsection{Early Observations}
- Ancient Knowledge: Initial observations by ancient civilizations (e.g., Greeks, Romans) and their interpretations of coral structures. \\
- Exploration Era: The role of early explorers and naturalists (e.g., Captain James Cook, Alexander von Humboldt) in documenting coral reefs and their biodiversity.
\subsubsection{Scientific Discovery}
- 19th Century Advances: Key contributions from scientists such as Charles Darwin, who developed theories on coral reef formation. \\
- 20th Century Developments: Advances in marine biology and oceanography that improved understanding of coral reef ecosystems.

\subsection{Coral Reef Structure and Formation}
- ... 
\subsubsection{Coral Anatomy}
- Coral Polyps: Basic building blocks of coral reefs, their structure, and function. Each polyp is a tiny, soft-bodied organism that secretes calcium carbonate to form the reef structure. \\
- Coral Colonies: How individual polyps form colonies and contribute to the growth of the reef structure. 
\subsubsection{Types of Coral Reefs}
- Fringing Reefs: \\
** Definition: Reefs that are directly attached to a coastline, extending out from the shore. \\
** Characteristics: Shallow waters, often with a narrow reef crest and slope. \\
- Barrier Reefs: \\
** Definition: Reefs that run parallel to the coastline but are separated by a lagoon or deep channel. \\
** Characteristics: Typically found farther from the shore, with a more pronounced lagoon and deeper water. \\
- Atolls: \\
** Definition: Circular or oval reefs that encircle a lagoon, often formed around a submerged volcanic island. \\
** Characteristics: Reefs form a ring around a central lagoon, with no land in the center. 
\subsubsection{Reef Building Processes}
- Calcium Carbonate Deposition: The process by which corals and other organisms secrete calcium carbonate to build the reef structure. \\
- Bioerosion: The natural process by which reef structures are eroded by marine organisms such as parrotfish and sea urchins.

\subsection{Importance in Biodiversity}
- ... 
\subsubsection{Species Diversity}
- Marine Life: Coral reefs are among the most diverse ecosystems in the world, supporting thousands of marine species including fish, invertebrates, and algae. \\
- Endemism: Many species are found exclusively in coral reef environments, contributing to global biodiversity.
\subsubsection{Ecosystem Services}
- Habitat Provision: Coral reefs provide essential habitats for a wide range of marine species, including commercially important fish and invertebrates. \\
- Nutrient Cycling: Reefs play a critical role in nutrient cycling, supporting the productivity of marine ecosystems. 
\subsubsection{Economic and Cultural Value}
- Fishing and Tourism: Coral reefs support important fisheries and generate significant revenue through tourism and recreational activities. \\
- Cultural Significance: Many coastal communities have cultural and spiritual connections to coral reefs, incorporating them into traditions and practices.

\subsection{Threats to coral reefs}
- ... 
\subsubsection{Climate Change}
- Coral Bleaching: Caused by elevated sea temperatures, leading to the expulsion of symbiotic algae (zooxanthellae) and subsequent coral bleaching. \\
- Ocean Acidification: Reduced calcification rates due to increased \ch{CO2} levels, impacting coral growth and reef structure.
\subsubsection{Pollution}
- Nutrient Runoff: Increased nutrient levels from agricultural and urban runoff can lead to algal blooms that smother corals and disrupt reef balance. \\
- Marine Debris: Pollution from plastics and other debris that can damage coral reefs and harm marine life.
\subsubsection{Overfishing}
- Depletion of Fish Stocks: Overfishing can lead to the loss of key reef species and disrupt ecological balance. \\
- Destructive Fishing Practices: Practices such as blast fishing and cyanide fishing cause direct physical damage to reefs and harm marine biodiversity. 
\subsubsection{Coastal Development}
- Habitat Destruction: Development activities such as dredging, land reclamation, and construction can destroy or degrade coral reef habitats. \\
- Sedimentation: Increased sedimentation from coastal construction and deforestation can smother corals and reduce light availability.

\subsection{Conservation efforts}
- ... 
\subsubsection{Marine Protected Areas (MPAs)}
- Designated Zones: Areas where human activities are regulated or restricted to protect coral reefs and promote recovery. \\
- Success Stories: Examples of successful MPA implementations and their positive impacts on reef health and biodiversity.
\subsubsection{Restoration Projects}
- Coral Gardening: Techniques for growing and replanting corals to restore damaged reef areas. \\
- Artificial Reefs: Creation of artificial structures to provide new habitats for marine life and promote reef recovery.
\subsubsection{Community Involvement}
- Local Engagement: Involving local communities in reef conservation through education, stewardship, and sustainable practices. \\
- Citizen Science: Encouraging public participation in monitoring and research efforts to support reef conservation. 
\subsubsection{Policy and Legislation}
- International Agreements: Global and regional agreements aimed at protecting coral reefs and addressing climate change impacts (e.g., the Convention on Biological Diversity). \\
- National Policies: Policies and regulations at the national level to safeguard coral reefs and manage marine resources sustainably. 

\subsection{Future Research and Monitoring}
- ... 
\subsubsection{Innovative Technologies}
- Remote Sensing: Use of satellite imagery and drones to monitor reef health and track changes over time. \\
- Genomics: Applying genetic research to understand coral resilience and adaptation to environmental stressors. 
\subsubsection{Adaptive Management}
- Dynamic Strategies: Developing flexible management approaches that can adapt to changing conditions and emerging threats. \\
- Collaboration: Promoting collaboration among scientists, policymakers, and local communities to address complex challenges facing coral reefs.
\subsubsection{Education and Awareness}
- Public Outreach: Raising awareness about the importance of coral reefs and the actions individuals can take to support conservation efforts. \\
- Training Programs: Providing training for scientists, managers, and local stakeholders to enhance reef management and restoration practices.
