% \chapter{State of the art}
% \label{chap:state-of-the-art}
% \minitoc
\resetgraphicspath
\appendtographicspath{{./Introduction/figures}}
\appendtographicspath{{./Chapter 1/figures/} }
\appendtographicspath{{./Chapter 1/islands} }
\appendtographicspath{ {./Chapter 3/figures/erosion/}{./Chapter 3/figures/terrain_representations/}{./Chapter 3/results/}{./Chapter 3/otherPapersRepro/} {./Chapter 3/images/erosion-processes/}}

\chapter{Background}
\label{chap:background}
\minitoc

\section{Procedural terrain generation}
\label{sec:state-of-the-art_procedural-generation}

Procedural generation refers to algorithmically producing content rather than authoring it manually, leveraging rules, mathematical models, or data-driven methods. Its primary benefits include reducing storage needs (since content is generated on-the-fly), enabling rapid creation of large or varied content, and supporting both reproducibility and controlled variability. In terrain contexts, these advantages allow creation of vast landscapes or seascapes with relatively small parameter sets and facilitate interactive pipelines where artists or researchers can iterate quickly without storing every variant.

\AltTextImage{
    The conceptual roots of procedural techniques in graphics trace back to fractal geometry and noise functions. Mandelbrot's work on fractals demonstrated how complex, self-similar structures can arise from simple rules and randomness, influencing landscape modeling approaches that mimic natural roughness and multi-scale detail \cite{Mandelbrot1983}. Perlin's introduction of gradient noise provided a practical mechanism to generate coherent pseudo-random patterns for textures and terrain features \cite{Perlin1985}. These foundational ideas underpin many terrain algorithms, where noise at multiple scales produces hills, valleys, and finer details.
}{noises.pdf}{Example of procedural noises}{fig:intro-noises}

\AltTextImage{
    Procedural paradigms are often categorized as deterministic (repeatable given fixed inputs), stochastic (introducing randomness for variation), or hybrid (structure defined by deterministic rules, with randomness filling details). Rule-based systems, where content emerges via iterative application of predefined rules or grammars, also play a role, particularly in generating structured features (e.g., river networks, vegetation patterns). In terrain generation, deterministic components ensure global coherence (e.g., major landmass shapes), while stochastic elements add natural variability (e.g., subtle elevation perturbations). Later chapters leverage these paradigms: for instance, \cref{chap:coral-island}'s sketch-based island shape combined with noise and data-driven augmentation yields varied yet controlled outputs, and \cref{chap:semantic-representation}'s ecosystem simulation is governed by rule-based systems with stochastic sampling.
}{Snowflake5-Siertriangle11.jpg}{Fractal geometry: Sierpinsky Triangle in Koch Snowflack, from Larry Riddle}{fig:intro-fractal}

\AltTextImage{
    Recent advances incorporate machine learning to learn distributions of terrain patches or entire landscapes from data, enabling synthesis of realistic terrain patterns beyond hand-tuned noise parameters, using especially generative models like GANs or autoencoders. \cref{chap:coral-island} uses a conditional GAN to diversify coral island heightmaps. However, a detailed description of learning-based terrain methods appears in that chapter's State-of-the-Art.
}{Graph-coloring-Kim2020.png}{Rule-based graph coloring from \cite{Kim2020}}{fig:intro-rule-based-Kim2020}

In interactive and production pipelines, procedural systems must balance realism, speed, and user control. Fast evaluation (often parallelized and/or using GPU) supports real-time or near-interactive feedback for artist-driven editing. Parameter design and UI tools (e.g., sketching, semantic objects, masking) are critical to let users influence generation without overwhelming them. \cref{chap:semantic-representation}'s semantic \glosses{EnvObj} framework exemplifies user-centric design, enabling multi-scale editing and expert knowledge integration.

\AltTextImage{
    Procedural terrain generation also encompasses physical simulation techniques such as erosion and sediment transport models, that refine initial shapes to enhance plausibility. These simulations are typically more expensive but can run offline or via parallel algorithms. \cref{chap:erosion} presents a particle-based erosion method applicable across representations (height fields, voxels, etc.), reflecting this integration of procedural and physical approaches.
}{CFD-Sphynx.png}{Fluid simulation over the Great Sphynx}{fig:intro-cfd}






\section{Terrain representations}
Terrain refers to the physical features and configuration of a specific area of land. It includes the elevation, slope, and the overall topography, such as mountains, valleys, and plains. Terrain is often used to describe the surface characteristics of the land, focusing on the natural contours and the geographical aspects that define a region's physical form.

While the term "terrain" describes the physical characteristics of land, it does not include the natural elements that shape an area's identity. Elements like vegetation, water bodies, and climatic conditions, such as snow cover, are essential to how we perceive and understand a landscape. Therefore, when discussing procedural generation in virtual environments, "landscape generation" is a more fitting term, as it integrates these natural elements along with the topographical features.

In addition to "terrain generation," other terms such as "landscape generation," "world generation," and "environment generation" can be used to describe the creation of virtual landscapes. These terms are interchangeable and can all refer to the process of generating physical terrain along with natural and artificial elements. However, by convention and for simplicity, the term "terrain generation" is most commonly used in the field. Despite its original focus on the physical features of the land, "terrain generation" has evolved to encompass a broader range of environmental elements, making it a convenient and widely accepted term for describing the comprehensive process of creating virtual environments.

A terrain can be represented in various ways, each of them suited for a given application. In the different chapters of this thesis, we will move from one representation to another depending on the scope of the work, so we will provide a short description of the representations used here. More details can be found in \cite{Galin2019}.

\subsection{Elevation models}

\begin{figure}[H]
    \centering
    \includegraphics[width = 0.8 \linewidth]{elevation_representation.png}
    \caption{Elevation functions}
    \label{fig:erosion-elevation-representation}
\end{figure}

Elevation models are a fundamental approach in terrain representation, widely used in procedural generation due to their simplicity and efficiency. These models define the terrain as a function $h : \R^2 \to \R$, where each point in a 2D plane is mapped to an elevation value. This approach is particularly effective for representing terrains where the elevation is the only varying factor, such as hills, valleys, and plateaus, and it is best suited for terrains without complex 3D features like overhangs or caves. While we visualize elevation models in three dimensions, they are mathematically considered two-dimensional functions. In the domain of terrain generation, we will name them 2.5D models and are closely related to image processing.

Elevation models are widely used in industries where large-scale terrain representation is crucial. In video games, they provide the foundation for creating vast open-world environments. In geographic information systems (GIS) and remote sensing, height fields are used to represent real-world terrain data, offering a practical means of visualizing and analyzing geographical features. The ability to manipulate and control terrain features procedurally makes elevation models a common choice for applications that require efficient terrain generation and rendering.

They offer a powerful method for representing terrains in procedural generation, combining simplicity with flexibility. While they have limitations in representing complex 3D structures, their efficiency and compatibility with existing algorithms make them indispensable in a variety of applications.

\subsubsection{Implicit height fields}

\begin{wrapfigure}{L}{0.4\textwidth}
    \includegraphics[width=\linewidth]{primitives_representation.png}
    \caption{Primitives composition}
    \label{fig:erosion-primitives-representation}
\end{wrapfigure}

Implicit height fields represent the terrain as a mathematical function that provides a height value at any given point in the domain. These functions can be procedural or closed-form expressions, allowing for compact storage and infinite precision in theory. The elevation function allows for easy manipulation of terrain features, making it ideal for generating terrains that require smooth, continuous surfaces. However, the primary disadvantage is the computational complexity involved in evaluating the function, especially for large or highly detailed terrains. The challenge lies in constructing functions that can realistically represent large-scale terrains with complex landforms.

This representation, inherently multi-scale from their mathematical nature, will be the main support for our \cref{chap:semantic-representation}'s environment generations.

\subsubsection{Discrete height fields}
Discrete height fields, or explicit height fields, are one of the most prevalent methods for terrain representation. These models consist of a 2D grid where each cell contains a height value, representing the elevation at that point. Height fields are particularly advantageous because they are simple to implement and are directly compatible with many rendering techniques and hardware, but also due to their closeness with image processing, a domain studied for many decades now. As the deep learning architectures used in \cref{chap:coral-island} are designed for image generation, the first contribution of this manuscript uses exclusively this representation.

The main advantage of height fields is their ability to handle large datasets efficiently, providing a balance between memory usage and detail. However, they are limited by their inability to represent terrains with overhangs or caves, as each point on the grid can only hold a single elevation value. Additionally, height fields often require interpolation methods, such as bi-linear or bi-cubic interpolation, to reconstruct a continuous surface from the discrete grid points. 

\subsection{Volumetric models}

\begin{figure}[H]
    \centering
    \includegraphics[width = 0.8 \linewidth]{volumetric_representation.png}
    \caption{Volumetric functions}
    \label{fig:erosion-volume-representation}
\end{figure}

Volumetric models represent a more complex approach to terrain modeling, allowing for the depiction of 3D features that go beyond the simple surface-based representation provided by elevation models. These models capture not only the surface of the terrain but also its internal structure, making them ideal for representing terrains with overhangs, caves, and other subsurface features. 

Volumetric models, including layered materials, voxel grids, and implicit models, are essential in applications where terrain complexity and detail are primordial. In geological simulations, these models allow for accurate representation of subsurface structures and processes. Voxel models are widely used in games that require dynamic terrain deformation, providing a rich interactive environment for players. Implicit models are favored in situations where smooth, continuous surfaces are needed [FIND OTHER USE CASES].

The erosion simulation presented in \cref{chap:erosion} evolves in the 3D space and will present applications for discrete and implicit volumetric representations.

\subsubsection{Implicit volumetric models}
Implicit volumetric models describe the terrain's shape and features using an implicit function. The terrain is represented by a mathematical function $f: \R^3 \to \R$ that determines the terrain surface by evaluating to an isovalue, often zero. This function provides a continuous representation of the terrain, with points inside the terrain returning positive values and while points in the air evaluate to negative values. It allows for the seamless representation of complex terrain features, including caves, overhangs, and varying geological structures, which are impossible to represent with  elevation models.

One of the key advantages of implicit models is their ability to produce smooth surfaces without the need for discrete polygonal meshes, which can result in realistic and natural-looking terrains. However, the computational complexity of evaluating the implicit function, especially for large terrains, can be a significant drawback. Additionally, converting an implicit surface into a mesh for rendering can be challenging and resource-intensive \cite{Araujo2015}.

\subsubsection{Layered models}
Layered models are a type of volumetric representation that encode different material layers within the terrain and are defined by a function $\mu : \R^3 \to \material$, where $\material$ denotes the material type at any given point in 3D space. This allows for a detailed representation of the terrain's internal composition, which can be crucial for applications requiring realistic geological simulations. Each layer is defined by its thickness or elevation, and multiple layers can be stacked to represent complex geological formations. These layers might include materials like bedrock, sand, soil, or water, each contributing to the overall structure of the terrain. Layered models are particularly useful in simulations that involve processes like erosion or sedimentation, where the interaction between different material layers affects the physical process.

The primary advantage of layered models is their ability to represent a stratified terrain with distinct material properties, which can be manipulated individually. This makes them well-suited for simulations that require detailed geological accuracy. However, they are more complex to implement than simple elevation models and require additional computational resources to manage the interactions between layers. 


\subsubsection{Voxel grid models}
Voxel grids are a common method for representing 3D terrains in procedural generation, offering the ability to capture complex internal structures and features that are difficult or impossible to represent with surface-based models. In a voxel grid, the 3D space is divided into a regular grid of small, cube-shaped elements called voxels (volumetric pixels). Each voxel holds information about the material or properties of the terrain at that specific point in space. This approach allows for detailed modeling of features such as caves, tunnels, overhangs, and intricate underground networks. The regular grid structure allows for the use of image processing-oriented algorithms.

There are three primary types of voxel grids used in terrain representation: binary voxel grids, material voxel grids and density voxel grids. Each has distinct characteristics, advantages, and limitations, making them suitable for different applications. 

\subsubsubsection{Binary voxel grids}
Binary voxel grids are the simplest form of voxel representation. In these grids, defined $f: \R^3 \to [0, 1]$, each voxel is either "filled" or "empty," representing the presence or absence of material. This binary state is typically represented by a 1 (filled) or 0 (empty). Binary voxel grids are straightforward to implement and require much less memory compared to more complex voxel representations, making them ideal for applications where the primary concern is whether a space is occupied or not.

The simplicity of binary voxel grids is one of their main advantages. They are easy to understand and visualize, with each voxel requiring only a single bit of information to represent its state. Additionally, because only a binary state is stored, these grids can be memory-efficient when combined with compression techniques like Sparse Voxel Octrees (SVOs) \cite{Laine2010} or voxel Directed Acyclic Graphs (DAG) \cite{Villanueva2017,Careil2020}. The simplicity of the data structure also allows for quick processing, making binary voxel grids suitable for real-time applications where performance is required. However, the binary nature of these grids limits their ability to represent variations in material density or properties, or even smoothness, resulting in less detailed terrain models. This can lead to hard, blocky edges in the terrain, which may appear unnatural without additional smoothing or processing.

\subsubsubsection{Material voxel grids}
Material voxel grids, defined as $\mu: \R^3 \to \material$, are commonly used in applications where simple occupancy information is sufficient. For example, voxel-based games like \citeProgram{Minecraft2011} utilize material grids to create terrains composed of solid blocks with clear boundaries. These grids are also employed in scientific simulations where the primary concern is the presence or absence of materials, rather than detailed material properties.

\subsubsubsection{Density voxel grids}
Finally, density voxel grids allow each voxel to store a range of values, representing varying degrees of material presence with $f: \R^3 \to \R$. Instead of a simple discrete state, a density voxel grid assigns a continuous value to each voxel, which can represent material density, opacity, or other properties. This added complexity enables density voxel grids to represent subtle variations in terrain, such as gradual changes in material density or smooth transitions between solid and empty spaces, allowing for more realistic and natural-looking terrain models.

% The use of density voxel grids results in soft transitions and smooth surfaces, reducing the blockiness typically associated with binary voxel grids. They are versatile and can represent not only solid terrain but also phenomena like fog, fluid densities, or temperature gradients. However, the increased detail and realism come at the cost of greater complexity. Density voxel grids require more memory and computational power, making them more challenging to implement and manage. The additional data and processing required can also lead to slower performance, particularly in real-time applications.

Density voxel grids are often used in high-fidelity simulations where detail and realism are essential. They are found in applications such as medical imaging, scientific visualizations, and advanced terrain modeling for films and visual effects. These grids are also employed in procedural terrain generation systems that require smooth and natural transitions between different terrain features, such as caves, cliffs, and eroded landscapes.




\section{Interactivity}
% \subsection{Realism-speed-control balance}
Procedural algorithms are designed to find a balance between three main objectives: controllability, realism, and computation speed.
In our work, we positioned our focus on the inclusion of the user in the generation process, resulting in a need to focus our attention on controllability for authoring without frustration, and computation speed in order to visualize results quickly, allowing to interatively interact with the system to obtain the best possible outputs. In this section we will briefly present the tools available for interacting with a procedural system.

Procedural terrain generation algorithms usually take parameters into account in order to generate something that fits the user's needs. The tools that let the user interact with the different parameters are essential to include efficiently the user in the process. 

The parameters that a user can play with can have many nature, and require a large varety of visual tools to interact with them. In the different nature of parameters, we can typically find: 
\begin{Itemize}
    \Item{Noise parameters:} Abstract values used in the equation ruling the noise functions. 
    \Item{Physical constants:} Constants that are used in physic simulations. They are often tweaked by the user to exagerate certain features, force some phenomena, etc... As they are constants, the parametrization of these values is often done inside the source code, as spinboxes, or inside a configuration file. 
    \Item{Densities and distributions:} Marking areas that are affected by a process, or where a seed can be used is useful to control a procedural algorithm. This is refered as "masking", but due to its binary nature, we can find sharp and unatural transitions from the use of binary masks. We often use weighted masks to provide more control about probabilities that a seed appear at one point. The control of the masks is usually done through grayscale images or by a "brush tool" that can modify the value directly on the surface of a 2D or 3D object. As virtual terrains are generally represented by a rectangular base, the use of brushes is a facade to the manipulation of a grayscale image.
    \Item{Structures placement:} Placing a specific terrain feature inside a terrain may be useful to finalize the generation of a landscape, or to add obstacles that will affect a physic simulation. This is usually implemented by clicking on the surface of the terrain, to provide the initial $xyz$ position, and then use translation and rotation tools to adjust the placement. If many elements are added, we usually use the previous strategy, involving masking, to add all the structures.
    \Item{Manual manipulation of the terrain:} In order to customize at a lower level the surface of the terrain, the user may be able to alter the surface of the terrain by using brushes. Clicking or dragging a brush on the surface usually result in the rise or lowering of the surface level around the brush. This process modifies the field representing the terrain. Procedural brushes can imply more complex behaviours, and may use the velocity of the brush stroke in its input.
\end{Itemize}

Each developer uses different strategies for each type of parameters as the needs or objectives of each application is different. Providing the user with too many parameters can be overwhelming, while too few can limit the output possibilities. In the meantime, providing unintuitive parameters like dimensionless values often result in trial and error strategy, requiering to run the generation algorithm many times before finding the appropriate values. This implies that the algorithms must be able to be executed fast. On an opposite side, removing some user controls to use hard-coded constants instead can greatly increase the speed. Finally, an algorithm that aims to be fast or let the user many parameters to control may reduce the realism of the output. Most algorithms try to strike a balance between realism, speed and control.

% User interaction affects how users balance realism, speed, and control over generated content.

Realism refers to the extent to which generated content accurately represents real-world characteristics, such as visual details, physical processes, and natural patterns. This is especially important in simulations, visualizations, and training environments where plausibility [FIND BETTER TERM] is critical. Techniques to enhance realism include physical simulations, which model processes like erosion and sediment transport, and the integration of expert knowledge from fields such as geology and ecology. However, achieving realism can be challenging due to the complexity of detailed simulations and the need for specialized expertise, which can demand substantial computational resources and inputs. Usually, a realistic algorithm tends to have low user control and speed.

On another hand, speed is about the efficiency of content generation within acceptable time constraints. While no official categorisation has been set, James Gain proposed to describe levels of speed based on response time:
\begin{Itemize}
    \Item{Real-time:} generation in less than 30 milliseconds, essential for interactive applications like VR environments. 
    \Item{Interactive:} generation in under 3 seconds, suitable for user-driven customization in games and simulations. 
    \Item{Near-interactive:} generation in less than 5 minutes, applicable for larger-scale simulations where some delay is acceptable. 
    \Item{Offline:} generation that takes longer, often used for precomputed content or offline rendering.
\end{Itemize}

The execution speed of a procedural algorithm is crucial as the user will fine-tune the parameters before being satisfied.
Optimizing speed involves using efficient algorithms and parallel processing. Almost all recent works achieve fast execution time thanks to high parallelisation on GPU. Other types of algorithm may rely on a refinement paradigm, generating a coarse result at first and then iteratively adding finer details, such that the global output shape is available long time before the real final result.

Control refers to how much users can influence or direct the procedural generation process to meet their specific needs or preferences. This includes parameters fine-tuning, allowing users to modify parameters like the noise functions parameters or the simulation parameters, and artistic control tools, made for artists and designers, to guide the generation process with the use of masks and brushes, allowing for combination and mixing of different algorithms.

Managing diverse and sometimes conflicting user expectations, such as balancing creative freedom with the demand for realistic outcomes, requires careful consideration. Additionally, it is essential to balance the complexity of procedural systems to ensure they remain user-friendly, avoiding overwhelming users or producing results that feel artificial or inconsistent.




\section{Underwater landscapes}
Underwater landscapes, or "seascapes", displays large differences with landscapes above water level, we will call "aerial landscapes". 
In the field of terrain generation, particularly when it comes to modeling underwater landscapes, the challenge is to accurately represent complex environments that include both geological and biological features. From the material composition differences and the water density in comparison with air density, stems multiple challenges that can be ignored in aerial landscape generation such as the lower importance of gravity in opposition with the importance of water currents, the difficulties to observe continuously from a large-scale to a small-scale data, and the comprehension of physics and biology in these dynamic ecosystems. 

\subsection{3D Data for underwater landscapes}

One of the most challenging aspects of underwater landscape modeling is capturing the complexity of its structure. Taking the example of coral reefs,these environments feature highly intricate 3D structures, including branching corals, coral mounds, and encrusting forms. These formations are not just surface-level features; they extend into the terrain with voids and cavities, such as caves, grottos, and karst networks, which add layers of complexity to their representation. Accurate modeling of coral reefs demands that the porous and irregular nature of these formations is represented in high detail, which is a significant challenge due to the variability of their structures.

In addition to biological features, underwater landscapes are shaped by geological processes that must also be accurately represented. For example, sedimentary layers and underwater geological formations like seamounts and ridges are important components of the terrain. 

To gather the data needed for such detailed modeling, advanced measurement techniques are employed. Sonar, including multi-beam echo sounders, and LIDAR (Light Detection and Ranging) are commonly used to capture detailed 3D data of underwater terrains. These technologies allow for the mapping of areas that are otherwise difficult to access, providing the foundational data necessary for accurate terrain generation. Remote sensing technologies also play a major role, enabling the collection of data over large and remote areas, further enhancing the model's accuracy, but represent a significant challenge as the approvision of humans and hardware in these areas is far from easy to achieve.

\subsection{Interdisciplinary data integration}

Accurately modeling underwater landscapes requires more than just geological data; it necessitates an interdisciplinary approach that integrates biological and ecological data. Geological validation is an important step in this process, involving collaboration with geologists and marine scientists to ensure that the models are not only scientifically accurate but also reflective of real-world conditions. This validation process ensures that the generated models accurately depict the geological and biological features present in underwater environments.

Biological data is also essential, particularly when modeling coral reefs and other marine ecosystems. Incorporating data on coral species, marine biodiversity, and ecosystem dynamics allows for the creation of realistic and biologically accurate representations. Additionally, the impact of biological processes, such as coral growth patterns and marine erosion, must be considered, as these factors significantly influence the terrain shape over time.

However, one of the major challenges in modeling underwater landscapes is the scarcity of high-resolution data, especially in remote, protected or less studied areas. This data scarcity makes it difficult to create detailed and accurate models. Another challenge lies in the integration of diverse data types, geological, biological, hydrological, etc, into a cohesive and comprehensive model. Effective integration requires complex techniques to ensure that all aspects of the underwater environment are accurately represented.

\subsection{Multi-scale modeling}

Underwater landscapes are characterized by features that vary greatly in scale, from vast underwater mountains and ridges to small-scale elements like individual coral polyps and marine vegetation. This diversity in scale necessitates multi-scale modeling techniques that can accommodate both large and small features within the same model.

\citep{ParisThesis} classified the different geological structures in four different spatial scales: the \textit{megascale} describes landforms spread on more than \si{100}{km} such as continents or mountain ranges, the \textit{macroscale} between \si{1}{km} and \si{100}{km} like river or cave networks, the \textit{mesoscale} ranging from \si{10}{m} to \si{1}{km} contains most complex structures (arches, ravines, cliffs, ...) and finally the \textit{microscale} for features between \si{10}{cm} and \si{10}{m}, like sand ripples, ventifacts, rocks, etc. Generally, we consider as \textit{large-scale} features that can be seen from far away and \textit{small-scale} elements for which somebody has to be to close to to observe it.

Large-scale features, such as underwater mountains and ridges, define the macrostructure of the landscape. These must be integrated seamlessly with small-scale features, such as detailed sedimentary textures and small coral formations, to ensure that the terrain is represented accurately across different scales.

[TODO]

The procedural generation of underwater landscapes involves the integration of advanced 3D data collection techniques, interdisciplinary collaboration, and multi-scale modeling approaches. Modeling underwater environments, from the complex structures of coral reefs to the accurate representation of geological and biological features, requires innovative solutions that combine geological plausibility with visual realism.


\section{Coral biology}
\label{sec:state-of-the-art_biology}

Coral reefs, once misunderstood as sea plants or lifeless rocks, are now known to be biologically rich structures formed by colonies of marine invertebrates. Technological advances in the 20th century, including scuba diving and sonar mapping, enabled direct observation of reefs, revealing their complex ecological and geological roles. Today, tools like remote sensing and genetic analysis continue to enhance our understanding of their biodiversity and spatial patterns.

This section summarizes key biological and ecological features of corals, focusing on elements relevant to reef morphology and terrain modeling.


\subsection{Coral polyps and colonies}
Coral polyps are tiny, soft-bodied marine invertebrates that secrete calcium carbonate (\ch{CaCO_3}) to build an external exoskeleton. Over time, successive secretion by many polyps produces the rigid framework of a reef. Symbiosis with zooxanthellae (photosynthetic algae) supplies most of the energy for calcification; this makes coral growth strongly light-dependent and tied to shallow, clear-water zones. 

From a terrain generation perspective, the \ch{CaCO_3} secretion underlies the elevation of reef structures and light-dependence of the polyps implies depth-based constraints on where reef heights appears, relatively to water level. Moreover, coral-built substrate produced by accumulated skeleton, or "dead coral", increase landscape resistance to erosion.

% \subsection{Coral colony morphologies and zonation}
Coral colonies adopt varied growth forms adapted to local conditions (light, water motion, depth). The main shapes are listed below :

\begin{Itemize}
    \AltTextImage{
        \Item{Massive:} These corals form large boulder-like shapes that can reach several meters in diameter under calm conditions. Found on reef slopes and back-reefs in low to moderate energy, up to ~30 m depth, tolerating moderate to low light, their slow growth persists over centuries. They are foundational reef-builders of the reef, providing long-term structural stability and substrate for other organisms to settle, while hosting diverse fishes in its crevices. Their bulk buffers wave energy, alters flow to form sediment-depositing lee zones, stabilizes the substrate, and produces oxygen over large surfaces.
    }{graphics-oceanus_corals-_MG_9848_LuisLamar.jpg}{Large \textit{Porite labota} - Photo credit: Luis Lamar}{fig:intro-Porite}

    \AltTextImage{
        \Item{Branching:} These corals exhibit tree-like branches with lengths often exceeding 1-2 m. They occupy shallow reef surface, up to ~20m deep in high-light, high-energy environments. Fast-growing, they rapidly deposit carbonate, create 3D nurseries for juvenile fish, and support biodiversity. Their high rugosity increases turbulence and drag, altering currents, trapping sediment, and producing oxygen. Fragile branches break in storms, aiding dispersal via fragmentation.
    }{staghorn-coral-large.jpg}{A field of staghorn coral - Photo credit: NOAA}{fig:intro-Staghorn}

    \AltTextImage{
        \Item{Foliose:} These corals form leaf-like laminae often spanning over a meter wide. They inhabit back-reefs and reef flats between 2 to 20 m depth with moderate flow and light. They provide microhabitats, surface for symbionts, trap sediment, and contribute moderate carbonate. 
    }{Pavona_cactus_colonia_in_situ.jpg}{\textit{Pavona Cactus} - Photo credit: Benzoni, F.}{fig:intro-Pavona}

    \AltTextImage{
        \Item{Table/Plate:} These corals develop broad plate up to 3m wide in shallow, well-lit reef crest and upper fore-reef zones above 15m deep. They maximize photosynthesis and carbonate production, offer horizontal habitats and shaded refuges, and support encrusting life beneath which alter water flow by causing divergence above and shadows below, redistributing sediment, and producing oxygen. Fragile in storms, they form rubble that aids colonization when fragmented.
    }{Acropora_pulchra.jpg}{Example of an \textit{Acropora pulchra}, forming a large table - Photo credit: Albert Kok}{fig:intro-Acropora}

    \AltTextImage{
        \Item{Encrusting:} These corals grow as thin mats tightly adhering to substrate for about a meter wide and up to 40m deep, tolerating various flow regimes. By resisting breakage and binding rubbles, they stabilize the substrate and cohesion reducing erosion.
    }{Psammocora.jpg}{\textit{Psammocora} - Photo credit: Michael Paletta}{fig:intro-Psammocora}
\end{Itemize}

% Zonation: each form appears in depth- and energy-specific bands (massive on slopes; branching on crest/fore reef; foliose on flats/back-reef; table in shallow flats; encrusting in deeper/stable). This results in a necessity of depth/light/slope masks to assign morphology types; this yields realistic spatial patterns in generated reefs. Moreover, different morphologies define habitat templates and resistance fields; in EFT, they become entity types with distinct parameter sets.

\subsection{Ecological interactions}

% Corals act as dynamic ecological agents, engaging in feedback loops with their environment. Modeling them requires accounting for spatial diffusion via diverse reproductive modes, competition for resources, and bidirectional environmental influences. We will describe ecological agent modeling in \cref{chap:semantic-representation}'s state of the art section.

% Dispersal occurs at multiple scales through budding (local polyp duplication), brooding (larvae settle near parent), polyp bail-out (stress-induced short-range resettlement), fragmentation (medium-range reattachment after disturbance) and broadcast spawning (long-range larval drift). In simulation, these map to multi-scale seeding rules or probabilistic dispersal kernels triggered by environmental and stochastic parameters.

% Environmental stressors (such as elevated temperature, increased \ch{CO_2}, excess nutrients from runoff) heighten bleaching, reduce resilience and favor algal proliferation. Sea-level rise imposes vertical growth constraints: reefs may keep pace, lag then match, or fail to keep up. These factors translate into dynamic parameters affecting coral health or mortality state transitions (e.g., healthy - stressed - dead) and limiting maximum reef elevation relative to sea-level scenarios, enabling exploration of degradation or recovery trajectories.

% Competition for light, substrate and nutrients allows faster-growing algae (especially under high nutrients or low herbivory) to overtake corals, flattening topography and lowering complexity. Simulation can use local competition rules or cellular-automata-style state transitions driven by fields (nutrient, light, herbivory), representing algal zones as lower-relief substrate. Bioerosion by fish and invertebrates produces sediment that fills gaps, suggesting episodic sediment-deposition events balancing accretion and breakdown to shape net reef morphology.

% Coral structural complexity feeds back via functional proxies (e.g., habitat-suitability fields, flow-resistance modifiers) instead of full geometry. These guide hydrodynamic or movement simulations (modulating flow routing and energy dissipation) and inform reef-dependent agent distributions. Reef morphology (crest, slopes, flats) attenuates waves, stabilizes sediments and buffers adjacent habitats (e.g., seagrass, mangroves). In procedural terrain generation, integrate reef features at appropriate elevations with roughness parameters feeding erosion or hydrodynamic modules without resolving every crevice.





% Corals are seen as a whole dynamic ecological component inside its ecosystem as it behave in response to its environment and affect it in a feedback loop. In order to model its behaviour, we need to consider its ability to diffuse itself through reproduction, compete for ressources with other agents, affect and is affected by its environment in a feedback loop.

% % [CORAL DIFFUSE IN THE ENVIRONMENT]

Coral dispersal and colonization occurs at multiple scales through budding (local polyp duplication), brooding (larvae settle near parent), polyp bail-out (stress-induced short-range resettlement), fragmentation (medium-range reattachment after disturbance) and broadcast spawning (long-range larval drift).
% Coral dispersal and colonization combine sexual and asexual modes, spreading their offsprings very locally or extremely far depending on the process. From the closest to the farthest, we can note:

% \begin{Itemize}
%     \Item{Budding:} produces local expansion on the existing colony (fine-scale growth rule).
%     \Item{Brooding:} internal fertilization, larvae released when more developed, settling close to parent, promoting clustered growth.
%     \Item{Polyp bail-out:} happens occasionally when individual polyps detach under stress and resettle, short-range dispersal under adverse conditions.
%     \Item{Fragmentation:} occurs when broken fragments reattach elsewhere, enabling medium-range spread and recovery after disturbance.
%     \Item{Broadcast spawning:} mass synchronous release of eggs and sperm; external fertilization yields larvae (planulae) that drift with currents, allowing long-range dispersal of few kilometers.
% \end{Itemize}

% These modes shows a clear capacity of corals to diffuse in their environment by emitting sexual and asexual parts locally that drift with the water currents.

\AltTextImage{
    % [CORAL IS INFLUENCED BY THE ENVIRONMENT]
    This diffusion is balanced with their mortality and environmental factors limiting their spread. The most notable factors taken into account are elevation of temperature and \ch{CO_2} concentration, visible through the increase of their bleaching frequency and reduction of resilience, as well as pollution which favor algal blooms, overcoming and shading polyps. Another key factor is sea-level changes, in response to which reefs vertical accretion may "keep up" (the coral reef stays in the sunlight, just below water), "catch up" (growth slower but eventually matches), or "give up" (cannot match and becomes submerged).
}{Coral_bleaching.png}{Coral bleaching in the National Marine Sanctuary of America Samoa between 2014 and 2015 - Photo credit: NOAA/ XL Catlin Seaview Survey}{fig:intro-coral-bleaching}

\AltTextImage{
    % [CORAL COMPETE IN THE ENVIRONMENT]
    Coral and algae both compete for sunlight, substrate, and nutrients. However, algae grow faster, especially under elevated nutrients or reduced herbivory, leading to algal overgrowth that smothers coral, flattens reef topography, and reduces habitat complexity and biodiversity.
    % In ecosystem simulations, this behaviour include competitive interactions or state transitions where algae can overtake coral patches under certain field conditions (nutrient/light/herbivore proxies). Terrain impact: algal-dominated zones may be represented by lower-relief substrate without reef elevation.
    On a small scale level, other organisms (e.g., parrotfish, certain invertebrates) break down dead coral, producing sediment that can fill gaps and support new settlement. %This introduce occasional sediment deposition events in erosion model; balance between accretion and bioerosion influences net morphology over time.
}{Coral_Algae_competition.png}{Algae colonizing massive coral}{fig:intro-coral-algae}

\AltTextImage{
    % [CORAL INFLUENCE THE ENVIRONMENT]

    Structural complexity (rugosity, crevices) offers shelter for juveniles, spawning grounds, and supports diverse species ( about 25\% of marine species in < 0.1\% ocean area). True internal geometry of a reef is not currently completely understood but is know to influence the hydrodynamics, and by extension, biodiversity, of the whole biome. 
    %  in our work we abstract it by encoding "presence of complex structure" via functional parameters (e.g., habitat suitability fields, flow resistance modifiers) rather than modeling fine-scale surfaces. 

    These fine-scale properties have influence on the large-scale environment as they introduce turbulences and drag in the water flow and producing sediment deposition at small and large distances. % In fluid or movement simulations, incorporate rugosity/resistance parameters derived from morphology type rather than resolving every crevice. In EFT, these fields influence distribution and behavior of reef-dependent agents.
}{Coral_shelter.jpg}{Coral reefs are porous, resulting in biodiversity shelter, but also complex hydrodynamics}{fig:intro-coral-shelter}

For the same reasons, reef crest breaks incoming waves; morphological complexity (e.g., branching structures) slows water and reduces energy reaching shore, mitigating erosion.

The erosion process is directly impacted by the small-scale. Reef structures trap or stabilize sediments, protecting mangroves and seagrass beds. Simulation of the evolution of the landscape and fluid dynamics would require to take into account almost as importantly the large-scale and small-scale bathymetry, as well as the geological and biological aspect of the ecosystem, to reach accuracy.

% Procedural terrain should include reef features (crest, slopes, flats) at appropriate elevations and roughness to inform hydrodynamic or erosion simulations. Abstract resistance fields derived from morphology guide flow routing and energy dissipation in particle-based erosion models. Adjacent zones (seagrass/mangrove) can be placed based on reef buffering.


Overall, from these behaviours, we consider that corals act as important dynamic ecological agents, engaging in feedback loops with their environment. Modeling them requires accounting for spatial diffusion via diverse reproductive modes, competition for resources, and bidirectional environmental influences. We will describe ecological agent simulation in \cref{chap:semantic-representation}'s state of the art section.


\subsection{Coral reefs' shape}
In \citep{Darwin1842}, Darwin outlined his atoll reefs formation theory, inspired by his Beagle voyage. He proposed that Earth's crust movements under the oceans created atolls. Darwin described a three-stage process in atoll formation: a fringing reef forms around a sinking volcanic island, evolving into a barrier reef, and finally an atoll reef as subsidence continues.

\subsubsection{Fringing reefs}
Fringing reefs are the most common and widespread coral reefs. They form narrow bands near the shore and are usually attached directly to land. Sometimes, they are separated by a shallow, narrow lagoon. Their coastal location makes them accessible but also exposes them to land- and sea-based stressors.

\AltTextImage{
    Their structure includes the reef flat, reef crest, and reef slope. The reef flat is shallow, calm, and closest to shore. The reef crest is higher, exposed at low tide, and faces strong wave action. Only hardy corals survive there. The reef slope descends into deeper water where coral diversity and density increase. It offers habitat for many marine organisms due to more light and water movement.
    
    Fringing reefs grow in clear, shallow waters with high light, warmth, and moderate waves. Coastal currents bring nutrients, helping coral growth. Yet, their location makes them prone to sedimentation, runoff, and pollution. Sediment can smother corals and block sunlight. Runoff adds nutrients, causing algae blooms that compete with corals.
}{Moorea_fringing.jpg}{Fringing reef around Mo'orea island.}{fig:intro-fringing-reef-Moorea}
    
    Zonation defines coral distribution across the reef. On the flat, corals handle temperature shifts and sediment. On the crest, species tolerate waves and low-tide exposure. The slope supports more diverse corals, often forming large structures that shelter marine life.
    % Fringing reefs are the most common and widespread type of coral reef, typically found in close proximity to the shore, forming a narrow band that runs parallel to the coastline. These reefs are directly attached to the land, often with no significant separation between the reef and the shore, or they may be separated by a shallow, narrow lagoon. The proximity of fringing reefs to the coast makes them easily accessible, but it also exposes them to various environmental stressors from both the land and the sea.

    % The structure of fringing reefs is defined by several key components. The reef flat is the area closest to the shore, characterized by shallow waters where corals grow in a relatively calm environment. The reef flat often extends out to the reef crest, which is the highest point of the reef and is typically exposed at low tide. The reef crest is subjected to the full force of incoming waves, making it a zone of high energy where only the hardiest coral species can thrive. Beyond the reef crest, the reef slope descends into deeper water, where the diversity and density of coral species often increase. The reef slope provides a habitat for a wide variety of marine organisms, taking advantage of the increased light availability and water movement.

    % Fringing reefs form in areas where the environmental conditions are favorable for coral growth, including clear, shallow waters with high light availability, warm temperatures, and moderate wave action. These conditions allow corals to establish themselves close to the shore, where they can benefit from the nutrients carried by coastal currents. However, the proximity of fringing reefs to land also makes them more vulnerable to the impacts of sedimentation, freshwater runoff, and pollution from human activities. Sedimentation can smother corals, blocking the light they need for photosynthesis, while excess nutrients from runoff can lead to algal blooms that compete with corals for space and resources.

    % Within fringing reefs, there is a distinct pattern of zonation, where different coral species and marine organisms are distributed according to the environmental conditions across the reef. On the reef flat, where conditions are relatively calm, corals tend to be more resilient to temperature fluctuations and sedimentation. On the reef crest, the coral species are more adapted to withstand the wave action and exposure during low tides. The reef slope, with its deeper waters and more stable conditions, often supports a greater diversity of coral species, including those that form large, complex structures that provide habitat for a variety of marine life.

\AltTextImage{
    \subsubsection{Barrier reefs}
    These reefs typically develop from fringing reefs that have grown over long periods, during which the coastline has either subsided or sea levels have risen. As the land subsides or the sea level rises, the original fringing reef is gradually separated from the shore, forming a lagoon between the reef and the coastline.
    The depth and size of the lagoon, as well as the height of the reef crest, are influenced by factors such as wave action, currents, and the rate of coral growth. The lagoon often contains patch reefs, seagrass beds, and sandy areas. The zonation within barrier reefs creates distinct ecological niches, with different species of corals, fish, and invertebrates adapted to the specific conditions found in each zone. The fore reef slope, with its high-energy environment, supports species that are resilient to strong waves and currents. In contrast, the more sheltered back reef and lagoon provide habitats for species that prefer calmer waters and stable conditions.
}{Divine_Providencia_barrier_reef.jpg}{Aerial image of Providencia island}{fig:intro-barrier-reef-Providencia}

    Barrier reefs also play role in the connectivity between marine ecosystems. They act as a bridge between the open ocean and the coastal environments, facilitating the movement of marine species across different habitats. Many species of fish and invertebrates migrate between the reef and the lagoon during different life stages, using the barrier reef as a nursery or feeding ground.

    \subsubsection{Atolls}
    The formation of atolls is a process that unfolds over millions of years. As the island gradually subsides, mainly due to tectonic activity, the coral continues to grow upward toward the sunlight. Eventually, the island disappears entirely beneath the ocean's surface, leaving behind a ring-shaped coral reef that encircles the central lagoon. This theory of atoll formation was first proposed by Charles Darwin and remains one of the most widely accepted explanations for the development of atolls.
    
\AltTextImage{
    The lagoon varies greatly in depth, ranging from shallow areas with sandy bottoms to deeper sections where patch reefs and seagrass beds may develop. The sheltered waters of the lagoon provide a calmer environment compared to the outer reef slope, allowing a different community of marine species to develop. These species often include a mix of corals, fish, and invertebrates that are weakly tolerant of the high-energy conditions. The lagoon may also contain small coral reef islands, known as motus, which form from the accumulation of coral debris and sand.
}{Tetiaroa_from_sky.jpeg}{Aerial view of Tetiaroa atoll, French Polynesia - Photo credit: }{fig:intro-atoll-Tetiaroa}

    The outer reef slope, with its exposure to the open ocean, is home to species that are adapted to strong currents and waves, including large predatory fish and robust coral species. In contrast, the sheltered lagoon provides a haven for more delicate species, such as seagrasses, small fish, and invertebrates, which rely on the calmer waters for feeding and reproduction. This diversity of habitats within a relatively small area makes atolls particularly important for marine biodiversity.

\subsubsection{Other types of coral reefs and structures}
\AltTextImage{
    Patch reefs are small, isolated coral formations found in lagoons, between larger reefs, or in shallow coastal waters. They grow in sheltered areas that support coral but not larger reef structures. Simpler and more isolated than larger reefs, they are often circular or irregular. They form where water depth, light, and wave action favor coral growth. The formation of patch reefs is influenced by a combination of biological and physical factors. Coral larvae settle in suitable areas where they can establish colonies and grow over time. They often develop in locations where the water is clear, and the substrate is stable, such as rocky outcrops or fossil coral ridges.
    % Patch reefs are smaller, isolated coral formations that occur within lagoons, between larger reef systems, or in shallow coastal waters. These reefs are typically found in the sheltered environments of lagoons, where conditions are favorable for coral growth but do not support the development of larger reef structures.

    % Patch reefs are characterized by their simplicity and isolation compared to their larger counterparts. These small, often circular or irregularly shaped coral assemblages form in areas where the physical and environmental conditions, such as water depth, light availability, and wave action, allow for the localized growth of coral colonies.

    % The formation of patch reefs is influenced by a combination of biological and physical factors. Coral larvae settle in suitable areas where they can establish colonies and grow over time. They often develop in locations where the water is clear, and the substrate is stable, such as rocky outcrops fossil coral ridges.
}{coral-patch-belize.png}{Parch reefs behind the barrier reef of Belize}{fig:intro-patch-reef-Belize}

\AltTextImage{
    Spurs and grooves are alternating ridges (spurs) and channels (grooves) found on the fore reef slope, which is the part of the reef that faces the open ocean. This distinctive pattern is formed primarily by the action of waves and currents, which erode and shape the coral over time. Spurs, the ridges of coral that extend seaward, help to break up and dissipate the energy of incoming waves. The grooves, or channels between these ridges, serve as pathways for water and sediment to flow back into the ocean after waves break on the reef. This spurred and grooved topography is crucial for the reef's ability to protect the coastline, as it reduces the energy of waves before they reach the more fragile parts of the reef and the shore.
}{Spurs-and-groove-Broomfield-Reef.jpg}{Sprurs-and-grooves patterns in Broomfield Reef, Australia - Photo credit: AIMS}{fig:intro-spurs-and-grooves-Broomfield}

\AltTextImage{
    Pinnacles are vertical, column-like coral structures rising from the seafloor, found on fore reef slopes or in lagoons. They add three-dimensional complexity to reefs and vary from small outcrops to tall columns near the surface. Their formation depends on local conditions like strong currents and hard substrate for coral larvae.

% Bommies are isolated outcrops or large coral heads, also called coral knolls. They rise from the seafloor and occur in lagoons or on reef slopes. Often large and scattered across lagoons or near reef edges, they are key to the reef's topographic complexity.
    % Pinnacles are vertical, column-like coral formations that rise from the seafloor, often found on the fore reef slope or within lagoons. These towering structures add significant three-dimensional complexity to the reef system. Pinnacles can vary greatly in size, from small outcrops to large, towering columns that reach close to the water's surface. The formation of pinnacles is often influenced by localized environmental conditions, such as strong currents or the availability of hard substrate for coral larvae to settle on.

    % Bommies are isolated outcrops or large coral heads that rise from the seafloor, typically within lagoons or on the reef slope. These features, sometimes referred to as coral knolls, can be quite large and are often found scattered across the lagoon or near the outer edges of the reef. Bommies are important structural components of the reef as they contribute to the overall topographic complexity of the environment. 
}{Pinnacle-Chumphon.jpg}{Chumphon pinnacle, Kho Toa, Tailand - Photo credit: Allie Vautin}{fig:intro-pinnacle-Chumphon}

