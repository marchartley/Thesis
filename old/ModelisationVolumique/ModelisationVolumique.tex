\chapter{Modélisation de terrains volumiques}
\minitoc

- ...

\section{Multiples représentations de terrain}
- ...

\subsection{Conversions entre représentations}
%\subsubsection{Problèmes}
- Pertes d'informations \\
-> Propagation d'erreur sur la géométrie (approximations sur les normales, résolution Z, surface, etc...) \\
-> Perte d'informations sous-terraines \\
- ...

\section{Densité de matière}
- ...

\subsection{Granularité de matériaux}
- ...

\subsection{Soil triangle}
- ...


\section{Terrains implicites}
- ...

\subsection{Fonctions scalaires}
- ...

\subsection{Fonctions de mélange}
- ...

\subsection{Fonctions de placement}
- ...

\subsection{Utilisation de matériaux}
- ...

\subsubsection{Définition du matériau final}
- ...

\subsubsection{Post-processing : transformation de matériaux}
- ...


\section{Régénération}
- ...

\subsection{Actions manuelles}
- ...

\subsubsection{Problématiques de la régénération}
- ...

\subsubsection{...}
- ...


\chapter{Génération automatique d'îles coralliennes}
\minitoc

- Definition ile corallienne \\
- Presentation coraux \\
- Difference paysages normaux \\
** Notion de coraux \\
*** Evolution longue (ile) et courte (coraux) \\
**** Processus geologique affaissement de l'ile \\
- ...

\section{Théorie darwinienne}
- Plusieurs théories \\
- Impossibilité d'étudier aisément les environnements \\
** Utilisation d'observations \\
- Théorie réfutée par \cite{Droxler2021} \\
** Mais trop tôt pour juger \\
** Pratique dans notre cas. \\
- ...

\subsection{Multiples théories}
- 

\subsection{Voyage de Darwin}
- ...
\section{Génération d'exemples}
- ...
\subsubsection{Pipeline}
- ...
\subsubsection{Entrée}
- ...
\subsubsection{"Simulation"}
- ...
\subsubsection{Sortie}
- ...
\section{cGAN}
- ...
\subsection{Définition cGAN}
- ...
\subsection{Pourquoi un cGAN?}
- ...
\subsection{Entraînement}
- ...
\subsubsection{Utilisation de données synthétiques}
+ Problème des données synthétiques \\
- ...
\subsubsection{Augmentation de données}
- ...
\subsection{Utilisation du modèle}
- ...
\subsubsection{Génération par sketch}
- ...
\subsubsection{Temps interactifs}
- ...
\subsubsection{Réalisme}
- ...




\chapter{Génération de réseaux karstiques}
- ...

\section{Géologie}
- ...

\subsection{Formation}
- ...

\subsection{Intérêts}
- ...

\subsection{Place des karsts dans le projet}
- ...

\section{Contrôle utilisateur}
- ...

\subsection{Méthode existante}
- ...

\section{Ma méthode}
- ...

\subsection{Karst en tant qu'arbre?}
- ...

\subsection{Génération de végétation -> Space colonization}
- ...


