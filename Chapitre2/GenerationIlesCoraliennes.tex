\chapter{Génération automatique d'îles coralliennes}
\minitoc

- Definition ile corallienne \\
- Presentation coraux \\
- Difference paysages normaux \\
** Notion de coraux \\
*** Evolution longue (ile) et courte (coraux) \\
**** Processus geologique affaissement de l'ile \\
- ...

\section{Théorie darwinienne}
- Plusieurs théories \\
- Impossibilité d'étudier aisément les environnements \\
** Utilisation d'observations \\
- Théorie réfutée par \cite{Droxler2021} \\
** Mais trop tôt pour juger \\
** Pratique dans notre cas. \\
- ...

\subsection{Multiples théories}
- 

\subsection{Voyage de Darwin}
- ...
\section{Génération d'exemples}
- ...
\subsubsection{Pipeline}
- ...
\subsubsection{Entrée}
- ...
\subsubsection{"Simulation"}
- ...
\subsubsection{Sortie}
- ...
\section{cGAN}
- ...
\subsection{Définition cGAN}
- ...
\subsection{Pourquoi un cGAN?}
- ...
\subsection{Entraînement}
- ...
\subsubsection{Utilisation de données synthétiques}
+ Problème des données synthétiques \\
- ...
\subsubsection{Augmentation de données}
- ...
\subsection{Utilisation du modèle}
- ...
\subsubsection{Génération par sketch}
- ...
\subsubsection{Temps interactifs}
- ...
\subsubsection{Réalisme}
- ...

