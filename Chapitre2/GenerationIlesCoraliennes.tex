\chapter{Génération automatique d'îles coralliennes}
\minitoc

- Definition ile corallienne \\
- Presentation coraux \\
- Difference paysages normaux \\
** Notion de coraux \\
*** Evolution longue (ile) et courte (coraux) \\
**** Processus geologique affaissement de l'ile \\
- ...

\section{Théorie darwinienne}
- Plusieurs théories \\
- Impossibilité d'étudier aisément les environnements \\
** Utilisation d'observations \\
- Théorie réfutée par \cite{Droxler2021} \\
** Mais trop tôt pour juger \\
** Pratique dans notre cas. \\
- ...

\subsection{Multiples théories}
- Préciser autres théories : \\
** "Growth on Submarine Mountains Theory" (John Murray): Les récifs commencent sur les monts sous-marins et guyots, puis montent jusqu'à la surface petit à petit. \\
** "Sea Level Change Theory" (Reginald Daly) ["The Coral Reef Problem"] \cite{Daly1915}  : [A FOUILLER, JE L'AI PAS COMPRISE] \\
** "Erosion and Sedimentation Theory" (Maurice Ewing et William Donn) : [A FOUILLER] \\
** "Platform Reef Theory" (William Morris Davis) : [A FOUILLER] \\
- Les théories ne se contredisent pas forcement, elles peuvent être complémentaires pour expliquer des cas que d'autres n'expliquent pas. \\
- ...

\subsection{Voyage de Darwin}
- ...

\section{Génération d'exemples}
- On utilise la théorie de Darwin dans notre cas car génération plutôt simple. \\
- 2 étapes: \\
** Génération de l'ile \\
** Génération du récif \\
- Nombreuses assumptions : \\
** Ile a une forme relativement ronde \\
** 
- ... 

\subsubsection{Pipeline}
- ...

\subsubsection{Entrée}
- ...

\subsubsection{"Simulation"}
- ...

\subsubsection{Sortie}
- ...

\section{cGAN}
- ...

\subsection{Définition cGAN}
- ...

\subsection{Pourquoi un cGAN?}
- Flexibilité de l'entrée \\
- Sortir de la condition de l'entrée "radiale" \\
- Output même pour des données "incohérentes" (e.g. océan dans une ile) \\
- Pas de math, géométrie, géologie, ou choses compliquées à maitriser (hehe)

\subsection{Entraînement}
- ...

\subsubsection{Utilisation de données synthétiques}
+ Problème des données synthétiques \\
- ...

\subsubsection{Augmentation de données}
- ...

\subsection{Utilisation du modèle}
- ...

\subsubsection{Génération par sketch}
- ...

\subsubsection{Temps interactifs}
- ...

\subsubsection{Réalisme}
- ...

