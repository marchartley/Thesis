\chapter{Génération automatique d'îles coralliennes}
\label{chap:coral-island}
\minitoc

\section{Introduction}
\label{sec:coral-island_introduction}
- Definition ile corallienne \\
** Différents types d'iles coralliennes \\
*** Ici, iles volcaniques \\
- Presentation coraux \\
- Difference paysages normaux \\
** Notion de coraux \\
*** Evolution longue (ile) et courte (coraux) \\
**** Processus geologique affaissement de l'ile \\
- ...

\subsection{Théorie darwinienne}
- Plusieurs théories \\
- Impossibilité d'étudier aisément les environnements \\
** Utilisation d'observations \\
- Théorie réfutée par \cite{Droxler2021} \\
** Mais trop tôt pour juger \\
** Pratique dans notre cas. \\
- ...

\subsubsection{Multiples théories}
- Préciser autres théories : \\
** "Growth on Submarine Mountains Theory" (John Murray): Les récifs commencent sur les monts sous-marins et guyots, puis montent jusqu'à la surface petit à petit. \\
** "Sea Level Change Theory" (Reginald Daly) ["The Coral Reef Problem"] \cite{Daly1915}  : [A FOUILLER, JE L'AI PAS COMPRISE] \\
** "Erosion and Sedimentation Theory" (Maurice Ewing et William Donn) : [A FOUILLER] \\
** "Platform Reef Theory" (William Morris Davis) : [A FOUILLER] \\
- Les théories ne se contredisent pas forcement, elles peuvent être complémentaires pour expliquer des cas que d'autres n'expliquent pas. \\
- ...

\subsubsection{Voyage de Darwin}
- ...

\subsection{Overview}
- Outil proposé \\
** Sketching de l'île \\
** Sketching du profil \\
** Simulation de vents \\
- Automatisation des exemples \\
** Augmentation de données \\
- ...

\section{Related works}
\label{sec:coral-island_related-works}
- Perlin + falloff \\
** Mais manque de control \\
- Uplift [SCHOTT UPLIFT] \cite{Cordonnier2016,Cordonnier2017a} \\
** Mais proposons une méthode qui limite l'interaction demandée \\
- Sketching [PEYTAVIE SKETCHING, EMILIEN FIRST PERSON] \cite{Gain2009} \\
** Ajout de la contrainte radiale pour simplifier interaction \\
- Modelisation geologique \cite{Patel2021} \\
** Pour nous, pas facile à connaitre ce qui se passe sous terre (?) \\
- Contrairement à la litterature, integration de la partie sous-marine \\
** Integration de données d'observation dans notre processus \\
*** -> Forme type et profil type d'une ile \\
** Difficile d'intégrer des simulations physiques, car plutôt incertain \\
- Generation par couches permettrait d'améliorer le réalisme des exemples en considérant les interactions couches/environnement. \\
- ... 

\section{Génération d'exemples}
\label{sec:coral-island_example-generation}
- On utilise la théorie de Darwin dans notre cas car génération plutôt simple. \\
- 2 étapes: \\
** Génération de l'ile \\
** Génération du récif \\
- Nombreuses assumptions : \\
** Ile a une forme relativement ronde \\
** Corail pousse à une hauteur constante autour de l'ile \\
** Toutes "features" sont radiales \\
** Deformations des iles causées par les vents et vagues \\
** Les iles sont indépendantes les unes des autres \\
** Le profil est relativement identique tout autour de l'ile \\
- ... 

\subsubsection{Entrée}
- Sketching de l'ile vue de dessus + vue de profil \\
- Force des vents \\
- Niveau d'eau \\
- Taux d'affaissement \\
- ...

\subsubsection{"Simulation"}
- Affaissement calculé par simple scaling \\
** Attention aucune consistance geologique \\
*** Ici, utilisation d'un niveau zéro pour scale en Z \\
*** Pas de prise en compte des différents matériaux dans le sol \\
- ...

\subsubsection{Sortie}
- Height map de la surface de l'ile \\
- Zones d'ile et zones de corail \\
- Possibilité de recalculer la hauteur de sol et hauteur de corail \\
- ...

\section{cGAN}
\label{sec:coral-island_cGAN}
- ...

\subsection{Définition cGAN}
- ...

\subsection{Pourquoi un cGAN?}
- Flexibilité de l'entrée \\
- Sortir de la condition de l'entrée "radiale" \\
- Output même pour des données "incohérentes" (e.g. océan dans une ile) \\
- Pas de math, géométrie, géologie, ou choses compliquées à maitriser (hehe)

\subsection{Entraînement}
- ...

\subsubsection{Utilisation de données synthétiques}
+ Problème des données synthétiques \\
- ...

\subsubsection{Augmentation de données}
- ...

\subsection{Utilisation du modèle}
- ...

\subsubsection{Génération par sketch}
- ...

\subsubsection{Temps interactifs}
- ...

\subsubsection{Réalisme}
- ...

