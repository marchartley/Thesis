\chapter{Modélisation de terrains volumiques [PEUT ETRE A DEPLACER DANS INTO]}
\minitoc

- Volumique est important pour représenter les structures en 3D \\
- Permet de representer des cavités, des arches, des superpositions, ... \\
- Notion de matériaux permet d'inclure beaucoup plus d'informations pour les parties suivantes : amplification et rendu \\
** Amplification (eg. érosion) doit connaitre le type de sol à la surface et sous-terrain pour être réaliste \\
** Rendu doit connaitre le matériau à la surface pour afficher correctement des textures \\
- ...


\section{Terrains implicites avec matériaux}
- ...

\subsection{Densité de matière}
- ...

\subsubsection{Granularité de matériaux}
- ...

\subsubsection{Soil triangle}
- ...

\subsection{Fonctions scalaires}
- ...

\subsection{Fonctions de mélange}
- ...

\subsection{Fonctions de placement}
- ...

\subsection{Utilisation de matériaux}
- ...

\subsubsection{Définition du matériau final}
- ...

\subsubsection{Post-processing : transformation de matériaux}
- ...



\section{Représentation graphique des objets environnementaux}
- Surfaces implicites \\
- Maillages \\
- ...
