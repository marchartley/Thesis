
\subsection{Fluid simulations}

Fluid simulation constitutes an important aspect in Computer Graphics and in terrain generation, providing a physical basis for modelling water behaviour and dynamics for real-time applications and offline rendering \cite{Wang2024}. The mathematical foundations of fluid simulations lie in the Navier-Stokes equations, or in the case of hydrodynamics, the incompressible Navier-Stokes equations. An accurate computation of these dynamics implies an expensive computational cost; thus, various solver variants have been proposed, relying on grids, particles, or hybrid frameworks, each balancing trade-offs between simulation stability, dissipation control, computational scalability, and user control. In this section we propose an overview of different solvers and their characteristics.

\subsubsection{Governing equations}
The governing equations for fluid simulation are rooted in the conservation laws of mass and momentum for an incompressible Newtonian fluid. In three dimensions, these are expressed by the incompressible Navier-Stokes equations, which consist of the momentum equation:

\begin{align}
    \rho \left(\frac{\partial \mathbf{u}}{\partial t} + \mathbf{u}\cdot\nabla\mathbf{u}\right) = -\nabla p + \mu \nabla^2 \mathbf{u} + \rho \mathbf{g}
\end{align}

and the incompressibility constraint:

\begin{align}
    \nabla\cdot\mathbf{u} = 0
\end{align}

Here $\mathbf{u}(\mathbf{x},t)$ denotes the velocity field, $p(\mathbf{x},t)$ the pressure field, $\rho$ the fluid density, $\mu$ the dynamic viscosity, and $\mathbf{g}$ the gravitational acceleration vector. The momentum equation enforces conservation of momentum under advective transport, pressure gradient forces, viscous diffusion, and body forces; the divergence-free condition enforces mass conservation by ensuring volume preservation of fluid elements. In computer graphics contexts, these equations form the basis for physically grounded fluid behaviour, although various approximations or discretisation strategies are typically applied to achieve stability, efficiency, or control in practice.

When altitude variations are negligible compared to horizontal scales, a depth-averaged approximation known as the Shallow Water Equations (SWE) is often employed \cite{Parna2019}. Under the assumption that the fluid column height $\eta(x,y,t)$ varies slowly in the vertical direction, the (non-conservative) SWE are expressed as
\begin{align}
    \frac{\partial \mathbf{u}}{\partial t} + \mathbf{u} \cdot \nabla \mathbf{u} = - \tensor{g} \nabla \eta \\
    \frac{\partial \eta}{\partial t} + \nabla\cdot(\eta \mathbf{u}) = 0
\end{align}

where $\mathbf{u}(x,y,t)$ is the horizontal velocity at the surface, and $\tensor{g}$ denotes gravitational acceleration.
The first equation enforces conservation of mass in the depth-integrated sense, and the momentum equation balances advective transport and pressure forces arising from fluid depth.
This model offers substantial computational savings and is widely used for real-time or large-domain simulations when three-dimensional effects such as vertical vortices and breaking waves are not critical \cite{Parna2019}.

Modelling assumptions are made explicit in selecting the governing equations. The fluid is typically assumed to be Newtonian and incompressible, with constant density and viscosity. Body forces are often limited to gravity, and surface tension is neglected.
The incompressibility assumption simplifies the mathematical treatment and enhances numerical stability; its enforcement commonly involves a pressure projection step that ensures the velocity field remains divergence-free. Viscosity may be incorporated implicitly or explicitly depending on stability requirements, and external forces or boundary conditions are specified to match the intended scenario.

\subsubsection{Numerical solvers}

Hydrodynamics are continuous but numerical solvers are commonly organised according to their discretisation paradigm, with each category offering different trade-offs in stability, adaptivity, and computational cost. The principal categories comprise grid-based Eulerian methods, particle-based Lagrangian techniques, hybrid schemes that combine grid and particle representations, and reduced models tailored for simplified scenarios or interactive control.

On one hand, grid-based Eulerian methods discretise the fluid domain on a fixed grid and approximate the incompressible Navier-Stokes equations via operator splitting or projection schemes. Historically, the Marker-and-Cell (MAC) approach established the staggered-grid representation \cite{Harlow1965}, storing velocities on cell faces and pressure at cell centres to enforce divergence-free constraints accurately in free-surface flows. Subsequent developments in graphics introduced semi-Lagrangian advection with implicit viscosity integration \cite{Stam1999}, which achieves unconditional stability permitting large time steps at the expense of increased numerical dissipation. Lattice Boltzmann methods (LBM) offer a mesoscopic viewpoint on fluid dynamics, leveraging local collision and streaming operations on a lattice to recover macroscopic behaviour \cite{Chen1998}; they are notable for parallel efficiency and natural handling of moderate boundary complexity, though three-dimensional or highly irregular domains carry significant computational overhead. Finite-volume or finite-element variants appear in engineering CFD frameworks such as OpenFOAM and can be adapted for graphics applications, but typically require careful optimisation to remain feasible for large-domain or interactive contexts. % Free-surface representation in Eulerian solvers often relies on level-set or volume-of-fluid techniques, with volume loss and interface sharpness requiring corrective measures or hybrid augmentation.

On the other hand, particle-based Lagrangian methods represent fluid as discrete particles carrying mass, momentum, and other properties. Smoothed Particle Hydrodynamics (SPH) discretises the continuum via kernel-weighted interactions among particles \cite{Monaghan2005}, naturally handling free surfaces and complex boundaries without explicit interface tracking; however, SPH demands high particle counts for fidelity, and stability challenges (e.g., tensile instability, pressure oscillations) necessitate specialised corrective formulations \cite{Monaghan2005,Koschier2022}. Pure Particle-In-Cell (PIC) schemes map particle information to a grid for pressure projection but tend to introduce substantial numerical dissipation through frequent interpolation \cite{Harlow1962}. The Fluid-Implicit Particle (FLIP) method mitigates dissipation by updating particle velocities using grid-derived increments rather than full replacements, thereby preserving kinetic energy and small-scale turbulence \cite{Brackbill1988}; FLIP is well suited for detailed free-surface phenomena such as splashes yet requires careful tuning near boundaries to avoid instability. Other particle variants, including Moving Particle Semi-implicit (MPS) methods, extend the Lagrangian paradigm with implicit pressure solves, but share similar demands for neighbour search and stabilisation \cite{Koshizuka1996}.

Hybrid approaches aim to combine the stability and incompressibility enforcement of grid-based solvers with the adaptivity and natural boundary handling of particles. In common hybrid pipelines, particles carry momentum and advect passive quantities, while a background grid enforces the divergence-free condition via projection. Affine Particle-in-Cell (APIC) refines the transfer between particles and grid by representing particle velocity fields affinely, improving momentum conservation and reducing dissipation relative to FLIP \cite{Jiang2015}. Such hybrids leverage the strengths of each paradigm but introduce overhead in particle-grid transfers and require careful design to maintain consistency and numerical robustness in dynamic domains.

Reduced fluid models are employed when full three-dimensional simulation is unnecessary or prohibitively expensive. Depth-averaged shallow water equations capture large-scale horizontal flows over terrain under the assumption of negligible vertical variations; their lower-dimensional form yields significant computational savings and is widely used in real-time or large-domain scenarios where vertical vortices or breaking waves are not critical \cite{Vreugdenhil1994,Pan2012}. Potential flow approximations or other simplified models may be invoked for wave-like phenomena where vorticity and viscous effects can be ignored. Procedural or heuristic approximations, such as cellular automata-based flow or ad hoc velocity fields, support highly interactive or stylised effects by sidestepping full PDE solves, at the cost of physical fidelity. These reduced methods serve both as initial terrain-shaping tools and as fallback options for real-time applications where performance constraints dominate.

\subsubsection{Solver characteristics}

Numerical characteristics of solvers critically influence the realism, stability, and performance of fluid simulation. These include time integration and stability properties, processing of free surfaces and boundary conditions, control of numerical dissipation to preserve detail, and computational efficiency and scalability strategies.

\AltTextImage{
    Time integration schemes must balance stability and accuracy. Explicit methods compute updates directly from known states but impose restrictive time-step limits proportional to grid spacing ($\Delta x, \Delta y, \Delta z$) or particle spacing in relation with the computed velocity, rendering fine resolutions costly. The CFL condition states that the dimensionless Courant number $C$ should not exceed a maximal value $C_{max}$ given depending on the PDE to solve (typically $C_{max} = 1$ for explicit integrations):
    \begin{align}
        C = \Delta t \left( \sum_{i=1}^{n}{\frac{u_i}{\Delta x_i}} \right) \leq C_{max} \nonumber
    \end{align}
}{delta-time-Lu2023.jpg}{Results of a fluid simulation after 5$\mu s$, 10$\mu s$ and 15$\mu s$ with different spatial resolutions (from left to right: 0.14mm, 0.07mm, 0.05mm and 0.04mm) shows large discrepancies. }{fig:erosion-sota-dt}{Importance of spatial resolution in fluid simulations}

Thus, when looking at 2D and 3D explicit methods, the main constraint is finding a valid time-step $\Delta t$:
\begin{align}
    \Delta t_{\text{2D}} \leq \left( \frac{\Delta x}{u_x} + \frac{\Delta y}{u_y} \right) C_{max}
    \quad
    \Delta t_{\text{3D}} \leq \left( \frac{\Delta x}{u_x} + \frac{\Delta y}{u_y} + \frac{\Delta z}{u_z} \right) C_{max}
\end{align}

Implicit or semi-implicit treatments of diffusion or pressure terms allow larger time steps by solving linear or nonlinear systems ($C_{max} > 1$), as exemplified by semi-Lagrangian advection with implicit viscosity in \cite{Stam1999}. Pressure projection typically entails solving a Poisson equation, for which direct solvers offer accuracy but scale poorly to large grids, while iterative solvers (e.g., conjugate gradient, multigrid) afford scalability at the expense of convergence concerns that must be managed via preconditioning or adaptive tolerance. Time-stepping strategies may incorporate adaptive substepping or projection frequency adjustments to maintain stability without excessive computation.

Accurate handling of free surfaces and boundary conditions is essential for plausible fluid behaviour. Interface-capturing methods in Eulerian solvers, such as level-set or volume-of-fluid, track the free surface implicitly but can suffer volume loss or smearing; corrective reinitialisation or particle-based markers near the interface often mitigate these deficits. Particle-based schemes represent the free surface naturally through particle distribution but require surface reconstruction for rendering and may exhibit clumping or void regions without density control. 

Solid boundary conditions in both paradigms demand robust collision and pressure treatment: Eulerian grids enforce no-slip or free-slip via ghost cells or immersed boundary techniques, while particles interact with geometry via repulsion forces or dynamic boundary particles. Hybrid methods must synchronise interface representation between particles and grid, ensuring that evolving boundaries remain consistent with divergence-free constraints. Dynamic domains, such as moving obstacles or adaptive meshes, necessitate regridding or particle reseeding procedures that preserve mass and momentum.

% \comment{Here, have the obstacle/terrain refining}

Numerical dissipation and detail preservation influence the visual richness of simulated flows. Semi-Lagrangian advection and grid-particle interpolation in PIC introduce artificial smoothing that dampens small-scale vortices and surface detail. Techniques to mitigate dissipation include the FLIP update, which applies only the change in grid velocity to particles, and higher-order advection schemes that reduce numerical diffusion. Vorticity confinement or turbulence-enhancement terms may reintroduce fine-scale structures lost to dissipation. In Eulerian solvers, divergence-free interpolation schemes and improved projection methods help maintain kinetic energy. SPH and other particle methods can preserve detail inherently but may suffer from noise or instability if neighbour sampling is irregular; stabilisation strategies, such as density reinitialisation, kernel correction, or pressure regularisation, seek to retain fidelity without sacrificing stability. Machine learning-based super-resolution methods have recently emerged to reconstruct fine details atop coarse simulation outputs, however, care must be taken when applying them to simulations that differ significantly from the data they were trained on, as they may not produce reliable results.


\subsubsection{Non-physical fluid simulations}

\AltTextImage{
    Recent advances in fluid simulation increasingly integrate machine learning techniques to accelerate or augment traditional numerical methods. Surveys of machine learning applications in Computational Fluid Dynamics identify roles for data-driven methods, physics-informed models, and ML-assisted numerical solvers that improve convergence or predict intermediate quantities such as pressure fields and subgrid turbulence closures \cite{Huang2022NN}. 
}{smoke-control-Tang2021.png}{An incompressible force field is computed to guide the smoke simulation from a user-defined shape to another \cite{Tang2021}.}{fig:erosion-artistic-fluid-Tang2021}{An incompressible force field computed to guide a smoke simulation}
    
Neural surrogates have been proposed to approximate costly components of the solver, yielding substantial speed-ups while maintaining acceptable fidelity in graphics contexts; such approaches often leverage convolutional networks for pressure projection or learn residual corrections to coarse simulations \cite{Tompson2017,Sousa2024}. Physics-informed neural networks and related frameworks enable embedding governing equations into the learning process, facilitating generalisation and adherence to conservation laws, though their applicability to large-scale, real-time graphics remains an active research area \cite{Brunton2025}.

\AltTextImage{
    Procedural or artistic fluid control has been an emerging topic for the last few decades, striking a balance between user interaction and plausibility of the fluid behaviour without full PDE solving. \cref{chap:semantic-representation} used the Kelvinlet paradigm \cite{DeGoes2017}, an analytic formulation for material deformation derived from Kelvin's state using Green's function of linear elasticity of material, providing more control types than the fluid-specific Stokelets \cite{Chwang1976}. \cite{Wejchert1991} shows that these analytic fluid primitives are an efficient and lightweight approximation of highly controlled fluid simulations. Stroke-based fluid models are closer to the storyboard design of animation artists, making them intuitive to use, but are usually for subsets of frames of an animation, which become time-consuming for the price of high-level control \cite{Xing2016,Patel2005,Yan2020b,Pan2013}. 
}{fluid-control-Lu2019.png}{\cite{Lu2019} uses fluid rigging-skinning on water particles to guide the animation of particles. }{fig:erosion-artistic-Lu2019}{Water particles animated by rigging-skinning}

Procedural fluids are usually defined by noise functions to artificially introduce motion \cite{Bridson2007c}, but may also be added after computing an accurate fluid simulation to introduce more vortices \cite{Wang2025}. Vortices are usually too small for Eulerian simulations as the grid's cell width used may be larger than a vortex, and the dissipation removes their presence. Editing velocity fields procedurally may also come through the use of frequency-domain editing of forces as used in \cref{fig:erosion-artistic-fluid-Tang2021} \cite{Forootaninia2020, Tang2021}, blending and interpolation techniques to merge a predefined fluid with another \cite{Raveendran2014}, or even morphing techniques to deform a velocity or force field as shown in \cref{fig:erosion-artistic-Lu2019} \cite{Lu2019,Raveendran2012,Flynn2019}.

Non-physical fluid simulations include procedural and artistic fluids during or after a more computationally expensive simulation, as they introduce variation on many levels: force field and velocity field of Eulerian simulation methods, or the force, velocity, and even position of simulation particles of Lagrangian methods \cite{Sims1990}.




\midConclusion

Fluid simulations come in a large variety of forms, each with advantages and inconveniences: accuracy, energy preservation, user control, and boundary representations. Erosion on terrains, and more importantly submerged terrains, has to compute the motion of fluid in the environment, whether it represents air for wind-driven phenomena or water for hydraulic erosions. Erosion simulation algorithms proposed in literature do not offer alternatives to the method-specific fluid simulation, constraining the user to only a subset of possibilities for a given terrain representation.
