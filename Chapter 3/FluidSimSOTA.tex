
\subsection{Fluid simulations}

Fluid simulation constitues an important aspect in Computer Graphics and in terrain generation, providing a physical basis for modeling water behaviour and dynamics for real-time applications and offline rendering \cite{Wang2024}. The mathematic foundations of fluid simulations lies in the Navier-Stokes equations, or in the case of hydrodynamics, the incompressible Navier-Stokes equations. An accurate computation of these dynamics implies an expensive computational cost; thus, multiple solver variants have been proposed, relying on grid-based, particles, or hybrid frameworks, each balancing trade-offs between simulation stability, dissipation control, computational scability, user control. In this section we will overview the different solvers and their characteristics.

% Fluid simulation constitutes a foundational component in computer graphics and interactive terrain-related applications, providing a physically grounded basis for modeling water behavior, free-surface dynamics, and flow-driven visual effects in both offline rendering and real-time systems ([link.springer.com][1], [today.ucsd.edu][2]). The underlying mathematical framework derives from the incompressible Navier-Stokes equations, often supplemented by depth-averaged approximations such as the shallow water equations when vertical variations are negligible, and various interface-tracking or interface-capturing techniques are employed to represent free surfaces accurately ([link.springer.com][1], [research.rug.nl][3]). This section examines numerical solver categories-including Eulerian grid-based methods, Lagrangian particle-based schemes, and hybrid approaches-alongside considerations of stability, dissipation control, boundary handling, and computational scalability, as well as emerging machine learning-assisted and differentiable frameworks for fluid simulation ([link.springer.com][1], [arxiv.org][4]). By focusing exclusively on fluid dynamics methods and omitting erosion coupling, the discussion clarifies modeling assumptions, highlights trade-offs among solver designs, and establishes a structured foundation for selecting or extending fluid simulation techniques suited to the specific requirements of terrain-related or visual-effect applications.

% [1]: https://link.springer.com/article/10.1007/s41095-023-0368-y "Physics-based fluid simulation in computer graphics"
% [2]: https://today.ucsd.edu/story/this-new-advanced-method-produces-highly-realistic-simulations-of-fluid-dynamics "This New Advanced Method Produces Highly Realistic Simulations ..."
% [3]: https://research.rug.nl/en/publications/physics-based-fluid-simulation-in-computer-graphics-survey-resear "Physics-based fluid simulation in computer graphics"
% [4]: https://arxiv.org/abs/2408.12171 "Recent Advances on Machine Learning for Computational Fluid ..."


\subsubsection{Governing equations}
The governing equations for fluid simulation are rooted in the conservation laws of mass and momentum for an incompressible Newtonian fluid. In three dimensions, these are expressed by the incompressible Navier-Stokes equations, which consist of the momentum equation

$$
\rho \left(\frac{\partial \mathbf{u}}{\partial t} + \mathbf{u}\cdot\nabla\mathbf{u}\right) = -\nabla p + \mu \nabla^2 \mathbf{u} + \rho \mathbf{g},
$$

and the incompressibility constraint

$$
\nabla\cdot\mathbf{u} = 0.
$$

Here $\mathbf{u}(\mathbf{x},t)$ denotes the velocity field, $p(\mathbf{x},t)$ the pressure field, $\rho$ the fluid density, $\mu$ the dynamic viscosity, and $\mathbf{g}$ the gravitational acceleration vector. The momentum equation enforces conservation of momentum under advective transport, pressure gradient forces, viscous diffusion, and body forces; the divergence-free condition enforces mass conservation by ensuring volume preservation of fluid elements. In computer graphics contexts, these equations form the basis for physically grounded fluid behavior, although various approximations or discretization strategies are typically applied to achieve stability, efficiency, or control in practice.

When altitude variations are negligible compared to horizontal scales, a depth-averaged approximation known as the shallow water equations (SWE) is often employed \cite{Parna2019}. Under the assumption that the fluid column height $\eta(x,y,t)$ varies slowly in the vertical direction, the (non-conservative) SWE take the form

$$
\frac{\partial \eta}{\partial t} + \nabla\cdot(\eta \mathbf{u}) = 0, 
\quad
\frac{\partial \mathbf{u}}{\partial t} + \mathbf{u} \cdot \nabla \mathbf{u} = - g \nabla \eta
% \frac{\partial (\eta \mathbf{u})}{\partial t} + \nabla\cdot\Bigl(\eta\,\mathbf{u}\otimes\mathbf{u} + \tfrac12 g\,\eta^2 I\Bigr) = \mathbf{S},
$$

where $\mathbf{u}(x,y,t)$ is the horizontal velocity at the surface, and $g$ denotes gravitational acceleration. %, and $\mathbf{S}$ represents source or sink terms such as bottom friction or external inflows. 
The first equation enforces conservation of mass in the depth-integrated sense, and the momentum equation balances advective transport, and pressure forces arising from fluid depth. %, and any additional forcing. 
This model offers substantial computational savings and is widely used for real-time or large-domain simulations when three-dimensional effects such as vertical vortices and breaking waves are not critical \cite{Parna2019}. % ([rke.abertay.ac.uk][3], [en.wikipedia.org][4]).

Modeling assumptions are made explicit in selecting the governing equations. The fluid is typically assumed to be Newtonian and incompressible, with constant density and viscosity. Body forces are often limited to gravity, and surface tension is neglected. % Commonly, the fluid domain is bounded by solid geometry (terrain or obstacles) and a free surface whose motion must be tracked or captured. 
The incompressibility assumption simplifies the mathematical treatment and enhances numerical stability; its enforcement commonly involves a pressure projection step that ensures the velocity field remains divergence-free. Viscosity may be incorporated implicitly or explicitly depending on stability requirements, and external forces or boundary conditions are specified to match the intended scenario. %These assumptions underpin both full three-dimensional solvers and reduced models, providing a clear framework for subsequent discretization and algorithmic design ([cs.ubc.ca][5], [digipen.edu][1]).

% Accurate representation of the free surface (interface between two fluids) is crucial for visually plausible fluid behaviour. In Eulerian approaches, interface-capturing methods such as the volume-of-fluid technique track a scalar field representing fluid fraction in each cell, whereas level-set methods represent the surface implicitly as the zero contour of a signed-distance function and evolve it via advection and reinitialization procedures. Level-set approaches can yield smooth surfaces but may suffer volume loss without corrective measures. Particle-based surface tracking augments Eulerian fields by introducing Lagrangian markers near the interface to preserve detail. In purely Lagrangian schemes, such as Smoothed Particle Hydrodynamics (SPH), the fluid is represented entirely by particles, and the free surface emerges naturally from the particle distribution, though additional reconstruction is needed for rendering. Hybrid methods combine grid-based pressure solves with particle advection to enforce incompressibility while retaining interface fidelity. The choice among interface tracking or capturing strategies influences numerical stability, computational cost, and ease of coupling with boundary conditions and velocity solvers ([arxiv.org][6], [en.wikipedia.org][7]).

% Overall, the governing-equation stage establishes the physical and mathematical foundation for fluid simulation. It clarifies the assumptions regarding fluid properties, dimensionality reduction when applicable, and the requirements for free-surface representation. This foundation informs the design of discretization schemes, pressure-projection algorithms, advection treatments, and boundary-handling mechanisms that follow in numerical solver discussions.

% [1]: https://www.digipen.edu/sites/default/files/public/docs/theses/brian-trevethan-digipen-master-of-science-in-computer-science-thesis-physically-based-fluid-simulation-for-computer-graphics.pdf "[PDF] Physically-Based Fluid Simulation for Computer Graphics - DigiPen"
% [2]: https://cg.informatik.uni-freiburg.de/intern/seminar/gridFluids_fluid-EulerParticle.pdf "[PDF] Fluid Simulation For Computer Graphics: A Tutorial in Grid Based ..."
% [3]: https://rke.abertay.ac.uk/files/65237873/Parna_ShallowWaterEquations_PhD_2020_Redacted.pdf "[PDF] Shallow Water Equations in Real-Time Computer Graphics"
% [4]: https://en.wikipedia.org/wiki/Shallow_water_equations "Shallow water equations"
% [5]: https://www.cs.ubc.ca/~rbridson/fluidsimulation/fluids_notes.pdf "[PDF] FLUID SIMULATION - UBC Computer Science"
% [6]: https://arxiv.org/html/2405.20958v2 "Enhanced Level-Set Method for free surface flow applications - arXiv"
% [7]: https://en.wikipedia.org/wiki/Volume_of_fluid_method "Volume of fluid method"


\subsubsection{Numerical solvers}

Hydrodynamics are continuous but numerical solvers are commonly organized according to their discretization paradigm, with each category offering different trade-offs in stability, adaptivity, and computational cost. The principal families comprise grid-based Eulerian methods, particle-based Lagrangian techniques, hybrid schemes that combine grid and particle representations, and reduced models tailored for simplified scenarios or interactive control.

On one hand, grid-based Eulerian methods discretize the fluid domain on a fixed grid and approximate the incompressible Navier-Stokes equations via operator splitting or projection schemes. Historically, the Marker-and-Cell (MAC) approach established the staggered-grid representation \cite{Harlow1965}, storing velocities on cell faces and pressure at cell centers to enforce divergence-free constraints accurately in free-surface flows. Subsequent developments in graphics introduced semi-Lagrangian advection with implicit viscosity integration \cite{Stam1999}, which achieves unconditional stability permitting large time steps at the expense of increased numerical dissipation. Lattice Boltzmann methods (LBM) offer a mesoscopic viewpoint on fluid dynamics, leveraging local collision and streaming operations on a lattice to recover macroscopic behavior \cite{Chen1998}; they are notable for parallel efficiency and natural handling of moderate boundary complexity, though three-dimensional or highly irregular domains carry significant computational overhead. Finite-volume or finite-element variants appear in engineering CFD frameworks like OpenFOAM and can be adapted for graphics applications, but typically require careful optimization to remain feasible for large-domain or interactive contexts. % Free-surface representation in Eulerian solvers often relies on level-set or volume-of-fluid techniques, with volume loss and interface sharpness requiring corrective measures or hybrid augmentation.

On the other hand, particle-based Lagrangian methods represent fluid as discrete particles carrying mass, momentum, and other properties. Smoothed Particle Hydrodynamics (SPH) discretizes the continuum via kernel-weighted interactions among particles \cite{Monaghan2005}, naturally handling free surfaces and complex boundaries without explicit interface tracking; however, SPH demands high particle counts for fidelity, and stability challenges (e.g., tensile instability, pressure oscillations) necessitate specialized corrective formulations \cite{Monaghan2005,Koschier2022}. Pure Particle-In-Cell (PIC) schemes map particle information to a grid for pressure projection but tend to introduce substantial numerical dissipation through frequent interpolation \cite{Harlow1962}. The Fluid-Implicit Particle (FLIP) method mitigates dissipation by updating particle velocities using grid-derived increments rather than full replacements, thereby preserving kinetic energy and small-scale turbulence \cite{Brackbill1988}; FLIP is well suited for detailed free-surface phenomena such as splashes yet requires careful tuning near boundaries to avoid instability. Other particle variants, including Moving Particle Semi-implicit (MPS) methods, extend the Lagrangian paradigm with implicit pressure solves, but share similar demands for neighbor search and stabilization \cite{Koshizuka1996}.

Hybrid approaches aim to combine the stability and incompressibility enforcement of grid-based solvers with the adaptivity and natural boundary handling of particles. In common hybrid pipelines, particles carry momentum and advect passive quantities, while a background grid enforces the divergence-free condition via projection. Affine Particle-in-Cell (APIC) refines the transfer between particles and grid by representing particle velocity fields affinely, improving momentum conservation and reducing dissipation relative to FLIP \cite{Jiang2015}. Such hybrids leverage the strengths of each paradigm but introduce overhead in particle-grid transfers and require careful design to maintain consistency and numerical robustness in dynamic domains.

Reduced fluid models are employed when full three-dimensional simulation is unnecessary or prohibitively expensive. Depth-averaged shallow water equations capture large-scale horizontal flows over terrain under the assumption of negligible vertical variations; their lower-dimensional form yields significant computational savings and is widely used in real-time or large-domain scenarios where vertical vortices or breaking waves are not critical \cite{Vreugdenhil1994,Pan2012}. Potential flow approximations or other simplified models may be invoked for wave-like phenomena where vorticity and viscous effects can be ignored. Procedural or heuristic approximations, such as cellular automata-based flow or ad-hoc velocity fields, support highly interactive or stylized effects by sidestepping full PDE solves, at the cost of physical fidelity. These reduced methods serve both as initial terrain-shaping tools and as fallback options for real-time applications where performance constraints dominate.


\subsubsection{Solver characteristics}

Across all solver categories, certain numerical characteristics critically influence the realism, stability, and performance of fluid simulation. These include time integration and stability properties, treatment of free surfaces and boundary conditions, control of numerical dissipation to preserve detail, and computational efficiency and scalability strategies.

Time integration schemes must balance stability and accuracy. Explicit methods compute updates directly from known states but impose restrictive time-step limits proportional to grid spacing ($\Delta x, \Delta y, \Delta z$) or particle spacing in relation with the computed velocity, rendering fine resolutions costly. The CFL condition states that the dimensionless Courant number $C$ should not exceed a maximal value $C_{max}$ given depending on the PDE to solve (typically $C_{max} = 1$ for explicit integrations):
\begin{align}
    C = \Delta t \left( \sum_{i=1}^{n}{\frac{u_i}{\Delta x_i}} \right) \leq C_{max} \nonumber
\end{align}
Thus, we have for 2D and 3D explicit methods, greatly restricting the valid time-step $\Delta t$:
\begin{align}
    \Delta t_{\text{2D}} \leq \left( \frac{\Delta x}{u_x} + \frac{\Delta y}{u_y} \right) C_{max}
    \quad
    \Delta t_{\text{3D}} \leq \left( \frac{\Delta x}{u_x} + \frac{\Delta y}{u_y} + \frac{\Delta z}{u_z} \right) C_{max}
\end{align}

Implicit or semi-implicit treatments of diffusion or pressure terms permit larger time steps by solving linear or nonlinear systems ($C_{max} > 1$), as exemplified by semi-Lagrangian advection with implicit viscosity in Stable Fluids. Pressure projection typically entails solving a Poisson equation, for which direct solvers offer accuracy but scale poorly to large grids, while iterative solvers (e.g., conjugate gradient, multigrid) afford scalability at the expense of convergence concerns that must be managed via preconditioning or adaptive tolerance. Time-stepping strategies may incorporate adaptive substepping or projection frequency adjustments to maintain stability without excessive computation.

Accurate handling of free surfaces and boundary conditions is essential for plausible fluid behavior. Interface-capturing methods in Eulerian solvers, such as level-set or volume-of-fluid, track the free surface implicitly but can suffer volume loss or smearing; corrective reinitialization or particle-based markers near the interface often mitigate these deficits. Particle-based schemes represent the free surface naturally through particle distribution but require surface reconstruction for rendering and may exhibit clumping or void regions without density control. 

Solid boundary conditions in both paradigms demand robust collision and pressure treatment: Eulerian grids enforce no-slip or free-slip via ghost cells or immersed boundary techniques, while particles interact with geometry via repulsion forces or dynamic boundary particles. Hybrid methods must synchronize interface representation between particles and grid, ensuring that evolving boundaries remain consistent with divergence-free constraints. Dynamic domains, such as moving obstacles or adaptive meshes, necessitate regridding or particle reseeding procedures that preserve mass and momentum.

% \comment{Here, have the obstacle/terrain refining}

Numerical dissipation and detail preservation influence the visual richness of simulated flows. Semi-Lagrangian advection and grid-particle interpolation in PIC introduce artificial smoothing that dampens small-scale vortices and surface detail. Techniques to mitigate dissipation include the FLIP update, which applies only the change in grid velocity to particles, and higher-order advection schemes that reduce numerical diffusion. Vorticity confinement or turbulence-enhancement terms may reintroduce fine-scale structures lost to dissipation. In Eulerian solvers, divergence-free interpolation schemes and improved projection methods help maintain kinetic energy. SPH and other particle methods can preserve detail inherently but may suffer from noise or instability if neighbor sampling is irregular; stabilization strategies, such as density reinitialization, kernel correction, or pressure regularization, seek to retain fidelity without sacrificing stability. Machine learning-based super-resolution methods have recently emerged to reconstruct fine details atop coarse simulation outputs, but they require caution regarding generalization beyond training regimes.

% Computational efficiency and scalability determine the practicality of fluid simulation for large domains or high resolutions. Parallelization on modern hardware, especially GPUs, accelerates key operations such as advection, kernel evaluation in particle methods, and sparse linear solves in pressure projection. Adaptive or multi-resolution structures-octrees, quadtree grids, wavelet-based refinement-allocate computational resources to regions of interest (e.g., near free surfaces or obstacles) while coarsening elsewhere. Data structures like sparse voxel grids or tiled streaming support large-scale terrain scenarios by limiting memory footprints and enabling out-of-core processing. Hybrid CPU/GPU pipelines orchestrate workload distribution: the GPU handles intensive per-element computations, while the CPU manages dynamic data structures, I/O, and high-level control. Solver implementations often exploit asynchronous compute and memory transfers to hide latency. Precomputation or reduced-order surrogate models may be employed when repeated simulation is needed for interactive parameter tuning or inverse design tasks.

% Collectively, these numerical characteristics inform the design and selection of fluid solvers. Stability and time integration choices affect allowable time steps and influence dissipation; interface and boundary treatments impact visual fidelity and robustness; dissipation control methods preserve small-scale phenomena critical for realism; and efficiency strategies enable simulation at scales and resolutions aligned with application requirements. A thorough understanding of these attributes supports informed trade-offs when choosing between grid-based, particle-based, hybrid, or reduced models in terrain-related or visual-effect contexts.



% [1]: https://en.wikipedia.org/wiki/Fluid_animation "Fluid animation"
% [2]: https://pages.cs.wisc.edu/~chaol/data/cs777/stam-stable_fluids.pdf "[PDF] Stable Fluids - cs.wisc.edu"
% [3]: https://www.researchgate.net/publication/2486965_Stable_Fluids "(PDF) Stable Fluids - ResearchGate"
% [4]: https://dl.acm.org/doi/10.1145/311535.311548 "Stable fluids | Proceedings of the 26th annual conference on ..."
% [5]: https://www.scirp.org/reference/referencespapers?referenceid=2052162 "Chen, S. and Doolen, G.D. (1998) Lattice Boltzmann Method for ..."
% [6]: https://www.annualreviews.org/content/journals/10.1146/annurev.fluid.30.1.329 "LATTICE BOLTZMANN METHOD FOR FLUID FLOWS"
% [7]: https://cg.informatik.uni-freiburg.de/intern/seminar/animation%20-%20SPH%20survey%20-%202005.pdf "[PDF] Smoothed particle hydrodynamics"
% [8]: https://onlinelibrary.wiley.com/doi/abs/10.1111/cgf.14508 "A Survey on SPH Methods in Computer Graphics - Koschier - 2022"
% [9]: https://www.sciencedirect.com/science/article/abs/pii/0010465588900203 "Flip: A low-dissipation, particle-in-cell method for fluid flow"
% [10]: https://www.math.ucla.edu/~cffjiang/research/apic/paper.pdf "[PDF] The Affine Particle-In-Cell Method"
% [11]: https://www.researchgate.net/profile/Hadi_Muhammed/post/Any_suggestion_for_references_on_boundary_conditions_for_shallow_water_equations/attachment/59d63c9479197b80779998c3/AS%3A416283125403650%401476261039179/download/%5BC._B._Vreugdenhil.Numerical_Methods.pdf "[PDF] NUMERICAL METHODS FOR SHALLOW-WATER FLOW"
% [12]: https://en.wikipedia.org/wiki/Shallow_water_equations "Shallow water equations"


\subsubsection{Non-physical fluid simulations}

Recent advances in fluid simulation increasingly integrate machine learning techniques to accelerate or augment traditional numerical methods. Surveys of machine learning applications in computational fluid dynamics identify roles for data-driven methods, physics-informed models, and ML-assisted numerical solvers that improve convergence or predict intermediate quantities such as pressure fields and subgrid turbulence closures \cite{Huang2022NN}. Neural surrogates have been proposed to approximate costly components of the solver, yielding substantial speed-ups while maintaining acceptable fidelity in graphics contexts; such approaches often leverage convolutional networks for pressure projection or learn residual corrections to coarse simulations \cite{Tompson2017,Sousa2024}. Physics-informed neural networks and related frameworks enable embedding governing equations into the learning process, facilitating generalization and adherence to conservation laws, though their applicability to large-scale, real-time graphics remains an active research area \cite{Brunton2025}.

% Differentiable fluid simulation frameworks allow computation of gradients through the simulation pipeline, enabling inverse design and optimization of boundary conditions or initial configurations. Early work demonstrates the feasibility of training convolutional networks to solve pressure Poisson equations within an operator-splitting scheme, thereby accelerating Eulerian solvers while preserving incompressibility constraints ([arxiv.org][3], [en.wikipedia.org][5]). More recent efforts explore end-to-end differentiable pipelines that support gradient-based control of fluid features in animations or interactive tools. Challenges include maintaining stability of backpropagation through long time horizons and scaling differentiable components to three-dimensional domains with complex boundaries ([link.springer.com][7], [sciopen.com][8]).

Procedural or artistic fluid control has been an emerging topic for the last few decades, striking a balance between user interaction and plausibility of the fluid behaviour without full PDE solving. \cref{chap:semantic-representation} used the Kelvinlet paradigm \cite{DeGoes2017}, an analytic formulation for material deformation derived from the Kelvin's state using Green's function of linear elasticity of material, providing more control types than the fluid-specific Stokelets \cite{Chwang1976}. \citep{Wejchert1991} shows that these analytic fluid primitives are an efficient and lightweight approximation of highly controlled fluid simulations. Stroke-based fluid models are closer to the storyboard design of animation artists, making them intuitive to use, but are usually for subsets of frames of an animation, which become time consuming for the price of high-level control \cite{Xing2016,Patel2005,Yan2020b,Pan2013}. 

Procedural fluids are usually defined by noise functions to artificially introduce motion \cite{Bridson2007c}, but may also be added after computing an accurate fluid simulation to introduce more vortices \cite{Wang2025}. Vortices are usually too small for Eulerian simulations as the grid's cells width used may be larger than a vortex, and the dissipation can remove their presence. Editing velocity fields procedurally may also comes through the use of frequency-domain edition of forces \cite{Forootaninia2020, Tang2021}, blending and interpolation techniques to merge a predefined fluid with another \cite{Raveendran2014}, or even morphing techniques to deform a velocity or force field \cite{Lu2019,Raveendran2012,Flynn2019}

Procedural and artistic fluids can be included during or after a more computational expensive simulation, as they introduce variation on the many levels: force field and velocity field of Eulerian simulation methods, or the force, velocity and even position of simulation particles of Lagrangian methods \cite{Sims1990}.



% Procedural and artist-controlled methods continue to evolve, offering intuitive interfaces for shaping fluid behaviour without full PDE solves. Techniques based on primitive velocity fields, stokelets, or cellular automata provide stylized motion and rapid feedback suitable for interactive editing. These methods often embed simplified heuristics or small-scale physics-inspired adjustments to enhance plausibility while preserving responsiveness ([sciopen.com][8], [link.springer.com][7]). Hybrid pipelines that combine coarse fluid solves with procedural detail insertion or post-processing via neural super-resolution strike a balance between physical grounding and artistic control. The procedural layer may guide the large-scale flow pattern, after which a reduced or learned model infers fine-scale vortical structures consistent with the coarse simulation ([sciopen.com][8], [animation.rwth-aachen.de][9]).

% Multi-physics coupling beyond erosion remains a fertile area of research in graphics-oriented fluid simulation. Fluid-structure interaction methods enable simulation of fluid flows interacting with deformable or rigid bodies, supporting effects such as floating objects, cloth in water, or compliant obstacles. Thermal effects and buoyancy-driven flows are relevant for applications involving smoke or hot fluids, requiring coupling between temperature fields and momentum equations. Multiphase flows, including air-water interactions and bubbly phenomena, pose additional challenges in interface tracking and stability; recent works propose hybrid Eulerian-Lagrangian schemes or improved interface-capturing methods to handle such complexities in real-time or offline rendering contexts ([link.springer.com][7], [sciopen.com][8]). Advances in GPU hardware and parallel algorithms facilitate the implementation of these coupled simulations at higher fidelity, though achieving robust and efficient coupling across diverse phenomena continues to demand novel algorithmic designs.

% High-fidelity real-time fluid simulation benefits from hardware trends and algorithmic innovations aimed at reducing computational overhead while retaining visual richness. Sparse data structures, such as adaptive octrees or wavelet-based representations, concentrate computation near free surfaces or regions of interest, reducing overall workload. Asynchronous CPU/GPU pipelines and efficient memory streaming permit large-domain simulations with dynamic level-of-detail. Machine learning techniques further contribute by providing learned accelerations or detail synthesis, but their integration must be carefully managed to avoid artifacts outside training regimes. Emerging specialized accelerators and heterogeneous computing platforms offer new opportunities for real-time, high-resolution fluid simulation, motivating continued exploration of scalable algorithms and system-level optimizations ([link.springer.com][7], [arxiv.org][1]).


% [1]: https://arxiv.org/abs/2408.12171 "Recent Advances on Machine Learning for Computational Fluid Dynamics: A Survey"
% [2]: https://www.researchgate.net/publication/383308099_Recent_Advances_on_Machine_Learning_for_Computational_Fluid_Dynamics_A_Survey "Recent Advances on Machine Learning for Computational Fluid ..."
% [3]: https://arxiv.org/abs/1607.03597 "Accelerating Eulerian Fluid Simulation With Convolutional Networks"
% [4]: https://www.sciencedirect.com/science/article/pii/S004578252400389X "Enhancing CFD solver with Machine Learning techniques"
% [5]: https://en.wikipedia.org/wiki/Physics-informed_neural_networks "Physics-informed neural networks"
% [6]: https://www.annualreviews.org/doi/10.1146/annurev-fluid-010719-060214 "Machine Learning for Fluid Mechanics - Annual Reviews"
% [7]: https://link.springer.com/article/10.1007/s41095-023-0368-y "Physics-based fluid simulation in computer graphics"
% [8]: https://www.sciopen.com/article/10.1007/s41095-023-0368-y "Physics-based fluid simulation in computer graphics - SciOpen"
% [9]: https://animation.rwth-aachen.de/publication/0577/ "A Survey on SPH Methods in Computer Graphics"

% \subsubsection{Quality metrics and trade-offs}

% Evaluation of fluid simulation methods relies on both quantitative metrics and qualitative assessments of visual plausibility. Standard numerical benchmarks include scenarios such as dam break, vortex preservation, and flow around obstacles, which test the solver's ability to conserve mass, maintain stability, and capture transient phenomena accurately ([link.springer.com][7], [sciopen.com][8]). Quantitative measures often track divergence error, volume conservation over time, energy behavior (e.g., kinetic energy decay or preservation), and error relative to reference solutions or high-fidelity simulations. In particle-based methods, additional diagnostics such as particle density variation and neighbor distribution metrics inform stability and uniformity. Performance benchmarks measure time per simulation step or frame at varying resolutions and hardware configurations, comparing CPU, GPU, and hybrid implementations ([link.springer.com][7], [animation.rwth-aachen.de][9]). Reproducible benchmarking protocols are critical for fair comparison and for guiding optimization efforts.

% Trade-offs among stability, detail preservation, computational cost, and implementation complexity underpin the selection of a fluid solver for a given application. Explicit integration schemes offer simplicity but restrict time-step size due to CFL constraints; implicit or semi-implicit schemes allow larger steps but require solving linear or nonlinear systems, which may incur significant overhead. Semi-Lagrangian advection enhances stability yet introduces numerical dissipation that blurs fine-scale features; FLIP and affine-transfer variants mitigate dissipation but involve additional complexity in particle-grid coupling. Eulerian methods enforce mass conservation robustly but may struggle with intricate boundaries without sophisticated interface treatments; particle-based methods handle free surfaces naturally yet demand high particle counts for fidelity and may exhibit noise without stabilization. Reduced or analytical models deliver computational efficiency for large or interactive scenarios but sacrifice full three-dimensional effects. Machine learning-based accelerations can improve performance or enhance detail but necessitate careful validation to ensure generalization beyond training data. The choice of solver must align with application requirements: whether the priority is offline high-fidelity simulation, real-time responsiveness, or interactive control, and whether resources such as GPU hardware and development effort permit complex implementations. A clear understanding of these trade-offs, informed by quantitative benchmarks and qualitative evaluation, is essential for selecting or designing fluid simulation techniques suited to specific terrain-related or visual-effect contexts ([link.springer.com][7], [researchgate.net][2]).

\midConclusion

% Fluid simulation methods encompass a spectrum from fully three-dimensional Navier-Stokes solvers to reduced shallow-water or procedural models, each presenting distinct advantages and limitations. Classical grid-based Eulerian approaches provide robust enforcement of incompressibility and stable behaviour but may incur numerical dissipation and require elaborate interface treatments. Particle-based and hybrid schemes offer natural boundary handling and enhanced detail preservation, particularly for free-surface phenomena, yet demand careful stabilization and higher computational resources. Machine learning-assisted techniques and differentiable frameworks present promising avenues for accelerating simulations, enabling inverse design, and synthesizing fine-scale detail, but their reliability depends on training data quality and embedding of physical constraints. Procedural and heuristic methods support interactive or stylized applications where responsiveness supersedes full physical fidelity. Evaluation through standardized benchmarks and quantitative metrics is indispensable for understanding solver performance and guiding optimization.

% Looking ahead, continued integration of machine learning with fluid solvers is expected to yield more efficient and adaptive methods, provided that physics-informed architectures and rigorous validation guard against artifacts. Differentiable simulation pipelines will facilitate automated control and optimization workflows, fostering new interactive design paradigms. Advances in hardware, including GPUs and specialized accelerators, coupled with sparse and adaptive data structures, will enable high-fidelity real-time simulations on increasingly large domains. Improved interface-capturing and multiphase coupling techniques will extend the realism of complex phenomena such as fluid-structure interactions and multiphase flows. Finally, robust strategies for managing uncertainty and ensuring generalization of learned components are essential to deploy ML-augmented solvers in diverse scenarios. By synthesizing these developments and balancing trade-offs among stability, detail, efficiency, and control, researchers and practitioners can select or design fluid simulation techniques that meet the evolving demands of terrain-related applications and visual-effect pipelines, while also charting pathways for future innovation.

\midConclusion

Fluid simulations comes in a large variety of forms, each with advantages and inconvenients: accuracy, energy-preservation, user control, and boundary representations. Erosion on terrains, and more importantly submerged terrains, have to compute the motion of fluid in the environment, whever it represents air for wind-driven phenomena or water for hydraulic erosions. Erosion simulation algorithms proposed in literature do not propose alternative to the method-specific fluid simulation, constraining the user to only a subset of possibilities for a given terrain representation.