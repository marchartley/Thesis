
\subsection{Erosion processes}
Driven by an array of natural forces and processes, erosion varies significantly across environments, from the intense carving of river valleys to the subtle reshaping of slopes in arctic regions. In this section, we present the primary types of erosion (thermal, hydraulic, and wind erosion) alongside other significant processes that contribute to landscape change.  
Each erosion type not only influences distinct terrain forms but also varies in applicability depending on terrain representation in simulations. Notably, not all erosion types are easily adaptable to all forms of terrain representation due to inherent limitations in data resolution and computational methods.

\subsubsection{Gravity-driven erosion}
Gravity-driven erosion encompasses processes that involve the downslope movement of soil, rock, or debris due to the force of gravity. These processes, including landslides and talus slope formation, play a crucial role in reshaping landscapes by redistributing materials across slopes, valleys, and cliffs, influencing terrain stability and morphology.


\subsubsubsection{Thermal erosion}
\AltTextImage{
    % \wrapFig{freeze-thaw.jpg}{0.25}{fig:erosion-freeze_thaw}{Freeze-thaw process from \cite{Wang2017}}

    Freeze-thaw weathering, also known as frost wedging, frost shattering, or more simply thermal erosion in Computer Graphics, is a process that occurs when water infiltrates cracks and pores within rocks and then freezes (\cref{fig:erosion-freeze-thaw-example}). As the water freezes, it expands and exerts pressure on the rock, causing it to fracture and break apart over time. This cycle of freezing and thawing is especially prevalent in regions with large temperature variations between day and night or between seasons, such as alpine and polar climates. Over time, freeze-thaw weathering contributes to the breakdown of large rocks into smaller fragments, creating loose rock material that can accumulate and gradually move downslope.
}{freeze-thaw.jpg}{Freeze-thaw process \cite{Wang2017}}{fig:erosion-freeze-thaw-example}

Freeze-thaw weathering plays an important role in shaping landscapes, as it weakens rock faces and cliff edges, contributing to the formation of loose rock debris that eventually becomes part of other erosive processes, such as the development of talus slopes. This process, involving freeze-thaw cycles, can quickly fragment rock surfaces, with cumulative landscape impacts such as the formation of talus slopes observable over decades to centuries.

% \wrapFigR{talus.png}{}{}{} % {Mark A. Wilson (Department of Geology, The College of Wooster)}{fig:erosion-talus}
\AltTextImage{
    \subsubsubsection{Talus slopes}
    Talus slopes, also known as scree slopes, are accumulations of loose, angular rock debris at the base of cliffs, steep slopes, or mountainous areas. These slopes form as fragments of rock break off due to weathering processes such as freeze-thaw, and gravity pulls them downslope, where they accumulate in a cone-shaped deposit (\cref{fig:erosion-talus-example}). Talus slopes are common in high-altitude or cold regions where physical weathering of rock faces is intense, and they contribute to the visual ruggedness of mountainous landscapes. Formed by the accumulation of rock debris from higher elevations, talus slopes develop gradually as materials accumulate, influencing terrain over centuries.
}{talus.png}{Talus cones on north shore of Isfjord, Svalbard, Norway - Photo credit: Mark A. Wilson}{fig:erosion-talus-example}

The procedural terrain generation domain blurs the definition of thermal erosion introduced by \cite{Musgrave1989}, simplifying it to a material stabilisation problem. 

\AltTextImage{
    \subsubsubsection{Landslides}
    Landslides encompass a range of processes where rock, soil, or debris moves downslope due to gravity. These events vary in scale and speed, ranging from rapid, sudden rockfalls to slow, gradual soil creep. They can be triggered by factors such as heavy rainfall, seismic activity, or thawing permafrost, which destabilise slopes and initiate movement. Key types of landslides include:
    \begin{Itemize}
        \Item{Rockfalls:} Sudden detachment of rock from steep faces, often triggered by weathering, freeze-thaw cycles, or seismic activity, leading to rapid downslope movement. 
        \Item{Soil creep:} Slow, continuous downslope movement of soil and rock, caused by repeated cycles of expansion and contraction due to changes in moisture and temperature, often imperceptible over short timescales.
        \Item{Mudflows and debris flows:} Rapid flows of water-saturated soil and debris, typically triggered by heavy rainfall or snowmelt, which transport large volumes of material downslope in a short period.
    \end{Itemize}

    Landslides are a major force in landscape evolution, rapidly reshaping terrain and redistributing materials across slopes and valleys. Triggered by factors such as heavy rain or seismic activity, landslides can reshape landscapes almost instantaneously, though their frequency and impact may vary widely.
}{earthflow-erosions.jpg}{Different landslide processes.}{fig:erosion-landslides_processes}

\smallConclusion

Realistic simulation of these effects can be achieved by applying multi-flow computational fluid dynamics on the internal rock fragments or sediments, considering them as fluid particles with an evaporating viscosity \cite{Feng2024,Harmon2001,Lenaerts2009} or as inelastic frictional spheres \cite{Walton1993}, but at the cost of very high computation times.

In procedural terrain generation, however, freeze-thaw, talus slopes, and landslides are all similarly considered and generalised as "thermal erosion". The use of "thermal erosion" or "thermal weathering" in procedural generation, introduced by \cite{Musgrave1989}, is a misnomer initially used in opposition to "hydraulic erosion". However, the effect of freeze-thaw can be seen as a reduction in the ground resistance to detachment, as the critical shear stress is highly reduced in cold weather regions, while talus slopes and landslides involve finer soil particles or the mixing of liquid fluids, introducing viscosity into the system, resulting in a quite similar output \cite{Hudak2011}.

Thermal erosion can be simulated by redistributing rock fragments or particles to accumulate at the base of cliffs or steep inclines, taking into account only the surface of the terrain. This effect can be achieved by applying gravity-based algorithms that allow loose materials to fall and settle, forming natural slopes of debris at the base of rocky terrain \cite{Musgrave1989,Jones2010}. Initially proposed for discrete height field terrain representations, the thermal erosion simulation proposed by \cite{Musgrave1989} and improved for GPUs by \cite{Jako2011} iteratively displaces a small amount of the height at each cell of the terrain and redistributes it to its direct neighbours if a repose angle is not satisfied. \cite{Jones2010} adapts this definition of thermal erosion for density-voxel representations, and \cite{Benes2001} for layered representations. The importance of keeping track of scree areas allows for more detailed modelling \cite{Peytavie2009a,Paris2020}.

\subsubsection{Hydraulic erosion}
\begin{figure}
    \autofitgraphics[width = 0.8 \linewidth]{hydraulic_erosion.pdf}
    \caption{Hydraulic erosion is caused by the friction of water displacing sediments on a slope.}
    \label{fig:erosion-hydraulic-erosion}
\end{figure}

Hydraulic erosion is the process by which moving water dislodges and transports soil, sediment, and rock from the Earth's surface. Occurring in multiple forms, including river-based, rainfall-induced, and coastal erosion, hydraulic erosion is driven by factors such as water velocity, volume, and surface composition. This process plays a primary role in reshaping landforms, forming valleys, river channels, and coastlines, and significantly contributes to sediment redistribution in terrestrial and coastal environments.

% \AltTextImage{
    \subsubsubsection{Fluvial erosion}

    Fluvial erosion is the process by which rivers and streams reshape the landscape by eroding, transporting, and depositing sediment. This phenomenon occurs as the kinetic energy of moving water exerts mechanical forces on the riverbed and banks, dislodging soil, rock, and sediment particles (\cref{fig:erosion-hydraulic-erosion}). The intensity of fluvial erosion is influenced by factors such as water velocity, discharge (volume of water flowing per unit time), channel slope, and the composition of the riverbed and banks.
% }{river_erosion.pdf}{The erosion on the river bank is the combination of two processes: the detachment of the bottom of the sides, then the break of the upper coasts.}{fig:erosion-river-erosion-profile}

In steep, fast-flowing sections of a river, higher water velocities generate turbulent flow, which increases the river's capacity to dislodge and carry large particles. These particles, including gravel and pebbles, collide with the riverbed in a process called abrasion, grinding and wearing down the bedrock over time. Additionally, water exerts direct hydraulic pressure, especially in areas where currents are swift, prying apart rocks and sediment through hydraulic action. This is especially effective in widening channels and undercutting banks.

Fluvial erosion processes contribute to the dynamic reshaping of river channels, forming distinct landforms such as V-shaped valleys, canyons, and river meanders. Over time, rivers naturally balance their erosive energy with sediment transport and deposition, forming floodplains where sediment is deposited during seasonal overflows. In meandering rivers, erosion typically occurs on the outer curves of bends, where flow velocity is highest, while sediment deposition takes place on the inner curves, forming point bars. This continual interaction between erosion and deposition drives the lateral migration of meanders, altering the river's course across the landscape. The river's competence, or its ability to transport particles of a certain size, depends on the flow's velocity and discharge. During periods of high flow, such as after heavy rainfall or snowmelt, rivers gain greater erosive power, enabling them to transport larger particles and increase their erosion rates. River systems reshape landscapes methodically over years to millennia, deeply engraving river paths and altering regional topographies.



\AltTextImage{
    \subsubsubsection{Rainfalls}
    % \wrapFig{rillsGullies.jpg}{0.25}{fig:erosion-rills_gullies}{From \cite{Geertsema2010}}

    Rainfall-induced hydraulic erosion begins as raindrops strike exposed soil surfaces, causing splash erosion, where particles are dislodged and displaced by the impact of individual raindrops, creating tiny craters \cite{Valette2005,Li2024}. As rainfall accumulates and flows overland, it transitions into sheet erosion, where a thin layer of water, known as sheet flow, moves across the land surface. This process is often intensified on sloped terrain, where the water gains momentum as it descends, picking up and carrying loose particles downslope. Sheet erosion can remove a uniform layer of soil across a large area, gradually depleting soil fertility and weakening the structure of the soil surface.
}{rillsGullies.jpg}{Rill and gully erosion along the Chilcotin River \cite{Geertsema2010}}{fig:erosion-rills_gullies}

On steep or prolonged slopes, sheet flow may concentrate into small channels, initiating rill erosion \cite{Gatto2000}. Rills are narrow, shallow channels that cut into the soil as water flow converges, carving miniature stream-like paths down the slope visible in \cref{fig:erosion-rills_gullies}. As rills deepen and widen, they can evolve into larger channels in a process called gully erosion, where channels become deep and wide enough that normal agricultural or natural processes cannot easily repair them. Gullies disrupt the landscape, fragmenting ecosystems, and accelerating the removal of topsoil.

The extent of erosion depends on factors such as rainfall intensity and duration, soil type, vegetation cover, and slope steepness. Sandy soils, for instance, are more prone to erosion due to their low cohesion, whereas clay-rich soils, while more resistant to initial splash erosion, are highly susceptible to rill and gully formation once water begins to concentrate. Splash erosion and sheet erosion can cause significant soil degradation and landscape changes over months to decades.

\smallConclusion

Hydraulic erosion is the main focus of study in terms of erosion for terrain generation as its effects on the surface of terrains are visible on a large scale and are the most stable as we consider a constant amount of rain at every point.

In procedural terrain generation, fluvial erosion is focused on the generation of rivers and cascades defined as feature-curves \cite{Emilien2015}. First, the path of the river is drawn either by the user \cite{Hnaidi2010} or through simulation \cite{ParisThesis}. Second, the shape of the rivers' beds are modelled \cite{Genevaux2013}. It is then finally possible to provide a geometric representation of the water surface to take into account flow rate, depth, obstacles, and user-defined details \cite{Peytavie2019}.

The computation of rivers' properties is defined by understanding the flow of water, which is directly related to the amount of rainfall at each point of the terrain \cite{Kelley1988}, then approximated by simplified Shallow Water modelling \cite{Mei2007}, but is mainly achieved in acceptable computation time by considering all rivers as a drainage network, an oriented graph of rivers as introduced in \cite{Roudier1993}. Recent works focus on the acceleration of the computation of these graphs and the drainage area associated at each point \cite{Cordonnier2016,Schott2023}.

In most works, the amount of water falling at each point of the terrain can be taken into account. However, this strategy is divided among works that consider that rain falls uniformly on the whole terrain or using a random noise function, that rain is more present in regions of high altitude \cite{Neidhold2005}, or can use a more accurate weather simulation to define areas more prone to rainfall \cite{Wojtek2022}.

As the motion of water is easier to model on the surface of the terrain, almost all algorithms consider the terrain with a 2.5D representation. On a large scale, this assumption is beneficial but limits their scope to this type of representation, making them unavailable to volumetric models. Particle-based methods are then proposed to overcome this issue for smaller-scale terrains, using 3D Eulerian fluid simulations on voxel terrain representations \cite{Benes2006} or Lagrangian simulations on TIN terrains \cite{Kristof2009}, at the expense of much higher computational time.

\AltTextImage{
    \subsubsection{Chemical erosion and caves}

    Chemical erosion, also known as chemical weathering, involves the chemical reactions between water and rock that lead to the dissolution and alteration of minerals within the rock. This process is especially significant in river and coastal environments, where water, often containing dissolved carbon dioxide, interacts with rocks such as limestone and dolostone to form weak carbonic acid. This acid reacts with carbonate minerals, gradually breaking down the rock structure through a process called dissolution. Over time, this breakdown weakens rock formations, making them more susceptible to mechanical erosion by water.
}{seaCaveArch.jpg}{Akun Island basalt sea cave - Photo credit: Steve Hillebrand}{fig:erosion-corrosion-erosion}

Chemical erosion is particularly influential in the formation of karst landscapes, where the dissolution of carbonate rocks creates unique features such as caves, sinkholes, and underground drainage systems. As acidic water seeps into fractures within the rock, it slowly enlarges these cracks, leading to the creation of hollow spaces and intricate cave networks. In addition to carbonic acid, other dissolved ions, such as sulphur and organic acids, can also contribute to the chemical weathering of rocks in various environments. The dissolution of soluble rocks such as limestone forms extensive karst landscapes, including networks of underground caves, over thousands to millions of years.

Sea caves can form through the mechanical force of hydraulic action as waves continuously impact the shore. These sea caves develop along coastlines with cliffs, where wave energy focuses on weak points in the rock, such as fractures or softer rock layers. Over time, the pressure from waves and tides pries apart rock fragments, carving out hollow spaces within the cliffside or even a whole arch as presented in \cref{fig:erosion-corrosion-erosion}. Formed by the erosive power of waves against rock cliffs containing zones of weakness, development can be observed over hundreds to thousands of years.

% This hydraulic action often works in conjunction with other processes, such as abrasion, where sediment carried by waves further grinds down the rock surfaces. Sea caves are commonly found in areas with high tidal energy, with examples such as the Blue Grotto in Malta and the sea caves along California's coastline.
In desert environments, caves can also form through aeolian erosion, where wind-driven sand particles abrade rock surfaces. These aeolian caves are typically found in sandstone or other softer rocks, where strong, consistent winds gradually wear away the rock. Unlike chemical or hydraulic caves, aeolian caves are usually shallower and smaller, as the erosive force of wind is less powerful than water or acid-driven processes. However, these caves add unique features to desert landscapes, creating sheltered hollows that sometimes serve as habitats for desert wildlife. Created by wind erosion primarily in softer rock formations, these caves can develop over centuries, depending on the consistency and strength of wind.

% An example of aeolian caves can be seen in the Navajo Sandstone formations of the southwestern United States, where wind erosion has carved intricate cave-like recesses and arches.
\smallConclusion

In procedural terrain generation, chemical erosion has been rarely studied, as this interaction mostly occurs in shallow to deep waters and requires a volumetric representation, while most terrain generation algorithms are tailored to aerial landscapes and are limited by 2.5D representations. However, using 3D cellular automata can provide an explicit representation of this effect \cite{Menshutina2020} (and can be extended to a simulation of coastal erosion \cite{Hawick2014}). An approximation of this phenomenon is proposed in \cite{Beardall2007,Jones2010} on voxel grids by considering that the most vulnerable voxels of the terrain are the ones that have the most neighbours with air voxels, in a very similar manner to the computation of voxel-grid ambient occlusion. The simulation of sea caves for implicit terrains has been proposed by \cite{Paris2018} by considering a resistance function in the bedrock, usually a function of height, and adding spherical holes in the construction tree at the least resistant areas of the terrain. The inverse percolation method enables the simulation to dig longer cavities, resembling coastal karsts.

The generation of complete karst networks has been studied in different domains, such as hydrogeology, where the generation of conduits based on the hydrologic network and soil properties allowed the generation of the first examples of karst systems \cite{Jaquet2004,Pardo2012,Pytel2015}. However, these methods do not allow user control and the computation time is unfit for terrain generation. More recently, \cite{Paris2021} proposed to describe the karst network as a graph whose nodes are sampled points in the terrain and for which the shortest path (using soil resistance as the metric) between each point provides the path of each tunnel in the network. An improved algorithm, with more geological parameters, proposes more geological plausibility \cite{Gouy2024}.


\subsubsection{Wind erosion}

\begin{figure}
    \autofitgraphics[width = 0.8 \linewidth]{wind_process.png}
    \caption{Wind erosion includes the lifting of the sand, the transport through the wind, and its deposition \cite{Paris2019b}.}
    \label{fig:erosion-wind-erosion}
\end{figure}

Aeolian erosion, also known as wind erosion, is the process by which wind transports, dislodges, and deposits particles of soil, sand, and rock, particularly in arid and semi-arid regions where vegetation is sparse. This form of erosion is driven by the movement of loose particles across the surface and the abrasive impact of wind-driven sand. Aeolian erosion leads to distinctive landforms that shape desert landscapes, coastal areas, and regions downwind of deserts. Wind erosion typically occurs through three main processes illustrated in \cref{fig:erosion-wind-erosion}: deflation (removal of fine particles), saltation (bouncing movement of medium-sized particles), and abrasion (wearing down of rock surfaces by wind-driven particles).

\AltTextImage{
    \subsubsubsection{Sand dunes}
    One such feature is sand dunes, mounds or ridges of sand formed by the accumulation of wind-transported particles (\cref{fig:erosion-sand-dunes}). As wind moves sand across a surface, obstacles such as rocks or vegetation can slow the particles, causing them to settle and accumulate into dunes. The shape and type of dune depend on wind direction, strength, sand availability, and landscape features \cite{Paris2019b}. Common types of dunes include barchan dunes, which are crescent-shaped with tips pointing downwind; transverse dunes, long ridges perpendicular to the wind; and star dunes, which have multiple arms formed by shifting winds. These dunes are dynamic, migrating over time as wind continues to move sand particles, contributing to the constantly evolving desert landscape.
}{sandDunes.jpg}{Sand dunes \cite{Sun2006}}{fig:erosion-sand-dunes}

\AltTextImage{
    \subsubsubsection{Yardangs}
    Yardangs are elongated ridges formed where softer material is eroded faster than more resistant rock. The wind-carved ridges align with the prevailing wind direction, creating streamlined shapes with steep sides facing into the wind and gentler slopes on the leeward side as visible in \cref{ig:erosion-yardang}. Yardangs vary widely in size, from small ridges a few metres long to massive formations stretching for kilometres, as seen in desert regions of Iran and Egypt. They illustrate the power of wind erosion in shaping landscapes over long periods, with their formation largely dependent on rock hardness, wind intensity, and particle size. These streamlined rock formations are carved by persistent wind eroding softer material faster than harder material, typically forming over centuries.
}{Yardang.jpg}{A yardang near Meadow, Texas - Photo credit: U.S. Departement of Agriculture}{fig:erosion-yardang}

\AltTextImage{
    \subsubsubsection{Ventifacts}
    Ventifacts are rocks that have been shaped and polished by the abrasive action of wind-driven sand. These rocks typically exhibit flat, smooth surfaces that are often oriented in the same direction as the prevailing winds. Their formation occurs predominantly in arid environments where loose sand is available to act as a natural sandblasting tool. This erosive process highlights the power of wind as a geomorphic agent, capable of sculpting rocks into distinctive shapes over time as illustrated in \cref{fig:erosion-ventifact}. Rocks shaped by wind-driven sand, ventifacts can take decades to centuries to form, depending on local wind conditions and rock exposure.
}{ventifact.jpg}{Ventifact at White Desert National Park, Egypt - Photo credit: Christine Schultz}{fig:erosion-ventifact}

\smallConclusion

Wind erosion shifts material through wind force, notably impacting areas with fine surface particles such as deserts.  
Sand dune generation has been modelled on discrete height fields \cite{Roa2004} by mimicking sand's wind-driven trajectory and using thermal erosion to correct the slope. These algorithms consider the wind coming from a unique direction, but with a little more complexity in the wind simulation, new dune formations can emerge. \cite{Paris2019b}, adapted for layer-based representations in which the layers of sand and the layers of bedrock are defined, includes the definition of a wind simulation consisting of warping a uniform flow with the terrain slopes. This allows for the fast generation of large-scale desertic landscapes. More recently, \cite{Rosset2024} associated a fine-resolution 3D fluid simulation for wind modelling, enabling the simulation of dune formation at a small scale with high fidelity, albeit at a higher computational cost due to fluid dynamics.

% While the presented methods take into account the effect of an obstacle on a grain of sand, few consider the effect of the sand on obstacles. For the same reasons as in the simulation of chemical erosion, the abrasive factor of sand and wind is left behind. The formation of yardangs has not been studied to the best of my knowledge. 
% Although not exactly targeted at ventifacts, but at the generation of hoodoos, \cite{Beardall2007} and \cite{Jones2010} propose a density-voxel-based method for computing the weathering of rocks while taking the voxel exposure to air and resistance into account. The inclusion of an additional force field for computing the displacement of the detached matter could be the only missing feature for ventifact generation.
While most aeolian terrain models account for the effect of wind on sand transport, far fewer simulate the feedback of transported material abrading obstacles. Early voxel-based approaches such as \cite{Beardall2007} and \cite{Jones2010} applied analytical heuristics (respectively spheroidal weathering and curvature-driven removal) to progressively reshape volumetric rock, in the spirit of exposure-based methods on voxels and implicit fields \cite{Paris2019a}. By contrast, \cite{Krs2020} adopt a physics-inspired formulation: they convert polygonal meshes into a layered-terrain representation, erode its layer stacks when flow stresses exceed material resistance, advect the detached mass as particles within an SPH fluid, and finally redeposit it into the volume. This particle-fluid coupling parallels hydraulic erosion work \cite{Kristof2009}, but targets wind-driven abrasion and deposition. Despite their different strategies (analytic voxel criteria versus particle-fluid transport) all three approaches ultimately rely on volumetric discretizations to modify geometry. Taken together, they outline a design space where voxel analytics offer controllable stylization, while physics-based enables obstacle-sand interactions. %; yet comprehensive models of aeolian landforms such as yardangs remain largely unexplored in computer graphics.



% In multiple works, the simulation of sand and snow is interchangeable as both are considered small particles whose density varies slightly.

\subsubsection{Glacial erosion}
Glacial erosion is a dominant geomorphic force in cold regions, characterised by massive ice sheets and glaciers that sculpt the Earth's surface over thousands of years. As glaciers advance and retreat, they transform landscapes through two primary mechanisms: plucking and abrasion.

Plucking occurs when glaciers freeze to the bedrock and, as they move, exert tremendous force that tears away blocks of rock. This process is facilitated by water that infiltrates cracks in the rock, freezes, and expands, helping to dislodge rock pieces as the glacier flows. Abrasion happens concurrently as rocks and debris embedded in the ice act as sandpaper, grinding and smoothing the rock surface beneath the glacier. This abrasion not only polishes the rock but also carves deep grooves and striations (linear marks that align with the direction of ice movement).

These erosive processes give rise to distinctive glacial landforms. U-shaped valleys, unlike the V-shaped valleys formed by river erosion, are wide and deep with a flat bottom, shaped by the broad, sweeping movement of glacier ice. Fjords are deep, narrow inlets of the sea set between high cliffs, created by the deepening of U-shaped valleys by glaciers that then become submerged as sea levels rise. Cirques are amphitheatre-like hollows situated at the heads of glacial valleys, formed by the erosion caused by the rotational movement of ice within them. Moraines are accumulations of dirt and rocks scraped up and deposited by moving glaciers, typically appearing as ridges along the sides of glaciers or as mounds of debris left behind after a glacier retreats.

The timescale of glacial processes spans over thousands to millions of years, allowing glaciers to leave a lasting impact on the landscape.

Few works have focused their effort on the understanding and modelling of glacial erosion phenomena. In these conditions, \cite{Argudo2020} consider the glacier as a fluid with no inertia, called the Shallow-Ice Approximation, taking only the terrain surface gradient and ice thickness to deduce the velocity of an ice column. The shear stress applied beneath the ice sheet is directly computed by the estimation of the ice column and its velocity. This method results in V-shaped valleys, but when integrating more factors into the simulation, such as debris flow, fluvial erosion and talus slopes, the improved method presented in \cite{Cordonnier2023} generates a large variety of features present in glacial landscapes.

\subsubsection{Volcanic activity}
Volcanic activity is a powerful tectonic process that alters landscapes by creating new landforms through the eruption of magma from the Earth's mantle \cite{Ramalho2013}. Volcanoes form primarily at tectonic boundary zones, either where plates diverge, allowing magma to rise and fill the gaps, or where one plate subducts beneath another, melting into magma due to high pressure and temperature.

The surface changes caused by volcanic activity are substential and varied. When a volcano erupts, it deposits layers of lava that solidify into new rock formations, gradually building the volcanic cones and mountainous structures that are characteristic of volcanic islands and mountain ranges \cite{Woodroffe2003}.

\AltTextImage{
    Pyroclastic flows are another aspect of volcanic activity that dramatically changes the terrain. These fast-moving currents of hot gas and volcanic matter can race down the sides of a volcano, destroying everything in their path and laying down thick deposits that can form natural barriers or change drainage patterns by damming rivers and creating lakes almost instantaneously (see \cref{fig:erosion-volcano-python-fournaise}).

    Volcanic activity also leads to the formation of calderas, large depressions that occur when the summit of a volcano collapses into the emptied magma chamber below after an eruption. Calderas can be several kilometres in diameter and significantly alter the local landscape.
}{Python-fournaise-erupt.jpg}{Piton de la Fournaise, Réunion Island, in eruption - Photo credit: Richard Roquejoffre}{fig:erosion-volcano-python-fournaise}



\subsubsection{Biological processes}
Biological processes play a crucial role in shaping terrestrial and aquatic environments through mechanisms that either promote or inhibit erosion. Known collectively as bioerosion, the activities of various organisms have significant impacts on the stability and durability of habitats.

% \subsubsubsection{Animals}
In terrestrial ecosystems, animals contribute to erosion. Burrowing animals, such as rodents, moles, and earthworms, play a significant role by disturbing soil structures. Their burrowing actions create tunnels and voids in the soil, increasing porosity and altering water infiltration rates, which can accelerate soil erosion under certain conditions. These disturbances also bring subsurface soils to the surface, making them more vulnerable to wind and water erosion. 

Grazing animals including deer, cattle, and sheep can have dual effects on the landscape. Indeed, these animals reduce vegetation cover, which increase the soil's exposure to erosive forces such as rain splash and surface runoff. Additionally, their trampling compacts the soil, reducing its porosity and permeability, which increases runoff and potential erosion. 

Very few mathods have focused on the effects of desire paths from human trailing and vehicle paths \cite{Cordonnier2018,Jaiswal2019}, human footsteps \cite{Alvarado2024} or animal paths \cite{Ecormier-Nocca2021} as the resulting spatial scale may become very small in comparison to other types of erosion previously cited.

In coastal and marine environments, bioerosion is a significant force affecting rocky shores and coral reefs. Organisms such as barnacles, sea urchins, and various molluscs attach to rocks and coral structures, physically breaking down these materials as they feed. Parrotfish, for example, erode coral structures by scraping off coral polyps to ingest the algae living on them. Over years, this contributes to the gradual wearing down of coral reefs, influencing the structural complexity and ecological dynamics of these habitats.

% \subsubsubsection{Vegetation}
Vegetation profoundly influences erosion control, hydrological cycles, and soil stability across diverse landscapes. The mechanisms through which plants impact erosion are varied and complex, ranging from the stabilisation provided by root systems to the protection offered by canopy cover.

Root systems form the foundation of soil stability. Deep-rooted plants, including many trees and shrubs, secure themselves deep within the soil, anchoring it firmly and preventing landslides and soil slips on slopes. These roots extend into subsoil layers, binding the soil particles together and maintaining the integrity of the slope beneath the surface. In contrast, grasses with fibrous root systems spread a dense network of fine roots through the topsoil, effectively creating a mat that holds the soil in place. This root mat not only prevents topsoil erosion but also enhances soil structure, promoting water infiltration and reducing runoff.

The canopy of vegetation acts as a first line of defence against the erosive force of rain. By intercepting raindrops, tree canopies distribute the water's impact over a wider area and decrease the velocity at which water hits the ground. This dispersion reduces the kinetic energy of raindrops, which lessens their ability to dislodge soil particles. Beneath the main canopy, smaller shrubs and herbaceous plants catch residual droplets, providing a secondary layer of protection that minimises splash erosion and soil displacement.

The presence of vegetative cover significantly modifies surface runoff dynamics. Dense plant life adds physical barriers that increase the land's surface roughness and impede the flow of runoff water. This roughness slows down water movement, allowing more time for the water to seep into the soil rather than washing it away. Additionally, the process of transpiration, where plants absorb water from the ground and release it into the atmosphere, reduces the soil's moisture level, thereby increasing its capacity to absorb further rainfall. This cycle plays a critical role in maintaining soil hydration balance and reducing the volume of runoff during precipitation events.

\midConclusion

In conclusion, various erosion processes occur on terrains over very different timescales. However, each procedural generation algorithm is mimicking a small number of these processes at once. The choice of the terrain representation is a limiting factor in the possibilities offered to the developer. However most of the erosion processes are very similar and are composed of three main steps, even if their names differ depending on the phenomenon: the detachment of matter, its displacement, and the deposition in smaller sediments. The displacement is guided by the environment in which the erosion occurs. It may be only the force of gravity, the water flow, the wind or the trajectories of glaciers.
