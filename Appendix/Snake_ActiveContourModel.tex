\chapter{Snake - Active Contour Model}

We want to symbolize a dead coral area as a \gloss{EnvObj} "reef." To do this, the coral objects continuously deposit a quantity of "dead coral" material. This material, stored in a discrete scalar field, contains high intensities where a reef should supposedly exist. 
We need to draw a curve to represent this new object. The constraints of this curve are that it must pass through the points of highest intensity while maintaining a given length.

The Snake algorithm, or Active Contour Model, approaches this application. The algorithm proposes to give an energy to the curve, which then tries to minimize it through gradient descent.

The energy, in the initial paper, is defined by
\begin{align}
    \Esnake^{*} = \int \limits _{0}^{1} { \Esnake(\mathbf {v} (s))\,ds } = \int \limits _{0}^{1} { \Eintern (\mathbf {v} (s)) + \Eextern (\mathbf {v} (s)) } \,ds
\end{align}

The internal energy represents the properties of the curve while the external energy represents properties of the field it lies in. In the original paper they are described as: 
\begin{align}
    \Eintern &= \alpha \Econt + \beta \Ecurv \\
    \Eextern &= \gamma \Eimage
\end{align}

The continuity cost $\Econt$, originally defined as the minimization of the spacing between the points $\left\|{\frac {d{\bar {v}}}{ds}}(s)\right\| ^{2}$, does not make much sense in the discrete form of the algorithm. In its discrete form, we seek to maintain a regular interval between the points by applying $\Econt = \left(\tilde{d} - \left\|p_i - p_{i-1} \right\| \right)^2$ with $\tilde{d}$ being the average distance between each point.

The curvature cost $\Ecurv$ seeks to minimize the oscillations of the curve and can thus be defined as the squared second derivative $\left\|{\frac {d^{2}{\bar {v}}}{ds^{2}}}(s)\right\| ^{2}$.

The discrete form $\Ecurv^{*} = \left\| p_{i-1} - 2 p_i + p_{i+1} \right\| ^2$ is not necessary with splines, due to their closed form.

The image cost $\Eimage$ tries to attract the points of the curve to a local maximum of the image gradient. It is defined as $\Eimage = - \left\| \nabla I \right\| $.

We want to see our curve maintain a given length $L$. We then modify the formulation of the continuity cost to become $\Econt = \left(\tilde{l} - \left\|p_i - p_{p-1} \right\| \right)^2$ with $l = \frac{L}{n - 1}$, knowing $n$ is the number of vertices of the curve. Additionally, we want a curve that follows points of high intensity rather than the gradient, which leads to modifying the image cost $\Eimage = -I$.

The calculation of the gradient $\nabla \Esnake$ remains trivial in parts:
\begin{align}
    \frac{\partial \Eimage}{\partial p_i} &= - \nabla I(p_i) \\
    \frac{\partial \Ecurv^{*}}{\partial p_i} &= -\frac{2 \left( p_{i-1} - 2 p_i + p_{i+1} \right) }{ \left\| p_{i-1} - 2 p_i + p_{i+1} \right\| } \\
    \frac{\partial \Econt^{*}}{\partial p_i} &= 2 \left(l - \left\|p_i - p_{i-1} \right\| \right) \cdot \frac{p_i - p_{i-1}}{ \left\| p_i - p_{i-1} \right\| }
\end{align}

We then have
\begin{align}
    \Esnake &= \alpha \left(l - \left\|p_i - p_{i-1} \right\| \right)^2 + \beta \left\| p_{i-1} - 2 p_i + p_{i+1} \right\| ^2 - \gamma I \\
    \nabla \Esnake &= 2 \alpha \left(l - \left\|p_i - p_{i-1} \right\| \right) \cdot \frac{p_i - p_{i-1}}{ \left\| p_i - p_{i-1} \right\| } - \beta \frac{2 \left( p_{i-1} - 2 p_i + p_{i+1} \right) }{ \left\| p_{i-1} - 2 p_i + p_{i+1} \right\| } - \gamma \nabla I(p_i)
\end{align}

It should be noted that the calculation of $\Econt^{*}$ uses the distance $\left\|p_i - p_{i - 1} \right\|$. For $i = 0$, the distance $\left\| p_i - p_{i + 1} \right\|$ is used.

If all the points of the curve are at a distance greater than $l$, the optimization will push each of these points to get closer to its predecessor. Point $p_0$, itself, will move very little, so the entire curve aligns towards point $p_0$. By using the distance to the successor $\left\| p_i - p_{i+1} \right\|$, the curve moves towards point $p_N$.
It is then possible to converge towards the median point by alternating the use of the distance with the predecessor and with the successor, at the cost of slower convergence.

The active contour model algorithm is highly sensitive to the initial curve placement. In cases where a portion of the curve is in an area with a very low gradient on $\Eimage$, the vertices of the curve will simply optimize $\Eintern$, resulting in a straight segment in a low-intensity area, while the rest of the curve optimizes correctly.

To mitigate this problem, we propose adapting the Snake algorithm into Caterpillar: throughout the gradient descent, the target length $L$ is artificially reduced and then increased. In this way, a portion of the curve blocked in a region without possible optimization on external energy will be attracted by the optimized curve until it falls on a strong gradient. The dead portion can take the place of optimized vertices. By returning the target length $L$ to its initial value, the optimization continues with fewer vertices in the dead zone. Repeating this process gradually brings all points into an optimizable area. However, a too-rapid change in the target length can prevent the vertices from optimizing $\Eextern$ by amplifying $\Eintern$ too much. Additionally, this algorithm can lead to numerical errors and slower convergence.
