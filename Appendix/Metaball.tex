\chapter{Computation of a metaball}
\label{sec:erosion-appendix_metaball}

\shortAbstract{
    In this section we develop in more detail the formulation used for metaballs used in \cref{chap:erosion} for the simulation of particle erosion on implicit terrains.
}

We use the following formula to evaluate a metaball in space with a center $c$ and of radius $\Radius$:
$$ g(\p) = 1 - \frac{||\p - c||}{\Radius} $$
using the euclidean distance.

We have a total amount $\totalErosion$ to define in this space, so the final metaball function $f$ needs to satisfy the equations \eqref{eq:erosion-function_f_is_metaball} and \eqref{eq:erosion-int_function_f_is_erosion}:
\begin{align}
\label{eq:erosion-function_f_is_metaball}
f(\p) &= \lambda g(\p) \\
\label{eq:erosion-int_function_f_is_erosion}
\int_{\p \in V_{3D}}{f \, dp} &= \totalErosion
\end{align}

First, let's exploit the radial symmetry of the metaball and rewrite $g(\p) = 1 - r$ by using the polar coordinates of the point $\p - c$.

We can then integrate $g$ over the volume $V_{3D}$ as 
\begin{align}
&\int_{0}^{1}{ \int_{0}^{\pi}{ \int_{0}^{2\pi}{ g(r) r^2 \sin(\angl)\, dr} \, d\angl} \, d\anglTwo} \nonumber \\
= &\int_{0}^{1}{ \int_{0}^{\pi}{ \int_{0}^{2\pi}{ (1 - r) r^2 \sin(\angl)\, dr} \, d\angl} \, d\anglTwo} \nonumber \\
&= \int_{0}^{1}{ (1 - r)r^2 \, dr} \times \int_{0}^{\pi}{ \sin{\angl} \, d\angl } \times \int_{0}^{2\pi}{ 1 \, d\anglTwo} \nonumber
\end{align}

We then break down the integrals one by one such that 
$$ \int_{0}^{1}{ (1 - r)r^2 \, dr} = \frac{1}{12} \nonumber$$ 
$$ \int_{0}^{\pi}{ \sin{\angl} \, d\angl } = 2 \nonumber$$ 
$$ \int_{0}^{2\pi}{ 1 \, d\anglTwo} = 2 \pi \nonumber$$

By combining all these integrals, we get $\int{g} = \frac{1}{12} \times 2 \times 2\pi = \frac{\pi}{3}$.

So given $\int{f} = \erosionAmount$ and $\int{f} = \lambda \int{g}$, we can deduce that $\lambda = \frac{\totalErosion}{\int{g}} = \frac{3}{\pi}\totalErosion$.

From \eqref{eq:erosion-function_f_is_metaball} we finally get 
\begin{align} 
\label{eq:erosion-proofErosionMetaball}
f(\p) = \frac{3 \totalErosion}{\pi} \left(1 - \frac{||\p - c||}{\Radius} \right)
\end{align}
, representing the rate of change on the evaluation function of the terrain surface.

The integration in the voxel space is out of the scope of this paper and a numerical solution is instead proposed in \cref{sec:erosion-application_on_voxels}.