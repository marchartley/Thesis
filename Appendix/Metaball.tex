\resetgraphicspath
\appendtographicspath{{./Appendix/figures/Metaball}}


% \chapter{Computation of a metaball}
% \label{sec:erosion-appendix_metaball}

% \shortAbstract{
%     In this section we develop in more detail the formulation used for metaballs used in \cref{chap:erosion} for the simulation of particle erosion on implicit terrains.
% }

% We use the following formula to evaluate a metaball in space with a center $\Center$ and of radius $\Radius$:
% $$ g(\p) = 1 - \frac{||\p - \Center||}{\Radius} $$
% using the euclidean distance.

% We have a total amount $\totalErosion$ to define in this space, so the final metaball function $f$ needs to satisfy the equations \eqref{eq:erosion-function_f_is_metaball} and \eqref{eq:erosion-int_function_f_is_erosion}:
% \begin{align}
% \label{eq:erosion-function_f_is_metaball}
% f(\p) &= \lambda g(\p) \\
% \label{eq:erosion-int_function_f_is_erosion}
% \int_{\p \in V_{3D}}{f \, dp} &= \totalErosion
% \end{align}

% First, let's exploit the radial symmetry of the metaball and rewrite $g(\p) = 1 - r$ by using the spherical coordinates of the point $\p - \Center$.

% We can then integrate $g$ over the volume $V_{3D}$ as:
% \begin{align}
% &\int_{0}^{1}{ \int_{0}^{\pi}{ \int_{0}^{2\pi}{ g(r) r^2 \sin(\angl)\, dr} \, d\angl} \, d\anglTwo} \nonumber \\
% = &\int_{0}^{1}{ \int_{0}^{\pi}{ \int_{0}^{2\pi}{ (1 - r) r^2 \sin(\angl)\, dr} \, d\angl} \, d\anglTwo} \nonumber \\
% &= \int_{0}^{1}{ (1 - r)r^2 \, dr} \times \int_{0}^{\pi}{ \sin{\angl} \, d\angl } \times \int_{0}^{2\pi}{ 1 \, d\anglTwo} \nonumber
% \end{align}

% We then break down the integrals one by one such that 
% $$ \int_{0}^{1}{ (1 - r)r^2 \, dr} = \frac{1}{12} \nonumber$$ 
% $$ \int_{0}^{\pi}{ \sin{\angl} \, d\angl } = 2 \nonumber$$ 
% $$ \int_{0}^{2\pi}{ 1 \, d\anglTwo} = 2 \pi \nonumber$$

% By combining all these integrals, we get $\int{g} = \frac{1}{12} \times 2 \times 2\pi = \frac{\pi}{3}$.

% So given $\int{f} = \erosionAmount$ and $\int{f} = \lambda \int{g}$, we can deduce that $\lambda = \frac{\totalErosion}{\int{g}} = \frac{3}{\pi}\totalErosion$.

% From \eqref{eq:erosion-function_f_is_metaball} we finally get 
% \begin{align} 
% \label{eq:erosion-proofErosionMetaball}
% f(\p) = \frac{3 \totalErosion}{\pi} \left(1 - \frac{||\p - \Center||}{\Radius} \right)
% \end{align}
% , representing the rate of change on the evaluation function of the terrain surface.

% The integration in the voxel space is out of the scope of this paper and a numerical solution is instead proposed in \cref{sec:erosion-application_on_voxels}.


\chapter{Computation of a metaball}
\label{sec:erosion-appendix_metaball}

\shortAbstract{
    In this section we develop in more detail the formulation used for metaballs in \cref{chap:erosion} for the simulation of particle erosion on implicit terrains.
}

\section{Linear metaball derivation}

We use the following formula to evaluate a metaball in space with a center $\Center$ and radius $\Radius$:
\begin{align}
    \eqlabel{eq:metaball-formula}{Definition of the linear metaball field}
    g(\p) = \max\!\left(1 - \frac{\abs{\p - \Center}}{\Radius},\,0\right),
\end{align}
where the Euclidean distance is used, and $g$ is clamped to zero outside the sphere of radius $\Radius$.

We have a total amount $\totalErosion$ to define in this space, so the final metaball function $f$ must satisfy
\begin{align}
    \eqlabel{eq:erosion-function_f_is_metaball}{Linear erosive metaball definition and conservation condition}
    f(\p) &= \lambda\, g(\p), 
    \quad
    % \eqlabel{eq:erosion-int_function_f_is_erosion}{Erosive metaball integral condition}
    \int_{\p \in B_{\Radius}(\Center)} f(\p) \, d\p &= \totalErosion,
\end{align}
where $B_{\Radius}(\Center)$ denotes the ball of radius $\Radius$ centered at $\Center$.

Exploiting radial symmetry, we set $r = \frac{\abs{\p - \Center}}{\Radius} \in [0,1]$ and write $g(\p) = 1 - r$.  
In spherical coordinates of the normalised point $\frac{(\p - \Center)}{\Radius}$, the volume element is $\Radius^3\, r^2 \sin(\angl)\, dr\, d\angl\, d\anglTwo$.  
The integral of $g$ over $B_{\Radius}(\Center)$ becomes
\begin{align}
    \eqlabel{eq:metaball-linear-metaball-integral}{Integration of the linear metaball over its support}
    &\Radius^3 \int_{0}^{1} \int_{0}^{\pi} \int_{0}^{2\pi} (1 - r) \, r^2 \, \sin(\angl)\, dr\, d\angl\, d\anglTwo \nonumber \\
    &= \Radius^3 \left[ \int_{0}^{1} (1 - r) r^2 \, dr \right]
       \left[ \int_{0}^{\pi} \sin{\angl} \, d\angl \right]
       \left[ \int_{0}^{2\pi} 1 \, d\anglTwo \right] \nonumber \\
    &= \Radius^3 \times \frac{1}{12} \times 2 \times 2\pi \nonumber \\
    &= \frac{\pi}{3} \, \Radius^3.
\end{align}

Given $\int f = \totalErosion$ from \cref{eq:erosion-int_function_f_is_erosion} and $\int f = \lambda \int g$, we obtain
\begin{align}
    \eqlabel{eq:metaball-scaling-computation}{Computation of the linear metaball scaling factor}
    \lambda = \frac{\totalErosion}{\int g} = \frac{3}{\pi\,\Radius^3}\,\totalErosion.
\end{align}

From \eqref{eq:erosion-function_f_is_metaball}, the normalised linear metaball is
\begin{align} 
    \eqlabel{eq:erosion-proofErosionMetaball}{Final expression of the erosive linear metaball}
    f(\p) = \frac{3 \,\totalErosion}{\pi\,\Radius^3} \;
    \max\!\left(1 - \frac{\abs{\p - \Center}}{\Radius},\,0\right),
\end{align}
representing the rate of change applied to the terrain's evaluation function.

\subsection*{Derivatives}
Let $\rho = \abs{\p - \Center}$, $r=\frac{\rho}{\Radius}$, and $\vec{u} = \frac{(\p - \Center)}{\rho}$ (the unit radial vector).  
For $0 < r < 1$,
\begin{align}
    \eqlabel{eq:metaball-derivative}{Gradient and Laplacian of the erosive linear metaball}
    \nabla f(\p) = -\frac{\lambda}{\Radius} \, \vec{u},
    \qquad
    \Delta f(\p) = -\frac{2\lambda}{\Radius^2\, r}.
\end{align}
We define by convention $\nabla f(\Center) = 0$.  
The function $f$ is only $C^0$: the derivative $\phi'(r)$ of its radial profile is discontinuous at $r=0$ and $r=1$, and the Laplacian diverges as $r \to 0$.

The integration in voxel space is out of the scope of this appendix; a numerical solution is instead proposed in \cref{sec:erosion-application_on_voxels}.


\section{Other possible radial falloff functions}
\label{sec:erosion-appendix-other-falloffs}

The linear falloff in \cref{eq:erosion-proofErosionMetaball} is only one possible choice of radial profile. We write
\begin{align*}
    x = \frac{\abs{\p - \Center}}{\Radius}, \qquad
    g(\p) = \phi(x),
\end{align*}
where $\phi: [0,\infty) \to \R_{\ge 0}$ is compactly supported, i.e. $\phi(x) = 0$ for $x \ge 1$. If a smoother boundary is desired, pick $\phi$ such that $\phi(1) = 0$ and $\phi'(1) = 0$ (or higher-order vanishing derivatives).

We want to decompose the final function as $f(\p) = \lambda\,\phi(x)$ with $\lambda$ a scaling factor. Using spherical coordinates and radial symmetry,
\begin{align}
    \eqlabel{eq:metaball-general-formula}{General integral form for a radial metaball kernel}
    \int_{\p\in\R^3} g(\p)\,d\p
    = 4\pi\,\Radius^3 \underbrace{\int_{0}^{1} \phi(x)\,x^2\,dq}_{J_\phi},
\end{align}
so that, given the total amount $\totalErosion$, the normalisation reads
\begin{align}
    \eqlabel{eq:metaball-general-J}{Normalisation of a general metaball kernel}
    \lambda = \frac{\totalErosion}{4\pi\,\Radius^3\,J_\phi},
    \qquad
    f(\p) = \lambda\,\phi\!\left(\frac{\abs{\p - \Center}}{\Radius}\right).
\end{align}

For implementation, it is convenient to record the radial derivatives once. Let $\rho=\abs{\p - \Center}$, $x=\frac{\rho}{\Radius}$, and $\vec{u}=\frac{(\p-\Center)}{\rho}$ (defined for $\rho>0$). For $0<x<1$,
\begin{align}
    \eqlabel{eq:metaball-general-derivatives}{Gradient and Laplacian of a general metaball kernel}
    \nabla f(\p) = \frac{\lambda}{\Radius}\,\phi'(x)\,\vec{u},
    \qquad
    \Delta f(\p) = \frac{\lambda}{\Radius^2}\!\left[\phi''(x) + \frac{2}{x}\,\phi'(x)\right].
\end{align}
By convention, we set $\nabla f(\Center)=0$. The smoothness at $x=0$ and $x=1$ depends on $\phi$; if $\phi'(0)\neq 0$ or $\phi'(1)\neq 0$, the gradient is discontinuous at the corresponding point.

\newcommand{\kernelcard}[8]{%
\begin{minipage}[t]{0.45\linewidth}
    \vspace{0pt}
    \textbf{#1} (\cref{fig:metaball-alternatives-#2})\\[0.25em]
    \setlength{\abovedisplayskip}{4pt}\setlength{\belowdisplayskip}{4pt}
    \begin{align}
        \eqlabel{eq:metaball-alternatives-#2}{Analytic expressions for "#1" metaball kernel}
        \phi(x)   &\,=\, #3 \nonumber\\
        \phi'(x)  &\,=\, #4 \nonumber\\
        \phi''(x) &\,=\, #5 \nonumber\\
        J_\phi    &\,=\, #6 \nonumber\\
        \lambda   &\,=\, #7
    \end{align}
    \small \emph{#8}
\end{minipage}\hfill
\begin{minipage}[t]{0.5\linewidth}
    \begin{figure}[H]
      \centering
    \vspace{0pt}
    \autofitgraphics{%
      metaball_#2_Q-0-1_R-0-75.png,%
      depos_#2_Q-0-1_R-0-75.png}
    \autofitgraphics{
      erosion_#2_Q-0-5_R-0-75.png,%
      slope_#2.png%
    }
    \caption[#1 profile]{Top left: #1 profile. Top right: deposition on a plane $f_\text{plane}(x, y)=-y$ with $\totalErosion=0.1$ and $\Radius=0.75$. Bottom left: erosion with $\totalErosion=-0.5$ and $\Radius=0.75$. Bottom right: multiple eroding metaballs on an inclined plane. }
    \label{fig:metaball-alternatives-#2}
    % \caption{#1 profile. Visuals are 2D slices through $\Center$; image parameters use $Q\equiv\totalErosion$ and $R\equiv\Radius$.}
  \end{figure}
  \end{minipage}
}


\kernelcard{Linear}{linear}
{ \max(1-x,0) }
{ -1 }
{ 0 }
{ \tfrac{1}{12} }
{ \dfrac{3\,\totalErosion}{\pi\,\Radius^3} }
{Continuity: $C^0$; gradient discontinuous at $x=0$ and $x=1$. A standard piecewise-linear choice.}

\kernelcard{Smoothstep}{smoothstep}
{ \max\left(1-3x^2+2x^3, 0\right) }
{ -6x+6x^2 }
{ -6+12x }
{ \tfrac{1}{15} }
{ \dfrac{15\,\totalErosion}{4\pi\,\Radius^3} }
{Continuity: $C^1$ with $\phi'(0)=\phi'(1)=0$; widely used in graphics as the cubic smoothstep \cite{Perlin2002}.}

\kernelcard{Smootherstep}{smootherstep}
{ \max\left(1-10x^3+15x^4-6x^5, 0\right) }
{ -30x^2+60x^3-30x^4 }
{ -60x+180x^2-120x^3 }
{ \tfrac{5}{84} }
{ \dfrac{21\,\totalErosion}{5\pi\,\Radius^3} }
{Continuity: $C^2$ with $\phi'(0)=\phi'(1)=0$; quintic smootherstep popularized by Perlin for higher-order smoothness \cite{Perlin2002}.}

\kernelcard{Wendland}{wendland}
{ \max\left((1-x)^4(1+4x), 0\right) }
{ -20x(1-x)^3 }
{ -20(1-x)^2(1-4x) }
{ \tfrac{1}{42} }
{ \dfrac{21\,\totalErosion}{2\pi\,\Radius^3} }
{Continuity: $C^2$ in 3D; a compactly supported, positive definite RBF of minimal degree \cite{Wendland1995}.}

\kernelcard{Biweight kernel}{biweight}
{ \max\left((1-x^2)^2, 0\right) }
{ -4x(1-x^2) }
{ -4+12x^2 }
{ \tfrac{8}{105} }
{ \dfrac{105\,\totalErosion}{32\pi\,\Radius^3} }
{Continuity: $C^1$ at $x=1$ (and $\phi'(0)=0$). A quartic kernel from density estimation \cite{Silverman1998, Wand1993}).}

\kernelcard{Triweight kernel}{triweight}
{ \max\left((1-x^2)^3, 0\right) }
{ -6x(1-x^2)^2 }
{ -6(1-x^2)(1-5x^2) }
{ \tfrac{16}{315} }
{ \dfrac{315\,\totalErosion}{64\pi\,\Radius^3} }
{Continuity: $C^2$ at $x=1$ (and $\phi'(0)=0$). A sextic kernel in the KDE literature \cite{Silverman1998,Wand1993}.}

\kernelcard{Raised cosine}{cosine}
{ \max\left(\tfrac{1+\cos(\pi x)}{2}, 0\right) }
{ -\tfrac{\pi}{2}\sin(\pi x) }
{ -\tfrac{\pi^2}{2}\cos(\pi x) }
{ \tfrac{1}{6}-\tfrac{1}{\pi^2} }
{ \dfrac{3\pi\,\totalErosion}{(2\pi^2-12)\,\Radius^3} }
{Continuity: $C^\infty$ on $(0,1)$ with $\phi'(0)=\phi'(1)=0$; a standard compactly supported kernel in statistics \cite{Soh2013}.}

Any other compactly supported $\phi$ can be used: we just need to compute $J_\phi$ once (analytically or numerically) and plug it into \eqref{eq:metaball-general-J}.